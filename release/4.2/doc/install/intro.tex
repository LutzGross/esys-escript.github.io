%%%%%%%%%%%%%%%%%%%%%%%%%%%%%%%%%%%%%%%%%%%%%%%%%%%%%%%%%%%%%%%%%%%%%%%%%%%%%%
% Copyright (c) 2003-2015 by The University of Queensland
% http://www.uq.edu.au
%
% Primary Business: Queensland, Australia
% Licensed under the Open Software License version 3.0
% http://www.opensource.org/licenses/osl-3.0.php
%
% Development until 2012 by Earth Systems Science Computational Center (ESSCC)
% Development 2012-2013 by School of Earth Sciences
% Development from 2014 by Centre for Geoscience Computing (GeoComp)
%
%%%%%%%%%%%%%%%%%%%%%%%%%%%%%%%%%%%%%%%%%%%%%%%%%%%%%%%%%%%%%%%%%%%%%%%%%%%%%%

\chapter{Introduction}
This document describes how to install \emph{esys-Escript}\footnote{For the rest of the document we will drop the \emph{esys-}} on to your computer.
To learn how to use \esfinley please see the Cookbook, User's guide or the API documentation.
If you use the Debian or Ubuntu and you have installed the \texttt{python-escript-doc} package then the documentation 
will be available in the directory\\
\file{/usr/share/doc/python-escript-doc}, otherwise (if you haven't done so already) you can download the documentation bundle 
from launchpad.



\esfinley is primarily developed on Linux desktop, SGI ICE and \macosx systems.
It can be installed in two ways:
\begin{enumerate}
  \item Binary packages -- ready to run with no compilation required. These are available for recent Debian and Ubuntu distributions.
  \item From source -- that is, it must be compiled for your machine.
  This will be required if you are running anything other than Debian/Ubuntu 
  or if extra functionality is required.
\end{enumerate}

See the site \url{https://answers.launchpad.net/escript-finley} for online help.
Chapter~\ref{chap:bin} describes how to install binary packages on Debian/Ubuntu systems.
Chapter~\ref{chap:source} covers installing from source.



