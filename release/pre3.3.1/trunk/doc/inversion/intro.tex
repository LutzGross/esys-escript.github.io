
%%%%%%%%%%%%%%%%%%%%%%%%%%%%%%%%%%%%%%%%%%%%%%%%%%%%%%%%%%%%%%%%%%%%%%%%%%%%%%
% Copyright (c) 2003-2013 by University of Queensland
% http://www.uq.edu.au
%
% Primary Business: Queensland, Australia
% Licensed under the Open Software License version 3.0
% http://www.opensource.org/licenses/osl-3.0.php
%
% Development until 2012 by Earth Systems Science Computational Center (ESSCC)
% Development since 2012 by School of Earth Sciences
%
%%%%%%%%%%%%%%%%%%%%%%%%%%%%%%%%%%%%%%%%%%%%%%%%%%%%%%%%%%%%%%%%%%%%%%%%%%%%%%


\chapter*{Overview}\label{sec:Intro}
this document provides some documentation for the 
usage of the \downunder module for \python. The module is 
designed to perform the inversion of geophysical data such as gravity and magnetic 
data using parallel supercomputer. The solution approach bases entirely on the
finite element method and is therefore different from the usual approach based on Green's functions
and linear algebra techniques. 

\downunder is implemented on using the \escript solver environment for \python and 
is distributed as part of the \escript package through \url{https://launchpad.net/escript-finley}. We refer 
to the \escript documentation~\cite{ESCRIPT} for installation. An basic introduction
into \escript can be found in~\cite{ESCRIPTCOOKBOOK}

This document has to parts: In part~\ref{part1} we give an introduction on how to solve inversion
using the \downunder module for the inversion of gravity and of magnetic data
for the joint inversion of gravity and magnetic data. The part is designed for high-level users and does not
require any programming skills. 

Part~\ref{part2} gives more details on the mathematical methods being used and the 
program infrastructure. It is the intention of the part to given users a deeper understanding of how
\downunder is implemented and also open the ddor for users with more experience in \python programming 
to build their own inversion programs using \downunder components and the \escript infrastructure. 




%\begin{figure}
%    \centering
%    \includegraphics{components.pdf}
%    \caption{The main components of the inversion toolbox}
%    \label{fig:components}
%\end{figure}

