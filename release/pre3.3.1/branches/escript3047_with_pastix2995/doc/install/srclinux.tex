%%%%%%%%%%%%%%%%%%%%%%%%%%%%%%%%%%%%%%%%%%%%%%%%%%%%%%%%
%
% Copyright (c) 2003-2010 by University of Queensland
% Earth Systems Science Computational Center (ESSCC)
% http://www.uq.edu.au/esscc
%
% Primary Business: Queensland, Australia
% Licensed under the Open Software License version 3.0
% http://www.opensource.org/licenses/osl-3.0.php
%
%%%%%%%%%%%%%%%%%%%%%%%%%%%%%%%%%%%%%%%%%%%%%%%%%%%%%%%%

\section{Installing from source for \linux}
\label{sec:srclinux}

\subsection{Preliminaries}

The following instructions assume you are running the \filename{bash} shell.
Comments are indicated with \# characters.

Make sure you have the following installed:
\begin{itemize}
 \item \filename{g++} and associated tools.
 \item \filename{make}
 % I suspect that these are only needed by VTK and if we aren't using it anymore they could be removed
%  \item \filename{libXext.so}\footnote{In Debian this is in the libXext-dev package.}
%  \item \filename{libxt.so}\footnote{In Debian this is in the libxt-dev package.}
\end{itemize}

To compile matplotlib you will also need the following\footnote{For Debian and Ubuntu users, installing \filename{libfreetype6-dev} and \filename{libpng-dev} will be sufficient.} (if your distribution separates development files,
make sure to get the development packages):
\begin{itemize}
 \item \filename{freetype2}
\item \filename{zlib}
\item \filename{libpng}
\end{itemize}



You will also need a copy of the \esfinley source code.
If you retrieved the source using subversion, don't forget that one can use the export command instead of checkout to get a smaller copy.
For additional visualization functionality see \Sec{sec:addfunc}.

These instructions will produce the following directory structure:
\begin{itemize}
 \item[] \filename{stand} \begin{itemize}
  \item[] \filename{escript.d}
  \item[] \filename{pkg}
  \item[] \filename{pkg_src}
  \item[] \filename{build}
  \item[] \filename{doc}
 \end{itemize}
\end{itemize}

Before you start copy the \esfinley source into the \filename{escript.d} directory.
The following instructions refer to software versions in the \filename{escript-support-3-src} bundle.
If you download your own versions of those packages substitute their version numbers and names as appropriate.
There are a number of uses of the \filename{make} command in the following instructions.
If your computer has multiple cores/processors you can speed up the compilation process by adding -j 2 after the make command.
For example to use all processors on a computer with 4 cores:
\begin{shellCode}
 make
\end{shellCode}
becomes
\begin{shellCode}
 make -j 4
\end{shellCode}

\begin{shellCode}
mkdir stand
cd stand
mkdir build doc pkg pkg_src
export PKG_ROOT=$(pwd)/pkg
\end{shellCode}

\subsection{Building the dependencies}

Copy the compressed sources for the packages into \filename{stand/pkg_src}.
If you are using the support bundles, decompress them in the stand directory:
\begin{shellCode}
tar -xjf escript-support-3-src.tar.bz2
\end{shellCode}

Copy documentation files into \filename{doc} then unpack the archives:

\begin{shellCode}
cd build
tar -jxf ../pkg_src/Python-2.6.2.tar.bz2
tar -jxf ../pkg_src/boost_1_39_0.tar.bz2
tar -zxf ../pkg_src/scons-1.2.0.tar.gz
tar -zxf ../pkg_src/numpy-1.3.0.tar.gz
tar -zxf ../pkg_src/netcdf-4.0.tar.gz
tar -zxf ../pkg_src/matplotlib-0.98.5.3.tar.gz
\end{shellCode}

\begin{itemize}

\item Build Python:
\begin{shellCode}
cd Python*
./configure --prefix=$PKG_ROOT/python-2.6.2 --enable-shared 2>&1 \
  | tee tt.configure.out
make 
make install 2>&1 | tee tt.make.out

cd ..

export PATH=$PKG_ROOT/python/bin:$PATH
export PYTHONHOME=$PKG_ROOT/python
export LD_LIBRARY_PATH=$PKG_ROOT/python/lib:$LD_LIBRARY_PATH

pushd ../pkg
ln -s python-2.6.2/ python
popd
\end{shellCode}

Run the new python executable to make sure it works.

\item Now build NumPy:
\begin{shellCode}
cd numpy-1.3.0
python setup.py build
python setup.py install --prefix $PKG_ROOT/numpy-1.3.0
cd ..
pushd ../pkg
ln -s numpy-1.3.0 numpy
popd
export PYTHONPATH=$PKG_ROOT/numpy/lib/python2.6/site-packages:$PYTHONPATH
\end{shellCode}

% \begin{shellCode}
% cd numarray-1.5.2
% 
% python setup.py install \
%  --gencode --install-lib=$PKG_ROOT/numarray-1.5.2/lib \
%  --install-headers=$PKG_ROOT=$PKG_ROOT/numarray-1.5.2/include/numarray \ 
%    2>&1 | tee tt.install.out
% 
% 
% export PYTHONPATH=$PKG_ROOT/numarray/lib:$PYTHONPATH
% cd ..
% pushd ../pkg
% ln -s numarray-1.5.2 numarray
% popd
% \end{shellCode}

\item Next build scons:
\begin{shellCode}
cd scons-1.2.0
python setup.py install --prefix=$PKG_ROOT/scons-1.2.0

export PATH=$PKG_ROOT/scons/bin:$PATH
cd ..
pushd ../pkg
ln -s scons-1.2.0 scons
popd
\end{shellCode}

\item The Boost libraries...:
\begin{shellCode}
pushd ../pkg
mkdir boost_1_39_0
ln -s boost_1_39_0 boost
popd
cd boost_1_39_0
./bootstrap.sh --with-libraries=python --prefix=$PKG_ROOT/boost
./bjam
./bjam install --prefix=$PKG_ROOT/boost --libdir=$PKG_ROOT/boost/lib
export LD_LIBRARY_PATH=$PKG_ROOT/boost/lib:$LD_LIBRARY_PATH
cd ..
pushd ../pkg/boost/lib/
ln *.so.* libboost_python.so
popd
\end{shellCode}

\item ...and NetCDF:
\begin{shellCode}
cd netcdf-4.0
CFLAGS="-O2 -fPIC -Df2cFortran" CXXFLAGS="-O2 -fPIC -Df2cFortran" \
FFLAGS="-O2 -fPIC -Df2cFortran" FCFLAGS="-O2 -fPIC -Df2cFortran" \
./configure --prefix=$PKG_ROOT/netcdf-4.0

make 
make install

export LD_LIBRARY_PATH=$PKG_ROOT/netcdf/lib:$LD_LIBRARY_PATH
cd ..
pushd ../pkg
ln -s netcdf-4.0 netcdf
popd
\end{shellCode}

\item Finally matplotlib:
\begin{shellCode}
cd matplotlib-0.98.5.3
python setup.py build
python setup.py install --prefix=$PKG_ROOT/matplotlib-0.98.5.3
cd ..
pushd ../pkg
ln -s matplotlib-0.98.5.3 matplotlib
popd
cd ..
\end{shellCode}

\end{itemize}

% \subsection{VTK support}
% VTK is only required for pyvisi. To build it you need CMake and Mesa.
% The packages can be downloaded independently or in the \filename{escript-support-visi-3-src}.
% If you will not be using pyvisi, then skip to \Sec{sec:compileescriptlinux}
% 
% Copy the compressed sources for the packages into \filename{stand/pkg_src}.
% If you are using the support bundles, decompress them in the stand directory.
% \begin{shellCode}
% tar -xjf escript-support-visi-3-src.tar.bz2
% \end{shellCode}
% 
% \begin{shellCode}
% cd build
% tar -jxf ../pkg_src/MesaLib-7.2.tar.bz2
% tar -zxf ../pkg_src/vtk-5.2.1.tar.gz
% tar -zxf ../pkg_src/vtkdata-5.2.1.tar.gz
% tar -zxf ../pkg_src/cmake-2.6.3.tar.gz
% \end{shellCode}
% 
% \begin{itemize}
% 
% \item Build CMake:
% \begin{shellCode}
% cd cmake-2.6.3
% ./configure --prefix=$PKG_ROOT/cmake-2.6.3 2>&1 | tee tt.configure
% make 
% make install
% 
% export PATH=$PKG_ROOT/cmake/bin:$PATH
% cd ..
% pushd ../pkg
% ln -s cmake-2.6.3 cmake
% popd
% \end{shellCode}
% 
% \item Build Mesa:
% \begin{shellCode}
% cd Mesa-7.2
% ./configure --prefix=$PKG_ROOT/mesa-7.2 --enable-gl-osmesa --with-driver=xlib
% 
% make 
% make install
% 
% export LD_LIBRARY_PATH=$PKG_ROOT/mesa/lib:$LD_LIBRARY_PATH
% cd ..
% pushd ../pkg
% ln -s mesa-7.2 mesa
% popd
% \end{shellCode}
% These instructions do not compile MesaDemos or GLUT.
% If you need to check if Mesa compiled correctly, then the demos are a good test.
% 
% \item Finally, build VTK:
% \begin{shellCode}
% cd VTK
% cmake .
% \end{shellCode}
% 
% Now edit the \filename{CMakeCache.txt} file and make the following changes.
% Where .... appears please replace it with the absolute path to the pkg directory.
% For example, replace \filename{CMAKE_INSTALL_PREFIX:PATH=..../vtk-5.2.1} with
% \filename{CMAKE_INSTALL_PREFIX:PATH=/home/bob/stand/pkg/vtk-5.2.1}
% (Search for the text before the =).
% \begin{shellCode}
% BUILD_EXAMPLES:BOOL=OFF
% BUILD_SHARED_LIBS:BOOL=ON
% CMAKE_INSTALL_PREFIX:PATH=..../vtk-5.2.1
% CMAKE_VERBOSE_MAKEFILE:BOOL=TRUE
% VTK_OPENGL_HAS_OSMESA:BOOL=TRUE
% VTK_USE_64BIT_IDS:BOOL=ON
% VTK_WRAP_PYTHON:BOOL=ON
% VTK_USE_MANGLED_MESA:BOOL=OFF
% \end{shellCode}
% 
% Now rerun cmake (it won't work but it adds some variables you need).
% 
% \begin{shellCode}
% cmake .
% \end{shellCode}
% 
% Edit \filename{CMakeCache.txt} and change the following variables:
% 
% \begin{shellCode}
% VTK_USE_OFFSCREEN:BOOL=ON
% VTK_USE_TK:BOOL=OFF
% OSMESA_INCLUDE_DIR:PATH=..../mesa/include
% OSMESA_LIBRARY:FILEPATH=..../mesa/lib/libOSMesa.so
% PYTHON_INCLUDE_PATH:PATH=..../python/include/python2.6
% PYTHON_LIBRARY:FILEPATH=..../python/lib/libpython2.6.so
% OPENGL_INCLUDE_DIR:PATH=..../mesa/include
% OPENGL_gl_LIBRARY:FILEPATH=..../mesa/lib/libGL.so
% \end{shellCode}
% 
% The following steps will take a while so grab a coffee while it compiles.
% \begin{shellCode}
% cmake .
% make
% chmod +w Utilities/vtktiff/tif_fax3sm.c
% make install
% 
% cd ../../pkg
% ln -s vtk-5.2.1 vtk
% cd ..
% \end{shellCode}
% 
% \end{itemize}

\subsection{Compiling escript}\label{sec:compileescriptlinux}

Change to the directory containing your escript source (\filename{stand/escript.d}), then:

\begin{shellCode}
cd escript.d/scons
cp linux_standalone_options_example.py YourMachineName_options.py

echo $PKG_ROOT
\end{shellCode}
Where \texttt{YourMachineName} is the name of your computer as returned by the hostname command.
If the name contains non-alphanumeric characters, then you will need to replace them with underscores.
For example the options file for \texttt{bob-desktop} would be named \filename{bob_desktop_options.py}.

Edit the options file and put the value of PKG_ROOT between the quotes in the PKG_ROOT= line.
For example:
\begin{shellCode}
PKG_ROOT="/home/bob/stand/pkg"
\end{shellCode}

\begin{shellCode}
cd ../bin
\end{shellCode}

Modify the STANDALONE line of \filename{run-escript} to read:
 
STANDALONE=1

Start a new terminal and go to the \filename{stand} directory.

\begin{shellCode}
export PATH=$(pwd)/pkg/scons/bin:$PATH
cd escript.d
eval $(bin/run-escript -e)
scons
\end{shellCode}

If you wish to test your build, then you can do the following. 
Note this may take a while if you have a slow processor and/or less than 1GB of RAM.
\begin{shellCode}
scons all_tests
\end{shellCode}

\subsection{Cleaning up}
Once you are satisfied, the \filename{escript.d/build} and \filename{\$PKG_ROOT/build} directories can be removed.

If you \emph{really} want to save space and do not wish to be able to edit or recompile \esfinley, you can remove the following:
\begin{itemize}
 \item From the \filename{escript.d} directory:\begin{itemize}
\item Everything except: \filename{bin}, \filename{include}, \filename{lib}, \filename{esys},
\filename{README_LICENSE}.
\item Hidden files, which can be removed using
\begin{shellCode}
find . -name '.?*' | xargs rm -rf
\end{shellCode}
in the \filename{escript.d} directory.
\end{itemize}
\item from the \filename{pkg} directory:
\begin{itemize}
\item  \filename{scons}, \filename{scons-1.2.0}, \filename{cmake-2.6.3} and \filename{cmake}
\end{itemize}
\item \filename{package\_src}\footnote{Do not remove this if you intend to redistribute.}.
\end{itemize}

Please note that removing all these files may make it more difficult for us to diagnose problems.


