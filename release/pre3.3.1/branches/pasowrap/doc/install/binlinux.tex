%%%%%%%%%%%%%%%%%%%%%%%%%%%%%%%%%%%%%%%%%%%%%%%%%%%%%%%%
%
% Copyright (c) 2003-2010 by University of Queensland
% Earth Systems Science Computational Center (ESSCC)
% http://www.uq.edu.au/esscc
%
% Primary Business: Queensland, Australia
% Licensed under the Open Software License version 3.0
% http://www.opensource.org/licenses/osl-3.0.php
%
%%%%%%%%%%%%%%%%%%%%%%%%%%%%%%%%%%%%%%%%%%%%%%%%%%%%%%%%

\section{Linux binary installation}
\label{sec:binlinux}

\esfinley can be installed as a stand-alone bundle, containing all the required dependencies.
Alternatively, if we have a package for your distribution you can use the standard tools to install.


For more information on using the \file{run-escript} command please see the User's Guide.

If you are using Debian~5.0(``Lenny''), Ubuntu~8.10(``Intrepid Ibex'') or greater, then see Section~\ref{sec:debian}.
For other linux distributions refer to Section~\ref{sec:standalonelinux}.

\subsection{Debian and Ubuntu}\label{sec:debian}

At the time of this writing we only produce deb's for the i386 and amd64 architectures.
The package file will be named \file{escript-X-D_A.deb} where \texttt{X} is the version, \texttt{D} is the distribution codename (eg ``\texttt{lenny}'' or ``\texttt{jaunty}'') and \texttt{A} is the architecture.
For example, \file{escript-3.0-1-lenny_amd64.deb} would be the file for lenny for 64bit processors.
To install \esfinley download the appropriate \file{.deb} file and execute the following commands as root (you need to be in the directory containing the file):

\begin{verse}
\textbf{(For users of Ubuntu~10.10 \textit{``Maverick Meercat''} only)}\\
You will need to either install \texttt{aptitude}\footnote{Unless you are short on disk space \texttt{aptitude} is recommended} or replace use \texttt{apt-get} where this guide uses \texttt{aptitude}.
\begin{shellCode}
sudo apt-get install aptitude
\end{shellCode}
\end{verse}

\begin{shellCode}
dpkg --unpack escript*.deb
aptitude install escript
\end{shellCode}

Installing escript should not remove any packages from your system.
If aptitude suggests removing escript, then choose 'N'.
It should then suggest installing some dependencies choose 'Y' here.
If it suggests removing escript-noalias then agree.

If you use sudo (for example on Ubuntu) enter the following instead:
\begin{shellCode}
sudo dpkg --unpack escript*.deb
sudo aptitude install escript
\end{shellCode}

This should install \esfinley and its dependencies on your system.
Please notify the development team if something goes wrong.

\subsection{Stand-alone bundle}\label{sec:standalonelinux}

If there is no package available for your distribution, you may be able to use one of our stand alone bundles.
These come in two parts: escript itself (\file{escript_3.2_i386.tar.bz2}) and a group of required programs (\file{escript-support_3.0_i386.tar.bz2}) (Note that the support bundle is version~3.0 not 3.2) . For $64$-bit Intel and Amd processors substitute \texttt{amd64} for \texttt{i386}.
This point release uses the same support bundle as previous releases so if you already have it you don't need a new version.
\begin{shellCode}
tar -xjf escript-support_3.0_i386.tar.bz2
tar -xjf escript_3.2_i386.tar.bz2
\end{shellCode}
This will produce a directory called \file{stand} which contains a stand-alone version of \esfinley and its dependencies.
You can rename or move it as is convenient to you, no installation is required.
Test your installation by running:
\begin{shellCode}
stand/escript.d/bin/run-escript
\end{shellCode}
This should give you a normal python shell.
If you wish to save on typing you can add \file{x/stand/escript.d/bin}\footnote{or whatever you renamed \texttt{stand} to.} to your \texttt{PATH} variable (where x is the absolute path to your install).

