

%%%%%%%%%%%%%%%%%%%%%%%%%%%%%%%%%%%%%%%%%%%%%%%%%%%%%%%%
%
% Copyright (c) 2003-2008 by University of Queensland
% Earth Systems Science Computational Center (ESSCC)
% http://www.uq.edu.au/esscc
%
% Primary Business: Queensland, Australia
% Licensed under the Open Software License version 3.0
% http://www.opensource.org/licenses/osl-3.0.php
%
%%%%%%%%%%%%%%%%%%%%%%%%%%%%%%%%%%%%%%%%%%%%%%%%%%%%%%%%

\section*{Introduction}
This document describes how to install \esfinley on your computer.
To learn how to use \esfinley please see the User guide or, for
more detailed information, the API documentation.

\esfinley is developed primarily on Linux desktop systems,  SGI ICE and MacOS.
Binary distributions (discussed in Chapter~\ref{chap:bin}) are available for the following platforms:
\begin{itemize}
\item Debian and Ubuntu Linux distributions ($32$-bit i686) (.deb package)
\item Linux desktop systems with gcc (stand-alone bundle)
\item MacOS X Leopard systems with gcc (stand-alone bundle)
\item Windows systems with Visual Studio (stand-alone bundle) 
\end{itemize}

Compilation from source is discussed in Chapter~\ref{chap:src}.

You can test your installation using the Python scripts in \filename{examples.zip} or \filename{examples.tar.gz}
\footnote{These will should either be in \filename{escript.d/release/doc} or in the case of Debian, in \filename{/usr/share/doc/escript}.}.
A simple test is
\begin{shellCode}
 escript poission.py
\end{shellCode}
It should produce a file called \filename{u.xml} (which can be removed).
If this is successful, then the main features of \escript have been sucessfully installed.

For visualisation we suggest \filename{visit}\footnote{\url{https://wci.llnl.gov/codes/visit/}} or \filename{mayavi}\footnote{\url{http://mayavi.sourceforge.net}}.

For online help, see the site \url{https://answers.launchpad.net/escript-finley}.