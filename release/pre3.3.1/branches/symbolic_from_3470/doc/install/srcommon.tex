%%%%%%%%%%%%%%%%%%%%%%%%%%%%%%%%%%%%%%%%%%%%%%%%%%%%%%%%
%
% Copyright (c) 2003-2012 by University of Queensland
% Earth Systems Science Computational Center (ESSCC)
% http://www.uq.edu.au/esscc
%
% Primary Business: Queensland, Australia
% Licensed under the Open Software License version 3.0
% http://www.opensource.org/licenses/osl-3.0.php
%
%%%%%%%%%%%%%%%%%%%%%%%%%%%%%%%%%%%%%%%%%%%%%%%%%%%%%%%%

% This file contains material common to all src distributions.

% The original version of this content came from the esscc twiki page maintained by ksteube

This chapter describes how to build \esfinley from source assuming that the dependencies are already installed (for example using precompiled packages for your OS).
Section~\ref{sec:deps} describes the dependencies, while Section~\ref{sec:compilesrc} gives the compile instructions.

If you would prefer to build all the dependecies from source in the escript-support packages please see Chapter~\ref{chap:allsrc}.
\esfinley is known to compile and run on the following systems:
\begin{itemize}
 \item \linux using gcc\footnote{There are some problems with \openmp under gcc prior to version 4.3.2. Also do not link the gomp library with gcc 4.3.3 - it causes problems.}
\item \linux using icc on SGI ICE 8200. (At this stage, we do not recommend building with intel-11)\footnote{There is a bug in icpc-11 related to exception handling and openmp. This results in binaries which crash.}
\item \macosx using gcc
\item \winxp using the Visual C compiler (we do not specifically discuss Windows builds in this guide).
\end{itemize}

If you have compiled a previous version of \esfinley please note that the format of the \file{..._options.py} file has changed so you will not be able to reuse old options file for this build.


\section{External dependencies}
\label{sec:deps}
The following external packages are required in order to compile and run \esfinley.
Where version numbers are specified, more recent versions can probably be subsituted.
You can either try the standard/precompiled packages available for your operating system or you can download and build them from source.
The advantage of using existing packages is that they are more likely to work together properly.
You must take greater care if downloading sources separately.

\begin{itemize}
 \item python-2.5.1 (\url{http://python.org}) \\-
        Python interpreter (you must compile with shared libraries.)
 \item numpy 1.1.0 (\url{http://numpy.scipy.org}) \\-
        Arrays for Python
 \item boost-1.35 (\url{http://www.boost.org}) \\-
        Interface between C++ and Python
 \item scons-0.989.5 (\url{http://www.scons.org/}) \\-
        Python-based alternative to \texttt{make}.
\end{itemize}

The version numbers given here are not strict requirements, more recent (and in some cases older) versions are very likely to work.
The following packages should be sufficient (but not necessarily minimal) for Debian 5.0 (``Lenny''):
python-dev, libboost-python1.35-dev, scons, python-numpy, g++.

These packages may be required for some of the optional capabilities of the system:

\begin{itemize}
 \item netcdf-3.6.2 (\url{http://www.unidata.ucar.edu/software/netcdf}) \\-
        Used to save data sets in binary form for checkpoint/restart (must be compiled with -fPIC)
%  \item vtk-5.0.4 (\url{http://www.vtk.org}) \\-
%         Used to save VTK files for visualization
%   \begin{itemize}
%    \item cmake-2.4.6 (\url{http://www.cmake.org}) \\-
%         Required to build VTK
%    \item mesa-7.0.3 (\url{http://www.mesa3d.org})\\-
%         Free OpenGL replacement used by VTK
%   \end{itemize}
 \item netpbm (\url{http://netpbm.sourceforge.com}) \\-
        Tools for producing movies from images
 \item parmetis-3.1 (\url{http://glaros.dtc.umn.edu/gkhome/metis/parmetis/overview}) \\-
        Optimization of the stiffness matrix
 \item MKL \\(\url{http://www.intel.com/cd/software/products/asmo-na/eng/307757.htm}) \\-
        Intel's Math Kernel Library for use with their C compiler.
\item Lapack - Available in various versions from various places. \\ 
Currently only used to invert dense square matrices larger than 3x3. 
\end{itemize}

The following packages might be useful for mesh generation:
\begin{itemize}
 \item gmsh-2.2.0 (\url{http://www.geuz.org/gmsh}) \\-
        Mesh generation and viewing
  \begin{itemize}
   \item fltk-1.1.9 (\url{http://www.fltk.org}) \\-
        Required to build gmsh 
   \item gsl-1.10 (\url{http://www.gnu.org/software/gsl}) \\-
        Required to build gmsh 
  \end{itemize}
 \item triangle-1.6 (\url{http://www.cs.cmu.edu/~quake/triangle.html}) \\-
        Two-dimensional mesh generator and Delaunay triangulator.
\end{itemize}

Packages for visualization:
\begin{itemize}
 \item mayavi-1.5 (\url{http://mayavi.sourceforge.net}) \\-
        MayaVi is referenced in our User's Guide for viewing VTK files
 \item visit-1.11.2 (\url{https://wci.llnl.gov/codes/visit/}) \\-
        A featureful visualization system with movie-making capabilities.
\end{itemize}

The source code comes with an extensive set of unit tests. If you would like to
build those to verify your installation you need:
\begin{itemize}
 \item cppunit-1.12.1 (\url{http://cppunit.sourceforge.net})
\end{itemize}

\section{Compilation}\label{sec:compilesrc}
Throughout this section we will assume that the source code is uncompressed in a directory called \file{escript.d}.
You can call the directory anything you like, provided that you make the change before you compile.

You need to indicate where to find the external dependencies.
To do this, create a file in the \file{escript.d/scons} directory called \file{x_options.py} where ``x'' is the name of your computer (output of the \texttt{hostname} command).
Please note that if your hostname has non-alphanumeric characters in it (eg - ) you need to replace them with underscores.
For example the options file for \texttt{bob-desktop} would be named \file{bob_desktop_options.py}.

From now on all paths will be relative to the top level of the source.
As a starting point copy the contents of one of the following files into your options file:
\begin{itemize}
\item \file{scons/TEMPLATE_linux.py} (\linux and \macosx desktop)
\item \file{scons/TEMPLATE_windows.py} (\winxp)
\end{itemize}

This options file controls which features and libraries your build of escript will attempt to use.
For example to use OpenMP or MPI you will need to enable it here.
If you want to try escript out without customising your build, then change
directories to \file{escript.d} and enter
\begin{shellCode}
scons 
\end{shellCode}
If this works you can skip to Section~\ref{sec:diff}.
If not, then you will need to make some modications to the the file.
Read on.

The template files contain all available options with a comment explaining the
purpose of each.
Check through the file and ensure that the relevant paths and names are correct
for your system and that you enable optional components that you wish to use.
For example, to use netCDF, find the netcdf-related lines, uncomment them
(i.e. remove the \# at the beginning of the lines) and change them according
to your installation:
\begin{shellCode}
netcdf = True
netcdf_prefix = '/opt/netcdf4'
netcdf_libs = ['netcdf_c++', 'netcdf']
\end{shellCode}

In this example, netCDF \emph{header} files must be located in
\file{/opt/netcdf4/include}\footnote{or \ldots/include32 or \ldots/include64 or \ldots/inc}
and the \emph{libraries} in \file{/opt/netcdf4/lib}\footnote{or \ldots/lib32 or \ldots/lib64}.
If this scheme does not apply to your installation then you may also specify
the include-path and library-path directly like so:
\begin{shellCode}
netcdf_prefix = ['/usr/local/include/netcdf', '/usr/local/lib']
\end{shellCode}
The order is important: the first element in the list is the
\emph{include}-path, the second element is the \emph{library}-path and both
must be specified.

If a line in the options file is commented out and you do not require the
feature, then it can be ignored.
To actually compile (if you have $n$ processors, then you can use \texttt{scons -j$n$} instead):

\begin{shellCode}
cd escript.d
scons
\end{shellCode}

As part of its output, scons will tell you the name of the options file it used
as well as a list of features and whether they are enabled for your build.
If you enabled an optional dependency and the library or include files could
not be found you will be notified and the build will stop.

Note, that you can override all settings from the options-file on the scons
command line. For example, if you usually build an optimized version but would
like to build a debug version into a separate directory without changing your
default settings, you can use:
\begin{shellCode}
scons debug=1 prefix=debugbuild
\end{shellCode}
This will install the binaries and libraries built in debug mode into
directories underneath \file{./debugbuild}.

To run the unit test suite that comes with the source code issue
\begin{shellCode}
scons py_tests
\end{shellCode}
(If you have cppunit installed you can run additional tests using \texttt{scons all_tests}.

Grab a coffee or two while the tests compile and run.
An alternative method is available for running tests on \openmp and \mpi builds.

\subsection{Compilation with \openmp}
\openmp is generally enabled by setting compiler and linker switches. For the
most common compilers these are automatically set by build system and all you
have to do is set the \texttt{openmp} option to True in your options file. If
this does not work or your compiler is different, then consult your compiler
documentation for the precise switches to use and modify the \texttt{omp_flags}
and \texttt{omp_ldflags} variables in your options file.
For example, for gcc compilers which support \openmp use:
\begin{shellCode}
openmp = True
omp_flags = '-fopenmp'
omp_ldflags = '-fopenmp'
\end{shellCode}
(The two latter settings can also be left out as this is the default OpenMP on gcc.)

You can test your \openmp-enabled build, e.g. using 4 threads by issuing
\begin{shellCode}
export ESCRIPT_NUM_THREADS=4
scons py_tests
\end{shellCode}

\subsection{Compilation with \mpi}
You need to have \mpi preinstalled on your system.
There are a number of implementations so we do not provide any specific advice
here.
Set the following variables in your options file to according to your
installation:
\begin{itemize}
 \item \texttt{mpi} \\
    which \mpi implementation (flavour) is used. Valid values are
    \begin{itemize}
        \item[\texttt{none}] \mpi is disabled
        \item[\texttt{MPT}] SGI MPI implementation \\
            \url{http://techpubs.sgi.com/library/manuals/3000/007-3687-010/pdf/007-3687-010.pdf}
        \item[\texttt{MPICH}] Argonne's MPICH implementation \\
            \url{http://www.mcs.anl.gov/research/projects/mpi/mpich1/}
        \item[\texttt{MPICH2}] Argonne's MPICH version 2 implementation \\
            \url{http://www.mcs.anl.gov/research/projects/mpi/mpich2/}
        \item[\texttt{OPENMPI}] Open MPI \\
            \url{http://www.open-mpi.org/}
        \item[\texttt{INTELMPI}] Intel MPI \\
            \url{http://software.intel.com/en-us/intel-mpi-library/}
    \end{itemize}
 \item \texttt{mpi_prefix} \\
    where to find \mpi headers and libraries (see netCDF example above)
 \item \texttt{mpi_libs} \\
    which libraries to link to.
\end{itemize}

To test your build using 6 processes enter:
\begin{shellCode}
export ESCRIPT_NUM_PROCS=6
scons py_tests
\end{shellCode}
and on $2$ processes with $4$ threads each (provided \openmp is enabled)\footnote{Unless your system has $8$ cores expect this to be slow}:
\begin{shellCode}
export ESCRIPT_NUM_THREADS=4
export ESCRIPT_NUM_PROCS=2
scons py_tests
\end{shellCode}
Alternatively, you can give a hostfile
\begin{shellCode}
export ESCRIPT_NUM_THREADS=4
export ESCRIPT_HOSTFILE=myhostfile
scons py_tests
\end{shellCode}
Note that depending on your \mpi flavour it may be required to start a daemon
before running the tests under \mpi.

\subsection{Difficulties}\label{sec:diff}

\subsubsection{Mismatch of runtime and build libraries}
Most external libraries used by \esfinley are linked dynamically.
This can lead to problems if after compiling \esfinley these libraries are
updated.
The same applies to the installed Python executable and libraries.
Whenever these dependencies change on your system you should recompile
\esfinley to avoid problems at runtime such as load errors or segmentation
faults.

\subsubsection{\openmp builds segfault running examples}
One known cause for this is linking the \file{gomp} library with escript built using gcc 4.3.3.
While you need the \texttt{-fopenmp} switch you should not need to link \file{gomp}.


