\chapter{The Data Inversion Problem}
\label{APP: FITTER}

We solve the following nonlinear PDE for the unknown solution $u_i$:
\begin{equation} \label{APP FIT EQU 1a}
\int_{\Omega} v_{i,j} \cdot X_{ij} + v_{i} \cdot Y_{i} \; dx
+ \int_{\partial \Omega}  v_{i} \cdot y_{i} \; ds  = 0 
\end{equation}
for all smooth $v_i$ with $v_i=0$ where $q_i>0$ and
\begin{equation} \label{APP FIT EQU 1b}
u_i=r_i \mbox{ where } q_i>0
\end{equation}
where $X_{ij}$ and $Y_i$ are non-linear functions of the solution $u_k$ and its gradient $u_{k,l}$
and $y_i$ is a function of solution $u_k$. 

The equation may depend on set of parameters $p_i$. It is the task to 
give values of the parameters $p_i$ such that the solution gives the best 
approximation of given measurements. In mathematical terms we need to minimize
a so-called cost function $J$ which measures the distance of the solution
to the data over the set of valid parameters. A typical example 
for a cost function measuring the gradient of a scalar solution $u$ 
in a given direction $d_i$ against measured data $\hat{g}$ is given as
\begin{equation}\label{APP FIT EQU 2a}
J_{data}(u) = \frac{1}{2}\int_{\Omega}  \chi \cdot ( d_i u_{,i} - \hat{g})^2 dx
\end{equation} 
where $\chi$ is a weighting function which has a non-negative value where data are available. 
Typically, $\chi$ is to be the inverse of the square of the deviation of the measurements or zero.
If the parameter $p_i$ is spatially variable a regularization term needs to be
added into the cost function in order to get a unique solution. Typically,
for a scalar parameter $p$ the regularization term takes the from
\begin{equation}\label{APP FIT EQU 3}
J_{reg}(p) =  \int_{\Omega} \frac{1}{2} \cdot (p_{,i}p_{,i})^{2} dx
\end{equation} 
The cost function to be minimized then takes the from
\begin{equation}\label{APP FIT EQU 2a}
J(p,u) = \frac{1}{2}\int_{\Omega}  \chi \cdot ( d_i u_{,i} - \hat{g})^2
+ (p_{,i}p_{,i})^{2} \;
dx
\end{equation} 
A more general from is given as 
\begin{equation}\label{APP FIT EQU 4}
J(u,p) = \int_{\Omega} H\; dx +  \int_{\partial \Omega} h \; ds
\end{equation} 
where $H$ is a scalar, possibly spatially variable  function 
of the solution $u_i$ and the parameter $p_i$ and their gradients
and  $H$ is a scalar, possibly spatially variable  function 
of the solution $u_i$ and the parameter $p_i$. For example~(\ref{APP FIT EQU 2a})
one has
\begin{equation}\label{APP FIT EQU 4a}
H = \frac{1}{2} \chi \cdot (d_i u_{,i} - \hat{g})^2 +  \frac{1}{2} (p_{,i}p_{,i})^{2} 
\end{equation} 
So task is to minimize the cost function $J$ over $u$ and $p$  
subject to the PDE~(\ref{APP FIT EQU 1a} connection $p$ and $u$. The secondary condition
is mixed into the cost function using a Lagrangean multiplier before the variation is calculated:
\begin{equation}\label{APP FIT EQU 5}
J(u,p,\lambda) = \int_{\Omega} H \; dx + \int_{\partial \Omega} h \; ds
+ \int_{\Omega} \lambda_{i,j} \cdot X_{ij} + \lambda_{i} \cdot Y_{i} \; dx
+ \int_{\partial \Omega}  \lambda_{i} \cdot y_{i} \; ds
\end{equation}
Notice that the Lagrangean multiplier needs to fullfull the constraint
\begin{equation} \label{APP FIT EQU 5b}
\lambda_{i}=0 \mbox{ where } q_i>0
\end{equation}

We can rearrange $J$ to 
\begin{equation}\label{APP FIT EQU 5}
J(u,p,\lambda) = \int_{\Omega} Z \; dx 
+  \int_{\partial \Omega} z \; ds
\end{equation}
with 
\begin{align}\label{APP FIT EQU 6}
 Z =  H+ \lambda_{i,j} \cdot X_{ij} +  \lambda_{i} \cdot Y_{i} \\
z= h  + \lambda_{i} \cdot y_{i} 
\end{align}

We are taking variation along $p$:
\begin{equation}\label{APP FIT EQU 10}
\int_{\Omega} \fracp{Z}{p_{i,j}}  \cdot  (\delta p)_{i,j} + \fracp{Z}{p_{i}} \cdot  (\delta p)_{i}
\; dx +  \int_{\partial \Omega} \fracp{z}{p_{i}}  \cdot  (\delta p)_{i} \; ds =0
\end{equation}
along $u$:
\begin{align}\label{APP FIT EQU 11}
\int_{\Omega} \fracp{Z}{u_{i,j}}  \cdot (\delta u)_{i,j} + \fracp{Z}{u_{i}} \cdot  (\delta u)_{i}
\; dx +  \int_{\partial \Omega} \fracp{z}{u_{i}}  \cdot  (\delta u)_{i} \; ds =0
\end{align}
and $\lambda$:
\begin{equation}\label{APP FIT EQU 12}
\int_{\Omega} X_{ij}  \cdot (\delta \lambda)_{i,j}  +  Y_{i}  \cdot (\delta \lambda)_{i}  \; dx
+ \int_{\partial \Omega}   y_{i}  \cdot  (\delta \lambda)_{i}  \; ds = 0
\end{equation}
This defines a system of non-linear PDEs for the unknown solution $\widehat{u} = (p,u,\lambda)$. With 
$\widehat{v} = (\delta p,\delta \lambda, \delta u)$\footnote{Notice that in comparison to the solution
the corresponding components for $u$ and $\lambda$ are swapped in order to bring strong couplings into the main-diagonal.}
we can write equations~(\ref{APP FIT EQU 10})-(\ref{APP FIT EQU 12}) in the form:
\begin{equation} \label{APP FIT EQU 13}
\int_{\Omega} \widehat{v}_{i,j} \cdot \widehat{X}_{ij} + \widehat{v}_{i} \cdot \widehat{Y}_{i} \; dx
+ \int_{\partial \Omega}  \widehat{v}_{i} \cdot \widehat{y}_{i} \; ds  = 0 
\end{equation}
with
\begin{align}\label{APP FIT EQU 15}
\widehat{X}_{:j} = \left[ \fracp{Z}{p_{:,j}}, X_{:j}, \fracp{Z}{u_{:,j}} \right] \\
\widehat{Y}_{:} = \left[ \fracp{Z}{p_{:}}, Y_{:}, \fracp{Z}{u_{:}} \right] \\
\widehat{y}_{:} = \left[ \fracp{z}{p_{:}}, y_{:}, \fracp{z}{u_{:}} \right] 
\end{align}
In some cases values for the parameter are known. So similar to the constraint~(\ref{APP FIT EQU 1b}) fro the solution
we need to observe a constraint for the parameter $p_i$:
\begin{equation} \label{APP FIT EQU 12a}
p_i=rp_i \mbox{ where } qp_i>0
\end{equation}
So for the composed solution $\widehat{u} = (p,u,\lambda)$ we need to observe the constraint
\begin{equation} \label{APP FIT EQU 12b}
\widehat{u}_i=\widehat{r}_i \mbox{ where } \widehat{q}_i>0
\end{equation}
with 
\begin{align}\label{APP FIT EQU 12c}
\widehat{q}_{:j} = \left[ qp_{:}, q_{:}, q_{:} \right] \\
\widehat{r}_{:} = \left[ rp_{:}, r_{:}, 0\right] \\
\end{align}
