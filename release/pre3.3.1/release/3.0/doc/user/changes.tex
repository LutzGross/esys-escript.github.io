
%%%%%%%%%%%%%%%%%%%%%%%%%%%%%%%%%%%%%%%%%%%%%%%%%%%%%%%%
%
% Copyright (c) 2009 by University of Queensland
% Earth Systems Science Computational Center (ESSCC)
% http://www.uq.edu.au/esscc
%
% Primary Business: Queensland, Australia
% Licensed under the Open Software License version 3.0
% http://www.opensource.org/licenses/osl-3.0.php
%
%%%%%%%%%%%%%%%%%%%%%%%%%%%%%%%%%%%%%%%%%%%%%%%%%%%%%%%%

\section{Changes from previous releases}
\label{app:changes}

\subsection*{2.0 to 3.0}
\begin{itemize}
\item The major change here was replacing \module{numarray} with \numpy.
For general instructions on converting scripts to use numpy see \url{http://www.stsci.edu/resources/software_hardware/numarray/numarray2numpy.pdf}.
The specific changes to \escript are:
\begin{itemize}
  \item getValueOfDataPoint() which returned a \module{numarray}.array has been replaced by
 getTupleForDataPoint() which returns a \PYTHON tuple containing
the components of the data point. In the case of matricies or higher ranked data, the tuples will be nested. Use 
\numpy.array(data.getTupleForDataPoint()) if a \numpyNDA object is required.
 \item getValueOfGlobalDataPoint has similarly been replaced by getTupleForGlobalDataPoint().
 \item integrate(data) now returns a \numpyNDA instead of a \module{numarray}.array.
\end{itemize}
Any python methods which previously accepted \module{numarray} objects will accept \numpy objects instead.

\item
The way solver options are defined for \LinearPDE objects has been changed. There is now a \SolverOptions object attached to the \LinearPDE object which is handeling the options of solvers used to solve the PDE. The following changes apply:  
\begin{itemize}
\item The \method{setTolerance} and \method{setAbsoluteTolerance} methods have been removed. Use now \method{getSolverOptions().setTolerance} 
and \method{getSolverOptions().setAbsoluteTolerance}

\item The \method{setSolverPackage} and \method{setSolverMethod} methods have been removed. Use now \method{getSolverOptions().setPackage},
\method{getSolverOptions().setSolverMethod} and
\method{getSolverOptions().setPreconditioner}.

\item The \method{setSolverPackage} and \method{setSolverMethod} methods have been removed. Use now \method{getSolverOptions().setPackage},
\method{getSolverOptions().setSolverMethod} and
\method{getSolverOptions().setPreconditioner}.

\item The static class variables defining packages, solvers and preconditioners have been removed and are now accessed via the corresponding  static class variables in \SolverOptions. For instance use \method{SolverOptions.PCG} instead of 
\method{LinearPDE.PCG} to select the preconditioned conjugate gradient method.

\item The \method{getSolution} takes now no argument. Use the corresponding
methods of the \SolverOptions object returned by \method{getSolverOptions()} 
to set values, e.g. use 
\method{getSolverOptions().setVerbosityOn()} instead of argument \code{verbose=True}
and \method{getSolverOptions().setIterMax(1000)} instead of argument \code{iter_max=1000}
\end{itemize}

\item
The \pyvisi module from previous releases has been deprecated and will no longer be supported.
It is still present in the source code and can still be used if you compile \escript from source.
It will not be available in binary releases.
Its use is discouraged.
The documentation for \pyvisi can be found in Appendix~\ref{PYVISI CHAP}.

\end{itemize}
