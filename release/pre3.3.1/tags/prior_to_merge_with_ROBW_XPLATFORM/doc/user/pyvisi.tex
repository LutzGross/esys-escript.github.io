%
%           Copyright © 2006 by ACcESS MNRF
%               \url{http://www.access.edu.au
%         Primary Business: Queensland, Australia.
%   Licensed under the Open Software License version 3.0
%      http://www.opensource.org/license/osl-3.0.php
%
\chapter{The module \pyvisi}
\declaremodule{extension}{pyvisi}
\modulesynopsis{visualization interface}

The idea behind is to provide an easy to use interface and unified to a variety of
visualization tools like \VTK, \OpenDX and \GnuPlot.

The following script illustartes the usage of \pyvisi together with the 
\VTK library:
\begin{python}
from esys.pyvisi import *                # base level visualisation stuff
from esys.pyvisi.renderers.vtk import *  # vtk renderer module
from esys.escript import *
from esys.finley import Brick
# now make some data of some kind
domain = Brick(3,5,7)  # a Finley domain
vectorData = domain.getX()  # get vector data from the domain nodes
# define the scene object
scene = Scene()
# create an ArrowPlot object
plot = ArrowPlot(scene)
# add the plot to the scene
scene.add(plot)
# assign some data to the plot
plot.setData(vectorData)
# render the scene
scene.render()
# saving a scene
scene.save(file="example.jpg", format="jpeg")
\begin{python}
A \Scene is a container for all of the kinds of things you want to put into your plot,
for instance, images, domaines, arrow plots, contour plots, spheres etc.
The renderer is specified in the scene initialisation. In fact the 
\code{from esys.pyvisi.renderers.vtk import *} provides the specific implementation for 
\VTK


\begin{verbose}
class ArrowPlot3D(Plot):
    """
    Arrow field plot in three dimensions
    """
    def __init__(self, scene):
        """
        Initialisation of the ArrowPlot3D class
        
        @param scene: The Scene to render the plot in
        @type scene: Scene object
    def setData(self, *dataList, **options):
        """
        Set data to the plot

        @param dataList: List of data to set to the plot
        @type dataList: tuple

        @param options: Dictionary of extra options
        @type options: dict

        @param fname: Filename of the input vtk file
        @type fname: string

        @param format: Format of the input vtk file ('vtk' or 'vtk-xml')
        @type format: string

	@param vectors: the name of the vector data in the vtk file to use
	@type vectors: string
        """
class ArrowPlot(Plot):
    """
    Arrow field plot
    """
    def __init__(self, scene):
        """
        Initialisation of the ArrowPlot class
        
        @param scene: The Scene to render the plot in
        @type scene: Scene object
        """
    def setData(self, *dataList, **options):
        """
        Set data to the plot

        @param dataList: List of data to set to the plot
        @type dataList: tuple

	@param options: Dictionary of extra options
	@type options: dict

	@param fname: the name of the input vtk file
	@type fname: string

	@param format: the format of the input vtk file ('vtk' or 'vtk-xml')
	@type format: string

	@param vectors: the name of the vector data in the vtk file to use
	@type vectors: string
        """
class Axes(Plot):
    """
    Axes class
    """
    def __init__(self):
        """
        Initialisation of Axes object
        """
        debugMsg("Called Axes.__init__()")
        Plot.__init__(self)

class BallPlot(Plot):
    """
    Ball plot
    """
    def __init__(self, scene):

    def setData(self, points=None, 
            fname=None, format=None,
            radii=None, colors=None, tags=None):
        """
        Set data to the plot
        @param points: the array to use for the points of the sphere
        locations in space
        @type points: float array

        @param fname: the name of the input vtk file
        @type fname: string

        @param format: the format of the input vtk file ('vtk' or 'vtk-xml')
        @type format: string

        @param radii: the name of the scalar array in the vtk unstructured
        grid to use as the radii of the balls
        @type radii: float array

        @param colors: the name of the scalar array in the vtk unstructured
        grid to use as the colour tags of the balls
        @type colors: string

        @param tags: the name of the scalar array in the vtk unstructured
        grid to use as the colour of the tags of the balls
        @type tags: integer array
        """

class Box(Item):
    """
    Generic class for Box objects

    To define a box one specify one of three groups of things:
      - The bounds of the box: xmin, xmax, ymin, ymax, zmin, zmax
      - The dimensions and origin: width, height, depth and origin
      - The bottom left front and top right back corners: blf, trb
    """

    def __init__(self):
        """
        Initialisation of the Box object
        """
        debugMsg("Called Box.__init__()")
        Item.__init__(self)

        # define a box in many ways, either by its centre and width, height
        # and depth, or by its bounds, xmin, xmax, ymin, ymax, zmin, zmax,
        # or by its bottom left front and top right back points.

        # set the default bounds
        self.xmin = -0.5
        self.xmax = 0.5
        self.ymin = -0.5
        self.ymax = 0.5
        self.zmin = -0.5
        self.zmax = 0.5

        # set the default origin (the centre of the box)
        self.origin = ((self.xmin + self.xmax)/2.0, 
                (self.ymin + self.ymax)/2.0, 
                (self.zmin + self.zmax)/2.0)

        # set the default dimensions
        self.width = self.xmax - self.xmin
        self.height = self.ymax - self.ymin
        self.depth = self.zmax - self.zmin

        # set the default blf and trb points
        self.blf = (self.xmin, self.ymin, self.zmin)
        self.trb = (self.xmax, self.ymax, self.zmax)

        # tolerance for calculated variables checking purposes
        self.tolerance = 1e-8

    def setBounds(self, xmin, xmax, ymin, ymax, zmin, zmax):
        """
        Set the bounds of the box
        """
    def getBounds(self):
        """
        Get the current bounds of the box
        """

    def setOrigin(self, xo, yo, zo):
        """
        Set the origin of the box
        """
    def getOrigin(self):
        """
        Get the current origin of the box
        """
        debugMsg("Called Box.getOrigin()")
        return self.origin

    def setWidth(self, width):
        """
        Set the width of the box
        """
    def getWidth(self):
        """
        Get the current box width
        """
        debugMsg("Called Box.getWidth()")
        return self.width

    def setHeight(self, height):
        """
        Set the box height
        """

    def getHeight(self):
        """
        Get the current box height
        """
        debugMsg("Called Box.getHeight()")
        return self.height

    def setDepth(self, depth):
        """
        Set the box depth
        """

    def getDepth(self):
        """
        Get the current box depth
        """
        debugMsg("Called Box.getDepth()")
        return self.depth

    def setBLF(self, bottom, left, front):
        """
        Set the position of the bottom, left, front corner
        """

    def getBLF(self):
        """
        Get the current position of the bottom, left, front corner
        """
        debugMsg("Called Box.getBLF()")
        return self.blf

    def setTRB(self, top, right, back):
        """
        Set the position of the top, right, back corner
        """

    def getTRB(self):
        """
        Get the current position of the top, right, back corner
        """
        debugMsg("Called Box.getTRB()")
        return self.trb


class ClipBox(Box):
    """
    Clip box class: used to clip data sets with a box

    A box in this sense means three planes at right angles to one another
    """

    def __init__(self, plot):
        """
        Intialisation of the ClipBox object
        """

    def setInsideOut(self, insideOut):
        """
        Set the inside out flag
        """

    def getInsideOut(self):
        """
        Get the current value of the inside out flag
        """

class Camera(Item):
    """
    Camera class
    """
    def __init__(self, scene):
        """
        Initialisation of the Camera object

        @param scene: The Scene object to add the Camera object to
        @type scene: Scene object
        """
    def setPosition(self, *pos):
        """
        Set position of camera within scene

        @param pos: Position to set camera in terms of x,y,z coordinates
        @type pos: tuple
        """

    def getPosition(self):
        """
        Get the position of Camera within Scene

        Returns the position in a tuple of form (xPos, yPos, zPos)
        """
        debugMsg("Called Camera.getPosition()")

        return (self.xPos, self.yPos, self.zPos)

    def setFocalPoint(self, *pos):
        """
        Sets the focal point of the Camera with the Scene

        @param pos: Position to set the focal point
        @type pos: tuple
        """

    def getFocalPoint(self):
        """
        Get the position of the focal point of the Camera

        Returns the position of the focal point in a tuple of form 
        (xPos, yPos, zPos)
        """

    def setElevation(self, elevation):
        """
        Set the elevation angle (in degrees) of the Camera

        @param elevation: The elevation angle (in degrees) of the Camera
        @type elevation: float
        """

        return

    def getElevation(self):
        """
        Gets the elevation angle (in degrees) of the Camera
        """

    def setAzimuth(self, azimuth):
        """
        Set the azimuthal angle (in degrees) of the Camera

        @param azimuth: The azimuthal angle (in degrees) of the Camera
        @type azimuth: float
        """

    def getAzimuth(self):
        """
        Get the azimuthal angle (in degrees) of the Camera
        """
class ContourPlot(Plot):
    """
    Contour plot
    """
    def __init__(self, scene):
        """
        Initialisation of the ContourPlot class
        
        @param scene: The Scene to render the plot in
        @type scene: Scene object
        """
    def setData(self, *dataList, **options):
        """
        Set data to the plot

        @param dataList: List of data to set to the plot
        @type dataList: tuple

        @param options: Dictionary of extra options
        @type options: dict

        @param fname: the name of the input vtk file
        @type fname: string

        @param format: the format of the input vtk file ('vtk' or 'vtk-xml')
        @type format: string

        @param scalars: the scalar data in the vtk file to use
        @type scalars: string
        """

class EllipsoidPlot(Plot):
    """
    Ellipsoid plot
    """
    def __init__(self, scene):
        """
        Initialisation of the EllipsoidPlot class

        @param scene: The Scene to render the plot in
        @type scene: Scene object
        """
        debugMsg("Called EllipsoidPlot.__init__()")
        Plot.__init__(self, scene)

        self.renderer = scene.renderer
        self.renderer.addToInitStack("# EllipsoidPlot.__init__()")

        # labels and stuff
        self.title = None
        self.xlabel = None
        self.ylabel = None
        self.zlabel = None
        
        # default values for fname, format and tensors
        self.fname = None
        self.format = None
	self.tensors = None

	# default values for shared info
	self.escriptData = False
	self.otherData = False

        # add the plot to the scene
        scene.add(self)

    def setData(self, *dataList, **options):
        """
        Set data to the plot

        @param dataList: List of data to set to the plot
        @type dataList: tuple

        @param options: Dictionary of keyword options to the method
        @type options: dict

	@param fname: the name of the input vtk file
	@type fname: string

	@param format: the format of the input vtk file ('vtk' or 'vtk-xml')
	@type format: string

	@param tensors: the name of the tensor data in the vtk file to use
	@type tensors: string
        """

class Image(Item):
    """
    Image class.  Generic class to handle image data.
    """
    def __init__(self, scene=None):
        """
        Initialises the Image class object
        
        @param scene: The Scene object to add to
        @type scene: Scene object
        """
        debugMsg("Called Image.__init__()")
        Item.__init__(self)

        if scene is not None:
            self.renderer = scene.renderer
        
    def load(self, fname):
        """
        Loads image data from file.

        @param fname: The filename from which to load image data
        @type fname: string
        """
        debugMsg("Called Image.load()")

        fileCheck(fname)

        return

class JpegImage(Image):
    """
    Subclass of Image class to explicitly handle jpeg images
    """
    def __init__(self, scene=None):
        """
        Initialises the JpegImage class object

        @param scene: The Scene object to add to
        @type scene: Scene object
        """

    def load(self, fname):
        """
        Loads jpeg image data from file.

        @param fname: The filename from which to load jpeg image data
        @type fname: string
        """

class PngImage(Image):
    """
    Subclass of Image class to explicitly handle png images
    """
    def __init__(self, scene=None):
        """
        Initialises the PngImage class object

        @param scene: The Scene object to add to
        @type scene: Scene object
        """

    def load(self, fname):
        """
        Loads png image data from file.

        @param fname: The filename from which to load png image data
        @type fname: string
        """
class BmpImage(Image):
    """
    Subclass of Image class to explicitly handle bmp images
    """
    def __init__(self, scene=None):
        """
        Initialises the BmpImage class object

        @param scene: The Scene object to add to
        @type scene: Scene object
        """
    def load(self, fname):
        """
        Loads bmp image data from file.

        @param fname: The filename from which to load bmp image data
        @type fname: string
        """

class TiffImage(Image):
    """
    Subclass of Image class to explicitly handle tiff images
    """
    def __init__(self, scene=None):
        """
        Initialises the TiffImage class object

        @param scene: The Scene object to add to
        @type scene: Scene object
        """
    def load(self, fname):
        """
        Loads tiff image data from file.

        @param fname: The filename from which to load tiff image data
        @type fname: string
        """
class PnmImage(Image):
    """
    Subclass of Image class to explicitly handle pnm (ppm, pgm, pbm) images
    """
    def __init__(self, scene=None):
        """
        Initialises the PnmImage class object

        @param scene: The Scene object to add to
        @type scene: Scene object
        """
        
    def load(self, fname):
        """
        Loads pnm (ppm, pgm, pbm) image data from file.

        @param fname: The filename from which to load pnm image data
        @type fname: string
        """

class PsImage(Image):
    """
    Subclass of Image class to explicitly handle ps images
    """
    def __init__(self, scene=None):
        """
        Initialises the PsImage class object

        This object is B{only} used for generating postscript output

        @param scene: The Scene object to add to
        @type scene: Scene object
        """

    def load(self, fname):
        """
        Loads ps image data from file.

        B{NOT} supported by this renderer module

        @param fname: The filename from which to load ps image data
        @type fname: string
        """
        debugMsg("Called PsImage.load()")

        # need to check if the file exists
        fileCheck(fname)

        # this ability not handled by this renderer module
        unsupportedError()
        
        return

    def render(self):
        """
        Does PsImage object specific (pre)rendering stuff
        """
        debugMsg("Called PsImage.render()")

        return

class PdfImage(Image):
    """
    Subclass of Image class to explicitly handle pdf images
    """
    def __init__(self, scene=None):
        """
        Initialises the PdfImage class object

        This object is B{only} used for generating pdf output

        @param scene: The Scene object to add to
        @type scene: Scene object
        """

    def load(self, fname):
        """
        Loads pdf image data from file.

        B{NOT} supported by this renderer module

        @param fname: The filename from which to load pdf image data
        @type fname: string
        """

class IsosurfacePlot(Plot):
    """
    Isosurface plot
    """
    def __init__(self, scene):
        """
        Initialisation of the IsosurfacePlot class
        
        @param scene: The Scene to render the plot in
        @type scene: Scene object
        """
    def setData(self, *dataList, **options):
        """
        Set data to the plot

        @param dataList: List of data to set to the plot
        @type dataList: tuple

        @param options: Dictionary of keyword options to the method
        @type options: dict

	@param fname: the name of the input vtk file
	@type fname: string

	@param format: the format of the input vtk file ('vtk' or 'vtk-xml')
	@type format: string

	@param scalars: the name of the scalar data in the vtk file to use
	@type scalars: string
        """

class LinePlot(Plot):
    """
    Line plot
    """
    def __init__(self, scene):
        """
        Initialisation of the LinePlot class
        
        @param scene: The Scene to render the plot in
        @type scene: Scene object
        """

    def setData(self, *dataList, **options):
        """
        Set data to the plot

        @param dataList: List of data to set to the plot
        @type dataList: tuple

	@param options: Dictionary of extra options
	@type options: dict

	@param offset: whether or not to offset the lines from one another
	@type offset: boolean

	@param fname: Filename of the input vtk file
	@type fname: string

	@param format: format of the input vtk file ('vtk' or 'vtk-xml')
	@type format: string

	@param scalars: the name of the scalar data in the vtk file to use
	@type scalars: string
        """

class OffsetPlot(Plot):
    """
    Offset plot
    """
    def __init__(self, scene):
        """
        Initialisation of the OffsetPlot class
        
        @param scene: The Scene to render the plot in
        @type scene: Scene object
        """

    def setData(self, *dataList, **options):
        """
        Set data to the plot

        @param dataList: List of data to set to the plot
        @type dataList: tuple

        @param options: Dictionary of extra options
        @type options: dict

	@param fname: Filename of the input vtk file
	@type fname: string

	@param format: Format of the input vtk file ('vtk' or 'vtk-xml')
	@type format: string

	@param scalars: the name of the scalar data in the vtk file to use
	@type scalars: string
        """
class Plane(Item):
    """
    Generic class for Plane objects
    """

    def __init__(self, scene):
        """
        Initialisation of the Plane object
        """

    def setOrigin(self, x, y, z):
        """
        Set the origin of the plane
        """

    def getOrigin(self):
        """
        Get the current origin of the plane
        """

    def setNormal(self, vx, vy, vz):
        """
        Set the normal vector to the plane
        """

    def getNormal(self):
        """
        Get the current normal vector to the plane
        """

    def mapImageToPlane(self, image):
        # this really needs to go somewhere else!!!
        """
        Maps an Image object onto a Plane object
        """

class CutPlane(Plane):
    """
    Cut plane class: used to cut data sets with a plane

    Cut plane objects define a plane to cut a data set or plot by and return
    the data along the intersection between the data set or plot with the
    defined plane.
    """

    def __init__(self):
        """
        Intialisation of the CutPlane object
        """


class ClipPlane(Plane):
    """
    Class for planes used to clip datasets
    """

    def __init__(self):
        """
        Intialisation of the ClipPlane object
        """

    def setInsideOut(self, insideOut):
        """
        Set the inside out flag
        """

    def getInsideOut(self):
        """
        Get the current value of the inside out flag
        """

class Plot(Item):
    """
    Abstract plot class
    """
    def __init__(self, scene):
        """
        Initialisation of the abstract Plot class
        
        @param scene: The Scene to render the plot in
        @type scene: Scene object
        """

    def setData(self, *dataList, **options):
        """
        Set data to the plot

        @param dataList: List of data to set to the plot
        @type dataList: tuple

	@param options: Dictionary of extra options
	@type options: dict
        """

    def setTitle(self, title):
        """
        Set the plot title

        @param title: the string holding the title to the plot
        @type title: string
        """
        debugMsg("Called setTitle() in Plot()")


    def setXLabel(self, label):
        """
        Set the label of the x-axis

        @param label: the string holding the label of the x-axis
        @type label: string
        """

    def setYLabel(self, label):
        """
        Set the label of the y-axis

        @param label: the string holding the label of the y-axis
        @type label: string
        """

    def setZLabel(self, label):
        """
        Set the label of the z-axis

        @param label: the string holding the label of the z-axis
        @type label: string
        """

    def setLabel(self, axis, label):
        """
        Set the label of a given axis

        @param axis: string (Axis object maybe??) of the axis (e.g. x, y, z)
        @type axis: string or Axis object

        @param label: string of the label to set for the axis
        @type label: string
        """

class Renderer(BaseRenderer):
    """
    A generic object holding a renderer of a Scene().
    """

    def __init__(self):
        """
        Initialisation of Renderer() class
        """
        debugMsg("Called Renderer.__init__()")
        BaseRenderer.__init__(self)

        # initialise some attributes
        self.renderWindowWidth = 640
        self.renderWindowHeight = 480

        # what is the name of my renderer?
        self.name = _rendererName

        # the namespace to run the exec code
        self.renderDict = {}

        # initialise the evalstack
        self._evalStack = ""

        # keep the initial setup of the module for later reuse
        self._initStack = ""

        # initialise the renderer module
        self.runString("# Renderer._initRendererModule")
        self.addToInitStack("import vtk")
        self.addToInitStack("from numarray import *")

__revision__ = '$Revision: 1.33 $'

class Scene(BaseScene):
    """
    The main object controlling the scene.
    
    Scene object methods and classes overriding the BaseScene class.
    """

    def __init__(self):
        """
        The init function
        """

    def add(self, obj):
        """
        Add a new item to the scene

        @param obj: The object to add to the scene
        @type obj: object
        """

    def place(self, obj):
        """
        Place an object within a scene

        @param obj: The object to place within the scene
        @type obj: object
        """

    def render(self, pause=False, interactive=False):
        """
        Render (or re-render) the scene
        
        Render the scene, either to screen, or to a buffer waiting for a save

        @param pause: Flag to wait at end of script evaluation for user input
        @type pause: boolean

        @param interactive: Whether or not to have interactive use of the output
        @type interactive: boolean
        """

    def save(self, fname, format):
        """
        Save the scene to a file

        Possible formats are:
            - Postscript
            - PNG
            - JPEG
            - TIFF
            - BMP
            - PNM

        @param fname: Name of output file
        @type fname: string

        @param format: Graphics format of output file
        @type format: Image object or string
        """

    def setBackgroundColor(self, *color):
        """
        Sets the background color of the Scene

        @param color: The color to set the background to.  Can be RGB or CMYK
        @type color: tuple
        """

    def getBackgroundColor(self):
        """
        Gets the current background color setting of the Scene
        """

    def setSize(self, xSize, ySize):
        """
        Sets the size of the scene.

        This size is effectively the renderer window size.

        @param xSize: the size to set the x dimension
        @type xSize: float

        @param ySize: the size to set the y dimension
        @type ySize: float
        """

    def getSize(self):
        """
        Gets the current size of the scene

        This size is effectively the renderer window size.  Returns a tuple
        of the x and y dimensions respectively, in pixel units(??).
        """
class SurfacePlot(Plot):
    """
    Surface plot
    """
    def __init__(self, scene):
        """
        Initialisation of the SurfacePlot class
        
        @param scene: The Scene to render the plot in
        @type scene: Scene object
        """

    def setData(self, *dataList, **options):
        """
        Set data to the plot

        @param dataList: List of data to set to the plot
        @type dataList: tuple

        @param options: Dictionary of extra options
        @type options: dict

        @param fname: the name of the input vtk file
        @type fname: string

        @param format: the format of the input vtk file ('vtk' or 'vtk-xml')
        @type format: string

        @param scalars: the scalar data in the vtk file to use
        @type scalars: string
        """

class Text(Item):
    """
    Text
    """
    def __init__(self, scene):
        """
        Initialisation of the Text object

        @param scene: the scene with which to associate the Text object
        @type scene: Scene object
        """

    def setFont(self, font):
        """
        Set the current font

        @param font: the font to set
        @type font: string
        """

    def getFont(self):
        """
\end{verbose}
