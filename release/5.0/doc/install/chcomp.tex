%%%%%%%%%%%%%%%%%%%%%%%%%%%%%%%%%%%%%%%%%%%%%%%%%%%%%%%%%%%%%%%%%%%%%%%%%%%%%%
% Copyright (c) 2014-2016 by The University of Queensland
% http://www.uq.edu.au
%
% Primary Business: Queensland, Australia
% Licensed under the Apache License, version 2.0
% http://www.apache.org/licenses/LICENSE-2.0
%
% Development until 2012 by Earth Systems Science Computational Center (ESSCC)
% Development 2012-2013 by School of Earth Sciences
% Development from 2014 by Centre for Geoscience Computing (GeoComp)
%
%%%%%%%%%%%%%%%%%%%%%%%%%%%%%%%%%%%%%%%%%%%%%%%%%%%%%%%%%%%%%%%%%%%%%%%%%%%%%%

\chapter{Changing compilers for package managers}
\label{chap:chcomp}

Here are some brief instructions / hints for changing the compilers used by package managers in OSX and FreeBSD.
If you are intend to change the compiler or python versions, you should install the new compiler first,
then the new version of python.
After this, you can install the other dependencies.
\emph{Please note that neither homebrew, MacPorts nor BSD recommend changing the default $C$ compiler so
do so at your own risk.  If you want OpenMP though, there does not seem to be a choice.}

The following steps seemed to work when we tested them but we cannot make guarantees.

\subsubsection*{Changing compiler on FreeBSD(9.1)}
The following sequence passes our unit tests under OpenMP.
\begin{itemize}
 \item Install \texttt{gcc46} from ports.
 \item Modify \texttt{/etc/make.conf} to set the default compiler to be \texttt{gcc46}
 \item Install the remaining dependencies. 
 \item Configure \escript to build with \openmp.
\end{itemize}

We chose version $4.6$ rather than a later one because one of the optional dependencies (scipy) will try to install it anyway.

\subsubsection*{Changing compiler on MacPorts}
The following sequence passes our unit tests under OpenMP.
\begin{itemize}
 \item \texttt{port install gcc47}
 \item Set the default compiler for the command line with: \texttt{port select --set gcc mp-gcc47}
 \item Set the default compiler for macports by adding the following line to \hfill~\linebreak[4] \file{/opt/local/etc/macports/macports.conf}:
\begin{shellCode} 
#Added by user to coerce macports into using a newer compiler
default_compiler	macports-gcc-4.7 
\end{shellCode}
\item Now install the remainder of the dependencies using \texttt{port -s install X}.
This will build them from source rather than downloading precompiled versions.
(This will unfortunately mean larger downloads.).
In some cases, some dependencies will not build properly from source.
In those cases(where XYZ is the dependency that fails to build), \texttt{port clean XYZ} then \texttt{port install XYZ}.
Then \emph{try to install the rest from source}.

Installing from source via macports is necessary to convince macports to link against the libraries provided by 
your new compiler rather than the default one.
\end{itemize}




\subsubsection*{Changing compiler on Homebrew}
There is no configuration file that needs to be changed in order to use a different compiler.
You do need to do things in the correct order and make sure that you have made any required changes to 
your environment variables before moving on to the next step.
In particular, the compiler must be installed before anything else, followed by Python.
