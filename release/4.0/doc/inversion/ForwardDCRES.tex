%!TEX root = inversion.tex
%%%%%%%%%%%%%%%%%%%%%%%%%%%%%%%%%%%%%%%%%%%%%%%%%%%%%%%%%%%%%%%%%%%%%%%%%%%%%%
% Copyright (c) 2003-2014 by University of Queensland
% http://www.uq.edu.au
%
% Primary Business: Queensland, Australia
% Licensed under the Open Software License version 3.0
% http://www.opensource.org/licenses/osl-3.0.php
%
% Development until 2012 by Earth Systems Science Computational Center (ESSCC)
% Development 2012-2013 by School of Earth Sciences
% Development from 2014 by Centre for Geoscience Computing (GeoComp)
%
%%%%%%%%%%%%%%%%%%%%%%%%%%%%%%%%%%%%%%%%%%%%%%%%%%%%%%%%%%%%%%%%%%%%%%%%%%%%%%

\section{Dc resistivity inversion: 3D}\label{sec:forward DCRES}
This section will discuss dc resistivity\index{DC forward} forward modelling, as well as a escript
class which allows for solutions of these forward problems. The dc resistivity 
forward problem is modelled via the application of ohm's law to the flow of current
through the ground. When sources are treated as a point sources and ohms law 
is written in terms of the potential field, the equation becomes:
\begin{equation} \label{ref:dcres:eq1}
\nabla \cdot (\sigma \nabla \phi) = -I \delta(x-x_s) \delta(y-y_s) \delta(z-z_s)
\end{equation}
Where (X,Y,Z) and $X_s, Y_s, Z_s$ are the coordinates of the observation and source
points respectively. The total potential $\phi$ is split into primary and secondary 
potential $\phi = \phi_p + \phi_s$, where the primary potential is analytically calculated 
as a (flat) half-space background model with conductivity $\sigma_p$. 
The secondary potential is due to conductivity deviations 
from the background model and has its conductivity denoted $\sigma_s$. 
This approach effectively removes the singularities of the Dirac delta 
source and provides more accurate results \cite{rucker2006three}.
An analytical solution is available for the primary potential and is given by:
\begin{equation} \label{ref:dcres:eq2}
\phi_p = \frac{I}{2 \pi \sigma_1 R}
\end{equation}
Where I is the current, and R is the distance from the observation points to the source.
In escript the observation points are the nodes of the domain and R is given by
\begin{equation} \label{ref:dcres:eq3}
R = \sqrt{(x-x_s)^2+(y-y_s)^2 + z^2}
\end{equation}
The secondary potential $\phi_s$ is given by
\begin{equation}\label{ref:dcres:eq4}
-\mathbf{\nabla}\cdot\left(\sigma\,\nabla \phi_s \right)  = 
 \mathbf{\nabla}\cdot\left( \left(\sigma_p-\sigma\right)\,\nabla \phi_p  \right)
\end{equation} 
where $\sigma_p$ is the conductivity of the background half-space.
The weak form of above PDE is given by multiplication of a suitable test function $w$ and integrating over the domain $\Omega$:
\begin{multline}\label{ref:dcres:eq5}
-\int_{\partial\Omega} \sigma\,\nabla \phi_s  \cdot \hat{n} w\,ds +
 \int_{\Omega} \sigma\,\nabla \phi_s  \cdot \nabla w\,d\Omega =\\
-\int_{\partial\Omega} \left(\sigma_p-\sigma\right)\,\nabla \phi_p  
\cdot \hat{n} w\,ds + \int_{\Omega} \left(\sigma_p-\sigma\right)\,\nabla \phi_p  \cdot \nabla w\,d\Omega 
\end{multline}
The integrals over the domain boundary provides the boundary conditions which are
implemented as Dirichlet conditions (zero potential) at all interfaces except the
top, where Neumann conditions apply (no current flux through the air-earth interface).
From the integrals over the domain, the Escript coefficients can be deduced: the 
left-hand-side conforms to Escript coefficient $A$, whereas the right-hand-side agrees
with the coefficient $X$ (see User Guide.)

A number a of different configurations for electrode set-up are available \cite[pg 5]{LOKE2014}.
An escript class is provided for each of the following survey types.
\begin{itemize}
\item Wenner alpha
\item Pole-Pole
\item Dipole-Dipole
\item Pole-Dipole
\item Schlumberger
\end{itemize}
The configurations are comprised of one or more current and potential electrodes
separated by a distance a. Some configurations also specify a quantity n this 
quantity creates a separation distance of $n \times a$, in the classes that follow
the specified value of n is an upper limit. That is n will start at 1 and iterate
up to the value specified.

\subsection{Usage}
The Dc resistivity forward modelling classes are specified as follows.

\begin{classdesc}{WennerSurvey}{self, domain, primaryConductivity, secondaryConductivity,
current, a, midPoint, directionVector, numElectrodes}
\end{classdesc}

\begin{classdesc}{polepoleSurvey}{domain, primaryConductivity, secondaryConductivity, 
current, a, midPoint, directionVector, numElectrodes}
\end{classdesc}

\begin{classdesc}{DipoleDipoleSurvey}{self, domain, primaryConductivity, secondaryConductivity,
current, a, n, midPoint, directionVector, numElectrodes}
\end{classdesc}

\begin{classdesc}{PoleDipoleSurvey}{self, domain, primaryConductivity, secondaryConductivity,
current, a, n, midPoint, directionVector, numElectrodes}
\end{classdesc}

\begin{classdesc}{SchlumbergerSurvey}{self, domain, primaryConductivity, secondaryConductivity,
current, a, n, midPoint, directionVector, numElectrodes}
\end{classdesc}

\noindent Where:
\begin{itemize}
\item Domain is the domain which represent the half-space of interest. 
it is important that a node exists at the points where the electrodes will be placed.
\item Primaryconductivity is a data object which defines the primary conductivity
it should be defined on the ContinuousFunction function space.
\item secondaryconductivity is a data object which defines the secondary conductivity
it should be defined on the ContinuousFunction function space.
\item current is the value of the injection current to be used in amps this is a currently a
constant.
\item a is the electrode separation distance.
\item n is the electrode separation distance multiplier.
\item midpoint is the centre of the survey. Electrodes will spread from this point
in the direction defined by direction vector and in the opposite direction, placing
numElectrodes/2 electrodes on either side.
\item directionVector is defines the direction in which electrodes are spread.
\item numElectrodes is the number of electrodes to be used in the survey.
\end{itemize} 

When calculating the potentials the survey, is moved along the set of electrodes.
The process of moving the electrodes along is repeated for each consecutive value of n.
As n increases less potentials are calculated this is because a greater spacing is
required and hence some electrodes are skipped. The process of building up these
pseudo-sections is covered in greater depth in \cite[pg 19]{LOKE2014}.
These classes all share common member functions described below. For the surveys
where n is not specified only one list will be returned. 

\begin{methoddesc}[]{getPotential}{}
Returns 3 list each made up of a number of list containing primary, secondary and total
potentials differences. Each of the lists contain a list for each value of n.
\end{methoddesc}

\begin{methoddesc}[]{getElectrodes}{}
Returns a list containing the positions of the electrodes
\end{methoddesc}

\begin{methoddesc}[]{getApparentResistivityPrimary}{}
Returns a series of lists containing primary apparent resistivities one for each 
value of n.
\end{methoddesc}

\begin{methoddesc}[]{getApparentResistivitySecondary}{}
Returns a series of lists containing secondary apparent resistivities one for each 
value of n.
\end{methoddesc}

\begin{methoddesc}[]{getApparentResistivityTotal}{}
Returns a series of lists containing Total apparent resistivities one for each 
value of n. This is generally the result of interest.
\end{methoddesc}

The apparent resestivities are calculated by applying a geometric factor to the
measured potentials.