
%%%%%%%%%%%%%%%%%%%%%%%%%%%%%%%%%%%%%%%%%%%%%%%%%%%%%%%%%%%%%%%%%%%%%%%%%%%%%%
% Copyright (c) 2003-2014 by University of Queensland
% http://www.uq.edu.au
%
% Primary Business: Queensland, Australia
% Licensed under the Open Software License version 3.0
% http://www.opensource.org/licenses/osl-3.0.php
%
% Development until 2012 by Earth Systems Science Computational Center (ESSCC)
% Development 2012-2013 by School of Earth Sciences
% Development from 2014 by Centre for Geoscience Computing (GeoComp)
%
%%%%%%%%%%%%%%%%%%%%%%%%%%%%%%%%%%%%%%%%%%%%%%%%%%%%%%%%%%%%%%%%%%%%%%%%%%%%%%


\chapter*{Overview}\label{sec:Intro}
The \downunder module for \Python is designed to perform the inversion of
geophysical data such as gravity and magnetic anomalies using a parallel
supercomputer.
The solution approach bases entirely on the finite element method and is
therefore different from the usual approach based on Green's functions
and linear algebra techniques.
The module is implemented on top of the \escript solver environment for
\Python and is distributed as part of the \escript package through
\url{https://launchpad.net/escript-finley}.
We refer to the \escript documentation~\cite{ESCRIPT} for installation
instructions and to \cite{ESCRIPTCOOKBOOK} for a basic introduction to \escript.

This document is split into two parts:
Part~\ref{part1} provides a tutorial-style introduction to running inversions
with the \downunder module.
Users with minimal or no programming skills should be able to follow the
tutorial which demonstrates how to run inversions of gravity anomaly data,
magnetic anomaly data and the combination of both.
The scripts and data files used in the examples are provided with the \escript
distribution.

Part~\ref{part2} gives more details on the mathematical methods used and the
module infrastructure.
It is the intention of this part to give users a deeper understanding of how
\downunder is implemented and also to open the door for experienced \Python
programmers to build their own inversion programs using \downunder components
and the \escript infrastructure. 

The development project of \downunder is part of the AuScope Inversion Lab. The work is funded 
under Australian Geophysical Observing System, 
see~\url{http://auscope.org.au/site/agos.php}, through the Education Investment Fund of
the Australian Commonwealth (2011-2014) and under the AuScope Sustainability Funding (2011-12)
with the support of the School of Earth Sciences at the University of Queensland, see~\url{http://www.earth.uq.edu.au/}.
