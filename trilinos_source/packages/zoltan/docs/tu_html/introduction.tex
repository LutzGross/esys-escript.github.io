%
% What this document is, what prior knowledge is assumed, and
% brief outline of what is contained in it.
% 
% PLEASE FEEL FREE TO RE-WRITE ANY OF THIS!!
%
\chapter{Introduction}

This document provides an introduction to data partitioning
and load balancing using the Zoltan Toolkit (version 3.1).  
Zoltan is a parallel toolkit for load balancing
(and several other combinatorial scientific computing tasks) 
targeted to the high performance computing community.  
This tutorial only covers load balancing,
and other capabilities in Zoltan are barely mentioned.
The Zoltan library is written in C but can be
linked with C, C++ and Fortran90 applications.  
It requires MPI to run in parallel.
Most of the examples in this tutorial are C language examples.

In Chapter~\ref{cha:lb} we
discuss the load balancing problem in general and describe
common algorithms, which are also supported in Zoltan.
Zoltan's role in addressing the load balancing problem is to 
efficiently partition,
or repartition, the problem data across mulitple processes while an
application is running, when requested to do so by the application.

% In Chapter~\ref{cha:partitioning} we
% describe in more detail the partitioning
% methods available in Zoltan, and try provide a guide
% to choosing the method or methods best suited to your problem.

In Chapter~\ref{cha:using} we briefly explain the interface
through which your application employs Zoltan.  This topic is
convered in more detail in the
Zoltan User's Guide available at
\url{http://cs.sandia.gov/Zoltan/ug_html/ug.html}.
An example can be worth a thousand words, so you may wish to 
skip this chapter and look through the examples in the last section first.

The final chapter of this document (~\ref{cha:ex}) provides
source code examples for simple applications that use Zoltan.
These examples may be found in the \textbf{examples} directory
of the Zoltan source code.  Zoltan is open source and freely available.

You will find additional documents and publications at the
Zoltan web site at \url{http://cs.sandia.gov/Zoltan/}.
The User's Guide is a very useful reference and provides
many details we could not cover here.

