%%%%%%%%%%%%%%%%%%%%%%%%%%%%%%%%%%%%%%%%%%%%%%%%%%%%%%%%%%%%%%%%%%%%%%%%%%%%%%
% Copyright (c) 2003-2018 by The University of Queensland
% http://www.uq.edu.au
%
% Primary Business: Queensland, Australia
% Licensed under the Apache License, version 2.0
% http://www.apache.org/licenses/LICENSE-2.0
%
% Development until 2012 by Earth Systems Science Computational Center (ESSCC)
% Development 2012-2013 by School of Earth Sciences
% Development from 2014 by Centre for Geoscience Computing (GeoComp)
%
%%%%%%%%%%%%%%%%%%%%%%%%%%%%%%%%%%%%%%%%%%%%%%%%%%%%%%%%%%%%%%%%%%%%%%%%%%%%%%

\documentclass{esysdoc}

%%%%%%%%%%%%%%%%%%%%%%%%%%%%%%%%%%%%%%%%%%%%%%%%%%%%%%%%
%
% Copyright (c) 2003-2010 by University of Queensland
% Earth Systems Science Computational Center (ESSCC)
% http://www.uq.edu.au/esscc
%
% Primary Business: Queensland, Australia
% Licensed under the Open Software License version 3.0
% http://www.opensource.org/licenses/osl-3.0.php
%
%%%%%%%%%%%%%%%%%%%%%%%%%%%%%%%%%%%%%%%%%%%%%%%%%%%%%%%%

% Please do not use .ps only packages.
% As a minimum this document should build under pdflatex

\usepackage{color}
\usepackage{xspace}

%Ensures that latex doesn't have an error if we don't specify the version
\providecommand{\RepVersion}{Unknown\xspace}
% to set this value use:
% (pdf)latex '\newcommand{\RepVersion}{version\xspace}%%%%%%%%%%%%%%%%%%%%%%%%%%%%%%%%%%%%%%%%%%%%%%%%%%%%%%%%
%
% Copyright (c) 2003-2010 by University of Queensland
% Earth Systems Science Computational Center (ESSCC)
% http://www.uq.edu.au/esscc
%
% Primary Business: Queensland, Australia
% Licensed under the Open Software License version 3.0
% http://www.opensource.org/licenses/osl-3.0.php
%
%%%%%%%%%%%%%%%%%%%%%%%%%%%%%%%%%%%%%%%%%%%%%%%%%%%%%%%%

\documentclass{esysdoc}

%%%%%%%%%%%%%%%%%%%%%%%%%%%%%%%%%%%%%%%%%%%%%%%%%%%%%%%%
%
% Copyright (c) 2003-2010 by University of Queensland
% Earth Systems Science Computational Center (ESSCC)
% http://www.uq.edu.au/esscc
%
% Primary Business: Queensland, Australia
% Licensed under the Open Software License version 3.0
% http://www.opensource.org/licenses/osl-3.0.php
%
%%%%%%%%%%%%%%%%%%%%%%%%%%%%%%%%%%%%%%%%%%%%%%%%%%%%%%%%

% Please do not use .ps only packages.
% As a minimum this document should build under pdflatex

\usepackage{color}
\usepackage{xspace}

%Ensures that latex doesn't have an error if we don't specify the version
\providecommand{\RepVersion}{Unknown\xspace}
% to set this value use:
% (pdf)latex '\newcommand{\RepVersion}{version\xspace}\input{install.tex}'
% as your command-line

\usepackage{listings}        % add the listings package to pretty print the code output


% hyperref is incompatible with python.sty which defines \ispdf and \url
%\usepackage[colorlinks=true, pdftitle={Escript/Finley Installation Guide}, pdfauthor={Escript development team}, pdfdisplaydoctitle=true]{hyperref}
%pdfpagelayout=TwoColumnRight

\newcommand{\escript}{\module{esys.escript}\xspace}
\newcommand{\finley}{\module{esys.finley}\xspace}
\newcommand{\esys}{\module{esys}\xspace}

\newcommand{\esfinley}{Escript/Finley\xspace}
\newcommand{\linux}{Linux\xspace}
\newcommand{\macosx}{MacOS~X\xspace}
\newcommand{\winxp}{Windows~XP\xspace}
\newcommand{\openmp}{OpenMP\xspace}
\newcommand{\mpi}{MPI\xspace}

% hyref is defined in python.sty as an alternative to hyperref which we can't
% use
\newcommand{\Sec}[1] {\hyref{#1}{Section~\ref{#1}}}
\newcommand{\Chap}[1] {\hyref{#1}{Chapter~\ref{#1}}}
\newcommand*{\filename}[1]{\texttt{#1}\xspace}

% defines the colour for the background of code examples
\definecolor{LightGrey}{gray}{0.9}


\lstdefinestyle{myShell}{%
%\lstset{%
language=ksh,
showstringspaces=false,
basicstyle=\small\ttfamily,
commentstyle=\color{black}\ttfamily,
keywordstyle=\color{black}\ttfamily,
%identifierstyle=\color[named]{Blue}\ttfamily,
%functionstyle=\color[named]{Blue}\ttfamily,
%typestyle=\color[named]{ForestGreen}\ttfamily,
stringstyle=\color{black}\ttfamily,%
frame=none,%
backgroundcolor=\color{LightGrey}%
}

% this implements nicely formatted shell code
\lstnewenvironment{shellCode}[1][]{\lstset{style=myShell}\lstset{#1}}{}

\newcommand{\todo}[1]{\emph{TODO:}{\em #1}}



\title{Installation guide for Escript and Finley}

\author{Escript development team}
\authoraddress{
Earth Systems Science Computational Centre (ESSCC) \\
The University of Queensland \\
Brisbane, Australia \\
Email: \email{esys@esscc.uq.edu.au}
}
\date{\today}      
% \release{Revision: \RepVersion}
% \setreleaseinfo{development version} 
% \setshortversion{\RepVersion}

\ifpdf
\pdfinfo {
/Author (Escript development team)
/Title (escript install guide)
/Keywords (escript, PDEs)
}
\fi

\makeindex
\begin{document}

\maketitle
\etableofcontents

%%%%%%%%%%%%%%%%%%%%%%%%%%%%%%%%%%%%%%%%%%%%%%%%%%%%%%%%
%
% Copyright (c) 2003-2012 by University of Queensland
% Earth Systems Science Computational Center (ESSCC)
% http://www.uq.edu.au/esscc
%
% Primary Business: Queensland, Australia
% Licensed under the Open Software License version 3.0
% http://www.opensource.org/licenses/osl-3.0.php
%
%%%%%%%%%%%%%%%%%%%%%%%%%%%%%%%%%%%%%%%%%%%%%%%%%%%%%%%%

\chapter{Introduction}
This document describes how to install \emph{esys-Escript}\footnote{For the rest of the document we will drop the \emph{esys-}} on your computer.
To learn how to use \esfinley please see the Cookbook, User's guide or the API documentation.
If you use the Debian or Ubuntu packages to install then the documentation will be available in
\file{/usr/share/doc/escript}, otherwise (if you haven't done so already) you can download the documentation bundle 
from launchpad.

\esfinley is primarily developed on Linux desktop, SGI ICE and \macosx systems.
It is distributed in two forms:
\begin{enumerate}
  \item Binary bundles -- these are great for first time users or for those who want to start using 
    \esfinley immediately.
      Bundles are available for:
      \begin{itemize}
	  \item Debian and Ubuntu Linux distributions ($32$/$64$-bit i686) (.deb package)
	  \item Linux desktop systems with gcc (stand-alone bundle)
	  \item \macosx Leopard systems (also tested on Lion) with gcc (stand-alone bundle)
	  \item $32$bit Windows (requires some other packages to be installed).
      \end{itemize}    
    Please see Chapter~\ref{chap:bin} for instructions on how to install the binary bundles \esfinley.
  \item Source bundles -- these require compilation and should be used if the binary bundles 
    don't work on the target machine or if extra functionality is required such as \mpi parallelisation.
    See Chapter~\ref{chap:compiler} for detailed instructions.
\end{enumerate}

See the site \url{https://answers.launchpad.net/escript-finley} for online help.

\section{Significant changes since version 3.3}
\begin{itemize}
 \item \texttt{SymPy} is now required to compile or run \escript. 
    This means you will need to download sympy in addition to the support bundle from previous releases.
 \item The minimum Python version is now $2.6$.
\end{itemize}

% \noindent If you choose to compile from source your options are to
% \begin{itemize}
%     \item install dependencies (e.g. using your package manager) and only compile \esfinley, OR
%     \item compile everything from source.
% \end{itemize}
% Either way, please see Chapter~\ref{chap:compiler} for a discussion of compiler features.
% Compiling \esfinley when its dependencies are already installed is discussed in Chapter~\ref{chap:essrc}.
% To compile \esfinley and all dependencies from source please see Chapter~\ref{chap:allsrc}.
% The latter option takes a significant amount of time and is only required if the versions of the dependent libraries available on your system do not work with \esfinley.
% 
% Once everything is installed you can test your installation using the Python scripts in \file{examples.zip} or \file{examples.tar.gz}\footnote{These should either be in \file{escript.d/release/doc} or in the case of Debian, in \file{/usr/share/doc/escript}.}.
% Unpack the examples and try to run the following from a terminal:
% \begin{shellCode}
%  run-escript poission.py
% \end{shellCode}
% If this produces a VTK file called \file{u.vtu} then you are likely to have a functional \esfinley installation.
% You can try and visualize the VTK data or delete the file.
% For visualization we suggest using \file{VisIt}\footnote{\url{https://wci.llnl.gov/codes/visit/}} or \file{MayaVi}\footnote{\url{http://mayavi.sourceforge.net}} which are both freely available.






%%%%%%%%%%%%%%%%%%%%%%%%%%%%%%%%%%%%%%%%%%%%%%%%%%%%%%%%%%%%%%%%%%%%%%%%%%%%%%
% Copyright (c) 2003-2012 by University of Queensland
% http://www.uq.edu.au
%
% Primary Business: Queensland, Australia
% Licensed under the Open Software License version 3.0
% http://www.opensource.org/licenses/osl-3.0.php
%
% Development until 2012 by Earth Systems Science Computational Center (ESSCC)
% Development since 2012 by School of Earth Sciences
%
%%%%%%%%%%%%%%%%%%%%%%%%%%%%%%%%%%%%%%%%%%%%%%%%%%%%%%%%%%%%%%%%%%%%%%%%%%%%%%

% material common to all binary distributions.

\chapter{Binary releases}\label{chap:bin}

Binary distributions (no compilation required) are available for the following operating systems:
\begin{itemize}
 \item \linux~-- Section~\ref{sec:binlinux}
 \item \macosx~-- Section~\ref{sec:binmac}
 \item Windows~-- Section~\ref{sec:binwin}.
\end{itemize}

Note that only the Debian/Ubuntu binary packages support \openmp and \mpi.
If you need these features you will need to compile \esfinley from source (see Section~\ref{sec:compilesrc} 
and Section~\ref{sec:compileescriptlinux}.)


\input{binlinux}
		% material common to all binary installs
%%%%%%%%%%%%%%%%%%%%%%%%%%%%%%%%%%%%%%%%%%%%%%%%%%%%%%%%%%%%%%%%%%%%%%%%%%%%%%
% Copyright (c) 2003-2013 by University of Queensland
% http://www.uq.edu.au
%
% Primary Business: Queensland, Australia
% Licensed under the Open Software License version 3.0
% http://www.opensource.org/licenses/osl-3.0.php
%
% Development until 2012 by Earth Systems Science Computational Center (ESSCC)
% Development since 2012 by School of Earth Sciences
%
%%%%%%%%%%%%%%%%%%%%%%%%%%%%%%%%%%%%%%%%%%%%%%%%%%%%%%%%%%%%%%%%%%%%%%%%%%%%%%

\section{\macosx binary installation}
\label{sec:binmac}

The standalone release for OSX has been tested on \macosx 10.5 (``Leopard'')\footnote{It \emph{should} work on 
``Snow Leopard'' but has not been tested.} and 10.7 (``Lion'').

You will need to download both escript (\file{escript_3.4_osx.dmg}) and the support files (\file{escript-support_3.0_osx.dmg}).
This point release uses the same support bundle as previous releases so if you already have it you don't need a new version.
You will also need to download the sympy source code from \url{sympy.org} (You are looking for a \texttt{.tar.gz} file).

\begin{itemize}
\item Create a folder to hold escript (no spaces in the name please).
\item Open the \file{.dmg} files and copy the contents to the folder you just created.
\item Copy the sympy file into the same directory.
\end{itemize}

To use escript, open a terminal\footnote{If you do not know how to open a terminal on Mac, then just type \texttt{terminal} in the spotlight (search tool on the top of the right corner) and once found, just click on it.} and type
\begin{shellCode}
eval `x/escript.d/bin/run-escript -e`
\end{shellCode}
where \textit{x} is the absolute path to your install.

\noindent Now we need to install sympy (substitue the version number of sympy you have):
\begin{shellCode}
tar -xzf sympy-0.7.1.tar.gz
cd sympy-0.7.1
python setup.py install --prefix ../stand/pkg
\end{shellCode}

You cay test your install with:
\begin{shellCode}
run-escript
\end{shellCode}

You may now remove the sympy files from the starting directory and ``eject'' the \texttt{.dmg} files.

If you wish to save on typing you can add \file{x/escript.d/bin} to your PATH variable 
(where \textit{x} is the absolute path to your install). 



%%%%%%%%%%%%%%%%%%%%%%%%%%%%%%%%%%%%%%%%%%%%%%%%%%%%%%%%
%
% Copyright (c) 2003-2008 by University of Queensland
% Earth Systems Science Computational Center (ESSCC)
% http://www.uq.edu.au/esscc
%
% Primary Business: Queensland, Australia
% Licensed under the Open Software License version 3.0
% http://www.opensource.org/licenses/osl-3.0.php
%
%%%%%%%%%%%%%%%%%%%%%%%%%%%%%%%%%%%%%%%%%%%%%%%%%%%%%%%%

\section{Windows binary installation}
\label{sec:binwin}

There is no automated install/uninstall procedure for \esfinley on Windows at this time.
However, the following procedure appears to work for the previous release\footnote{Thanks to Peter Hornby and Frederick Roger for this report.}

Please ensure you have the following software installed: 
\begin{itemize}
 \item pythonxy (\url{http://www.pythonxy.com})
%\item numarray 1-5-2win32py2.5  for  win32
\end{itemize}

From the escript zip file:
\begin{itemize}
\item 
 copy the \filename{esys} directory to your Python 2.5 site-packages folder (usually \filename{C:\textbackslash Python25\textbackslash Lib\textbackslash site-packages}).
\item 
 copy the \filename{.dll} files from \filename{esys\_dlls} to a directory on your PATH. For example copy the directory to \filename{C:\textbackslash Python25\textbackslash libs\textbackslash esys\_dlls} and add  \filename{C:\textbackslash Python25\textbackslash libs\textbackslash esys\_dlls} to your PATH.
\end{itemize}


\chapter{Building escript from source}
\label{chap:essrc}
%%%%%%%%%%%%%%%%%%%%%%%%%%%%%%%%%%%%%%%%%%%%%%%%%%%%%%%%%%%%%%%%%%%%%%%%%%%%%%
% Copyright (c) 2003-2014 by University of Queensland
% http://www.uq.edu.au
%
% Primary Business: Queensland, Australia
% Licensed under the Open Software License version 3.0
% http://www.opensource.org/licenses/osl-3.0.php
%
% Development until 2012 by Earth Systems Science Computational Center (ESSCC)
% Development 2012-2013 by School of Earth Sciences
% Development from 2014 by Centre for Geoscience Computing (GeoComp)
%
%%%%%%%%%%%%%%%%%%%%%%%%%%%%%%%%%%%%%%%%%%%%%%%%%%%%%%%%%%%%%%%%%%%%%%%%%%%%%%

% This file contains material common to all src distributions.

% The original version of this content came from the esscc twiki page maintained by ksteube

This chapter describes how to build \esfinley from source assuming that the dependencies are already installed (for example using precompiled packages for your OS).
Section~\ref{sec:deps} describes the dependencies, while Section~\ref{sec:compilesrc} gives the compile instructions.

If you would prefer to build all the dependencies from source in the escript-support packages please see Chapter~\ref{chap:allsrc}.
\esfinley is known to compile and run on the following systems:
\begin{itemize}
 \item \linux using gcc
\item \linux using icc on SGI ICE 8200. (We do not recommend building with intel-11)
\item \macosx using gcc or clang
\item \winxp using the Visual C compiler (we do not specifically discuss Windows builds in this guide).
\end{itemize}

If you have compiled a previous version of \esfinley, the \file{..._options.py} file has the same format
as in the previous release, so you can reuse it.


\section{External dependencies}
\label{sec:deps}
The following external packages are required in order to compile and run \esfinley.
Where version numbers are specified, more recent versions can probably be substituted.
You can either try the standard/precompiled packages available for your operating system or you can download and build them from source.
The advantage of using existing packages is that they are more likely to work together properly.
You must take greater care if downloading sources separately.

\begin{itemize}
 \item python $\geq 2.6$ (\url{http://python.org}) \\-
        Python interpreter (you must compile with shared libraries.)
 \item numpy $\geq 1.1.0$ (\url{http://numpy.scipy.org}) \\-
        Arrays for Python
 \item boost $\geq 1.35$ (\url{http://www.boost.org}) \\-
        Interface between C++ and Python
 \item scons $\geq 0.989.5$ (\url{http://www.scons.org/}) \\-
        Python-based alternative to \texttt{make}.
\end{itemize}

The version numbers given here are not strict requirements, more recent (and in some cases older) versions are very likely to work.
The following packages should be sufficient (but not necessarily minimal) for Debian 6.0 (``Squeeze''):
\texttt{libboost-python-dev, scons, python-numpy, python-sympy, g++}.

\noindent The following packages may be required for some of the optional capabilities of the system:
\begin{itemize}
 \item sympy $\geq$ (\url{http://sympy.org}) \\-
        Used by \texttt{esys.escript.symbolic}.
 \item netcdf $\geq 3.6.2$ (\url{http://www.unidata.ucar.edu/software/netcdf}) \\-
        Used to save data sets in binary form for checkpoint/restart (must be compiled with -fPIC)
 \item parmetis $\geq 3.1$ (\url{http://glaros.dtc.umn.edu/gkhome/metis/parmetis/overview}) \\-
        Optimization of the stiffness matrix
 \item MKL \\(\url{http://www.intel.com/cd/software/products/asmo-na/eng/307757.htm}) \\-
        Intel's Math Kernel Library for use with their C compiler.
\item Lapack - Available in various versions from various places. \\ 
Currently only used to invert dense square matrices larger than 3x3. 
 \item gmsh $\geq 2.2.0$ (\url{http://www.geuz.org/gmsh}) \\-
        Mesh generation and viewing [esys.pycad uses this]
\end{itemize}

\noindent Mesh generation: as well as \texttt{gmsh} above you could also use:
\begin{itemize}
 \item triangle $\geq 1.6$ (\url{http://www.cs.cmu.edu/~quake/triangle.html}) \\-
        Two-dimensional mesh generator and Delaunay triangulator.
\end{itemize}

Packages for visualization:
\begin{itemize}
 \item mayavi $\geq 1.5$ (\url{http://mayavi.sourceforge.net}) \\-
        MayaVi is referenced in our User's Guide for viewing VTK files
 \item visit $\geq 1.11.2$ (\url{https://wci.llnl.gov/codes/visit/}) \\-
        A powerful visualisation system with movie-making capabilities.
\end{itemize}



The source code comes with an extensive set of unit tests. If you would like to
build those to verify your installation you need:
\begin{itemize}
 \item cppunit $\geq 1.12.1$ (\url{http://cppunit.sourceforge.net})
\end{itemize}

\section{Compilation}\label{sec:compilesrc}
Throughout this section we will assume that the source code is uncompressed in a directory called \file{escript.d}.
You can call the directory anything you like, provided that you make the change before you compile.

You need to indicate where to find the external dependencies.
To do this, create a file in the \file{escript.d/scons} directory called \file{x_options.py} where ``x'' is the name of your computer (output of the \texttt{hostname} command).
Please note that if your hostname has non-alphanumeric characters in it (eg - ) you need to replace them with underscores.
For example the options file for \texttt{bob-desktop} would be named \file{bob_desktop_options.py}.

From now on all paths will be relative to the top level of the source.
As a starting point copy the contents of one of the following files into your options file:
\begin{itemize}
\item \file{scons/TEMPLATE_linux.py} (\linux and \macosx desktop)
\item \file{scons/TEMPLATE_windows.py} (\winxp)
\end{itemize}

This options file controls which features and libraries your build of escript will attempt to use.
For example to use OpenMP or MPI you will need to enable it here.
If you want to try escript out without customising your build, then change
directories to \file{escript.d} and enter
\begin{shellCode}
scons 
\end{shellCode}
If this works you can skip to Section~\ref{sec:diff}.
If not, then you will need to make some modications to the file.
Read on.

The template files contain all available options with a comment explaining the
purpose of each.
Check through the file and ensure that the relevant paths and names are correct
for your system and that you enable optional components that you wish to use.
For example, to use netCDF, find the netcdf-related lines, uncomment them
(i.e. remove the \# at the beginning of the lines) and change them according
to your installation:
\begin{shellCode}
netcdf = True
netcdf_prefix = '/opt/netcdf4'
netcdf_libs = ['netcdf_c++', 'netcdf']
\end{shellCode}

In this example, netCDF \emph{header} files must be located in
\file{/opt/netcdf4/include}\footnote{or \ldots/include32 or \ldots/include64 or \ldots/inc}
and the \emph{libraries} in \file{/opt/netcdf4/lib}\footnote{or \ldots/lib32 or \ldots/lib64}.
If this scheme does not apply to your installation then you may also specify
the include-path and library-path directly like so:
\begin{shellCode}
netcdf_prefix = ['/usr/local/include/netcdf', '/usr/local/lib']
\end{shellCode}
The order is important: the first element in the list is the
\emph{include}-path, the second element is the \emph{library}-path and both
must be specified.

If a line in the options file is commented out and you do not require the
feature, then it can be ignored.
To actually compile (if you have $n$ processors, then you can use \texttt{scons -j$n$} instead):

\begin{shellCode}
cd escript.d
scons
\end{shellCode}

As part of its output, scons will tell you the name of the options file it used
as well as a list of features and whether they are enabled for your build.
If you enabled an optional dependency and the library or include files could
not be found you will be notified and the build will stop.

Note, that you can override all settings from the options-file on the scons
command line. For example, if you usually build an optimized version but would
like to build a debug version into a separate directory without changing your
default settings, you can use:
\begin{shellCode}
scons debug=1 prefix=debugbuild
\end{shellCode}
This will install the binaries and libraries built in debug mode into
directories underneath \file{./debugbuild}.

To run the unit test suite that comes with the source code issue
\begin{shellCode}
scons py_tests
\end{shellCode}
(If you have cppunit installed you can run additional tests using \texttt{scons all_tests}.

Grab a coffee or two while the tests compile and run.
An alternative method is available for running tests on \openmp and \mpi builds.

\subsection{Compilation with \openmp}
\openmp is generally enabled by setting compiler and linker switches. For the
most common compilers these are automatically set by build system and all you
have to do is set the \texttt{openmp} option to True in your options file. If
this does not work or your compiler is different, then consult your compiler
documentation for the precise switches to use and modify the \texttt{omp_flags}
and \texttt{omp_ldflags} variables in your options file.
For example, for gcc compilers which support \openmp use:
\begin{shellCode}
openmp = True
omp_flags = '-fopenmp'
omp_ldflags = '-fopenmp'
\end{shellCode}
(The two latter settings can also be left out as this is the default OpenMP on gcc.)

You can test your \openmp-enabled build, e.g. using 4 threads by issuing
\begin{shellCode}
export ESCRIPT_NUM_THREADS=4
scons py_tests
\end{shellCode}

\subsection{Compilation with \mpi}
You need to have \mpi preinstalled on your system.
There are a number of implementations so we do not provide any specific advice
here.
Set the following variables in your options file to according to your
installation:
\begin{itemize}
 \item \texttt{mpi} \\
    which \mpi implementation (flavour) is used. Valid values are
    \begin{itemize}
        \item[\texttt{none}] \mpi is disabled
        \item[\texttt{MPT}] SGI MPI implementation \\
            \url{http://techpubs.sgi.com/library/manuals/3000/007-3687-010/pdf/007-3687-010.pdf}
        \item[\texttt{MPICH}] Argonne's MPICH implementation \\
            \url{http://www.mcs.anl.gov/research/projects/mpi/mpich1/}
        \item[\texttt{MPICH2}] Argonne's MPICH version 2 implementation \\
            \url{http://www.mcs.anl.gov/research/projects/mpi/mpich2/}
        \item[\texttt{OPENMPI}] Open MPI \\
            \url{http://www.open-mpi.org/}
        \item[\texttt{INTELMPI}] Intel MPI \\
            \url{http://software.intel.com/en-us/intel-mpi-library/}
    \end{itemize}
 \item \texttt{mpi_prefix} \\
    where to find \mpi headers and libraries (see netCDF example above)
 \item \texttt{mpi_libs} \\
    which libraries to link to.
\end{itemize}

To test your build using 6 processes enter:
\begin{shellCode}
export ESCRIPT_NUM_PROCS=6
scons py_tests
\end{shellCode}
and on $2$ processes with $4$ threads each (provided \openmp is enabled)\footnote{Unless your system has $8$ cores expect this to be slow}:
\begin{shellCode}
export ESCRIPT_NUM_THREADS=4
export ESCRIPT_NUM_PROCS=2
scons py_tests
\end{shellCode}
Alternatively, you can give a hostfile
\begin{shellCode}
export ESCRIPT_NUM_THREADS=4
export ESCRIPT_HOSTFILE=myhostfile
scons py_tests
\end{shellCode}
Note that depending on your \mpi flavour it may be required to start a daemon
before running the tests under \mpi.

\subsection{Difficulties}\label{sec:diff}

\subsubsection{Mismatch of runtime and build libraries}
Most external libraries used by \esfinley are linked dynamically.
This can lead to problems if after compiling \esfinley these libraries are
updated.
The same applies to the installed Python executable and libraries.
Whenever these dependencies change on your system you should recompile
\esfinley to avoid problems at runtime such as load errors or segmentation
faults.

\subsubsection{\openmp builds segfault running examples}
One known cause for this is linking the \file{gomp} library with escript built using gcc 4.3.3.
While you need the \texttt{-fopenmp} switch you should not need to link \file{gomp}.


		% material common to all source installs
%%%%%%%%%%%%%%%%%%%%%%%%%%%%%%%%%%%%%%%%%%%%%%%%%%%%%%%%
%
% Copyright (c) 2003-2008 by University of Queensland
% Earth Systems Science Computational Center (ESSCC)
% http://www.uq.edu.au/esscc
%
% Primary Business: Queensland, Australia
% Licensed under the Open Software License version 3.0
% http://www.opensource.org/licenses/osl-3.0.php
%
%%%%%%%%%%%%%%%%%%%%%%%%%%%%%%%%%%%%%%%%%%%%%%%%%%%%%%%%

\chapter{Building escript and dependencies from source}
\label{chap:allsrc}

This chapter describes how to build escript and its dependencies from the source code in the escript support packages.
You can also use these instructions if you have gathered the various sources yourself.

\input{srclinux}
\input{srcmac}
\input{srcadditional}
 
\end{document}
'
% as your command-line

\usepackage{listings}        % add the listings package to pretty print the code output


% hyperref is incompatible with python.sty which defines \ispdf and \url
%\usepackage[colorlinks=true, pdftitle={Escript/Finley Installation Guide}, pdfauthor={Escript development team}, pdfdisplaydoctitle=true]{hyperref}
%pdfpagelayout=TwoColumnRight

\newcommand{\escript}{\module{esys.escript}\xspace}
\newcommand{\finley}{\module{esys.finley}\xspace}
\newcommand{\esys}{\module{esys}\xspace}

\newcommand{\esfinley}{Escript/Finley\xspace}
\newcommand{\linux}{Linux\xspace}
\newcommand{\macosx}{MacOS~X\xspace}
\newcommand{\winxp}{Windows~XP\xspace}
\newcommand{\openmp}{OpenMP\xspace}
\newcommand{\mpi}{MPI\xspace}

% hyref is defined in python.sty as an alternative to hyperref which we can't
% use
\newcommand{\Sec}[1] {\hyref{#1}{Section~\ref{#1}}}
\newcommand{\Chap}[1] {\hyref{#1}{Chapter~\ref{#1}}}
\newcommand*{\filename}[1]{\texttt{#1}\xspace}

% defines the colour for the background of code examples
\definecolor{LightGrey}{gray}{0.9}


\lstdefinestyle{myShell}{%
%\lstset{%
language=ksh,
showstringspaces=false,
basicstyle=\small\ttfamily,
commentstyle=\color{black}\ttfamily,
keywordstyle=\color{black}\ttfamily,
%identifierstyle=\color[named]{Blue}\ttfamily,
%functionstyle=\color[named]{Blue}\ttfamily,
%typestyle=\color[named]{ForestGreen}\ttfamily,
stringstyle=\color{black}\ttfamily,%
frame=none,%
backgroundcolor=\color{LightGrey}%
}

% this implements nicely formatted shell code
\lstnewenvironment{shellCode}[1][]{\lstset{style=myShell}\lstset{#1}}{}

\newcommand{\todo}[1]{\emph{TODO:}{\em #1}}



\usepackage{upquote} %This is to allow single quotes in python blocks to be left alone (able to be be copy and pasted)
%Previous LaTeX was replacing them with non python slanty quotes
\usepackage{comment}


%\newcommand{\relver}{4.1}
%\newcommand{\reldate}{24 July 2015}

\newcommand{\relver}{development}
\newcommand{\reldate}{\today}



\title{Installation guide for \emph{esys-Escript}}

\author{Escript development team}
\authoraddress{
Centre for Geoscience Computing (GeoComp) \\
The University of Queensland \\
Brisbane, Australia \\
Email: \email{esys@esscc.uq.edu.au}
}
\date{\reldate}      
\release{\relver}
%\release{development}

\ifpdf
\pdfinfo {
/Author (Escript development team)
/Title (escript install guide)
/Keywords (escript, PDEs)
}
\fi

\makeindex
\begin{document}

\maketitle

%%%%%%%%%%%%%%%%%%%%%%%%%%%%%%%%%%%%%%%%%%%%%%%%%%%%%%%%%%%%%%%%%%%%%%%%%%%%%%
% Copyright (c) 2003-2015 by The University of Queensland
% http://www.uq.edu.au
%
% Primary Business: Queensland, Australia
% Licensed under the Open Software License version 3.0
% http://www.opensource.org/licenses/osl-3.0.php
%
% Development until 2012 by Earth Systems Science Computational Center (ESSCC)
% Development 2012-2013 by School of Earth Sciences
% Development from 2014 by Centre for Geoscience Computing (GeoComp)
%
%%%%%%%%%%%%%%%%%%%%%%%%%%%%%%%%%%%%%%%%%%%%%%%%%%%%%%%%%%%%%%%%%%%%%%%%%%%%%%

\clearpage
\vbox{}
\vfill
\begin{center}
\textbf{\Large Guide to Documentation}\pdfbookmark[0]{Documentation guide}{documentation guide}
\vspace{0.5cm}

Documentation for \module{esys.escript} comes in a number of parts.
Here is a rough guide to what goes where.

\vspace{1cm}
\hrule
\vspace{1cm}

\begin{tabular}{rp{11cm}}
 \textbf{install.pdf} & ``Installation guide for \emph{esys-Escript}'': 
 Instructions for compiling \emph{escript} for your system from its
 source code. 
 Also briefly covers installing \texttt{.deb} packages for Debian and Ubuntu.\\
 &\\
 \textbf{cookbook.pdf} & ``The \textit{escript} COOKBOOK'':
 An introduction to \emph{escript} for new users from a geophysics perspective.\\
 &\\ 
 \textbf{user.pdf} & ``\emph{esys-Escript} User's Guide: Solving Partial
 Differential Equations with Escript and Finley'': Covers main \emph{escript} concepts.\\
 & \\ 
 \textbf{inversion.pdf} & ``\module{esys.downunder}: Inversion with \emph{escript}'':
 Explanation of the inversion toolbox for \emph{escript}.\\
 & \\ 
 \textbf{sphinx_api directory} & Documentation for \emph{escript} {\it Python} libraries.\\
 & \\ 
 \textbf{escript_examples(.tar.gz)/(.zip)} & Full example scripts referred to
 by other parts of the documentation.\\
 & \\ 
 \textbf{doxygen directory} & Documentation for C++ libraries (mostly only of
 interest for developers).\\
\end{tabular}
\end{center}
\vfill
\vbox{}
\pagebreak


\cleardoublepage\pdfbookmark[0]{Contents}{contents}%
\tableofcontents

#--------------
\chapter{Introduction}
This document describes how to install \emph{esys-Escript}\footnote{For the rest of the document we will drop the \emph{esys-}} on to your computer.
To learn how to use \esfinley please see the Cookbook, User's guide or the API documentation.

\esfinley is primarily developed on Linux desktop, SGI ICE and \macosx systems.
It can be installed in several ways:
\begin{enumerate}
  \item Binary packages -- ready to run with no compilation required.
  These are available in Debian and Ubuntu repositories, so just use your normal package manager (so you don't need this guide). They are also available for Anaconda Python 3.
  \item Using flatpak
  \item From source -- that is, it must be compiled for your machine.
This is the topic of this guide.
\end{enumerate}

See the site \url{https://answers.launchpad.net/escript-finley} for online help.
Chapter~\ref{chap:source} covers installing from source.
Appendix~\ref{app:cxxfeatures} lists some c++ features which your compiler must support in order to compile escript.
This version of escript has the option of using \texttt{Trilinos} in addition to our regular solvers.
Appendix~\ref{app:trilinos} covers features of \texttt{Trilinos} which escript needs.

#--------------
chapter{Installing from Flatpak}\label{chap:flatpak}

To install \escript on any linux distribution using flatpak\footnote{For most linux distributions this can be installed from the repository. Otherwise, flatpak is available at \url{https://flathub.org/home}}, type
\begin{shellCode}
flatpak install flathub au.edu.uq.esys.escript
\end{shellCode}

This wil download and install \escript on your machine. The \escript build installed utilises both the Trilinos and PASO solver libraries, with openMP but without OPENMPI.

After flatpak has finished downloading and installing \escript, you can launch an \escript window from the menu or run \escript in a terminal with the command:
\begin{shellCode}
flatpak run au.edu.uq.esys.escript [other arguments to pass to escript]
\end{shellCode}

Finally, to uninstall \escript from your machine, type
\begin{shellCode}
flatpak uninstall escript
\end{shellCode}

#--------------
\chapter{Installing inside Anaconda}\label{chap:conda}

There are precompiled binaries of \escript available for Anaconda python (for Linux), available in the conda-forge channel on Anaconda cloud. They can be installed using the command
\begin{shellCode}
conda install esys-escript -c conda-forge
\end{shellCode}
#--------------

\chapter{Installing from Docker}\label{chap:docker}

To install an \escript Docker container on your machine, first install Docker and then type:
\begin{shellCode}
docker pull esysescript/esys-escript
\end{shellCode}

\vspace{1cm}\
Once installed, you can launch an escript session using the command:
\begin{shellCode}
docker run -ti esysescript/esys-escript run-escript
\end{shellCode}

If you would also like to mount a folder, you can use the command:
\begin{shellCode}
docker run -ti -v $(pwd):/app/ esysescript/esys-escript run-escript
\end{shellCode}

Additionally, if you wish to run an escript named test_program.py, you can use the command:
\begin{shellCode}
docker run -ti -v $(pwd):/app/ esysescript/esys-escript \
					run-escript [path to test_program.py]
\end{shellCode}

# ------
chapter{Installing from Source}\label{chap:source}

This chapter describes installing \escript from source on unix/posix like
systems (including MacOSX) and Windows 10.

\section{Parallel Technologies}\label{sec:par}
It is likely that the computer you run \escript on, will have more than one processor core.
Escript can make use of multiple cores [in order to solve problems more quickly] if it is told to do so,
but this functionality must be enabled at compile time.
Section~\ref{sec:needpar} gives some rough guidelines to help you determine what you need.

There are two technologies which \escript can employ here.
\begin{itemize}
 \item OpenMP -- the more efficient of the two [thread level parallelism].
 \item MPI -- Uses multiple processes (less efficient), needs less help from
   the compiler (not supported on Windows).
\end{itemize}

Escript is primarily tested on recent versions of the GNU and Intel suites
(``g++'' / ``icpc''), and Microsoft Visual C++ (MSVC).  However, it also passes
our tests when compiled using ``clang++''.  Escript now requires compiler
support for some features of the C++11 standard.  See
Appendix~\ref{app:cxxfeatures} for a list.


Our current test compilers include:
\begin{itemize}
 \item g++ 10.2
 \item clang++ 11.0
 \item intel icpc v17
 \item MSVC 2017 or 2019
\end{itemize}

Note that:
\begin{itemize}
 \item OpenMP will not function correctly for g++ $\leq$ 4.2.1 (and is not currently supported by clang).
 \item icpc v11 has a subtle bug involving OpenMP and C++ exception handling, so this combination should not be used.
\end{itemize}

\subsection{What parallel technology do I need?}\label{sec:needpar} If you are
using any version of Linux released in the past few years, then your system
compiler will support \openmp with no extra work (and give better performance);
so you should use it. MSVC 2017 and 2019 also have \openmp support on Windows
(\openmp 2.0). You will not need MPI unless your computer is some form of
cluster. MPI is not recommended on Windows as it will interfer with Jupyter.

If you are using BSD or MacOSX and you are just experimenting with \escript, then performance is
probably not a major issue for you at the moment so you don't need to use either \openmp or MPI.
This also applies if you write and polish your scripts on your computer and then send them to a cluster to execute.
If in the future you find escript useful and your scripts take significant time to run, then you may want to recompile
\escript with more options.



Note that even if your version of \escript has support for \openmp or MPI, you will still need to tell the system to
use it when you run your scripts.
If you are using the \texttt{run-escript} launcher, then this is controlled by
the \texttt{-t}, \texttt{-p}, and \texttt{-n} flags.
If not, then consult the documentation for your MPI libraries (or the compiler documentation in the case of OpenMP
\footnote{It may be enough to set the \texttt{OMP\_NUM\_THREADS} environment variable.}).

If you are using MacOSX, then see the next section, if not, then skip to Section~\ref{sec:build}.

\section{MacOS}
This release of \escript has only been tested on OSX 10.13.
For this section we assume you are using either \texttt{homebrew} or \texttt{MacPorts} as a package
manager\footnote{Note that package managers will make changes to your computer based on programs configured by other people from
various places around the internet. It is important to satisfy yourself as to the security of those systems.}.
You can of course install prerequisite software in other ways.
For example, we have had \emph{some} success changing the default
compilers used by those systems. However this is more complicated and we do not provide a guide here.

\noindent Both of those systems require the XCode command line tools to be installed\footnote{As of OSX10.9, the
command \texttt{xcode-select --install} will allow you to download and install the commandline tools.}.

\section{Building}\label{sec:build}

\esfinley is built using \textit{SCons}. To simplify the installation process, we have prepared \textit{SCons} \textit{_options.py} files for a number of common systems\footnote{These are correct a time of writing but later versions of those systems may require tweaks.
Also, these systems represent a cross section of possible platforms rather than meaning those systems get particular support.}.
The options files are located in the \textit{scons/templates} directory. We suggest that the file most relevant to your OS
be copied from the templates directory to the scons directory and renamed to the form XXXX_options.py where XXXX
should be replaced with your computer's (host-)name.
If your particular system is not in the list below, or if you want a more customised
build,
see Section~\ref{sec:othersrc} for instructions.
\begin{itemize}
 \item Debian - \ref{sec:debsrc}
 \item Ubuntu - \ref{sec:ubsrc}
 \item Mint - \ref{sec:mintsrc}
 \item Arch Linux - \ref{sec:archsrc}
 \item OpenSuse - \ref{sec:susesrc}
 \item Centos - \ref{sec:centossrc}
 \item Fedora - \ref{sec:fedorasrc}
 \item MacOS (macports) - \ref{sec:macportsrc}
 \item MacOS (homebrew) - \ref{sec:homebrewsrc}
 \item FreeBSD - \ref{sec:freebsdsrc}
 \item Windows - \ref{sec:windowssrc}
\end{itemize}

Once these are done proceed to Section~\ref{sec:cleanup} for cleanup steps.

\noindent All of these instructions assume that you have obtained the \escript source (and uncompressed it if necessary).
\subsection{Debian}\label{sec:debsrc}
\noindent These instructions were prepared on Debian 10 \textit{Buster}.

\noindent As a preliminary step, you should install the dependencies that \esfinley requires from the repository.
If you intend to use Python 2.7, then you should install the following
\begin{shellCode}
sudo apt-get install python-dev python-numpy
sudo apt-get install python-sympy python-matplotlib python-scipy
sudo apt-get install libboost-python-dev libboost-random-dev
sudo apt-get install lib
sudo apt-get install scons lsb-release libsuitesparse-dev gmsh
\end{shellCode}

\noindent If you intend to use Python 3.0+, then you should install the following
\begin{shellCode}
sudo apt-get install python3-dev python3-numpy
sudo apt-get install python3-sympy python3-matplotlib python3-scipy
sudo apt-get install libboost-python-dev libboost-random-dev libhdf5-serial-dev
sudo apt-get install libsuitesparse-dev scons lsb-release gmsh
\end{shellCode}

\noindent In the source directory execute the following (substitute \textit{buster_py2} or \textit{buster_py3} for XXXX):
\begin{shellCode}
scons -j4 options_file=scons/templates/XXXX_options.py
\end{shellCode}

\noindent If you wish to test your build, you can use the following:
\begin{shellCode}
scons -j4 py_tests options_file=scons/templates/XXXX_options.py
\end{shellCode}

% \begin{optionalstep}
% If for some reason, you wish to rebuild the documentation, you would also need the following:
% \begin{shellCode}
% sudo aptitude install python-sphinx doxygen python-docutils texlive
% sudo aptitude install zip texlive-latex-extra latex-xcolor
% \end{shellCode}
% \end{optionalstep}

\subsection{Ubuntu}\label{sec:ubsrc}
These instructions were prepared on Ubuntu 20.04 LTS \textit{Focal Fossa}. \newline


% \noindent As a preliminary step, you should install the dependencies that \esfinley requires from the repository.
% If you intend to use Python 2.7, then you should install the following packages:
% \begin{shellCode}
% sudo apt-get install python-dev python-numpy
% sudo apt-get install python-sympy python-matplotlib python-scipy libhdf5-serial-dev
% sudo apt-get install libboost-random-dev libboost-python-dev libboost-iostreams-dev
% sudo apt-get install scons lsb-release libsuitesparse-dev
% \end{shellCode}

%For Python 3.0+, you should instead install the following packages:
\noindent As a preliminary step, you should install the dependencies that \esfinley requires from the repository.
\begin{shellCode}
sudo apt-get install python3-dev python3-numpy
sudo apt-get install python3-sympy python3-matplotlib python3-scipy libhdf5-serial-dev
sudo apt-get install libboost-random-dev libboost-python-dev libboost-iostreams-dev
sudo apt-get install scons lsb-release libsuitesparse-dev
\end{shellCode}

% \begin{optionalstep}
% If for some reason, you wish to rebuild the documentation, you would also need the following:
% \begin{shellCode}
% sudo aptitude install python-sphinx doxygen python-docutils texlive
% sudo aptitude install zip texlive-latex-extra latex-xcolor
% \end{shellCode}
% \end{optionalstep}

% \noindent Then navigate to the source directory and execute the following (substitute \textit{focus_py2} or \textit{focus_py3} as appropriate for XXXX):
% \begin{shellCode}
% scons -j4 options_file=scons/templates/XXXX_options.py
% \end{shellCode}

\noindent Then navigate to the source directory and execute the following
\begin{shellCode}
scons -j4 options_file=scons/templates/focus_options.py
\end{shellCode}

% \noindent If you wish to test your build, you can use the following:
% \begin{shellCode}
% scons -j4 py_tests options_file=scons/templates/XXXX_options.py
% \end{shellCode}

\subsection{Mint}\label{sec:mintsrc}
These instructions were prepared on Mint 20.3. \newline

\noindent As a preliminary step, you should install the dependencies that \esfinley requires from the repository.
\begin{shellCode}
sudo apt-get install python3-dev python3-numpy
sudo apt-get install python3-sympy python3-matplotlib python3-scipy
sudo apt-get install libhdf5-serial-dev
sudo apt-get install libboost-random-dev libboost-python-dev libboost-iostreams-dev
sudo apt-get install scons lsb-release libsuitesparse-dev
\end{shellCode}

\noindent Then navigate to the source directory and execute the following
\begin{shellCode}
scons -j4 options_file=scons/templates/mint_options.py
\end{shellCode}

\subsection{Arch}\label{sec:archsrc}
These instructions were prepared on Arch Linux. \newline

First, install the dependencies that escript uses:
\begin{shellCode}
pacman -Sy --noconfirm python python-numpy python-scipy
pacman -Sy --noconfirm community/hdf5-serial
pacman -Sy --noconfirm extra/boost extra/boost-libs suitesparse
\end{shellCode}

Now you can compile \escript using the command
\begin{shellCode}
scons options_file=scons/templates/arch_py3_options.py -j4 build_full
\end{shellCode}

\subsection{OpenSuse}\label{sec:susesrc}
These instructions were prepared using OpenSUSE Leap 15.2. \newline

\noindent As a preliminary step, you should install the dependencies that \esfinley requires from the repository.
\noindent  If you intend to use Python 2.7, then you should install the following packages
\begin{shellCode}
sudo zypper in python-devel python2-numpy
sudo zypper in python2-scipy python2-sympy python2-matplotlib
sudo zypper in gcc gcc-c++ scons hdf5-devel
sudo zypper in libboost_python-py2_7-1_66_0-devel libboost_numpy-py2_7-1_66_0-devel
sudo zypper in libboost_iostreams1_66_0-devel suitesparse-devel
\end{shellCode}

\noindent If you intend to use Python 3.0, then you should instead install the following packages
\begin{shellCode}
sudo zypper in python3-devel python3-numpy
sudo zypper in python3-scipy python3-sympy python3-matplotlib
sudo zypper in gcc gcc-c++ scons hdf5-devel
sudo zypper in libboost_python-py3-1_66_0-devel libboost_numpy-py3-1_66_0-devel
sudo zypper in libboost_random1_66_0-devel libboost_iostreams1_66_0-devel
sudo zypper in suitesparse-devel
\end{shellCode}

\noindent Now to build escript itself.
\noindent In the escript source directory execute the following (substitute \textit{opensuse_py2} or \textit{opensuse_py3} as appropriate for XXXX):
\begin{shellCode}
scons -j4 options_file=scons/templates/XXXX_options.py
\end{shellCode}

\noindent If you wish to test your build, you can use the following:
\begin{shellCode}
scons -j4 py_tests options_file=scons/templates/XXXX_options.py
\end{shellCode}

\noindent Now go to Section~\ref{sec:cleanup} for cleanup.

\subsection{CentOS}\label{sec:centossrc}
It is possible to install \escript on both CentOS releases $7$ and $8$. We include separate instructions for each of these CentOS releases in this section.
\subsubsection{CentOS release $7$}
The core of escript works, however some functionality is not available because the default packages for some dependencies in CentOS are too old.
At present, it is not possible to compile \escript using Python 3.0+ on CentOS $7$ as Python 3.0+ versions of many of the dependencies are not currently available in any of the CentOS repositories, but this may change in the future.
In this section we only outline how to install a version of \escript that uses Python 2.7.

\noindent First, add the \texttt{EPEL} repository.
\begin{shellCode}
yum install epel-release.noarch
\end{shellCode}

\noindent Install packages:
\begin{shellCode}
yum install hdf5-devel
yum install python-devel numpy scipy sympy python2-scons
yum install python-matplotlib gcc gcc-c++ boost-devel
yum install boost-python suitesparse-devel
\end{shellCode}

\noindent Now to build escript itself.
In the escript source directory:
\begin{shellCode}
scons -j4 options_file=scons/templates/centos7_0_options.py
\end{shellCode}

\noindent Now go to Section~\ref{sec:cleanup} for cleanup.

\subsubsection{CentOS release $8$}
The core of escript works in CentOS $8$, however some functionality is not available because the default packages for some dependencies in CentOS are too old. This install is for Python 3.

First, add the EPEL, PowerTools and Okay repositories:
\begin{shellCode}
yum update
yum install epel-release.noarch dnf-plugins-core
yum config-manager --set-enabled PowerTools
rpm -ivh http://repo.okay.com.mx/centos/8/x86_64/release/okay-release-1-3.el8.noarch.rpm
yum update
\end{shellCode}

Now, install the packages:
\begin{shellCode}
yum install python3-devel python3-numpy python3-scipy
yum install boost-devel boost-python3 boost-python3-devel
yum install gcc gcc-c++ scons
yum install suitesparse suitesparse-devel
\end{shellCode}

Finally, you can compile \escript with the command
\begin{shellCode}
scons -j4 options_file=scons/templates/centos8_0_options.py
\end{shellCode}

\subsection{Fedora}\label{sec:fedorasrc}
These instructions were prepared using Fedora $31$ Workstation.

\noindent To build the a version of \escript that uses Python 2.7, install the following packages:
\begin{shellCode}
yum install gcc-c++ scons suitesparse-devel
yum install python2-devel boost-python2-devel
yum install python2-scipy
yum install hdf5-devel
\end{shellCode}

\noindent To build the a version of \escript that uses Python 3.0+, install the following packages:
\begin{shellCode}
yum install gcc-c++ scons suitesparse-devel
yum install python3-devel boost-python3-devel
yum install python3-scipy python3-matplotlib
yum install boost-python3 boost-numpy3 boost-iostreams boost-random
yum install hdf5-devel
\end{shellCode}

\noindent In the source directory execute the following (substitute \textit{fedora_py2} or \textit{fedora_py3} for XXXX):
\begin{shellCode}
scons -j4 options_file=scons/templates/XXXX_options.py
\end{shellCode}

\noindent Now go to Section~\ref{sec:cleanup} for cleanup.

\subsection{MacOS 10.10 and later (macports)}\label{sec:macportsrc}

The following will install the capabilities needed for the \texttt{macports_10.10_options.py} file (later versions can use the same options file).

\begin{shellCode}
sudo port install scons
sudo port select --set python python27
sudo port install boost
sudo port install py27-numpy
sudo port install py27-sympy
sudo port select --set py-sympy py27-sympy
sudo port install py27-scipy
sudo port install hdf5
sudo port install silo
\end{shellCode}

\begin{shellCode}
scons -j4 options_file=scons/templates/macports_10.10options.py
\end{shellCode}


\subsection{MacOS 10.13 and later (homebrew)}\label{sec:homebrewsrc}

The following will install the capabilities needed for the \texttt{homebrew_10.13_options.py} file.

\begin{shellCode}
brew install scons
brew install boost-python
brew install hdf5
\end{shellCode}

There do not appear to be formulae for \texttt{sympy}  so if you wish to use those features, then
you will need to install them separately.


\begin{shellCode}
scons -j4 options_file=scons/templates/homebrew_10.13_options.py
\end{shellCode}


\subsection{FreeBSD}\label{sec:freebsdsrc}

At time of writing, \texttt{numpy} does not install correctly on FreeBSD.
Since \texttt{numpy} is a critical dependency for \escript, we have been unable to test on FreeBSD.

\begin{comment}
\subsubsection{Release 10.0}

Install the following packages:
\begin{itemize}
 \item python
 \item scons
 \item boost-python-libs
 \item bash
 \item hdf5
 \item silo
 \item py27-scipy
 \item py27-matplotlib
 \item py27-sympy
\end{itemize}

\noindent Next choose (or create) your options file.
For the setup as above the escript source comes with a prepared file in
\texttt{scons/templates/freebsd10.0_options.py}.
Finally to build escript issue the following in the escript source directory
(replace the options file as required):
\begin{shellCode}
scons -j4 options_file=scons/templates/freebsd10.0_options.py
\end{shellCode}

\emph{Note:} Some packages installed above are built with gcc 4.7. Somewhere
in the toolchain a system-installed gcc library is pulled in which is
incompatible with the one from version 4.7 and would prevent escript from
executing successfully. As explained in the FreeBSD
documentation\footnote{see \url{http://www.freebsd.org/doc/en/articles/custom-gcc/article.html}}
this can be fixed by adding a line to \texttt{/etc/libmap.conf}:
\begin{shellCode}
libgcc_s.so.1 gcc47/libgcc_s.so.1
\end{shellCode}

\end{comment}
\subsection{Windows}\label{sec:windowssrc}

\noindent These instructions were prepared for Microsoft Windows 10.

\noindent Start by installing \escript dependencies.

\begin{itemize}
\item Microsoft Visual Studio
\begin{enumerate}
\item Download the Microsoft Visual Studio Community 2017 (or VS 2019 if
preferred) installer from
\begin{itemize}
\item[] \url{https://visualstudio.microsoft.com}.
\end{itemize}
\item Launch the Visual Studio installer, selecting Individual components:
\begin{itemize}
\item VS 2017: \textbf{VC++ 2017 latest v141 tools} \\
VS 2019: \textbf{MSVC v142 - VS 2019 C++ build tools}
\item \textbf{Windows 10 SDK}
\item \textbf{MSBuild}
\item \textbf{Visual C++ tools for CMake}
\end{itemize}
\end{enumerate}
\item Anaconda
\begin{enumerate}
\item Download the Python 3.7 64-Bit Graphical Installer for Windows from
\begin{itemize}
\item[] \url{https://anaconda.org/}.
\end{itemize}
\item Launch the Anaconda installer, selecting installation type: \textbf{Just
Me} and destination folder: \newline \verb!C:\Users\%USERNAME%\Anaconda3!.
\end{enumerate}
\end{itemize}

\noindent Next, open Windows Command Prompt (\file{cmd.exe}) and set-up the
\escript dependencies.

\begin{itemize}
\item Conda environment
\begin{enumerate}
\item Create and activate a new environment
\begin{shellCode}
C:\Users\%USERNAME%\Anaconda3\Scripts\activate
conda create --name escript python=3.7
conda deactivate
C:\Users\%USERNAME%\Anaconda3\Scripts\activate escript
\end{shellCode}
\item Install required conda modules
\begin{shellCode}
conda install numpy==1.15.4 matplotlib==2.2.2 sympy==1.1.1
    boost git scons scipy m2-patch mumps gmsh
    -c defaults -c conda-forge
\end{shellCode}
\end{enumerate}
\item Vcpkg
\begin{enumerate}
\item Build vcpkg package manager
\begin{shellCode}
cd C:\Users\%USERNAME%
git clone https://github.com/Microsoft/vcpkg.git
cd vcpkg
bootstrap-vcpkg
\end{shellCode}
\item Install the CppUnit vcpkg package
\begin{shellCode}
vcpkg install cppunit:x64-windows
\end{shellCode}
\end{enumerate}
\end{itemize}

\noindent Once the dependencies are installed and set-up, you can download and
build \escript from source.

\begin{enumerate}
\item Activate the conda environment (if not active).
\begin{shellCode}
C:\Users\%USERNAME%\Anaconda3\Scripts\activate escript
\end{shellCode}
\item Set-up the Command Prompt for the 64-bit MSVC command line build environment.
\begin{shellCode}
"C:\Program Files (x86)\Microsoft Visual Studio\2017\
    Community\VC\Auxiliary\Build\vcvars64.bat"
\end{shellCode}
\item Add CppUnit to the Windows System Path.
\begin{shellCode}
set PATH=%PATH%;C:\Users\%USERNAME%\vcpkg\packages\cppunit_x64-windows\bin
\end{shellCode}
\item Download the \escript source code tarball from
\begin{itemize}
\item[] \url{https://launchpad.net/escript-finley}
\end{itemize}
Extract the tarball to \verb!C:\Users\%USERNAME%\escript!
\item Build and install the netCDF-4 C++ library before starting the \escript
build.  Download the netCDF-4 C++ v4.3.1 source code tarball from
\begin{itemize}
\item[] \url{https://github.com/Unidata/netcdf-cxx4/archive/v4.3.1.tar.gz}
\end{itemize}
Extract the tarball to \verb!C:\Users\%USERNAME%\escript!
\item Apply the provided patch.
\begin{shellCode}
cd C:\Users\%USERNAME%\escript\netcdf-cxx4-4.3.1
patch -p1 < ..\src\tools\anaconda\Anaconda3\netcdf-cxx4.patch
\end{shellCode}
\item Configure, build, and install netcdf-cxx4.
\begin{shellCode}
mkdir build
cd build
cmake -G "Visual Studio 15 2017 Win64" -DBUILD_SHARED_LIBS=OFF
    -DCMAKE_INSTALL_PREFIX="%CONDA_PREFIX%\Library"
    -DCMAKE_LIBRARY_PATH="%CONDA_PREFIX%\Library\lib"
    -DCMAKE_PREFIX_PATH="%CONDA_PREFIX%\Library"
    -DNETCDF_LIB_NAME="netcdf" -DHDF5_LIB_NAME="hdf5" ..
cmake --build . --config Release
cmake --build . --config Release --target install
\end{shellCode}
\item Kick-off the \escript build when the netCDF-4 C++ install is complete.
\begin{shellCode}
cd C:\Users\%USERNAME%\escript\src
scons -j4 build_full options_file=scons/templates/windows_msvc141_options.py
\end{shellCode}
\item Once the build completes successfully, you can validate \escript using
the provided test script.
\begin{shellCode}
python utest.py C:\Users\%USERNAME%\escript\src\build -t8
\end{shellCode}
\end{enumerate}

\subsection{Other Systems / Custom Builds}\label{sec:othersrc}

\escript has support for a number of optional packages.
Some, like \texttt{netcdf} need to be enabled at compile time, while others, such as \texttt{sympy} and the projection packages
used in \downunder are checked at run time.
For the second type, you can install them at any time (ensuring that python can find them) and they should work.
For the first type, you need to modify the options file and recompile with scons.
The rest of this section deals with this.

To avoid having to specify the options file each time you run scons, copy an existing \texttt{_options.py} file from the
\texttt{scons/} or \texttt{scons/templates/} directories. Put the file in the \texttt{scons} directory and name
it \textit{yourmachinename}\texttt{_options.py}.\footnote{If the name
has - or other non-alpha characters, they must be replaced with underscores in the filename}.
For example: on a machine named toybox, the file would be \texttt{scons/toybox_options.py}.

Individual lines can be enabled/disabled, by removing or adding \# (the python comment character) to the beginning of the line.
For example, to enable OpenMP, change the line
\begin{verbatim}
#openmp = True
\end{verbatim}
to
\begin{verbatim}
openmp = True
\end{verbatim}

If you are using libraries which are not installed in the standard places (or have different names) you will need to
change the relevant lines.
A common need for this would be using a more recent version of the boost::python library.
You can also change the compiler or the options passed to it by modifying the relevant lines.

\subsubsection*{MPI}
If you wish to enable or disable MPI, or if you wish to use a different implementation of MPI, you can use the \texttt{mpi}
configuration variable.
You will also need to ensure that the \texttt{mpi_prefix} and \texttt{mpi_libs} variables are uncommented and set correctly.
To disable MPI use, \verb|mpi = 'none'|.


\subsubsection{Testing}
As indicated earlier, you can test your build using \texttt{scons py_tests}.
Note however, that some features like \texttt{netCDF} are optional for using \escript, the tests will report a failure if
they are missing.

\section{Cleaning up}
\label{sec:cleanup}

Once the build (and optional testing) is complete, you can remove everything except:
\begin{itemize}
 \item bin
 \item esys
 \item lib
 \item doc
 \item CREDITS
 \item LICENSE
 \item README
\end{itemize}
The last three aren't strictly required for operation.
The \texttt{doc} directory is not required either but does contain examples of escript scripts.

You can run escript using \texttt{\textit{path_to_escript_files}/bin/run-escript}.\\
Where \texttt{\textit{path_to_escript_files}} is replaced with the real path.

\begin{optionalstep}
You can add the escript \texttt{bin} directory to your \texttt{PATH} variable.
The launcher will then take care of the rest of the environment.
\end{optionalstep}

\section{Optional Extras}

Some other packages which might be useful include:
\begin{itemize}
 \item Lapack and UMFPACK --- direct solvers (install the relevant libraries and enable them in the options file).
 \item support for the Silo file format (install the relevant libraries and enable them in the options file).
 \item VisIt --- visualisation package. Can be used independently but our \texttt{weipa} library can make a Visit
plug-in to allow direct visualisation of escript simulations.
 \item gmsh --- meshing software used by our \texttt{pycad} library.
 \item Mayavi2 --- another visualisation tool.
\end{itemize}


\subsection{Trilinos}
\escript now has some support for Trilinos\footnote{\url{https://trilinos.org/}}
solvers and preconditioners.
The most significant limitation is that the current Trilinos release does not
support block matrices so \escript can only use Trilinos solvers for single
PDEs (i.e. no PDE systems).

If your distribution does not provide Trilinos packages you can build a working
version from source. (See Appendix~\ref{app:trilinos})


\section{Testing \escript}\label{chap:utest}

\escript has extensive testing that can be used to confirm that the program is working correctly. To run the unit testing, compile \escript with the flag \texttt{build_full}. This will build \escript normally and then create a shell script named \texttt{utest.sh}. Once this file has been created, you can run unit testing using the command
\begin{shellCode}
sh utest.sh path_to_build_folder '-tT -nN -pP'
\end{shellCode}
where \texttt{T}, \texttt{N} and \texttt{P} represent the number of threads, nodes and processes to run the testing on. Some of these terms can be omitted. For example, to run the testing in serial, you would run
\begin{shellCode}
sh utest.sh path_to_build_folder '-t1'
\end{shellCode}

Note that a careless selection of these parameters may cause the testing program to skip many of the tests. For example, if you compile \escript with OpenMP enabled but then instruct the testing program to run on a single thread, many of the OpenMP tests will not be run.


\esysappendix
\chapter{Required compiler features}
\label{app:cxxfeatures}

Building escript from source requires that your c++ compiler supports at least the following features:
\begin{itemize}
 \item \texttt{std::complex<>}
 \item Variables declared with type \texttt{auto}
 \item Variables declared with type \texttt{decltype(T)}
 \item \texttt{extern template class} to prevent instantiation of templates.
 \item \texttt{template class \textit{classname$<$type$>$};} to force instantiation of templates
 \item \texttt{isnan()} is defined in the \texttt{std::} namespace
\end{itemize}
The above is not exhaustive and only lists language features which are more recent that our previous baseline of c++99 (or which
we have recently begun to rely on).
Note that we test on up to date versions of \texttt{g++, icpc \& clang++} so they should be fine.

Note that in future we may use c++14 features as well.

\chapter{Trilinos}
\label{app:trilinos}

In order to solve PDEs with complex coefficients, escript needs to be compiled with \texttt{Trilinos} support.
This requires that your version of Trilinos has certain features enabled.
Since some precompiled distributions of \texttt{Trilinos} are not built with these features, you may
need to compile \texttt{Trilinos} yourself as well.

While we can't provide support for building \texttt{Trilinos}, we provide here two configuration files which seem to work for
Debian 10 ``buster'. One of these cmake script builds \texttt{Trilinos} with MPI support and one builds \texttt{Trilinos} without MPI support.

\section{Debian ``buster'' configuration}


\subsection{Required packages}

The following packages should be installed to attempt this build:
\begin{itemize}
\item[] cmake
\item[] g++
\item[] libsuitesparse-dev
\item[] libmumps-dev
\item[] libboost-dev
\item[] libscotchparmetis-dev
\item[] libmetis-dev
\item[] libcppunit-dev
\end{itemize}
\end{document}
