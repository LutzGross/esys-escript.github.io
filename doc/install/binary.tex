%%%%%%%%%%%%%%%%%%%%%%%%%%%%%%%%%%%%%%%%%%%%%%%%%%%%%%%%%%%%%%%%%%%%%%%%%%%%%%
% Copyright (c) 2003-2014 by University of Queensland
% http://www.uq.edu.au
%
% Primary Business: Queensland, Australia
% Licensed under the Open Software License version 3.0
% http://www.opensource.org/licenses/osl-3.0.php
%
% Development until 2012 by Earth Systems Science Computational Center (ESSCC)
% Development 2012-2013 by School of Earth Sciences
% Development from 2014 by Centre for Geoscience Computing (GeoComp)
%
%%%%%%%%%%%%%%%%%%%%%%%%%%%%%%%%%%%%%%%%%%%%%%%%%%%%%%%%%%%%%%%%%%%%%%%%%%%%%%



\chapter{Binary Installation}\label{chap:bin}

Binary distributions (no compilation required) are available for Windows(Section~\ref{sec:binwin}) 
and ``recent'' Debian and Ubuntu releases.

Debian (i386 or amd64):
\begin{itemize}
 \item $6$ --- \emph{Squeeze}
 \item $?$ --- \emph{Wheezy}
\end{itemize}

Ubuntu (i386 or amd64):
\begin{itemize}
 \item $11.10$ --- \emph{Oneiric} Ocelot
 \item $12.04$ --- \emph{Precise} Pangolin (LTS)
 \item $12.10$ --- \emph{Quantal} Queztal 
 \item $13.04$ --- \emph{Raring} Rabbit
\end{itemize}


We produce \texttt{.deb}s for the i386 and amd64 architectures for Debian stable(``squeeze'') and the 
following Ubuntu releases:
\begin{itemize}
 \item $11.10$ --- \emph{Oneiric} Ocelot
 \item $12.04$ --- \emph{Precise} Pangolin (LTS)
 \item $12.10$ --- \emph{Quantal} Queztal 
\end{itemize}

The package file will be named \file{escript-X-D_A.deb} where \texttt{X} is the version, \texttt{D} 
is the distribution codename (eg ``\texttt{squeeze}'' or ``\texttt{oneric}'') and \texttt{A} is the architecture.
For example, \file{escript-3.4-1-squeeze_amd64.deb} would be the file for squeeze for 64bit processors.
To install \esfinley download the appropriate \file{.deb} file and execute the following 
commands as root (you need to be in the directory containing the file):

\begin{verse}
\textbf{(For Ubuntu users)}\\
You will need to either install \texttt{aptitude}\footnote{Unless you are short on disk space \texttt{aptitude} is recommended} or substitute \texttt{apt-get} where this guide uses \texttt{aptitude}.
\begin{shellCode}
sudo apt-get install aptitude
\end{shellCode}
\end{verse}

\begin{shellCode}
dpkg --unpack escript*.deb
aptitude install escript
\end{shellCode}

Installing escript should not remove any packages from your system.
If aptitude suggests removing escript, then choose 'N'.
It should then suggest installing some dependencies choose 'Y' here.
If it suggests removing \texttt{escript-noalias} then agree.

If you use sudo (for example on Ubuntu) enter the following instead:
\begin{shellCode}
sudo dpkg --unpack escript*.deb
sudo aptitude install escript
\end{shellCode}

This should install \esfinley and its dependencies on your system.
Please notify the development team if something goes wrong.


%%%%%%%%%%%%%%%%%%%%%%%%%%%%%%%%%%%%%%%%%%%%%%%%%%%%%%%%
%
% Copyright (c) 2003-2008 by University of Queensland
% Earth Systems Science Computational Center (ESSCC)
% http://www.uq.edu.au/esscc
%
% Primary Business: Queensland, Australia
% Licensed under the Open Software License version 3.0
% http://www.opensource.org/licenses/osl-3.0.php
%
%%%%%%%%%%%%%%%%%%%%%%%%%%%%%%%%%%%%%%%%%%%%%%%%%%%%%%%%

\section{Windows binary installation}
\label{sec:binwin}

There is no automated install/uninstall procedure for \esfinley on Windows at this time.
However, the following procedure appears to work for the previous release\footnote{Thanks to Peter Hornby and Frederick Roger for this report.}

Please ensure you have the following software installed: 
\begin{itemize}
 \item pythonxy (\url{http://www.pythonxy.com})
%\item numarray 1-5-2win32py2.5  for  win32
\end{itemize}

From the escript zip file:
\begin{itemize}
\item 
 copy the \filename{esys} directory to your Python 2.5 site-packages folder (usually \filename{C:\textbackslash Python25\textbackslash Lib\textbackslash site-packages}).
\item 
 copy the \filename{.dll} files from \filename{esys\_dlls} to a directory on your PATH. For example copy the directory to \filename{C:\textbackslash Python25\textbackslash libs\textbackslash esys\_dlls} and add  \filename{C:\textbackslash Python25\textbackslash libs\textbackslash esys\_dlls} to your PATH.
\end{itemize}




