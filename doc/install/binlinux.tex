%%%%%%%%%%%%%%%%%%%%%%%%%%%%%%%%%%%%%%%%%%%%%%%%%%%%%%%%
%
% Copyright (c) 2003-2010 by University of Queensland
% Earth Systems Science Computational Center (ESSCC)
% http://www.uq.edu.au/esscc
%
% Primary Business: Queensland, Australia
% Licensed under the Open Software License version 3.0
% http://www.opensource.org/licenses/osl-3.0.php
%
%%%%%%%%%%%%%%%%%%%%%%%%%%%%%%%%%%%%%%%%%%%%%%%%%%%%%%%%

\section{Linux binary installation}
\label{sec:binlinux}

\esfinley can be installed as a stand-alone bundle, containing all the required dependencies.
Alternatively, if we have a package for your distribution you can use the standard tools to install.
Please note, however, that the current binary packages do not support \openmp\footnote{This is due to a bug related to gcc 4.3.2.} or \mpi\footnote{Producing packages for \mpi requires knowing something about your computer's configuration.}.
If you need these features you may need to compile \esfinley from source (see Section~\ref{sec:compilesrc} and Section~\ref{sec:compileescriptlinux}.)

For more information on using the \file{run-escript} command please see the User's Guide.

If you are using Debian (5.0 - ``Lenny'') or Ubuntu (8.10-``Intrepid Ibex'', 9.04-``Jaunty Jackalope'') then see Section~\ref{sec:debian}.
For Ubuntu 9.10-``karmic koala'' see Section~\ref{sec:karmic}.
For other Linux distributions refer to Section~\ref{sec:standalonelinux}.

\subsection{Debian 5.0(``Lenny''), Ubuntu 8.10(``Intrepid Ibex'') or 9.04(``Jaunty Jakalope'')}\label{sec:debian}

At the time of this writing we only produce deb's for the i386 and amd64 architectures.
The package file will be named \file{escript-X-D_A.deb} where \texttt{X} is the version, \texttt{D} is either ``\texttt{lenny}'' or ``\texttt{jaunty}'' and \texttt{A} is the architecture.
For example, \file{escript-3.0-1-lenny_amd64.deb} would be the file for lenny (and intrepid) for 64bit processors.
To install \esfinley download the appropriate \file{.deb} file and execute the following commands as root (you need to be in the directory containing the file):
\begin{shellCode}
dpkg --unpack escript*.deb
aptitude install escript
\end{shellCode}

If you use sudo (for example on Ubuntu) enter the following instead:
\begin{shellCode}
sudo dpkg --unpack escript*.deb
sudo aptitude install escript
\end{shellCode}

This should install \esfinley and its dependencies on your system.
Please notify the development team if something goes wrong.

\subsection{Ubuntu 9.10(``Karmic Koala'')}\label{sec:karmic}

You will need to download either \file{escript-noalias-3.1-1-lenny_i386.deb} (for 32bit processors) or \file{escript-noalias-3.1-1-lenny_amd64.deb} (for 64bit processors).

Type the following in the directory containing the file.
\begin{shellCode}
sudo dpkg --unpack escript-noalias*.deb
sudo aptitude install escript
\end{shellCode}



\subsection{Stand-alone bundle}\label{sec:standalonelinux}

If there is no package available for your distribution, you may be able to use one of our stand alone bundles.
These come in two parts: escript itself (\file{escript_3.0_i386.tar.bz2}) and a group of required programs (\file{escript-support_3.0_i386.tar.bz2}). For $64$-bit Intel and Amd processors substitute \texttt{amd64} for \texttt{i386}.
\begin{shellCode}
tar -xjf escript-support_3.0_i386.tar.bz2
tar -xjf escript_3.0_i386.tar.bz2
\end{shellCode}
This will produce a directory called \file{stand} which contains a stand-alone version of \esfinley and its dependencies.
You can rename or move it as is convenient to you, no installation is required.
Test your installation by running:
\begin{shellCode}
stand/escript.d/bin/run-escript
\end{shellCode}
This should give you a normal python shell.
If you wish to save on typing you can add \file{x/stand/escript.d/bin}\footnote{or whatever you renamed \texttt{stand} to.} to your \texttt{PATH} variable (where x is the absolute path to your install).

