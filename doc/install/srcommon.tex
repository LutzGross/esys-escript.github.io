%%%%%%%%%%%%%%%%%%%%%%%%%%%%%%%%%%%%%%%%%%%%%%%%%%%%%%%%
%
% Copyright (c) 2003-2009 by University of Queensland
% Earth Systems Science Computational Center (ESSCC)
% http://www.uq.edu.au/esscc
%
% Primary Business: Queensland, Australia
% Licensed under the Open Software License version 3.0
% http://www.opensource.org/licenses/osl-3.0.php
%
%%%%%%%%%%%%%%%%%%%%%%%%%%%%%%%%%%%%%%%%%%%%%%%%%%%%%%%%

% This file contains material common to all src distributions.

% The original version of this content came from the esscc twiki page maintained by ksteube

This chapter describes how to build \esfinley from source assuming that the dependencies are already installed (for example using precompiled packages for your OS).
\Sec{sec:deps} describes the dependencies, while \Sec{sec:compilesrc} gives the compile instructions.

If you would prefer to build all the dependecies from source in the escript-support packages please see \Chap{chap:allsrc}.
\esfinley is known to compile and run on the following systems:
\begin{itemize}
 \item \linux using gcc\footnote{There are some problems with \openmp under gcc prior to version 4.3.2. Also do not link the gomp library with gcc 4.3.3 - it causes problems.}
\item \linux using icc on SGI ICE 8200
\item \macosx using gcc
\item \winxp using the Visual C compiler (we do not specifically discuss Windows builds in this guide).
\end{itemize}

\section{External dependencies}
\label{sec:deps}
The following external packages are required in order to compile and run \esfinley.
Where version numbers are specified, more recent versions can probably be subsituted.
You can either try the standard/precompiled packages available for your operating system or you can download and build them from source.
The advantage of using existing packages is that they are more likely to work together properly.
You must take greater care if downloading sources separately.

\begin{itemize}
 \item python-2.5.1 (\url{http://python.org}) \\-
        Python interpreter (you must compile with shared libraries.)
 \item numpy 1.1.0 (\url{http://numpy.scipy.org}) \\-
        Arrays for python
 \item boost-1.35 (\url{http://www.boost.org}) \\-
        Interface between C++ and Python
 \item scons-0.989.5 (\url{http://www.scons.org/}) \\-
        Python-based alternative to ``make''.
\end{itemize}

The version numbers given here are not strict requirements, more recent (and in some cases older) versions are very likely to work.
The following packages should be sufficient (but not necessarily minimal) for Debian 5.0 (``Lenny''):
python-dev, libboost-python1.35-dev, scons, python-numpy, g++.

These packages may be required for some of the optional capabilities of the system:

\begin{itemize}
 \item netcdf-3.6.2 (\url{http://www.unidata.ucar.edu/software/netcdf}) \\-
        Used to save data sets in binary form for checkpoint/restart (must be compiled with -fPIC)
 \item vtk-5.0.4 (\url{http://www.vtk.org}) \\-
        Used to save VTK files for visualization
  \begin{itemize}
   \item cmake-2.4.6 (\url{http://www.cmake.org}) \\-
        Required to build VTK
   \item mesa-7.0.3 (\url{http://www.mesa3d.org})\\-
        Free OpenGL replacement used by VTK
  \end{itemize}
 \item netpbm (\url{http://netpbm.sourceforge.com}) \\-
        Tools for producing movies from images
 \item mpich2-1.0.7 (\url{http://www.mcs.anl.gov/research/projects/mpich2}) \\-
        Parallelization with \mpi
 \item parmetis-3.1 (\url{http://glaros.dtc.umn.edu/gkhome/metis/parmetis/overview}) \\-
        Optimization of the stiffness matrix
 \item MKL \\(\url{http://www.intel.com/cd/software/products/asmo-na/eng/307757.htm}) \\-
        Intel's Math Kernel Library for use with their C compiler.
\end{itemize}

The following packages might be useful for mesh generation:
\begin{itemize}
 \item gmsh-2.2.0 (\url{http://www.geuz.org/gmsh}) \\-
        Mesh generation and viewing
  \begin{itemize}
   \item fltk-1.1.9 (\url{http://www.fltk.org}) \\-
        Required to build gmsh 
   \item gsl-1.10 (\url{http://www.gnu.org/software/gsl}) \\-
        Required to build gmsh 
  \end{itemize}
 \item triangle-1.6 (\url{http://www.cs.cmu.edu/~quake/triangle.html}) \\-
        Two-dimensional mesh generator and Delaunay triangulator.
\end{itemize}

Packages for visualization:
\begin{itemize}
 \item mayavi-1.5 (\url{http://mayavi.sourceforge.net}) \\-
        MayaVi is referenced in our User's Guide for viewing VTK files
 \item visit-1.11.2 (\url{https://wci.llnl.gov/codes/visit/}) \\-
        A featureful visualization system with movie-making capabilities.
\end{itemize}

\section{Compilation}\label{sec:compilesrc}
Throughout this section we will assume that the source code is uncompressed in a directory called \filename{escript.d}.
You can call the directory anything you like, provided that you make the change before you compile.

You need to indicate where to find the external dependencies.
To do this, create a file in the \filename{escript.d/scons} directory called \filename{x_options.py} where ``x'' is the name of your computer (output of the \texttt{hostname} command).
From now on all paths will be relative to the top level of the source.
As a starting point copy the contents one of the following files :
\begin{itemize}
 \item \filename{scons/linux_options_example.py} (\linux desktop)
\item \filename{scons/mac_options_example.py} (\macosx desktop)
\item \filename{scons/ice_options_example.py} (SGI ICE 8200)
\item \filename{scons/winxp_options_example.py} (\winxp)
\end{itemize}

To actually compile (if you have $n$ processors, then you can use \texttt{scons -j$n$} instead):

\begin{shellCode}
cd escript.d
scons
\end{shellCode}

As part of its output, scons will tell you the name of the options file it used as well as a list of features 
and whether they are enabled for your build.

If you require debug versions of the libraries, use:
\begin{shellCode}
 scons usedebug=yes
\end{shellCode}
A note about scons: if you recompile later with different options (e.g. leaving out usedebug), scons will revert 
to its default values. If you wish to make a change more permanent, then modify your options file.


You can install the binaries/libraries in a different location with:
\begin{shellCode}
 scons prefix=some_dir
\end{shellCode}

You can test your build using 
\begin{shellCode}
scons all_tests
\end{shellCode}
Grab a coffee or two while the tests compile and run.
An alternative method is available for performing tests on \openmp and \mpi builds.

\subsection{Compilation with \openmp}
You will need to consult your compiler documentation for the precise switches to use to enable \openmp features.
Once you know the options, modify the omp_optim, omp_debug and omp_libs variables in your options.py file.

For example, for gcc compilers which support \openmp use:
\begin{shellCode}
omp_optim		= '-fopenmp'
omp_debug		= '-fopenmp'
omp_libs		= ['gomp']
\end{shellCode}
Depending on your version, the last change may not be required.
If you're unsure try without the gomp library first and add it if you get linker errors.

Then recompile.
\begin{shellCode}
 scons useopenmp=yes
\end{shellCode}

You can test your build, e.g. using 4 threads by issuing
\begin{shellCode}
export ESCRIPT_NUM_THREADS=4
scons all_tests
\end{shellCode}

\subsection{Compilation with \mpi}
You will need to have \mpi installed on your system.
There are a number of implementations so we do not provide any specific advice here.
You will need to modify the following variables in your options file.
\begin{itemize}
 \item \texttt{mpi_flavour} \\
	which \mpi implementation is used. Valid values are
    \begin{itemize}
        \item[\texttt{MPT}] SGI MPI implementation \\
            \url{http://techpubs.sgi.com/library/manuals/3000/007-3687-010/pdf/007-3687-010.pdf}
        \item[\texttt{MPICH2}] Argonne's MPICH version 2 implementation \\
            \url{http://www.mcs.anl.gov/research/projects/mpi/mpich2/}
        \item[\texttt{MPICH}] Argonne's MPICH implementation \\
            \url{http://www.mcs.anl.gov/research/projects/mpi/mpich1/}
        \item[\texttt{OPENMPI}] Open MPI \\
            \url{http://www.open-mpi.org/}
        \item[\texttt{INTELMPI}] Intel's MPI \\
            \url{http://software.intel.com/en-us/intel-mpi-library/}
    \end{itemize}
 \item \texttt{mpi_path} \\
	where to find \filename{mpi.h}
 \item \texttt{mpi_lib_path} \\
	where to find libraries for \mpi
 \item \texttt{mpi_libs} \\
	which libraries to link to.
\end{itemize}

Then compile with:
\begin{shellCode}
 scons usempi=yes
\end{shellCode}

As with debug and openmp, you can make this a more permanent setting by modifying your options file.

To test your build using 6 processors enter:
\begin{shellCode}
export ESCRIPT_NUM_NODES=6
scons usempi=yes all_tests
\end{shellCode}
and on 6 processors with 4 threads each using 
\begin{shellCode}
export ESCRIPT_NUM_THREADS=4
export ESCRIPT_NUM_NODES=6
scons usempi=yes all_tests
\end{shellCode}
Alternatively, you can give a hostfile
\begin{shellCode}
export ESCRIPT_NUM_THREADS=4
export ESCRIPT_HOSTFILE=myhostfile
scons usempi=yes all_tests
\end{shellCode}
Note that depending on your \mpi flavour it may be required to start a daemon before running the tests under \mpi.


\subsection{Difficulties}

%This is copied from Ken's notes on the old Twiki page
\subsubsection{``Bad magic number''}
This error usually indicates that the version of python used to run escript differs from the version used when installing escript (Use \texttt{which python} and \texttt{python --version} to check).

It is also possible that incompatible libraries were used when compiling \esfinley.
For example, if you run with Python2.4 but the software was compiled against Python2.5 then you will get unsatisfied externals or a large error message with a long traceback.
Another case is when Boost or Numarray was compiled against the wrong Python library.
To avoid these problems both builder and user must ensure they are using the same python libraries.

\subsubsection{\openmp builds segfault running examples}

One known cause for this is linking the \filename{gomp} library with escript built using gcc 4.3.3.
While you need the \texttt{-fopenmp} switch you should not need to link \filename{gomp}.

