
%%%%%%%%%%%%%%%%%%%%%%%%%%%%%%%%%%%%%%%%%%%%%%%%%%%%%%%%
%
% Copyright (c) 2003-2008 by University of Queensland
% Earth Systems Science Computational Center (ESSCC)
% http://www.uq.edu.au/esscc
%
% Primary Business: Queensland, Australia
% Licensed under the Open Software License version 3.0
% http://www.opensource.org/licenses/osl-3.0.php
%
%%%%%%%%%%%%%%%%%%%%%%%%%%%%%%%%%%%%%%%%%%%%%%%%%%%%%%%%

\section{Installing from source for \macosx}
\label{sec:srcmac}

First of all before you start installing from source you will need Mac OS X development tools installed on your Mac. This will ensure that you have  the following installed:
\begin{itemize}
\item \filename{g++} and associated tools.
\item \filename{make}
\end{itemize}

Here are the instruction on how to install them. 
\begin{enumerate}
\item Insert the Mac OS X v10.5 (Leopard) DVD. 
\item Double-click on XcodeTools.mpkg, located inside Optional Installs/Xcode Tools. 
\item Follow the instructions in the Installer. 
\item Authenticate as the administrative user. The first user you create when setting up Mac OS X has administrator privileges by default. 
\end{enumerate}

You will also need a copy of the \esfinley source code.
If you retrieved the source using subversion, don't forget that one can use the export command instead of checkout to get a smaller copy.
For additional visualisation functionality see Section~\ref{sec:macaddfunc}.

These instructions will produce the following structure:
\begin{itemize}
\item \filename{stand}: \begin{itemize}
\item \filename{escript.d}
\item \filename{packages}
\item \filename{package_src}
\item \filename{build}
\item \filename{doc}
  \end{itemize}
\end{itemize}

The build directory can be removed when you are finished.

The following instructions assume you are running the \filename{bash} shell.
Comments are indicated with \# characters. 

Open a terminal~\footnote{If you do not know how to open a terminal on Mac, then just type terminal in the spotlight (search tool on the top of the right corner) and once found just click on it.} and type

\begin{shellCode}
mkdir stand
cd stand
export PKG_ROOT=`pwd`/packages
\end{shellCode}

Copy compressed source bundles into \filename{stand/package_src}.
Copy documentation files into \filename{doc}.

\begin{shellCode}
mkdir packages
mkdir build
cd build
tar -jxf ../package_src/Python-2.6.1.tar.bz2
tar -jxf ../package_src/boost_1_38_0.tar.bz2
tar -jxf ../package_src/MesaLib-7.2.tar.bz2
tar -zxf ../package_src/netcdf-4.0.tar.gz
tar -zxf ../package_src/vtk-5.2.1.tar.gz
tar -zxf ../package_src/vtkdata-5.2.1.tar.gz
tar -zxf ../package_src/numarray-1.5.2.tar.gz
tar -zxf ../package_src/cmake-2.6.3.tar.gz
tar -zxf ../package_src/scons-1.2.0.tar.gz
\end{shellCode}

Build python.
\begin{shellCode}
cd Python*

./configure --prefix=$PKG_ROOT/python-2.6.1 \
  --exec-prefix=$PKG_ROOT/python-2.6.1 --enable-shared \
  --enable-framework=$PKG_ROOT/python-2.6.1 2>&1 | tee tt.configure.out
  
make -j2

make install 2>&1 | tee tt.make.out

cp ../../packages/python-2.6.1/Python.framework/Versions/2.6/Python \
     ../../packages/python-2.6.1/lib/libpython2.6.dylib

cp ../../packages/python-2.6.1/Python.framework/Versions/2.6/Python \
      ../../packages/python-2.6.1/lib/libpython.dylib

cp ../../packages/python-2.6.1/include/python2.6/pyconfig.h \
     ../../packages/python-2.6.1/Python.framework/Headers/

cd ..

export PATH=$PKG_ROOT/python/bin:$PATH
export PYTHONHOME=$PKG_ROOT/python
export LD_LIBRARY_PATH=$PKG_ROOT/python/lib:$LD_LIBRARY_PATH

pushd ../packages
ln -s python-2.6.1/ python
popd

\end{shellCode}

Run python to make sure~(check the version number) it works.
Now build numarray.

\begin{shellCode}
cd numarray-1.5.2

python setup.py install \
 --gencode --install-lib=$PKG_ROOT/numarray-1.5.2/lib \
 --install-headers=$PKG_ROOT=$PKG_ROOT/numarray-1.5.2/include/numarray \ 
   2>&1 | tee tt.install.out


export PYTHONPATH=$PKG_ROOT/numarray/lib:$PYTHONPATH
cd ..
pushd ../packages
ln -s numarray-1.5.2 numarray
popd
\end{shellCode}

Now we build scons.
\begin{shellCode}
cd scons-1.2.0
python setup.py install --prefix=$PKG_ROOT/scons-1.2.0

export PATH=$PKG_ROOT/scons/bin:$PATH
cd ..
pushd ../packages
ln -s scons-1.2.0 scons
popd
\end{shellCode}

...Boost libraries ...
\begin{shellCode}
cd boost_1_38_0

./configure --prefix=$PKG_ROOT/boost_1_38_0 --with-python-root=$PKG_ROOT/python \
  --with-python-version=2.6 --with-libraries=python

make -j2
make install
ln -s $PKG_ROOT/boost_1_38_0 $PKG_ROOT/boost
export LD_LIBRARY_PATH=$PKG_ROOT/boost/lib:$LD_LIBRARY_PATH
cd ..
pushd ../packages
ln -s boost_1_38_0 boost
popd
\end{shellCode}

... and netcdf.
\begin{shellCode}
cd netcdf-4.0
CFLAGS="-O2 fPIC -Df2cFortran" CXXFLAGS="-O2 fPIC -Df2cFortran" \
FFLAGS="-O2 fPIC -Df2cFortran" FCFLAGS="-O2 fPIC -Df2cFortran" \
./configure --prefix=$PKG_ROOT/netcdf-4.0

make -j2
make install

export LD_LIBRARY_PATH=$PKG_ROOT/netcdf/lib:$LD_LIBRARY_PATH
cd ..
pushd ../packages
ln -s netcdf-4.0 netcdf
popd
\end{shellCode}

CMake and Mesa are required for VTK. 
\begin{shellCode}
cd cmake-2.6.3
./configure --prefix=$PKG_ROOT/cmake-2.6.3 2>&1 | tee tt.configure
make -j 4
make install

export PATH=$PKG_ROOT/cmake/bin:$PATH
cd ..
pushd ../packages
ln -s cmake-2.6.3 cmake
popd
\end{shellCode}

These instructions do not compile MesaDemos or GLUT.
If you need to check if Mesa compiled correctly, then the demos are a good test.
\begin{shellCode}
cd Mesa-7.2
./configure --prefix=$PKG_ROOT/mesa-7.2 --enable-gl-osmesa

make -j 4
make install

export LD_LIBRARY_PATH=$PKG_ROOT/mesa:$LD_LIBRARY_PATH
cd ..
pushd ../packages
ln -s mesa-7.2 mesa
popd
\end{shellCode}

\begin{shellCode}
cd VTK
cmake .

#Edit the CMakeCache and make the following changes: 
#(Please replace .... with an absolute path to the stand directory)

#-----------------

BUILD_EXAMPLES should be OFF
BUILD_SHARED_LIBS should be ON

CMAKE_INSTALL_PREFIX	..../stand/packages/vtk-5.2.1
CMAKE_VERBOSE_MAKEFILE	TRUE

#check PYTHON_EXECUTABLE is correct.
#but it seems to be when I went through these steps

VTK_OPENGL_HAS_OSMESA	TRUE
VTK_USE_64BIT_IDS	ON
# That last one is marked as "May cause some bugs" in the original instructions

VTK_WRAP_PYTHON	ON
VTK_USE_MANGLED_MESA	OFF

#--------------------

cmake .
#It won't work but it will put some variables in that you need.

#Edit CMakeCache again and make the following changes

#----------------

VTK_USE_TK	OFF

OSMESA_INCLUDE_DIR	..../stand/packages/mesa/include

OSMESA_LIBRARY	..../stand/packages/mesa/lib/libOSMesa.dylib

PYTHON_INCLUDE_PATH	..../stand/packages/python/include/python2.6

PYTHON_LIBRARY	..../stand/packages/python/lib/libpython2.6.dylib

OPENGL_INCLUDE_DIR	..../stand/packages/mesa/include

OPENGL_gl_LIBRARY	..../stand/packages/mesa/lib/libGL.dylib

#----------------

cmake .
make
make install


cd ../../packages
ln -s vtk-5.2.1 vtk
cd ..
\end{shellCode}

Now copy the \esfinley source into an \filename{escript.d} directory in \filename{stand}.

\begin{shellCode}
cd scons
cp mac_options_example.py YourMachineName_options.py

#edit the options file and make the following changes:
#-----------------------------------------------------------------
declare a PKG_ROOT variable at the top of the file eg:
PKG_ROOT='/Users/artak/stand/packages'

python_path		= PKG_ROOT+'python/include/python2.6'
python_lib_path		= PKG_ROOT+'python/lib'
python_libs		= 'python2.6'

boost_path		= PKG_ROOT+'boost/include/boost-1_38'
boost_lib_path		= PKG_ROOT+'boost/lib'
boost_libs		= ['boost_python-gcc43-mt']
# You could simlink the boost python library to give a shorter 
# name but it's not worth it

usevtk		= 'yes'
#-------------------------------------------------------------------

ln -s $PKG_ROOT/vtk-5.2.1 $PKG_ROOT/vtk

Modify /scripts/finley_wrapper_template

STANDALONE=1

#Check to make sure the paths in the if [ $STANDALONE == 1 ]
# Section are correct

#-----------------------------------------------------------------

scons bin/escript

#start a new terminal
cd stand
export PATH=`pwd`/packages/scons/bin:$PATH
cd escript.d
eval `bin/escript -e`
scons
\end{shellCode}

If you wish to test your build, then you can do the following. 
Note this may take a while if you have a slow processor and/or less than 1Gb of RAM.
\begin{shellCode}
scons all_tests
\end{shellCode}

Once you are satisfied, the \filename{build} and \filename{\$PKG_ROOT/build} directories can be removed.
Within the \filename{packages} directory, the \filename{scons}, \filename{scons-1.2.0}, \filename{cmake-2.6.3} and \filename{cmake} entries can also be removed.
If you are not redistributing this bundle you can remove \filename{\$PKG_ROOT/package_src}.

If you do not plan to edit or recompile the source you can remove it.
The only entries which are required in \filename{escript.d} are:
\begin{itemize}
 \item \filename{bin}
\item \filename{esys}
\item \filename{include}
\item \filename{lib}
\item \filename{README_LICENSE}
\end{itemize}

Hidden files can be removed with
\begin{shellCode}
find . -name .?* | xargs rm -rf
\end{shellCode}

\section{Additional Functionality}\label{sec:macaddfunc}
To perform visualisations you will need some additional tools.
Since these do not need to be linked with any of the packages above, you can install versions available for your
system, or build them from source.
\begin{itemize}
\item \filename{ppmtompeg} and \filename{jpegtopnm} from the \filename{netpbm} suite. - To build from source 
you would also need \filename{libjpeg} and its headers as well as \filename{libpng}\footnote{libpng requires zlib to build} and its headers.
\item A tool to visualise VTK files. For example Mayavi or Visit.
\end{itemize}


\section{Installing from source for \macosx using Macports}
\label{sec:srcmacports}

As we mentioned in the Section~\ref{sec:srcmac}, before you start installing from source you will need Mac OS X development tools installed on your Mac. 

If you do not have Macports already, please install Macports from \url{www.macports.org} . You can also install porticus (GUI for Macports).
 
Once you have Macports working Install boost using porticus or from terminal 
\begin{shellCode}
sudo port install boost@1.35.0_2+complete
\end{shellCode}
Sometimes this fails due to unknown reasons, but to overcome this problem you need to run
\begin{shellCode}
sudo port clean boost@1.35.0_2+complete

sudo port install boost@1.35.0_2+complete
 \end{shellCode}
  
Download scons source scons-0.98.5.tar.gz from \url{www.scons.org}.
\begin{shellCode}
tar xfz scons-0.98.5.tar.gz
cd scons-0.98.5
python setup.py install
\end{shellCode}

Note: Do not try to install scons using porticus or sudo port install scons, because if automatically installs another python version and you are likely run into problems with different python versions.  
 
Download numarray-1.5.2.tar.gz from \url{http://www.stsci.edu/resources/software_hardware/numarray/numarray.html}. 
\begin{shellCode}
tar xfz numarray-1.5.2.tar.gz
cd numarray-1.5.2
python setup.py install --gencode  2>&1 | tee tt.install.out
\end{shellCode}

You can ran a test to check numarray installation by
\begin{shellCode}
python
import numarray.testall as testall
testall.test()
\end{shellCode}
 
Install netcdf, gsl and fltk using Macports 
\begin{shellCode}
sudo port install netcdf
sudo port install gsl
sudo port intall fltk
\end{shellCode}
Note: If this fails, download and install from sources. 
 
Downlaod gmsh-2.2.3-source.tar and install from sources.
\begin{shellCode}
./configure --with-gsl-prefix=/opt/local/  --with-fltk    --prefix=/usr/local/
\end{shellCode}
Note: if you install using porticus or sudo port  it automaticall installs in \filename{/opt/local/}, but if you install from sources it installs in /usr/local. So, make sure these paths are right.
\begin{shellCode}
sudo make �j2
sudo make install
\end{shellCode} 
 
Download and install Mesa-7.0.3 (required for VTK) from sources
\begin{shellCode}
tar xjf MesaDemos-7.0.3.tar.bz2
tar xjf MesaGLUT-7.0.3.tar.bz2
tar xjf MesaLib-7.0.3.tar.bz2
cd Mesa-7.0.3
 
sudo make -j 2
make install
\end{shellCode} 
 
Install vtk-5.0.4 from source. If you install from ports it won't configure to use shared libraries. Once you untar it you will have (assume user is john) /Users/john/Downloads/VTK folder, then run following:
 
\begin{shellCode}
sudo mkdir /usr/local/VTKBuild/
cd /usr/local/VTKBuild/
sudo ccmake /Users/john/Downloads/VTK/ 
# It will create CMakeCache.txt. Make sure  you use the following configurations.
 
#       Advanced options			ON
#       BUILD_EXAMPLES				ON
#       BUILD_SHARED_LIBS			ON
#       VTK_WRAP_PYTHON				ON
#       CMAKE_VERBOSE_MAKEFILE			TRUE
#       VTK_OPENGL_HAS_OSMESA			ON
#       VTK_USE_MANGLED_MESA			OFF
#       VTK_USE_OFFSCREEN			OFF
 
sudo make -j2
sudo make install
\end{shellCode}
 
Note: you need to set following ENV variables into your \filename{/Users/john/.profile} for VTK to work:
\begin{shellCode}
export LD_LIBRARY_PATH = /usr/local/VTKBuild/bin: \
             /usr/local/VTKBuild/bin:${LD_LIBRARY_PATH}
export PYTHONPATH = /usr/local/VTKBuild/Wrapping/Python:\ 
            /usr/local/VTKBuild/bin:${PYTHONPATH}
     
#     For testing you can run:
python
import vtk
\end{shellCode}
 
Left only to install triangle and netpbm (required for ppmtompeg) using Macports 
\begin{shellCode}
sudo port install triangle
sudo port install netpbm
\end{shellCode}
 


