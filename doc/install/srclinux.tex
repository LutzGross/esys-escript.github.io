
%%%%%%%%%%%%%%%%%%%%%%%%%%%%%%%%%%%%%%%%%%%%%%%%%%%%%%%%
%
% Copyright (c) 2003-2008 by University of Queensland
% Earth Systems Science Computational Center (ESSCC)
% http://www.uq.edu.au/esscc
%
% Primary Business: Queensland, Australia
% Licensed under the Open Software License version 3.0
% http://www.opensource.org/licenses/osl-3.0.php
%
%%%%%%%%%%%%%%%%%%%%%%%%%%%%%%%%%%%%%%%%%%%%%%%%%%%%%%%%

\section{Installing from source for \linux}
\label{sec:srclinux}
The following instructions assume you are running the \filename{bash} shell.
Comments are indicated with \# characters.

Make sure you have the following installed:
\begin{itemize}
 \item \filename{g++} and associated tools.
\item \filename{make}
\item \filename{libXext.so}\footnote{In Debian this is in the libXext-dev package.}
\item \filename{libxt.so}\footnote{In Debian this is in the libxt-dev package.}
\end{itemize}

You will also need a copy of the \esfinley source code.
If you retrieved the source using subversion, don't forget that one can use the export command instead of checkout to get a smaller copy.
For additional visualisation functionality see Section~\ref{sec:linaddfunc}.

These instructions will produce the following structure:
\begin{itemize}
\item \filename{stand}: \begin{itemize}
 \item \filename{escript.d}
\item \filename{packages}
\item \filename{package_src}
\item \filename{build}
\item \filename{doc}
  \end{itemize}
\end{itemize}

The build directory can be removed when you are finished.

\begin{shellCode}
mkdir stand
cd stand
export PKG_ROOT=`pwd`/packages
\end{shellCode}

Copy compressed source bundles into \filename{stand/package_src}.
Copy documentation files into \filename{doc}.

\begin{shellCode}
mkdir packages
mkdir build
cd build
tar -jxf ../package_src/Python-2.6.1.tar.bz2
tar -jxf ../package_src/boost_1_37_0.tar.bz2
tar -jxf ../package_src/MesaLib-7.2.tar.bz2
tar -zxf ../package_src/netcdf-4.0.tar.gz
tar -zxf ../package_src/vtk-5.2.1.tar.gz
tar -zxf ../package_src/vtkdata-5.2.1.tar.gz
tar -zxf ../package_src/numarray-1.5.2.tar.gz
tar -zxf ../package_src/cmake-2.6.3.tar.gz
tar -zxf ../package_src/scons-1.2.0.tar.gz
\end{shellCode}

Build python.
\begin{shellCode}
cd Python*
./configure --prefix=$PKG_ROOT/python-2.6.1 --enable-shared 2>&1 \
  | tee tt.configure.out
make install 2>&1 | tee tt.make.out

cd ..

export PATH=$PKG_ROOT/python/bin:$PATH
export PYTHONHOME=$PKG_ROOT/python
export LD_LIBRARY_PATH=$PKG_ROOT/python/lib:$LD_LIBRARY_PATH

pushd ../packages
ln -s python-2.6.1/ python
popd

\end{shellCode}

Run python to make sure it works.
Now build numarray.

\begin{shellCode}
cd numarray-1.5.2

python setup.py install \
 --gencode --install-lib=$PKG_ROOT/numarray-1.5.2/lib \
 --install-headers=$PKG_ROOT=$PKG_ROOT/numarray-1.5.2/include/numarray \ 
   2>&1 | tee tt.install.out


export PYTHONPATH=$PKG_ROOT/numarray/lib:$PYTHONPATH
cd ..
pushd ../packages
ln -s numarray-1.5.2 numarray
popd
\end{shellCode}

Now we build scons.
\begin{shellCode}
cd scons-1.2.0
python setup.py install --prefix=$PKG_ROOT/scons-1.2.0

export PATH=$PKG_ROOT/scons/bin:$PATH
cd ..
pushd ../packages
ln -s scons-1.2.0 scons
popd
\end{shellCode}

...Boost libraries ...
\begin{shellCode}
cd boost_1_37_0

./configure --prefix=$PKG_ROOT/boost_1_37_0 --with-python-root=$PKG_ROOT/python \
  --with-python-version=2.6 --with-libraries=python

make
make install
ln -s $PKG_ROOT/boost_1_37_0 $PKG_ROOT/boost
export LD_LIBRARY_PATH=$PKG_ROOT/boost/lib:$LD_LIBRARY_PATH
cd ..
pushd ../packages
ln -s boost_1_37_0 boost
popd
\end{shellCode}

... and netcdf.
\begin{shellCode}
cd netcdf-4.0
CFLAGS="-O2 fPIC -Df2cFortran" CXXFLAGS="-O2 fPIC -Df2cFortran" \
FFLAGS="-O2 fPIC -Df2cFortran" FCFLAGS="-O2 fPIC -Df2cFortran" \
./configure --prefix=$PKG_ROOT/netcdf-4.0

make -j2
make install

export LD_LIBRARY_PATH=$PKG_ROOT/netcdf/lib:$LD_LIBRARY_PATH
cd ..
pushd ../packages
ln -s netcdf-4.0 netcdf
popd
\end{shellCode}

CMake and Mesa are required for VTK. 
\begin{shellCode}
cd cmake-2.6.3
./configure --prefix=$PKG_ROOT/cmake-2.6.3 2>&1 | tee tt.configure
make -j 4
make install

export PATH=$PKG_ROOT/cmake/bin:$PATH
cd ..
pushd ../packages
ln -s cmake-2.6.3 cmake
popd
\end{shellCode}

These instructions do not compile MesaDemos or GLUT.
If you need to check if Mesa compiled correctly, then the demos are a good test.
\begin{shellCode}
cd Mesa-7.2
./configure --prefix=$PKG_ROOT/mesa-7.2 --enable-gl-osmesa --with-driver=xlib

make -j 4
make install

export LD_LIBRARY_PATH=$PKG_ROOT/mesa:$LD_LIBRARY_PATH
cd ..
pushd ../packages
ln -s mesa-7.2 mesa
popd
\end{shellCode}

\begin{shellCode}
cd VTK
cmake .

#Edit the CMakeCache and make the following changes: 
#(Please replace .... with an absolute path to the stand directory)

#-----------------

BUILD_EXAMPLES should be OFF
BUILD_SHARED_LIBS should be ON

CMAKE_INSTALL_PREFIX	..../stand/packages/vtk-5.2.1
CMAKE_VERBOSE_MAKEFILE	TRUE

#check PYTHON_EXECUTABLE is correct.
#but it seems to be when I went through these steps

VTK_OPENGL_HAS_OSMESA	TRUE
VTK_USE_64BIT_IDS	ON
# That last one is marked as "May cause some bugs" in the original instructions

VTK_WRAP_PYTHON	ON
VTK_USE_MANGLED_MESA	OFF

#--------------------

cmake .
#It won't work but it will put some variables in that you need.

#Edit CMakeCache again and make the following changes

#----------------

VTK_USE_TK	OFF

OSMESA_INCLUDE_DIR	..../stand/packages/mesa/include

OSMESA_LIBRARY	..../stand/packages/mesa/lib/libOSMesa.so

PYTHON_INCLUDE_PATH	..../stand/packages/python/include/python2.6

PYTHON_LIBRARY	..../stand/packages/python/lib/libpython2.6.so

OPENGL_INCLUDE_DIR	..../stand/packages/mesa/include

OPENGL_gl_LIBRARY	..../stand/packages/mesa/lib/libGL.so

#----------------

cmake .

chmod +w Utilities/vtktiff/tif_fax3sm.c

make
make install


cd ../../packages
ln -s vtk-5.2.1 vtk
cd ..
\end{shellCode}

Now copy the \esfinley source into an \filename{escript.d} directory in \filename{stand}.

\subsection{Compiling escript}\label{sec:compileescriptlinux}

Change to the directory containing your escript source (\filename{escript.d}), then:

\begin{shellCode}
cd scons
cp linux_options_example.py YourMachineName_options.py

#edit the options file and make the following changes:
#-----------------------------------------------------------------
declare a PKG_ROOT variable at the top of the file eg:
PKG_ROOT='/home/jfenwick/stand/packages'

python_path		= PKG_ROOT+'python/include/python2.6'
python_lib_path		= PKG_ROOT+'python/lib'
python_libs		= 'python2.6'

boost_path		= PKG_ROOT+'boost/include/boost-1_37'
boost_lib_path		= PKG_ROOT+'boost/lib'
boost_libs		= ['boost_python-gcc43-mt']
# You could simlink the boost python library to give a shorter 
# name but it's not worth it

usevtk		= 'yes'
#-------------------------------------------------------------------

ln -s $PKG_ROOT/vtk-5.2.1 $PKG_ROOT/vtk

Modify bin/escript

STANDALONE=1

#Check to make sure the paths in the if [ $STANDALONE == 1 ]
#block are correct

#-----------------------------------------------------------------

#start a new terminal
cd stand
export PATH=`pwd`/packages/scons/bin:$PATH
cd escript.d
eval `bin/escript -e`
scons
\end{shellCode}

If you wish to test your build, then you can do the following. 
Note this may take a while if you have a slow processor and/or less than 1Gb of RAM.
\begin{shellCode}
scons all_tests
\end{shellCode}

\subsection{Cleaning up}
Once you are satisfied, the \filename{escript.d/build} and \filename{\$PKG_ROOT/build} directories can be removed.

If you \emph{really} want to save space and do not wish to be able to edit or recompile escript, you can remove the following:
\begin{itemize}
 \item From the \filename{escript.d} directory:\begin{itemize}
\item Everything except: \filename{bin}, \filename{include}, \filename{lib}, \filename{esys},
\filename{README_LICENSE}.
\item Hidden files, which can be removed using
\begin{shellCode}
find . -name .?* | xargs rm -rf
\end{shellCode}
in the \filename{escript.d} directory.
\end{itemize}
\item from the \filename{packages} directory:
\begin{itemize}
\item  \filename{scons}, \filename{scons-1.2.0}, \filename{cmake-2.6.3} and \filename{cmake}
\end{itemize}
\item \filename{package\_src}\footnote{Do not remove this if you intend to redistribute.}.
\end{itemize}

Please note that removing all these files may make it more difficult for us to diagnose problems.




\section{Additional Functionality}\label{sec:linaddfunc}
To perform visualisations you will need some additional tools.
Since these do not need to be linked with any of the packages above, you can install versions available for your
system, or build them from source.
\begin{itemize}
\item \filename{ppmtompeg} and \filename{jpegtopnm} from the \filename{netpbm} suite. - To build from source 
you would also need \filename{libjpeg} and its headers as well as \filename{libpng}\footnote{libpng requires zlib to build} and its headers.
\item A tool to visualise VTK files. For example Mayavi or Visit.
\end{itemize}