\clearpage
\vbox{}
\vfill
\begin{center}
\textbf{\Large Guide to Documentation}\pdfbookmark[0]{Documentation guide}{documentation guide}
\vspace{0.5cm}

Documentation for \module{esys.escript} comes in a number of parts.
Here is a rough guide to what goes where.

\vspace{1cm}
\hrule
\vspace{1cm}

\begin{tabular}{rp{11cm}}
 \textbf{install.pdf} & ``Installation guide for \emph{esys-Escript}'': 
Instructions for compiling \module{esys.escript} for your system from its source code. 
 Also briefly covers installing \texttt{.deb} packages for Debian and Ubuntu. \\
 & \\
 
 \textbf{cookbook.pdf} & ``The \textit{escript} COOKBOOK'':
 A introduction to \module{esys.escript} for new users from a geophysics perspective.\\
 & \\ 
 \textbf{user.pdf} & ``\emph{esys-Escript} User's Guide: Solving Partial Differential Equations with Escript and Finley'':
 Covers main \emph{escript} concepts.\\
 & \\ 
 \textbf{inversion.pdf} & ``\module{esys.downunder}: Inversion with \module{esys.escript}'':
 Explanation of the inversion toolbox for \module{esys.escript}.\\
 & \\ 
 \textbf{sphinx_api directory} & Documentation for \emph{escript} Python libraries.\\
 & \\ 
 \textbf{escript_examples(.tar.gz)/(.zip)} & Full example scripts referred to by other parts of the documentation.\\
 & \\ 
 \textbf{doxygen directory} & Documentation for C++ libraries (mostly only of interest for developers).\\
\end{tabular}
\end{center}
\vfill
\vbox{}
\pagebreak