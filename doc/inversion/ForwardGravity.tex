
%%%%%%%%%%%%%%%%%%%%%%%%%%%%%%%%%%%%%%%%%%%%%%%%%%%%%%%%%%%%%%%%%%%%%%%%%%%%%%
% Copyright (c) 2003-2016 by The University of Queensland
% http://www.uq.edu.au
%
% Primary Business: Queensland, Australia
% Licensed under the Apache License, version 2.0
% http://www.apache.org/licenses/LICENSE-2.0
%
% Development until 2012 by Earth Systems Science Computational Center (ESSCC)
% Development 2012-2013 by School of Earth Sciences
% Development from 2014 by Centre for Geoscience Computing (GeoComp)
%
%%%%%%%%%%%%%%%%%%%%%%%%%%%%%%%%%%%%%%%%%%%%%%%%%%%%%%%%%%%%%%%%%%%%%%%%%%%%%%

\section{Gravity Inversion}\label{sec:forward gravity}
For the gravity inversion we use the anomaly of the gravity acceleration~\index{gravity acceleration} of the Earth.
The controlling material parameter is the density~\index{density} $\rho$ of
the rock.
If the density field $\rho$ is known the gravitational potential $\psi$ is
given as the solution of the PDE
\begin{equation}\label{ref:GRAV:EQU:100}
-\psi_{,ii} = -4\pi G \cdot  \rho
\end{equation}
where $G=6.6730 \cdot 10^{-11}  \frac{m^3}{kg \cdot s^2}$ is the gravitational
constant.
The gravitational potential is set to zero at the top of the
domain $\Gamma_0$.
On all other faces the normal component of the gravity acceleration anomaly
$g_i$ is set to zero, i.e. $n_i \psi_{,i} = 0$ with outer normal field $n_i$.
The gravity force $g_i$ is given as the negative of the gradient of the gravity
potential $\psi$:
\begin{equation}\label{ref:GRAV:EQU:101}
 g_i = - \psi_{,i} 
\end{equation} 
From the gravitational potential we can calculate the gravity acceleration
anomaly via Equation~(\ref{ref:GRAV:EQU:101}) to obtain the defect to the
given data.
If $g^{(s)}_i$ is a measurement of the gravity acceleration anomaly for
survey $s$ and $\omega^{(s)}_i$ is a weighting factor the data defect
$J^{grav}(k)$ in the notation of Chapter~\ref{chapter:ref:inversion cost function} is given as
\begin{equation}\label{ref:GRAV:EQU:9}
J^{grav}(k) = \frac{1}{2}\sum_{s} \int_{\Omega} ( \omega^{(s)}_i \cdot (g_{i}- g^{(s)}_i) ) ^2 dx
\end{equation} 
Summation over $i$ is performed. 
The cost function kernel\index{cost function!kernel} is given as
\begin{equation}\label{ref:GRAV:EQU:10}
K^{grav}(\psi_{,i},k) = \frac{1}{2}\sum_{s} ( \omega^{(s)}_i \cdot (\psi_{,i}+ g^{(s)}_i) ) ^2
\end{equation} 
In practice the gravity acceleration $g^{(s)}$ is measured in vertical
direction $z$ with a standard error deviation $\sigma^{(s)}$ at certain
locations in the domain.
In this case one sets the weighting factors $\omega^{(s)}$ as
\begin{equation}\label{ref:GRAV:EQU:11}
\omega^{(s)}_i 
= \left\{
\begin{array}{lcl}
f \cdot  \frac{\delta_{iz}}{\sigma^{(s)}} & & \mbox{data are available} \\
& \mbox{ where } & \\
0 & & \mbox{ otherwise } \\
\end{array}
\right.
\end{equation} 
With the objective to control the 
gradient of the cost function 
the scaling factor $f$ is chosen in the way that
\begin{equation}\label{ref:GRAV:EQU:12}
\sum_{s} \int_{\Omega} ( \omega^{(s)}_i g^{(s)}_i ) \cdot ( \omega^{(s)}_j \frac{1}{L_j} ) \cdot 4\pi G L^2 \cdot \rho' \;  dx =\alpha
\end{equation} 
where $\alpha$ defines a scaling factor which is typically set to one and $L$ is defined by equation~(\ref{ref:EQU:REG:6b}). $\rho'$ is considering the 
derivative of the density with respect to the level set function. 


\subsection{Usage}

\begin{classdesc}{GravityModel}{domain, 
w, g,
\optional{, coordinates=\None}
\optional{, fixPotentialAtBottom=False},
\optional{, tol=1e-8}
}
opens a gravity forward model over the \Domain \member{domain} with 
weighting factors \member{w} ($=\omega^{(s)}$) and measured gravity acceleration anomalies \member{g} ($=g^{(s)}$).
The weighting factors and the  measured gravity acceleration anomalies must be vectors
where components refer to the components 
$(x_0,x_1,x_2)$ for the Cartesian coordinate system 
and to $(\phi, \lambda, h)$ for the geodetic coordinate system. 
If \member{reference} defines the reference coordinate system to be used, see Chapter~\ref{Chp:ref:coordinates}.
\member{tol} set the tolerance for the solution of the PDE~(\ref{ref:GRAV:EQU:100}).
If \member{fixPotentialAtBottom} is set to  \True, the gravitational potential 
at the bottom is set to zero in addition to the potential on the top. 
\member{coordinates} set the reference coordinate system to be used. By the default the 
Cartesian coordinate system is used.
\end{classdesc}

\begin{methoddesc}[GravityModel]{rescaleWeights}{
        \optional{scale=1.}
 \optional{rho_scale=1.}}
rescale the weighting factors such condition~(\ref{ref:GRAV:EQU:12}) holds where 
\member{scale} sets the scale $\alpha$
and \member{rho_scale} sets $\rho'$. This method should be called before any inversion is started
in order to make sure that all components of the cost function are appropriately scaled.
\end{methoddesc}


\subsection{Gradient Calculation}
This section briefly explains how the gradient
$\frac{\partial J^{grav}}{\partial \rho}$ of the cost function $J^{grav}$ with
respect to the density $\rho$ is calculated. We follow the concept as outlined in section~\ref{chapter:ref:inversion cost function:gradient}.
The gravity potential $\psi$ from PDE~(\ref{ref:GRAV:EQU:100}) is solved in
weak form:
\begin{equation}\label{ref:GRAV:EQU:201}
\int_{\Omega} q_{,i} \psi_{,i} \; dx  = - \int_{\Omega}  4\pi G \cdot q \rho\; dx 
\end{equation} 
for all $q$ with $q=0$ on $\Gamma_0$.
In the following we set $\Psi[\cdot]=\psi$ for a given density $\cdot$ as
solution of the variational problem~(\ref{ref:GRAV:EQU:201}).
If $\Gamma_{\rho}$ denotes the region of the domain where the density is known
and for a given direction $p$ with $p=0$ on $\Gamma_{\rho}$ one has
\begin{equation}\label{ref:GRAV:EQU:201aa}
\int_{\Omega}   \frac{\partial J^{grav}}{\partial \rho} \cdot p \; dx  =  \int_{\Omega}  
\sum_{s} (\omega^{(s)}_j \cdot 
(g^{(s)}_j-g_{j}) ) \cdot ( \omega^{(s)}_i \Psi[p]_{,i})  \; dx  
\end{equation} 
with 
\begin{equation}\label{ref:GRAV:EQU:202c}
Y_i[\psi]=  \sum_{s} (\omega^{(s)}_j \cdot 
(g^{(s)}_j-g_{j}) ) \cdot  \omega^{(s)}_i
\end{equation} 
This is written as 
\begin{equation}\label{ref:GRAV:EQU:202cc}
\int_{\Omega}   \frac{\partial J^{grav}}{\partial \rho} \cdot p \;  dx  = \int_{\Omega}  
Y_i[\psi] \Psi[p]_{,i} \; dx  
\end{equation} 
We then set $Y^*[\psi]$ as the solution of the equation 
\begin{equation}\label{ref:GRAV:EQU:202d}
\int_{\Omega} r_{,i} Y^*[\psi]_{,i} \; dx  =  \int_{\Omega} r_{,i} ,Y_i[\psi]  \; dx  \mbox{ for all } p \mbox{ with } r=0 \mbox{ on } \Gamma_{top}
\end{equation} 
with $Y^*[\psi]=0$ on $\Gamma_0$. With $r=\Psi[p]$ we get
\begin{equation}\label{ref:GRAV:EQU:202dd}
\int_{\Omega} \Psi[p]_{,i} Y^*[\psi]_{,i} \; dx  =  \int_{\Omega} \Psi[p]_{,i} ,Y_i[\psi]  \; dx
\end{equation} 
and from Equation~(\ref{ref:GRAV:EQU:201}) with $q=Y^*[\psi]$ we get
\begin{equation}\label{ref:GRAV:EQU:20e}
\int_{\Omega} Y^*[\psi]_{,i}  \Psi[p]_{,i} \; dx  = - \int_{\Omega}  4\pi G \cdot Y^*[\psi] \cdot  p\;  dx  
\end{equation}
which leads to 
\begin{equation}\label{ref:GRAV:EQU:20ee}
\int_{\Omega} \Psi[p]_{,i} ,Y_i[\psi]  \; dx  = - \int_{\Omega}  4\pi G \cdot Y^*[\psi] \cdot  p \; dx  
\end{equation}
and finally
\begin{equation}\label{ref:GRAV:EQU:201a}
\int_{\Omega}   \frac{\partial J^{grav}}{\partial \rho} \cdot p \;  dx  = - \int_{\Omega}  
4\pi G \cdot Y^*[\psi] \cdot  p \; dx  
\end{equation} 
or 
\begin{equation}\label{ref:GRAV:EQU:201b}
\frac{\partial J^{grav}}{\partial \rho}  =- 4\pi G \cdot Y^*[\psi]
\end{equation} 

\subsection{Geodetic Coordinates }
For geodetic coordinates $(\phi, \lambda, h)$, see Chapter~\ref{Chp:ref:coordinates}, the solution process needs to be slightly modified.
Observations are recorded along the geodetic coordinates axes $\alpha$ rather than the Cartesian axes $i$. In fact we
have in equation~\ref{ref:GRAV:EQU:9}:
\begin{equation}\label{ref:GRAV:EQU:300}
\omega^{(s)}_i \cdot (g_{i}- g^{(s)}_i) = \omega^{(s)}_{\alpha} \cdot (g_{{\alpha}}- g^{(s)}_{\alpha}) 
\end{equation} 
where now $g^{(s)}_{\alpha}$ are the observational data with weighting factors $\omega^{(s)}_{\alpha}$.  Using the 
fact that $g_{{\alpha}} = - d_{\alpha \alpha} \psi_{,\alpha}$ 
equation~\ref{ref:GRAV:EQU:10} translates to 
\begin{equation}\label{ref:GRAV:EQU:301}
J^{grav}(k) = \frac{1}{2}\sum_{s} \int_{\widehat{\Omega}} 
( \omega^{(s)}_{\alpha} \cdot (d_{\alpha \alpha}  \psi_{,\alpha} + g^{(s)}_{\alpha} ) ) ^2 \; v \; d\widehat{x}
\end{equation} 
where $\widehat{\Omega}$ and $d\widehat{x}$ refer to integration over the geodetic coordinates axes. This can be rearranged to 
\begin{equation}\label{ref:GRAV:EQU:301bb}
J^{grav}(k) = \frac{1}{2}\sum_{s} \int_{\widehat{\Omega}} 
(  \omega^{(s)}_{\alpha} v^{\frac{1}{2}} d_{\alpha \alpha} \cdot ( \psi_{,\alpha} + \frac{1}{d_{\alpha \alpha}} g^{(s)}_{\alpha} ) ) ^2 \; d\widehat{x}
=\frac{1}{2}\sum_{s} \int_{\widehat{\Omega}} 
(  {\widehat{\omega}}^{(s)}_{\alpha}\cdot ( \psi_{,\alpha} + \widehat{g}^{(s)}_{\alpha} ) ) ^2 \; d\widehat{x}
\end{equation} 
with 
\begin{equation}\label{ref:GRAV:EQU:301b}
 \widehat{\omega}^{(s)}_{\alpha} =\omega^{(s)}_{\alpha} v^{\frac{1}{2}} d_{\alpha \alpha} \mbox{ and }
\widehat{ g}^{(s)}_{\alpha}=
\frac{1}{d_{\alpha \alpha}} g^{(s)}_{\alpha}
\end{equation} 
which means one can apply the Cartesian formulation to the geodetic coordinates using modified data. 
The gravity potential is calculated from 
\begin{equation}\label{ref:GRAV:EQU:302}
\int_{\widehat{\Omega}} v \; d_{\alpha \alpha}^2 q_{,\alpha} \psi_{,\alpha} \;  d\widehat{x}  
= - \int_{\widehat{\Omega}}  (4\pi G v) \cdot q \rho\;  d\widehat{x} 
\end{equation} 
see equation~\ref{ref:GRAV:EQU:201}, and the adjoint function $Y^*[\psi]$ for $Y_{\alpha}[\psi]$ is given from
\begin{equation}\label{ref:GRAV:EQU:303}
\int_{\widehat{\Omega}} v  \; d_{\alpha \alpha}^2 q_{,\alpha} Y^*[\psi]_{,\alpha } \;d\widehat{x}  =
  \int_{\widehat{\Omega}} q_{,\alpha} Y_{\alpha}[\psi]  \; d\widehat{x} 
\end{equation} 
and finally 
\begin{equation}\label{ref:GRAV:EQU:310}
 \frac{\partial J^{grav}}{\partial \rho}  = - 
(4\pi G v) \cdot Y^*[\psi]
\end{equation} 
