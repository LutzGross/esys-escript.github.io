\chapter{Magnetic Inversion}\label{Chp:cook:magnetic inversion}


\begin{figure}
\centering
\includegraphics[width=0.7\textwidth]{QLDWestMagneticDataPlot.png}
\caption{Magentic Anomaly Data in $nT$ from Western Queensland, Australia
    (file \examplefile{data/QLDWestMagnetic.nc}). Data obtained from Geoscience Australia.}
\label{FIG:P1:MAG:0}
\end{figure}

Magnitic data report the observed magnetic flux density over a region above the surface of the Earth. 
Similar to the gravity case the data are given as deviation from an expected, background magnetic flux density $B^b$
of the Earth. Example data in units of $nT$ (nano Tesla) are shown in Figure~\ref{FIG:P1:MAG:0} .
It is the task of the inversion to recover the susceptibility distribution $k$ from the magnetic
data collected. The approach for inverting magnetic is almost identical to the one used for gravity data. 
In fact the \downunder script~\code{code: magnetic1} used for the magnetic inversion is very similar to the 
script~\ref{code: gravity1} for gravity inversion.   


\begin{pyc}\label{code: magnetic1}
\
\begin{python}
# Header:
from esys.downunder import *
from esys.weipa import *
from esys.escript import unitsSI as U


# Step 1: set up domain
dom=DomainBuilder()
dom.setVerticalExtents(depth=40.*U.km, air_layer=6.*U.km, num_cells=25)
dom.setFractionalPadding(pad_x=0.2, pad_y=0.2)
B_b= [ -41000*U.Nano * U.Tesla, 31000*U.Nano * U.Tesla, 0.]
dom.setBackgroundMagneticFluxDensity(B_b)
dom.fixSusceptibilityBelow(depth=40.*U.km)

# Step 2: read gravity data
source0=NetCdfData(NetCdfData.MAGNETIC, 'MagneticSmall.nc', scale_factor=U.Nano * U.Tesla)
dom.addSource(source0)

# Step 3: set up inversion
inv=MagneticInversion()
inv.setSolverTolerance(1e-4)
inv.setSolverMaxIterations(50)
inv.setup(dom)

# Step 4: run inversion 
inv.getCostFunction().setTradeOffFactorsModels(0.1) 
k = inv.run()

# Step 5: write reconstructed susceptibility to file
saveVTK("result.vtu", susceptibility=k)
\end{python}
\end{pyc}

\begin{figure}
\centering
\includegraphics[width=0.7\textwidth]{QLDMagContourMu01.png}
\caption{Contour plot of susceptibility from three-dimensional magnetic inversion (with $\mu=0.1$).
Colours represent values of susceptibility where high values are represented by
    blue and low values are represented by red.}
\label{FIG:P1:MAG:1}
\end{figure}


The structure of the script is identical to the gravity case. We have the header section importing the necessary modules. In the first step
the domain of the inversion is defined. In step two the data are read and added to the domain builder. In step three the inversion is 
set up and - in step four - is run. Finally in step five the result is written to the result file, here \file{result.vtu} in the \VTK format.
Results are shown in Figure~\ref{FIG:P1:MAG:1}.

Although scripts for magnetic and gravity inversion are largely identical there are a few small differences which we are
going to highlight now. The magnetic inversion requires data about the background  magnetic flux density which is added to the domain
by the statements 
\begin{verbatim}
B_b= [ -41000*U.Nano*U.Tesla, 31000*U.Nano * U.Tesla, 0.]
dom.setBackgroundMagneticFluxDensity(B_b)
\end{verbatim}
Here it is assumed that the background magnetic flux density is constant across the domain and is given as the list
\begin{verbatim}
B_b= [ B_N,  B_E, B_V ]
\end{verbatim}
in units of Tesla (T) where 
\member{B_N}, \member{B_E} and \member{B_V} refer to 
the north, east and vertical component of the magnetic flux density.
Values for the magnetic flux density can be obtained by the International
Geomagnetic Reference Field (IGRF)~\cite{IGRF} 
(or the Australian Geomagnetic Reference Field (AGRF)~\cite{AGRF} via \url{http://www.ga.gov.au/oracle/geomag/agrfform.jsp}).
The 



\begin{verbatim}
dom.fixSusceptibilityBelow(depth=40.*U.km)
\end{verbatim}

\begin{verbatim}
source0=NetCdfData(NetCdfData.MAGNETIC, 'MagneticSmall.nc', \
     scale_factor=U.Nano * U.Tesla)
\end{verbatim}

\begin{verbatim}
inv=MagneticInversion()
\end{verbatim}






 B_r, B_theta, 0.



Although Magnetic and gravity methods are almost the same, 
Magnetic has its own complexity, elaboration and instability and it is very localized.
Outer core of the Earth has a convection current which produce a magnetic field through the earth. 
Magnetic fields are not central and their directions vary with azimuth. Its north pole is in the south of 
the Earth and south pole is in north of the earth. Meantime magnetic poles and its axis are not 
exactly coinciding with geographical one. The lines of magnetic field come out from south magnetic pole and 
go into north magnetic field. Also poles are shifted continuously.

The basic magnetic field or magnetic flux density in any medium is $B$. M
eanwhile $H$ is a parameter proportional to $B$ in non magnetizable material.
 In magnetizable material $H$ is describe how $B$ is changed with polarization or magnetization.

All material magnetic behavior, refer on magnetic moments of atoms or its ions, have a character. 
The ability of material to be magnetized in an external magnetic field, introduces as magnetic susceptibility. 
Based on their magnetic susceptibility, material is compartmented in three main classes: diamagnetism, paramagnetism and ferromagnetism.

Increasing magnetic field anomalies over subsurface geological structure illustrate 
contrast between magnetization in neighboring rock properties.

Some particular ions in atmosphere release electrical currents so this external magnetic field acquire in the surface of the magnetic observation. Also in day time sun heating cause more motion in these particle. This time related changes of magnetic field are the diurnal variation which depends on the latitude of observation point. 

Magnetic field intensity differs in latitude, longitude and altitude. The vertical gradient of magnetic field gives the elevation correction. It is varied from magnetic equator to magnetic poles which is generally small. Latitude correction is zero in magnetic poles and equator and reaches a maximum at intermediate latitude.

The shape of the magnetic anomaly is distinguished with the form and the depth of the structure and depends on magnetization contrast and the objects orientation in the earth.

International Geomagnetic Reference Field (IGRF) is a mathematical description of Global magnetic field and it  is provided each 5 year.

In comparison to the correction of gravity observation, magnetic survey needs very few corrections. After compensation of diurnal effect, latitude and elevation corrections are applied. Then global magnetic field should be subtracted from data. Finally magnetic anomaly is used in geophysical processing.



\section{Input File} 
This section of inversion package needs two input files which contain magnetic anomalies and some constraint factors. The firs file includes magnetic anomalies, in which all corrections were applied previously, the location (latitude and longitude) and elevation of the observed place precisely.

The next file which consists some factors to figure the inversion escripts. Again in magnetic inversion, a padding area present around the real data to smooth the effects of cutting data in boundaries.
 
A small part of sample of run_mag2D:

\begin{verbatim}
mu=10
THICKNESS=20.*U.km
DATASET='NSW_east.nc'
PAD_X = 0.2
PAD_Y = 0.2
l_air = 6. * U.km
n_cells_v = 25
\end{verbatim}

Almost all of constraints factors are the same as gravity instead of Mu factor.\\

\begin{description} 	

\item[mu]
It is defined in accordance with the noise of data and it has a wide range to select from 0.0001 to 100. Also its does not have same value for 2D and 3D inversion.

\end{description}

\section{Output File}
After inversion completion, an output file with silo extension is created, which is consisted inversion result. This file shows the input data as magnetic anomaly and inverted susceptibility separately. The objective is indeed to  predict a susceptibility model with having a best fit with input data. The inversion carry on to attain an acceptable volume for error in its mathematical function. 





\begin{figure}
    \begin{center}
        \subfigure[$\mu=0.001$]{%
            \label{FIG:P1:MAG:10 MU0001}
            \includegraphics[width=0.45\textwidth]{QLDMagContourMu0001.png}
        }%
        \subfigure[$\mu=0.01$]{%
            \label{FIG:P1:MAG:10 MU001}
            \includegraphics[width=0.45\textwidth]{QLDMagContourMu001.png}
        }\\ %  ------- End of the first row ----------------------%
        \subfigure[$\mu=0.1$]{%
            \label{FIG:P1:MAG:10 MU01}
            \includegraphics[width=0.45\textwidth]{QLDMagContourMu01.png}
        }%
        \subfigure[$\mu=1.$]{%
            \label{FIG:P1:MAG:10 MU1}
            \includegraphics[width=0.45\textwidth]{QLDMagContourMu1.png}
        }\\ %  ------- End of the second row ----------------------%
        \subfigure[$\mu=10.$]{%
            \label{FIG:P1:MAG:10 MU10}
            \includegraphics[width=0.45\textwidth]{QLDMagContourMu10.png}
        }%
    \end{center}
    \caption{3-D contour plots of magnetic inversion results with data from
    Figure~\ref{FIG:P1:MAG:0} for various values of the model trade-off
    factor $\mu$. Visualization has been performed in \VisIt.
    \AZADEH{check images.}}
    \label{FIG:P1:MAG:10}
\end{figure}

\begin{figure}
    \begin{center}
        \subfigure[$\mu=0.001$]{%
            \label{FIG:P1:MAG:11 MU0001}
            \includegraphics[width=0.45\textwidth]{QLDMagDepthMu0001.png}
        }%
        \subfigure[$\mu=0.01$]{%
            \label{FIG:P1:MAG:11 MU001}
            \includegraphics[width=0.45\textwidth]{QLDMagDepthMu001.png}
        }\\ %  ------- End of the first row ----------------------%
        \subfigure[$\mu=0.1$]{%
            \label{FIG:P1:MAG:11 MU01}
            \includegraphics[width=0.45\textwidth]{QLDMagDepthMu01.png}
        }%
        \subfigure[$\mu=1.$]{%
            \label{FIG:P1:MAG:11 MU1}
            \includegraphics[width=0.45\textwidth]{QLDMagDepthMu1.png}
        }\\ %  ------- End of the second row ----------------------%
        \subfigure[$\mu=10.$]{%
            \label{FIG:P1:MAG:11 MU10}
            \includegraphics[width=0.45\textwidth]{QLDMagDepthMu10.png}
        }%
    \end{center}
    \caption{3-D slice plots of magnetic inversion results with data from
    Figure~\ref{FIG:P1:MAG:0} for various values of the model trade-off
    factor $\mu$. Visualization has been performed \VisIt.
    \AZADEH{check images.}}
    \label{FIG:P1:MAG:11}
\end{figure}


% \section{Reference}
% 
% As some examples there are several inversions which have ran in some synthetic magnetic data sets. Here comparisons between synthetic susceptibility and inverted one are shown.
% 
% Some of the presumptions are the same for all of the examples to simplify the situation to make a logical comparison between synthetic input and output. which is as followed:
% 
% \begin{verbatim}
% depth_offset=0.*U.km
% l_data = 100 * U.km
% l_pad=40*U.km
% THICKNESS=20.*U.km
% l_air=6*U.km
% \end{verbatim}
% 
% The others assumptions comes with each example.
% 
% \begin{enumerate}
% \item A 2D magnetic susceptibility area is created with one maximum and one minimum in two sides. After inversion the inside of main boundary of our dataset have a desirable simulation.(\ref{fig:mag2D2}) 
% \begin{verbatim}
% n_cells_in_data=100
% n_humbs_h= 2
% n_humbs_v=1
% mu=1.
% \end{verbatim}
% 
% \begin{figure}
% \centering
% \includegraphics[width=\textwidth]{mag2D2.png}
% \caption{2D magnetic inversion up) the reference model  down)the inverted model}
% \label{fig:mag2D2}
% \end{figure}
% 
% \item A 2D magnetic area with two maximum and two minimum intermittent is suggested. In this initial model two of the humps are located in the padding area which is not important after inversion, is omitted then. so in the result just two humps in middle of the boundary is observable.(\ref{fig:mag2D4})
% 
% \begin{verbatim}
% n_cells_in_data=100
% n_humbs_h= 4
% n_humbs_v=1
% mu=1.
% \end{verbatim}
% 
% \begin{figure}
% \centering
% \includegraphics[width=\textwidth]{mag2D4.png}
% \caption{2D magnetic model up) the reference model  down)the inverted model}
% \label{fig:mag2D4}
% \end{figure}
% 
% \item A 3D magnetic model with one humbs in the middle of the area is proposed that surrounded all main and padding. After inversion just the main area is objective which have a good result for inversion.(\ref{fig:mag3D1-ref} and \ref{fig:mag3D1})
% 
% \begin{verbatim}
% n_humbs_h=4
% n_humbs_v=1
% mu=0.0001
% n_cells_in_data=50
% \end{verbatim}
% 
% \begin{figure}
% \centering
% \includegraphics[width=\textwidth]{mag3D1-ref.png}
% \caption{3D magnetic reference model with one maximum susceptibility}
% \label{fig:mag3D1-ref}
% \end{figure}
% 
% \begin{figure}
% \centering
% \includegraphics[width=\textwidth]{mag3D1.png}
% \caption{3D magnetic inversion result}
% \label{fig:mag3D1}
% \end{figure}
% \end{enumerate}

