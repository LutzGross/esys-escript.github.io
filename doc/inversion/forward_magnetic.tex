
%%%%%%%%%%%%%%%%%%%%%%%%%%%%%%%%%%%%%%%%%%%%%%%%%%%%%%%%%%%%%%%%%%%%%%%%%%%%%%
% Copyright (c) 2003-2012 by University of Queensland
% http://www.uq.edu.au
%
% Primary Business: Queensland, Australia
% Licensed under the Open Software License version 3.0
% http://www.opensource.org/licenses/osl-3.0.php
%
% Development until 2012 by Earth Systems Science Computational Center (ESSCC)
% Development since 2012 by School of Earth Sciences
%
%%%%%%%%%%%%%%%%%%%%%%%%%%%%%%%%%%%%%%%%%%%%%%%%%%%%%%%%%%%%%%%%%%%%%%%%%%%%%%


\section{Magnetic Inversion}\label{chp:magneticinversion}

We want to calculate the susceptibility $k$ over a rectangular domain as described in Chapter~\ref{chp:gravityinversion} from 
the measured magnetic field $B_i$. Under the assumption of linear, isotropic material the magnetic field is 
given as 
\begin{equation}\label{EQU:Hb}
B_i = H_i + k \cdot B^b_i 
\end{equation}
where  $H_i$ is the unknown is the magnetizing field and
$B^b_i$ is the background  magnetic field of the Earth.
\footnote{Notation: the superscript ``b'' following a symbol indicates 
a background field quantity.}. We assume that the background  magnetizing field is constant across the domain 
where for a domain at latitude $theta$ the latitude and radial components of magnetizing field are given as  
\begin{equation}\label{EQU:Hb 2}
\begin{array}{rcl}
B^b_{\theta}  & = & \displaystyle{ \frac{\mu_0 \cdot m}{4 \pi \cdot R_{earth}^3} sin(\theta) }  \\
B^b_r & = & \displaystyle{ \frac{\mu_0 \cdot m}{1 \pi \cdot R_{earth}^3} cos(\theta) }
\end{array}
\end{equation}
with the vacuum permeability $\mu_0 = 4 \pi \cdot 10^{-7} \frac{Vs}{Am}$, 
the magnetic dipole moment of Earth $m= 8.22 \cdot 10^22 Am^2$ and earth radius $R_{earth}= 6378137m$. We assume a flat domain
and associate where longitude is associated with $x_0$-direction,
the latitude is associated with $x_1$-direction and the radial direction is associated with $x_2$-direction:
\begin{equation}\label{EQU:Hb 3}
\begin{array}{rcl}
B^b_0  & = & 0 \\
B^b_1  & = & -B^b_{\theta} \\
B^b_2  & = & -B^b_r \\
\end{array}
\end{equation}
We introduce the potential $\psi$ for the deviation of the magnetizing field from the background magnetizing field: 
\begin{equation}\label{EQU:Hb 4}
H_i = - \psi_{,i} + B^b
\end{equation}
With the is notation we get from equation~(\ref{EQU:Hb}):
\begin{equation}\label{EQU:Hb 5}
B_i = (1+k) B^b_i  - \psi_{,i} 
\end{equation} 
As the magnetic field $B$ is divergence free we get the PDE 
\begin{equation}\label{EQU:Hb 5}
- \psi_{,ii} = - ((1+k) B^b_i)_{,i} 
\end{equation} 
for the magnetic potential $\psi$. 

The problem is to calculate the susceptibility $k$ from the magnetic field $B$ known at some parts of the region of interest 
$\Omega$. In fact we want to minimize the value
\begin{equation}\label{GRAV:EQU:102}
J_{data}(\psi) = \frac{1}{2}\sum_{s} \int_{\Omega} \chi^{(s)}_i \cdot (  B_{i}- B^{(s)}_i)^2 dx
\footnote{Summation over $i$ is performed by Einstein notation.}
\end{equation} 
where $B^{(s)}_i$ is a measurement of the gravity force $B_i = (1+k) H^b_i  - \psi_{,i} $
and $\chi^{(s)}_i$ is a weighting factor. 
$s$ indexes the surveys included in the inversion.



\section{Solution methods}
To apply the Nonlinear Conjugate Gradient method (NLCG), see Appendix~\ref{sec:NLCG} or the L-BFGS method, see Appendix~\ref{sec:LBFGS} we need
to define an inner product $<.,.>$ and need to calculate the gradient of of $f$. 

As inner product we use 
\begin{equation}\label{MAG:EQU:200}
<p,q> = \int_{\Omega} p \cdot q \; dx
\end{equation} 
With this notation the magnetic potential is given as $\psi=\Psi[1+k]$ where
\begin{equation}\label{MAG:EQU:201}
< q_{,i},\Psi[p]_{,i} > = < q_{,i} , p \cdot H^b_i> \mbox{ for all } q \mbox{ with } q=0 \mbox{ on } \Gamma_{z}
\end{equation} 
where $\Gamma_{z}$ denotes the top and bottom surface of the domain for case (D)
and the top surface of the domain for case (N). 



For any $q$ with $q=0$ on $\Omega^{ref}$ we get 
\begin{equation}\label{MAG:EQU:202}
< \nabla f(m),q> = < \nabla J_{data}(\Psi[1+C[m]]), q>  +  \mu \cdot < \nabla J_{reg}(m),q>
\end{equation} 
with 
\begin{equation}\label{MAG:EQU:202a}
< \nabla J_{reg}(m),q> = 
\int_{\Omega} 2 \omega q \cdot (m-m^{ref}) + 2 \omega_i \cdot L_i^2 \cdot q_{,i} (m-m^{ref})_{,i} dx
\end{equation} 
and 
\begin{equation}\label{MAG:EQU:202b}
< \nabla J_{data}(\Psi[C[m]]), q> =  \sum_{s} \int_{\Omega} \chi^{(s)}_i \cdot (  B_i  -  B^{(s)}_i ) 
\cdot ( \frac{dC}{dm} \cdot q \cdot B^b_i  - \Psi[\frac{dC}{dm} \cdot q]_{,i}) dx 
\end{equation} 
where $k=C(m)$ and $B_i=(1+k) H^b_i  - \Psi[1+k]_{,i}$
With
\begin{equation}\label{MAG:EQU:202c}
Y_i[\psi]= \sum_{s} \chi^{(s)}_i \cdot (  B_i  -  B^{(s)}_i ) 
\end{equation} 
we get 
\begin{equation}\label{MAG:EQU:202bb}
< \nabla J_{data}(\Psi[C[m]]), q> = 
< Y_i[\psi] B^b_i, \frac{dC}{dm} q> - < Y_i[\psi] ,  \Psi[\frac{dC}{dm} \cdot q]_{,i} >
\end{equation} 
and $Y^*[\psi]$ as the solution of the equation 
\begin{equation}\label{MAG:EQU:202d}
< p_{,i},Y^*[\psi]_{,i} > =  < p_{,i} ,Y_i[\psi] > \mbox{ for all } p \mbox{ with } p=0 \mbox{ on } \Gamma_{z}
\end{equation} 
with $Y^*[\psi]=0$ on $\Gamma_{z}$. 
By stetting $p=\Psi[\frac{dC}{dm} \cdot q]$ we get
\begin{equation}\label{MAG:EQU:202e}
<(\Psi[\frac{dC}{dm} \cdot q]) _{,i},Y^*[\psi]_{,i} > =  < (\Psi[\frac{dC}{dm} \cdot q])_{,i} ,Y_i[\psi] >
\end{equation} 
and from~\ref{MAG:EQU:201} with $q=Y^*[\psi]$ we get
\begin{equation}\label{MAG:EQU:20e}
< (Y^*[\psi])_{,i},(\Psi[\frac{dC}{dm} \cdot q])_{,i} > = <(Y^*[\psi])_{,i} , \frac{dC}{dm} \cdot q \cdot H^b_i>
\end{equation}
which leads to 
\begin{equation}\label{MAG:EQU:202f}
< \nabla J_{data}(\Psi[C[m]]), q> = < ( Y_i[\psi] - (Y^*[\psi])_{,i} ) H^b_i,  \frac{dC}{dm} \cdot q>
\end{equation} 
Putting this all together we get
\begin{align}
< \nabla f(m),q> = &\mu \cdot < 2 \omega \cdot (m-m^{ref}), q>\nonumber\\
+ &\mu \cdot < 2 \omega_i \cdot L_i^2 \cdot (m-m^{ref})_{,i}, q_{,i}>\nonumber\\
+ &< ( Y_i[\psi] - (Y^*[\psi])_{,i} ) B^b_i,  \frac{dC}{dm} \cdot q>\label{MAG:EQU:202g}
\end{align}  


