
%%%%%%%%%%%%%%%%%%%%%%%%%%%%%%%%%%%%%%%%%%%%%%%%%%%%%%%%%%%%%%%%%%%%%%%%%%%%%%
% Copyright (c) 2003-2012 by University of Queensland
% http://www.uq.edu.au
%
% Primary Business: Queensland, Australia
% Licensed under the Open Software License version 3.0
% http://www.opensource.org/licenses/osl-3.0.php
%
% Development until 2012 by Earth Systems Science Computational Center (ESSCC)
% Development since 2012 by School of Earth Sciences
%
%%%%%%%%%%%%%%%%%%%%%%%%%%%%%%%%%%%%%%%%%%%%%%%%%%%%%%%%%%%%%%%%%%%%%%%%%%%%%%

\chapter{Gravity Inversion}
We want to recover the density field $\rho$ from an acceleration field $g_i$ known on certain locations within the domain $\Omega$. It is
assumed that
\begin{equation}
\Omega = [x^{min}_0, x^{max}_0] \times
 [x^{min}_1, x^{max}_1] \times
 [x^{min}_2, x^{max}_2] 
\end{equation} 
If the density field $\rho$ is known the gravitational potential $\psi$ is given
as the solution of the PDE 
\begin{equation}\label{GRAV:EQU:100}
\psi_{,ii} = 4\pi \cdot G \cdot  \rho
\end{equation}
where $G=6.67300 \cdot 10^{-11}  \frac{m^3}{kg \cdot s^2}$ is the gravitational constant.  
The gravity force $g_i$ is given
as the negative of the gradient of the gravity potential $\psi$:
\begin{equation}\label{GRAV:EQU:101}
 g_i = - \psi_{,i} 
\end{equation} 
The boundary conditions for PDE~\ref{GRAV:EQU:100} are set as follows:
On vertical faces of the domain we assume that the horizontal gravity forces are equal to zero:
\begin{equation}\label{GRAV:EQU:101a}
\begin{array}{ll}
g_0=0 & \mbox{ for } x_0=x^{min}_0 \mbox{ or } x_0=x^{max}_0 \\
g_1=0 & \mbox{ for } x_1=x^{min}_1 \mbox{ or } x_1=x^{max}_1 \\
\end{array}
\end{equation}
which translates to
\begin{equation}\label{GRAV:EQU:101aa}
n_i \cdot  \psi_{,i} = 0
\end{equation} 
on faces $x_0=x^{min}_0$, 
$x_0=x^{max}_0$,
$x_1=x^{min}_1$ or 
$x_1=x^{max}_1$. On the bottom surface we set 
\begin{equation}\label{GRAV:EQU:101b}
\psi = 0 \mbox{ for } x_2=x^{min}_2
\end{equation} 
which sets all horizontal gravity forces to zero $g_0=g_1=0$. 
For the boundary conditions on the top surface we consider two options
\begin{itemize}
 \item[(D)] 
On the top surface we set 
\begin{equation}\label{GRAV:EQU:101 BC D}
\psi = 0 \mbox{ for } x_2=x^{max}_2
\end{equation} 
which sets all horizontal gravity forces to zero $g_0=g_1=0$.
 \item[(N)] 
On the top surface we set the vertical gravity forces to zero:
\begin{equation}\label{GRAV:EQU:101 BC N}
n_i \cdot  \psi_{,i} = 0 \mbox{ for } x_2=x^{max}_2
\end{equation} 
\end{itemize}
The PDE~\ref{GRAV:EQU:100} together with the boundary conditions~\ref{GRAV:EQU:101aa}, ~\ref{GRAV:EQU:101b}
and~\ref{GRAV:EQU:101 BC D} or~\ref{GRAV:EQU:101 BC N}
has a unique solution. For a given density $\rho$ we denote the unique solution $\psi$ by $\Psi[\rho]$.


The problem is to calculate the density distribution $\rho$ from the gravity force known at some parts of the region of interest 
$\Omega$. In fact we want to minimize the value
\begin{equation}\label{GRAV:EQU:102}
J_{data}(\psi) = \frac{1}{2}\sum_{s} \int_{\Omega} \chi^{(s)}_i \cdot (  g_{i}- g^{(s)}_i)^2 dx
\footnote{Summation over $i$ is performed by Einstein notation.}
\end{equation} 
where $g^{(s)}_i$ is a measurement of the gravity force $g_i=-\psi_{,i}$ and $\chi^{(s)}_i$ is a weighting factor. 
$s$ indexes the surveys included in the inversion. 
We assume that the gravity force $g^{(s)}_i$ is measured as deviation from a background gravity force $g^{bg}_i$,
i.e. the actually present gravity force is given as  $g^{(s)}_i + g^{bg}$~\footnote{Notice that we use the same reference gravity force for all surveys.}.
This can be added to the 
gravitational potential $\psi$ 
\begin{equation}\label{GRAV:EQU:102x}
\psi^{total} = \psi + g^{bg}_i \cdot ( x_i - x^{min}_i ) 
\end{equation}
The total gravitational potential $\psi^{total}$ still fulfills the PDE~\ref{GRAV:EQU:100} 
with inhomogeneous versions of the boundary conditions. For the most common case of a purely vertical
background gravity force (i.e. $g^{bg}_0=g^{bg}_1=0$) the boundary conditions at the top of the domain
take the form $\psi^{total} =g^{bg}_2 \cdot ( x_2^{max} - x^{min}_2 )$ for case (D) or
$n_i \cdot  \psi^{total}_{,i} =g^{bg}_2$ for case (N). Notice that the first case does not enforce
the background gravity force on the top surface but enforces the vertical gravity forces to be zero. 
The second condition for case (D) enforces the vertical gravity force to the background gravity force but allows 
for a non-zero vertical gravity force. In most cases the case (N) is the appropriate boundary condition
assuming the top boundary of the domain is sufficiently far away from any density variation.   
 

In field observations  $g^{(s)}_i$
are known on a sub-domain of the domain only so one will set
\begin{equation}\label{GRAV:EQU:105}
\chi^{(s)}_i 
= \left\{
\begin{array}{lcl}
\frac{1}{(\sigma^{(s)}_i)^2} & & g^{(s)}_i \mbox{ is available} \\
& \mbox{ where } & \\
0 & & \mbox{ otherwise } \\
\end{array}
\right.
\end{equation} 
where $\sigma^{(s)}_i$ is the standard error deviation of measurement $g^{(s)}_i$, see~\cite{A}.
With this setting the data $g^{(s)}_i$ can be extended to any arbitrary value (typically to zero) on
all locations where no measurements of $g^{(s)}_i$ are available.  

We need to make sure that the density $\rho$ takes physically meaningful values. 
To achieve this we use a parametrization $C(m)$ for the $\rho$ where $C$ is a given function. We think as 
$m$ being a dimensionless value between $-\infty$ to $\infty$. Possible settings for the function $C$ are:
\begin{itemize}
 \item[(I)] We can set $\rho = C(m)= \rho^{scale} \cdot m$ where $\rho^{scale}$ is a reference density.
 \item[(B)] To restrict the density $\rho$ between 
a minimum density $\rho^{min}$ and a maximum density $\rho^{max}$ one sets 
\begin{equation} 
\rho = C(m) = \frac{\rho^{max} +\rho^{min}}{2} + \frac{\rho^{max} -\rho^{min}}{2} \cdot \tanh(m)
\end{equation} 
 \item[(L)] To use a logarithmic scale which ensures a positive density one sets
\begin{equation} 
\rho = C(m) = \rho^{scale} \cdot e^{m}
\end{equation} 
\end{itemize}
In the following the exact form of the parametrization $C$ is not relevant, however we will make use of the 
fact that $C$ is sufficiently smooth and invertible, i.e. for any given valid density $\rho$ there is a unique
 value $m$ in the parameter space such that $\rho=C(m)$. For the inverse function we use the notation $C^{-1}$.
In the minimization problem we will search for the value of $m$ providing the best fit to the data 
rather than the density $\rho$. 


The regularization term is added into the minimization problem so the problem has a unique answer. This
takes the form 
\begin{equation}\label{GRAV:EQU:104}
J_{reg}(m) =  \int_{\Omega} \omega \cdot (m-m^{ref})^2 + \omega_i \cdot L_i^2 \cdot [ (m -m^{ref})_{,i}] ^2 dx
\end{equation} 
where $\omega$ and $\omega_i$ are weighting factors, $L_i$ is the smoothing length (e.g. $L_i= x^{max}_i-x^{min}_i$)
and $m^{ref}=C^{-1}(\rho^{ref})$ defines a reference parametrization for a given reference 
density $\rho^{ref}$. 
If the density (and therefore the parameter $m$) is known to be exactly $\rho^{ref}$ on a certain subregion $\Omega^{ref}$ of
the domain this information can be used
to directly constraint the problem, or alternatively one can choose the weights $\omega$. 

The task is now to find $m$ which minimizes the cost function
\begin{equation}\label{GRAV:EQU:103}
f(m) =  J_{data}(\Psi[C[m]]) +  \mu \cdot J_{reg}(m)
\end{equation} 
where $\mu>0$ is a fixed penalty factor subject to $m=m^{ref}$ on $\Omega^{ref}$.

\section{Solution methods}
To apply the Nonlinear Conjugate Gradient method (NLCG), see Appendix~\ref{sec:NLCG} or the L-BFGS method, see Appendix~\ref{sec:LBFGS} we need
to define an inner product $<.,.>$ and need to calculate the gradient of of $f$. 

As inner product we use 
\begin{equation}\label{GRAV:EQU:200}
<p,q> = \int_{\Omega} p \cdot q \; dx
\end{equation} 
With this notation the gravity potential $\phi$ is given as
\begin{equation}\label{GRAV:EQU:201}
< q_{,i},\psi_{,i} > = - 4\pi \cdot G \cdot  < q , \rho > \mbox{ for all } q \mbox{ with } q=0 \mbox{ on } \Gamma_{z}
\end{equation} 
where $\Gamma_{z}$ denotes the top and bottom surface of the domain for case (D)
and the top surface of the domain for case (N). 


For any $q$ with $q=0$ on $\Omega^{ref}$ we get 
\begin{equation}\label{GRAV:EQU:202}
< \nabla f(m),q> = < \nabla J_{data}(\Psi[C[m]]), q>  +  \mu \cdot < \nabla J_{reg}(m),q>
\end{equation} 
with 
\begin{equation}\label{GRAV:EQU:202a}
< \nabla J_{reg}(m),q> = 
\int_{\Omega} 2 \omega q \cdot (m-m^{ref}) + 2 \omega_i \cdot L_i^2 \cdot q_{,i} (m-m^{ref})_{,i} dx
\end{equation} 
and 
\begin{equation}\label{GRAV:EQU:202b}
< \nabla J_{data}(\Psi[C[m]]), q> = 
\sum_{s} \int_{\Omega} \chi^{(s)}_i \cdot (  \psi_{,i} +  g^{(s)}_i)  \cdot (\Psi[\frac{dC}{dm} \cdot q])_{,i} dx 
\end{equation} 
where $\rho=C(m)$ and $\psi=\Psi[\rho]$. We then set 
\begin{equation}\label{GRAV:EQU:202c}
Z_i[\psi]= \sum_{s} \chi^{(s)}_i \cdot (  \psi_{,i} +  g^{(s)}_i) 
\end{equation} 
and $Z^*[\psi]$ as the solution of the equation 
\begin{equation}\label{GRAV:EQU:202d}
< p_{,i},Z^*[\psi]_{,i} > =  < p_{,i} ,Z_i[\psi] > \mbox{ for all } p \mbox{ with } p=0 \mbox{ on } \Gamma_{z}
\end{equation} 
with $Z^*[\psi]=0$ on $\Gamma_{z}$. By stetting $p=\Psi_0[\frac{dC}{dm} \cdot q]$ we get
\begin{equation}\label{GRAV:EQU:202e}
<(\Psi_0[\frac{dC}{dm} \cdot q]) _{,i},Z^*[\psi]_{,i} > =  < (\Psi_0[\frac{dC}{dm} \cdot q])_{,i} ,Z_i[\psi] >
\end{equation} 
and from~\ref{GRAV:EQU:201} with $q=Z^*[\psi]$ we get
\begin{equation}\label{GRAV:EQU:20e}
< (Z^*[\psi])_{,i},(\Psi_0[\frac{dC}{dm} \cdot q])_{,i} > = - 4\pi \cdot G \cdot  < Z^*[\psi], \frac{dC}{dm} \cdot q >
\end{equation}
which leads to 
\begin{equation}\label{GRAV:EQU:202f}
< \nabla J_{data}(\Psi[C[m]]), q> = < (\Psi_0[\frac{dC}{dm} \cdot q])_{,i} ,Z_i[\psi] > 
=  - 4\pi \cdot G \cdot  < \frac{dC}{dm} Z^*[\psi] , q > 
\end{equation} 
Putting this all together we get
\begin{equation}\label{GRAV:EQU:202g}
< \nabla f(m),q> = 
\mu \cdot < 2 \omega \cdot (m-m^{ref}), q>
+ \mu \cdot < 2 \omega_i \cdot L_i^2 \cdot (m-m^{ref})_{,i}, q_{,i}> - 4\pi \cdot G \cdot  < \frac{dC}{dm} Z^*[\psi] , q > 
\end{equation}  

