\chapter{Regularization}\label{Chp:ref:regularization}

The general cost function $J_{total}$ to be minimized has some of the cost function $J_{forward}$ measuring the defect of the result from the  
forward model with the data and the cost function $J_{reg}$ introducing the regularization into the problem and makes sure 
that a unique answer exists. The regularization term is a function of possibly, vector-valued level set function  $m$ which 
represents the physical properties to be represented and is - from a mathematical point of view - the unknown
of the inversion problem. It is the intention that the values of $m$ are between zero and one and that
actual physical values are created from a mapping before being fed into a forward model. In general the cost function $J_{reg}$ is 
defined as 
\begin{equation}\label{EQU:REG:1}
J_{reg}(m) = \frac{1}{2} \int_{\Omega} 
 \sum_{k} \mu^{(0)}_k \cdot s^{(0)}_k \cdot m_k^2 +  \mu^{(1)}_{ki} \cdot s^{(1)}_{ki} \cdot L_i^2  \cdot m_{k,i} \cdot m_{k,i}
+  \sum_{l<k} \mu^{(c)}_{lk} \cdot s^{(c)}_{lk} \cdot L^4  \cdot  \sigma(m_l,m_k) dx 
\end{equation} 
where $s^{(0)}_k$, $s^{(1)}_{ki}$ and $s^{(c)}_{lk}$ are scaling factors with values between
zero and one (limits including). They may vary with there location within the domain $\Omega$.  $L_i$ is the width of the domain in $x_i$ direction and $L^2=L_i \cdot L_i$.
$sigma$ is a given symmetric, non-negative cross correlation function. We use
\begin{equation}\label{EQU:REG:4}
 \sigma(a,b) =  ( a_{,i} \cdot a_{,i}) \cdot ( b_{,i} \cdot b_{,i}) -   ( a_{,i} \cdot b_{,i})^2 
\end{equation} 
Notice that cross correlation function is measuring the angle between the surface normals of contours of level set functions. So 
minimizing the cost function will make align the surface normals of the contours.
The additional weight factors $\mu^{(0)}_k$, $ \mu^{(1)}_{ki}$ and $\mu^{(c)}_{lk}$ are between zero and one and constant across the 
domain. They are potentially modified during the inversion in order to improve the balance between the different terms 
in the cost function. Notice that values for $\mu^{(0)}_k$, $ \mu^{(1)}_{ki}$ and $\mu^{(c)}_{lk}$ are relevant for which 
the values of the corresponding entries in scaling factors $s^{(0)}_k$, $s^{(1)}_{ki}$ and $s^{(c)}_{lk}$ are non-zero.
Notice that the factors $L^4$ and $L_i^2$ are take care of any length scale changes. With the notation
\begin{equation}\label{EQU:REG:2}
\begin{array}{rcl}
\widehat{s}^{(0)}_k & = & \mu^{(0)}_k \cdot w^{(0)}_k \\
\widehat{s}^{(1)}_{ki} & = & \mu^{(1)}_{ki} \cdot w^{(1)}_{ki} \cdot L_i^2  \\
\widehat{s}^{(c)}_{lk} & = & \mu^{c}_{lk} \cdot w^{(c)}_{lk} \cdot L^4  \\
\end{array}
\end{equation} 
with $k<l$ we can write
\begin{equation}\label{EQU:REG:1b}
J_{reg}(m) = \frac{1}{2} \int_{\Omega} \left(
 \sum_{k} \widehat{s}^{(0)}_k \cdot m_k^2 + \widehat{s}^{(1)}_{ki}  \cdot m_{k,i} \cdot m_{k,i}
+  \sum_{l<k} \widehat{s}^{(c)}_{lk} \cdot  \sigma(m_l,m_k) \right) dx 
\end{equation} 
We need to provide the derivative $\frac{ \partial J_{reg}}{\partial q}$  of the cost function $J_{reg}$ with respect to a given direction $q$ which equals zero at locations
where $m$ is assumed to be zero. The derivative is given as 
\begin{equation}\label{EQU:REG:3}
\frac{ \partial J_{reg}}{\partial q}(m) =
 \int_{\Omega} Y_k \cdot q_k + X_{k,i} q_{k,i} dx 
\end{equation} 
where



 For a single-valued 
level set function th is takes the form 
\begin{equation}\label{EQU:REG:3}
\frac{ \partial J_{reg}}{\partial q}(m) =
\mu^{reg} \int_{\Omega} \omega \cdot m \cdot q  + \omega_i \cdot L_i^2 \cdot m_{,i} \cdot q_{,i} dx
\end{equation} 
So we can represent the gradient $\nabla J_{reg}$ of the cost function $J_{reg}$ by the pair of values $(Y,X)$ where we set   
\begin{equation}\label{EQU:REG:3b}
Y=\mu^{reg} \cdot \omega \cdot m \mbox{ and } X_i = \mu^{reg} \cdot \omega_i \cdot L_i^2 \cdot m_{,i}
\end{equation} 
and 
\begin{equation}\label{EQU:REG:3c}
\frac{ \partial J_{reg}}{\partial q}(m) = [ \nabla J_{reg}(m), q ] =
\int_{\Omega} Y  \cdot q  + X_i  \cdot q_{,i} dx
\end{equation} 
where $[.,.]$ is called the dual product.



==============================================================

where $<.,.>$ is an inner product. If the level set function  $m$ has several components $m_j$ the inner product $<.>$ is given
in the form 
$\omega^{(k)}$ and $\omega^{(k)}_i$ are fixed non-negative weighting factors and  $\mu^{reg}_k$ are weighting factors
which may be modified during the inversion.  $L_i$ is the length of the domain in $x_i$ direction. In the special case that 
the level set function  $m$ has a single component the inner product takes the form
\begin{equation}\label{EQU:REG:2b}
<p,q> = 
\mu^{reg} \int_{\Omega} \omega \cdot p \cdot q  + \omega_i \cdot L_i^2 \cdot p_{,i} \cdot q_{,i} dx
\end{equation} 
In practice it is assumed that the level set function is known to be zero in certain regions in the domain. Typically these regions
corresponds to region above the surface or regions explored by drilling.

We need to provide the derivative of the cost function $J_{reg}$ with respect to a given direction $q$ which equals zero at locations
where $m$ is assumed to be zero. For a single-valued 
level set function th is takes the form 
\begin{equation}\label{EQU:REG:3}
\frac{ \partial J_{reg}}{\partial q}(m) =
\mu^{reg} \int_{\Omega} \omega \cdot m \cdot q  + \omega_i \cdot L_i^2 \cdot m_{,i} \cdot q_{,i} dx
\end{equation} 
So we can represent the gradient $\nabla J_{reg}$ of the cost function $J_{reg}$ by the pair of values $(Y,X)$ where we set   
\begin{equation}\label{EQU:REG:3b}
Y=\mu^{reg} \cdot \omega \cdot m \mbox{ and } X_i = \mu^{reg} \cdot \omega_i \cdot L_i^2 \cdot m_{,i}
\end{equation} 
and 
\begin{equation}\label{EQU:REG:3c}
\frac{ \partial J_{reg}}{\partial q}(m) = [ \nabla J_{reg}(m), q ] =
\int_{\Omega} Y  \cdot q  + X_i  \cdot q_{,i} dx
\end{equation} 
where $[.,.]$ is called the dual product.

For a multi-valued level set function an additional correlation term is introduced into the cost function $J_{total}$:
\begin{equation}\label{EQU:REG:1c}
J_{reg}(m)  =  \frac{1}{2} < m,   m > + \frac{1}{2}  \sum_{k,l} \mu_{kl}^{sec} \cdot \int_{\Omega} \sigma(m_k,m_l) dx
\end{equation} 
where $sigma$ is a given symmetric, non-negative correlation function, and $\mu_{kl}^{sec}$ are symmetric, weighting factors 
($\mu_{kl}^{sec} = \mu_{lk}^{sec}$, $\mu_{kk}^{sec}=0$) which may 
be altered during the inversion. We use the correlation function 
\begin{equation}\label{EQU:REG:4}
 \sigma(a,b) = \frac{L^2}{2} \cdot ( ( a_{,i} \cdot a_{,i}) \cdot ( b_{,i} \cdot b_{,i}) -   ( a_{,i} \cdot b_{,i})^2 )
\end{equation} 
with $L=L_i \cdot L_i$. Minimizing $J_{reg}(m)$ is minimizing the angle between the surface normals of the contours formed by
two level set function. the derivative of the cost function $J_{reg}$ with respect to a given direction $q$  which equals zero at locations
where $m$ is assumed to be zero:
\begin{equation}\label{EQU:REG:5}
\begin{array}{ll}
\displaystyle{\frac{ \partial J_{reg}}{\partial q}(m)} =
\displaystyle{\sum_{k} \mu^{reg}_k \int_{\Omega} \omega^{(k)} \cdot m_k \cdot q_k  + \omega^{(k)}_i \cdot L_i^2 \cdot m_{k,i} \cdot q_{k,i} dx } \\
+ \displaystyle{\sum_{k,l} \mu_{kl}^{sec} \cdot {L^2}  \int_{\Omega}  ( m_{k,i} \cdot q_{k,i}) \cdot ( m_{l,j} \cdot m_{l,j}) -   ( m_{k,j} \cdot m_{l,j}) \cdot  ( q_{l,i} \cdot m_{k,i}) } dx 
\end{array}
\end{equation} 
Similar to the single-case we can represent  
the gradient $\nabla J_{reg}$ of the cost function $J_{reg}$ by the pair of values $(Y,X)$ where we set   
\begin{equation}\label{EQU:REG:6}
Y_k= \mu^{reg}_k  \cdot \omega^{(k)} \cdot m_k
\end{equation} 
and 
\begin{equation}\label{EQU:REG:6b}
X_{ki} = \mu^{reg}_k \cdot  \omega^{(k)} \cdot L_i^2 \cdot m_{k,i} +
\sum_{l} \mu_{kl}^{sec} \cdot {L^2} \cdot ( ( m_{l,j} \cdot m_{l,j}) \cdot m_{k,i}  -   ( m_{l,j} \cdot m_{k,j}) \cdot m_{l,i} ) 
\end{equation} 
and 
\begin{equation}\label{EQU:REG:7}
\frac{ \partial J_{reg}}{\partial q}(m) = [ \nabla J_{reg}(m), q ] =
\int_{\Omega} Y_j  \cdot q_j  + X_{ki}  \cdot q_{k,i} dx
\end{equation} 
where $[.,.]$ is the dual product.

We also need to provide an approximation of the inverse of the Hessian operator which provides a 
level set function $h$ for a given value $r$ represented by the pair of values $(Y,X)$. If one ignores the correlation function
the inner product defines the Hessian operator of the cost function. In this approach we set 
\begin{equation}\label{EQU:REG:8}
 < p,   h > = [p, r]
\end{equation} 
for all $p$. This problem can be solved using \escript \class{LinearPDE} class by setting
\begin{equation}\label{EQU:REG:8b}
\begin{array}{rcl}
 A_{ij} & =&  (\omega_i \cdot L_i^2) \cdot \delta_{ij}  \\
D & = & \mu^{reg} \cdot \omega
\end{array}
\end{equation} 
and $X$ and $Y$ as defined by $r$ for the case of a single-valued level set function.
For a vector-valued level-set function one sets:
\begin{equation}\label{EQU:REG:8c}
\begin{array}{rcl}
 A_{kilj} & = & (\mu^{reg}_l \omega^{(l)}_i L_i^2) \cdot  \delta_{kl} \cdot  \delta_{ij}  \\
D_{kl} & =  & \mu^{reg}_l  \cdot \omega^{(l)} \delta_{kl}  \omega
\end{array}
\end{equation} 
====================================================
\begin{classdesc}{Regularization}{domain, \optional{s0=\None,  \optional{s1=\None, \optional{sc=\None, 
\optional{location_of_set_m=Data(), 
\optional{numLevelSets=1, 
\optional{useDiagonalHessianApproximation=\True, \optional{tol=1e-8}}}}}}}}

opens a linear, steady, second order PDE on the \Domain \var{domain}.
The parameters \var{numEquations} and \var{numSolutions} give the number of
equations and the number of solution components.
If \var{numEquations} and \var{numSolutions} are non-positive, then the number
of equations and the number of solutions, respectively, stay undefined until a
coefficient is defined.
\end{classdesc}
 


\subsection{The general regularization class}
\begin{classdesc}{RegularizationBase}{}

\end{classdesc}