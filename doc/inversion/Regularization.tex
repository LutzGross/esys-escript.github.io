\chapter{Regularization}\label{Chp:ref:regularization}

The general cost function $J^{total}$ to be minimized has some of the cost
function $J^f$ measuring the defect of the result from the
forward model with the data, and the cost function $J^{reg}$ introducing the
regularization into the problem and makes sure that a unique answer exists.
The regularization term is a function of, possibly vector-valued, level set
function $m$ which represents the physical properties to be represented and is,
from a mathematical point of view, the unknown of the inversion problem.
It is the intention that the values of $m$ are between zero and one and that
actual physical values are created from a mapping before being fed into a
forward model. In general the cost function $J^{reg}$ is defined as 
\begin{equation}\label{EQU:REG:1}
J^{reg}(m) = \frac{1}{2} \int_{\Omega} \left(
 \sum_{k} \mu_k \cdot ( \omega^{(0)}_k \cdot m_k^2 + \omega^{(1)}_{ki}m_{k,i}^2 ) 
+  \sum_{l<k} \mu^{(c)}_{lk} \cdot \omega^{(c)}_{lk}  \cdot  \chi(m_l,m_k) \right) \; dx 
\end{equation} 
where summation over $i$ is performed. The additional trade--off factors 
$\mu_k$ and $\mu^{(c)}_{lk}$ ($l<k$) are between zero and one and constant across the 
domain. They are potentially modified during the inversion in order to improve the balance between the different terms 
in the cost function.

$\chi$ is a given symmetric, non-negative cross-gradient function\index{cross-gradient
}. We use
\begin{equation}\label{EQU:REG:4}
 \chi(a,b) =  ( a_{,i} a_{,i}) \cdot ( b_{,j} b_{,j}) -   ( a_{,i} b_{,i})^2 
\end{equation} 
where summations over $i$ and $j$  are performed. Notice that cross-gradient function
is measuring the angle between the surface normals of contours of level set functions. So 
minimizing the cost function will align the surface normals of the contours.

The coefficients $\omega^{(0)}_k$, $\omega^{(1)}_{ki}$ and $\omega^{(c)}_{lk}$ define weighting factors which 
may depend on their location within the domain. We assume that for given level set function $k$ the 
weighting factors $\omega^{(0)}_k$, $\omega^{(1)}_{ki}$ are scaled such that 
\begin{equation}\label{ref:EQU:REG:5}
\int_{\Omega} ( \omega^{(0)}_k  + \frac{\omega^{(1)}_{ki}}{L_i^2} ) \; dx = \alpha_k \cdot  vol(\Omega)
\end{equation} 
where $\alpha_k$ defines the scale which is typically set to one. $L_i$ is the width of the domain in $x_i$ direction.
Similarly we set for $l<k$ we set 
\begin{equation}\label{ref:EQU:REG:6}
\int_{\Omega} \frac{\omega^{(c)}_{lk}}{L^4} \; dx = \alpha^{(c)}_{lk} \cdot  vol(\Omega)
\end{equation} 
where $\alpha^{(c)}_{lk}$ defines the scale which is typically set to one and
\begin{equation}\label{ref:EQU:REG:6b}
\frac{1}{L^2} = \sum_i \frac{1}{L_i^2} \;.
\end{equation} 

In some cases values for the level set functions are known to be zero at certain regions in the domain. Typically this is the region 
above the surface of the Earths. This expressed using a
a characteristic function $q$ which varies with its location within the domain. The function $q$ is set to zero except for those 
locations $x$ within the domain where the values of the level set functions is known to be zero. For these locations $x$
$q$ takes a positive value. for a single level set function one has
\begin{equation}\label{ref:EQU:REG:7}
q(x) = \left\{ 
\begin{array}{rl}
  1 & \mbox{ if } m \mbox{ is set to zero at location } x \\
  0 & \mbox{ otherwise }
\end{array}
\right.
\end{equation} 
For multi-valued level set function the  characteristic function is set componentwise:
\begin{equation}\label{ref:EQU:REG:7b}
q_k(x) = \left\{ 
\begin{array}{rl}
  1 & \mbox{ if component } m_k \mbox{ is set to zero at location } x \\
  0 & \mbox{ otherwise }
\end{array}
\right.
\end{equation} 


\section{Usage}

\LG{Add example}

\begin{classdesc}{Regularization}{domain
        \optional{, w0=\None}
        \optional{, w1=\None}
        \optional{, wc=\None}
        \optional{, location_of_set_m=Data()}
        \optional{, numLevelSets=1}
        \optional{, useDiagonalHessianApproximation=\False}
        \optional{, tol=1e-8}
        \optional{, scale=\None}
        \optional{, scale_c=\None}
        }

  
initializes a regularization component of the cost function for inversion. 
\member{domain} defines the domain of the inversion. \member{numLevelSets}
sets the number of level set functions to be found during the inversion. 
\member{w0}, \member{w1} and  \member{wc} define the weighting factors
$\omega^{(0)}$,
$\omega^{(1)}$ and
$\omega^{(c)}$, respectively. A value for \member{w0} or \member{w1} or both must be given. 
If more then one level set function is involved  \member{wc} must be given. 
\member{location_of_set_m} sets the characteristic function $q$ 
to define locations where the level set function is set to zero, see equation~(\ref{ref:EQU:REG:7}).
\member{scale} and 
\member{scale_c} set the scales $\alpha_k$ in equation~(\ref{ref:EQU:REG:5}) and
$\alpha^{(c)}_{lk}$ in equation~(\ref{ref:EQU:REG:6}), respectively. By default, their values are set to one.
Notice that weighting factors are rescaled to meet the scaling conditions. \member{tol} sets the 
tolerance for the calculation of the Hessian approximation. \member{useDiagonalHessianApproximation} 
indicates to ignore coupling in the Hessian approximation produced by the 
cross-gradient term. This can speed-up an individual iteration step in the inversion but typically leads to more
inversion steps.
\end{classdesc}

\section{Gradient Calculation}


The cost function kernel\index{cost function!kernel} is given as
\begin{equation}\label{ref:EQU:REG:100}
K^{reg}(m) = \frac{1}{2}
\sum_{k} \mu_k \cdot ( \omega^{(0)}_k \cdot m_k^2 + \omega^{(1)}_{ki}m_{k,i}^2 ) 
+  \sum_{l<k} \mu^{(c)}_{lk} \cdot \omega^{(c)}_{lk}  \cdot  \chi(m_l,m_k)
\end{equation} 

We need to provide the gradient of the cost function $J^{reg}$  with respect to the level set functions $m$.
The gradient is represented by two functions $Y$ and $X$ which define the 
derivative of the cost function kernel with respect to $m$ and to the gradient $m_{,i}$, respectively:
\begin{equation}\label{ref:EQU:REG:101}
\begin{array}{rcl}
  Y_k & = & \displaystyle{\frac{\partial K^{reg}}{\partial m_k}} \\
   X_{k,i} & = & \displaystyle{\frac{\partial K^{reg}}{\partial m_{k,i}}} 
\end{array}
\end{equation} 
For $Y$ we get 
\begin{equation}\label{ref:EQU:REG:102}
Y_k = \mu_k \cdot \omega^{(0)}_k \cdot m_k
\end{equation} 
and for $X$ (and ignoring the cross gradient component):
\begin{equation}\label{ref:EQU:REG:103}
 X_{k,i} = \mu_k \cdot \omega^{(1)}_{ki} \cdot m_{k,i}
\end{equation}  


We also need to provide an approximation of the inverse of the Hessian operator which provides a 
level set function $h$ for a given value $r$ represented by the pair of values $(Y,X)$. If one ignores the correlation function
the inner product defines the Hessian operator of the cost function. In this approach we set 
\begin{equation}\label{EQU:REG:8}
 < p,   h > = [p, r]
\end{equation} 
for all $p$. This problem can be solved using \escript \class{LinearPDE} class by setting
\begin{equation}\label{EQU:REG:8b}
\begin{array}{rcl}
 A_{ij} & =&  \mu \cdot \omega^{(1)}_i \cdot \delta_{ij}  \\
D & = &  \mu \cdot \omega^{(0)} 
\end{array}
\end{equation} 
and $X$ and $Y$ as defined by $r$ for the case of a single-valued level set function.
For a vector-valued level-set function one sets:
\begin{equation}\label{EQU:REG:8c}
\begin{array}{rcl}
 A_{kilj} & = & \mu_k \cdot \omega^{(1)}_{ki} \cdot \delta_{ij} \cdot \delta_{kl}  \\
D_{kl} & =  &  \mu_k \cdot \omega^{(0)}_k \cdot \delta_{kl} 
\end{array}
\end{equation} 

