
%%%%%%%%%%%%%%%%%%%%%%%%%%%%%%%%%%%%%%%%%%%%%%%%%%%%%%%%%%%%%%%%%%%%%%%%%%%%%%
% Copyright (c) 2003-2016 by The University of Queensland
% http://www.uq.edu.au
%
% Primary Business: Queensland, Australia
% Licensed under the Apache License, version 2.0
% http://www.apache.org/licenses/LICENSE-2.0
%
% Development until 2012 by Earth Systems Science Computational Center (ESSCC)
% Development 2012-2013 by School of Earth Sciences
% Development from 2014 by Centre for Geoscience Computing (GeoComp)
%
%%%%%%%%%%%%%%%%%%%%%%%%%%%%%%%%%%%%%%%%%%%%%%%%%%%%%%%%%%%%%%%%%%%%%%%%%%%%%%


\section{Linear Magnetic Intensity Inversion}\label{sec:forward magnetic intensity}
An alternative target for then inversion is the intensity anomaly of the magnetic flux
density. In this case the difference $b$ of the length of the total flux
density $B^{tot}=B^b+B$ to the back ground flux density $B^b$~\index{magnetic flux density}:
\begin{equation}\label{ref:IMAG:EQU:0}
b=\|B^{tot}\| -\|B^{b}\|
\end{equation}
where is the Euclean norm is defined as  
\begin{equation}\label{ref:IMAG:EQU:1}
\|B\| =  \sqrt{B_i B_i}
\end{equation}
and we use the notations of the previous section~\ref{sec:forward magnetic}.   
We have 
\begin{equation}\label{ref:IMAG:EQU:1b}
\|B^{tot}\|^2 = \|B^{b}\|^2 + 2 B^{b}_i B_i + \|B\|^2 \approx  \|B^{b}\|^2 + 2 B^{b}_i B_i
\end{equation}
where the term $\|B\|^2$ of the anomaly can is assumed to be neglagable. 
This gives 
\begin{equation}\label{ref:IMAG:EQU:1bb}
\|B^{tot}\|  = \|B^{b}\| \sqrt{1 + \frac{2 B^{b}_i B_i}{\|B^{b}\|^2}} \approx  \|B^{b}\| \left(1 + \frac{B^{b}_i B_i}{\|B^{b}\|^2} \right) 
\end{equation}
and therefore
\begin{equation}\label{ref:IMAG:EQU:0b}
b= \frac{B^{b}_i B_i}{\|B^{b}\|}  = \hat{B}^{b}_i B_i \mbox{ with } \hat{B}^{b}_i = \frac{B^b_i}{\|B^{b}\|} 
\end{equation}
If $b^{(s)}_i$ is the measurement of the magnetic flux density intensity anomaly for
survey $s$ and $\omega^{(s)}$ is a weighting factor the data defect
$J^{mag}(k)$ in the notation of Chapter~\ref{chapter:ref:inversion cost function} is given as
\begin{equation}\label{ref:IMAG:EQU:9}
J^{mag}(k) = \frac{1}{2}\sum_{s} \int_{\Omega} \left( \omega^{(s)} \cdot ( \hat{B}^{b}_i B_i - b^{(s)} )\right)^2 dx
\end{equation} 

The cost function kernel\index{cost function!kernel} is given as
\begin{equation}\label{ref:IMAG:EQU:10}
K^{mag}(\psi_{,i},k) = \frac{1}{2}\sum_{s} \left( \omega^{(s)} \cdot (k \cdot  \|B^{b}\|  -  \hat{B}^{b}_i \psi_{,i} - b^{(s)}) \right)^2
\end{equation} 
Notice that if magnetic flux density is measured in air one can ignore the
$k\cdot B^b_i$ as the susceptibility is zero.

In practice the magnetic flux density intensity $b^{(s)}$ is measured 
with a standard error deviation $\sigma^{(s)}$ at certain locations in the domain.
In this case one sets the weighting
factors $\omega^{(s)}$ as
\begin{equation}\label{ref:IMAG:EQU:11}
\omega^{(s)} 
= \left\{
\begin{array}{lcl}
f \cdot  \frac{f}{\sigma^{(s)}} & & \mbox{data are available} \\
& \mbox{ where } & \\
0 & & \mbox{ otherwise } \\
\end{array}
\right.
\end{equation} 
With the objective to control the 
gradient of the cost function the scaling factor $f$ is chosen in the way that 
\begin{equation}\label{ref:IMAG:EQU:12}
\sum_{s} \int_{\Omega} ( \omega^{(s)}_i B^{(s)}_i ) 
 \cdot ( \omega^{(s)}_j \frac{1}{L_j} ) \cdot L^2 \cdot
( B^b_n \frac{1}{L_n} )
 \cdot k' \;
 dx =\alpha
\end{equation} 
where $\alpha$ defines a scaling factor which is typically set to one and $L$ is defined by equation~(\ref{ref:EQU:REG:6b}).
$k'$ is considering the 
derivative of the density with respect to the level set function. 



\subsection{Usage}

\begin{classdesc}{MagneticIntensityModel}{domain, w, b, background_field,
        \optional{, fixPotentialAtBottom=False},
        \optional{, tol=1e-8},
}
opens a magnetic forward model over the \Domain \member{domain} with 
weighting factors \member{w} ($=\omega^{(s)}$) and measured magnetic flux
density anomalies \member{b} ($=b^{(s)}$).
The weighting factors and the  measured magnetic flux density anomalies must be scalars.
\member{background_field} defines the background magnetic flux density $B^b$
as a vector with north, east and vertical components. 
\member{tol} sets the tolerance for the solution of the PDE~(\ref{ref:MAG:EQU:8}).
If \member{fixPotentialAtBottom} is set to  \True, the magnetic potential 
at the bottom is set to zero in addition to the potential on the top. 
\member{coordinates} set the reference coordinate system to be used. By the default the 
Cartesian coordinate system is used.
\end{classdesc}

\begin{methoddesc}[MagneticModel]{rescaleWeights}{
        \optional{scale=1.}
 \optional{k_scale=1.}}
rescale the weighting factors such condition~(\ref{ref:IMAG:EQU:12}) holds where 
\member{scale} sets the scale $\alpha$
and \member{k_scale} sets $k'$. This method should be called before any inversion is started
in order to make sure that all components of the cost function are appropriately scaled.
\end{methoddesc}


\subsection{Gradient Calculation}
This section briefly explains how the gradient
$\frac{\partial J^{mag}}{\partial k}$ of the cost function $J^{mag}$ with
respect to the susceptibility $k$ is calculated.  We follow the concept as outlined in section~\ref{chapter:ref:inversion cost function:gradient}.

The magnetic potential $\psi$ from PDE~(\ref{ref:MAG:EQU:8}) is solved in weak form:
\begin{equation}\label{ref:IMAG:EQU:201}
\int_{\Omega} q_{,i} \psi_{,i} \; dx  = \int_{\Omega}  k \cdot q_{,i}  B^r_i \; dx 
\end{equation} 
for all $q$ with $q=0$ on $\Gamma_{0}$.
In the following we set $\Psi[k]=\psi$ for a given susceptibility $k$ as
solution of the variational problem~(\ref{ref:MAG:EQU:201}).
If $\Gamma_{k}$ denotes the region of the domain where the susceptibility is
known and for a given direction $p$ with $p=0$ on $\Gamma_{k}$ one has
\begin{equation}\label{ref:IMAG:EQU:201aa}
\int_{\Omega}   \frac{\partial J^{mag}}{\partial k} \cdot p \; dx  = \int_{\Omega}  
\sum_{s} 
\omega^{(s)} \; \left(b^{(s)} - k \cdot  \|B^{b}\|  +  \hat{B}^{b}_i \psi_{,i}  \right)
\cdot \omega^{(s)} \; \left(\hat{B}^{b}_i \Psi[p]_{,i} - p \cdot  \|B^{b}\| \right) \; dx  
\end{equation}
With
\begin{equation}\label{ref:IMAG:EQU:202c}
Y_i[\psi]=   \sum_{s} 
(\omega^{(s)})^2 
\left(b^{(s)} - k \cdot  \|B^{b}\|  +  \hat{B}^{b}_i \psi_{,i} \right) \hat{B}^{b}_i
\end{equation} 
this is written as 
\begin{equation}\label{ref:IMAG:EQU:202cc}
\int_{\Omega}   \frac{\partial J^{mag}}{\partial k} \cdot p \;  dx  = \int_{\Omega}  
(Y_i[\psi] \Psi[p]_{,i} -p \cdot Y_i[\psi]B^b_i)   \; dx  
\end{equation} 
We then set adjoint function $Y^*[\psi]$ as the solution of the equation 
\begin{equation}\label{ref:IMAG:EQU:202d}
\int_{\Omega} r_{,i} Y^*[\psi]_{,i} \; dx  =  \int_{\Omega} r_{,i} Y_i[\psi]  \; dx  \mbox{ for all } r \mbox{ with } r=0 \mbox{ on } \Gamma_{0}
\end{equation} 
with $Y^*[\psi]=0$ on $\Gamma_{0}$. With $r=\Psi[p]$ we get
\begin{equation}\label{ref:IMAG:EQU:202dd}
\int_{\Omega} \Psi[p]_{,i} Y^*[\psi]_{,i} \; dx  =  \int_{\Omega} \Psi[p]_{,i} Y_i[\psi]  \; dx
\end{equation} 
and from Equation~(\ref{ref:IMAG:EQU:201}) with $q=Y^*[\psi]$ we get
\begin{equation}\label{ref:IMAG:EQU:20e}
\int_{\Omega} Y^*[\psi]_{,i}  \Psi[p]_{,i} \; dx  = \int_{\Omega}  p \cdot Y^*[\psi]_{,i}  B^r_i \; dx  
\end{equation}
which leads to 
\begin{equation}\label{ref:IMAG:EQU:20ee}
\int_{\Omega} \Psi[p]_{,i} ,Y_i[\psi]  \; dx  = \int_{\Omega}  p \cdot Y^*[\psi]_{,i}  B^r_i \; dx  
\end{equation}
and finally
\begin{equation}\label{ref:IMAG:EQU:201a}
\int_{\Omega}   \frac{\partial J^{mag}}{\partial k} \cdot p \;  dx  = \int_{\Omega}  
p \cdot (Y^*[\psi]_{,i}  B^r_i - Y_i[\psi] B^b_i) \; dx  
\end{equation} 
or 
\begin{equation}\label{ref:IMAG:EQU:201b}
\frac{\partial J^{mag}}{\partial k} = Y^*[\psi]_{,i}  B^r_i - Y_i[\psi] B^b_i
\end{equation}

\subsection{Geodetic Coordinates}
Not supported yet.
