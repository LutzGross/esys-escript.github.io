
%%%%%%%%%%%%%%%%%%%%%%%%%%%%%%%%%%%%%%%%%%%%%%%%%%%%%%%%%%%%%%%%%%%%%%%%%%%%%%
% Copyright (c) 2003-2012 by University of Queensland
% http://www.uq.edu.au
%
% Primary Business: Queensland, Australia
% Licensed under the Open Software License version 3.0
% http://www.opensource.org/licenses/osl-3.0.php
%
% Development until 2012 by Earth Systems Science Computational Center (ESSCC)
% Development since 2012 by School of Earth Sciences
%
%%%%%%%%%%%%%%%%%%%%%%%%%%%%%%%%%%%%%%%%%%%%%%%%%%%%%%%%%%%%%%%%%%%%%%%%%%%%%%


\section{Magnetic Inversion}\label{sec:forward magnetic}
For the magnetic inversion we use the anomaly of the magnetic flux density~\index{magnetic flux density} of the Earth.
The controlling material parameter
is the susceptibility~\index{susceptibility} $k$ of the rock. With magnetization $M$,
inducing magnetic field anomaly $H^s$ the magnetic flux density anomaly $B^s$ is given as
\begin{equation}\label{ref:EQU:1}
B_i = \mu_0 \cdot ( H^s_i  + M_i )
\end{equation}
where $\mu_0 = 4 \pi \cdot 10^{-7} \frac{Vs}{Am}$.
In this forward model we make the simplifying assumption that the magnetization is proportional to
the known geomagnetic
 flux density $B^b$:
\begin{equation}\label{ref:EQU:4}
\mu_0  \cdot M_i = k \cdot B^b_i \;. 
\end{equation}
Values for the magnetic flux density
can be obtained by the International Geomagnetic Reference Field (IGRF)~\cite{IGRF}
(or the Australian Geomagnetic Reference Field (AGRF)~\cite{AGRF}). A rough approximation at 
at latitude $\theta$ is given by 
\begin{equation}\label{ref:EQU:5}
\begin{array}{rcl}
B^b_{\theta}  & = & \displaystyle{ \frac{ \mu_0 \cdot m_{earth}}{4 \pi \cdot R_{earth}^3} sin(\theta) }  \\
B^b_r & = & \displaystyle{ \frac{\mu_0 \cdot  m_{earth}}{1 \pi \cdot R_{earth}^3} cos(\theta) }
\end{array}
\end{equation}
with the vacuum permeability\index{vacuum permeability}  $\mu_0 = 4 \pi \cdot 10^{-7} \frac{Vs}{Am}$, 
the magnetic dipole moment of Earth $m_{earth}= 8.22 \cdot 10^{22} Am^2$ and earth radius $R_{earth}= 6378137m$.
$B^b_r$ and $b^b_{\theta}$ denote the radial and latitudinal component of the geomagnetic
 flux density.
Notice that convention~(\ref{REF:EQU:INTRO 9}) applies if Cartesian coordinates\index{Cartesian coordinates} are used. 
In most cases it is reasonable to assume that that the background field is constant across the domain.

The magnetic field anomaly $H^s$ can be represented be the gradient of a magnetic scalar potential\index{scalar potential!magnetic}
$\psi$. We use the form 
\begin{equation}\label{ref:EQU:6}
\mu_0  \cdot H^s_i = - \psi_{,i}
\end{equation}
With this notation one gets from equations~(\ref{ref:EQU:1}) and~(\ref{ref:EQU:4}):
\begin{equation}\label{ref:EQU:7}
B_i = - \psi_{,i}  + k \cdot B^b_i
\end{equation}
As the $B^s$ magnetic flux density anomaly we obtain the PDE  
\begin{equation}\label{ref:EQU:8}
- \psi_{,ii} = - (k B^b_i)_{,i} 
\end{equation} 
which needs to be solved for a given susceptibility $k$. The magnetic scalar potential is set to zero 
at the top of the domain $\Gamma_{top}$. On all other faces the normal component of the magnetic flux density anomaly $B_i$
is set to zero, i.e. $n_i \psi_{,i}  = k \cdot n_i  B^b_i$ with outer normal field $n_i$).

From the magnetic scalar potential we can calculate the magnetic flux density anomaly via equation~(\ref{ref:EQU:8}) to
calculate the defect to the given data. If $B^{(s)}_i$ is a measurement of the magnetic flux density anomaly for 
survey $s$ and $\chi^{(s)}_i$ is a weighting factor the data defect $J^{mag}(k)$ in the notation of Chapter~\ref{Chp:ref:introduction}
is given as
\begin{equation}\label{ref:EQU:9}
J^{mag}(k) = \frac{1}{2}\sum_{s} \int_{\Omega} \chi^{(s)}_i \cdot (B_{i}- B^{(s)}_i)^2 dx
\end{equation} 
The cost function kernel\index{cost function!kernel} is given as
\begin{equation}\label{ref:EQU:10}
K^{mag}(\psi_{,i},k) = \frac{1}{2}\sum_{s} \int_{\Omega} \chi^{(s)}_i \cdot (k \cdot B^b_i - \psi_{,i} - B^{(s)}_i)^2
\end{equation} 



\subsection{Solution methods}
To apply the Nonlinear Conjugate Gradient method (NLCG), see Appendix~\ref{sec:NLCG} or the L-BFGS method, see Appendix~\ref{sec:LBFGS} we need
to define an inner product $<.,.>$ and need to calculate the gradient of of $f$. 

As inner product we use 
\begin{equation}\label{MAG:EQU:200}
<p,q> = \int_{\Omega} p \cdot q \; dx
\end{equation} 
With this notation the magnetic potential is given as $\psi=\Psi[1+k]$ where
\begin{equation}\label{MAG:EQU:201}
< q_{,i},\Psi[p]_{,i} > = < q_{,i} , p \cdot H^b_i> \mbox{ for all } q \mbox{ with } q=0 \mbox{ on } \Gamma_{z}
\end{equation} 
where $\Gamma_{z}$ denotes the top and bottom surface of the domain for case (D)
and the top surface of the domain for case (N). 



For any $q$ with $q=0$ on $\Omega^{ref}$ we get 
\begin{equation}\label{MAG:EQU:202}
< \nabla f(m),q> = < \nabla J_{data}(\Psi[1+C[m]]), q>  +  \mu \cdot < \nabla J_{reg}(m),q>
\end{equation} 
with 
\begin{equation}\label{MAG:EQU:202a}
< \nabla J_{reg}(m),q> = 
\int_{\Omega} 2 \omega q \cdot (m-m^{ref}) + 2 \omega_i \cdot L_i^2 \cdot q_{,i} (m-m^{ref})_{,i} dx
\end{equation} 
and 
\begin{equation}\label{MAG:EQU:202b}
< \nabla J_{data}(\Psi[C[m]]), q> =  \sum_{s} \int_{\Omega} \chi^{(s)}_i \cdot (  B_i  -  B^{(s)}_i ) 
\cdot ( \frac{dC}{dm} \cdot q \cdot H^b_i  - \Psi[\frac{dC}{dm} \cdot q]_{,i}) dx 
\end{equation} 
where $k=C(m)$ and $B_i=(1+k) H^b_i  - \Psi[1+k]_{,i}$
With
\begin{equation}\label{MAG:EQU:202c}
Y_i[\psi]= \sum_{s} \chi^{(s)}_i \cdot (  B_i  -  B^{(s)}_i ) 
\end{equation} 
we get 
\begin{equation}\label{MAG:EQU:202bb}
< \nabla J_{data}(\Psi[C[m]]), q> = 
< Y_i[\psi] H^b_i, \frac{dC}{dm} q> - < Y_i[\psi] ,  \Psi[\frac{dC}{dm} \cdot q]_{,i} >
\end{equation} 
and $Y^*[\psi]$ as the solution of the equation 
\begin{equation}\label{MAG:EQU:202d}
< p_{,i},Y^*[\psi]_{,i} > =  < p_{,i} ,Y_i[\psi] > \mbox{ for all } p \mbox{ with } p=0 \mbox{ on } \Gamma_{z}
\end{equation} 
with $Y^*[\psi]=0$ on $\Gamma_{z}$. 
By stetting $p=\Psi[\frac{dC}{dm} \cdot q]$ we get
\begin{equation}\label{MAG:EQU:202e}
<(\Psi[\frac{dC}{dm} \cdot q]) _{,i},Y^*[\psi]_{,i} > =  < (\Psi[\frac{dC}{dm} \cdot q])_{,i} ,Y_i[\psi] >
\end{equation} 
and from~\ref{MAG:EQU:201} with $q=Y^*[\psi]$ we get
\begin{equation}\label{MAG:EQU:20e}
< (Y^*[\psi])_{,i},(\Psi[\frac{dC}{dm} \cdot q])_{,i} > = <(Y^*[\psi])_{,i} , \frac{dC}{dm} \cdot q \cdot H^b_i>
\end{equation}
which leads to 
\begin{equation}\label{MAG:EQU:202f}
< \nabla J_{data}(\Psi[C[m]]), q> = < ( Y_i[\psi] - (Y^*[\psi])_{,i} ) H^b_i,  \frac{dC}{dm} \cdot q>
\end{equation} 
Putting this all together we get
\begin{align}
< \nabla f(m),q> = &\mu \cdot < 2 \omega \cdot (m-m^{ref}), q>\nonumber\\
+ &\mu \cdot < 2 \omega_i \cdot L_i^2 \cdot (m-m^{ref})_{,i}, q_{,i}>\nonumber\\
+ &< ( Y_i[\psi] - (Y^*[\psi])_{,i} ) H^b_i,  \frac{dC}{dm} \cdot q>\label{MAG:EQU:202g}
\end{align}  


