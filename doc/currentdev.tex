
%%%%%%%%%%%%%%%%%%%%%%%%%%%%%%%%%%%%%%%%%%%%%%%%%%%%%%%%%%%%%%%%%%%%%%%%%%%%%%
% Copyright (c) 2003-2014 by University of Queensland
% http://www.uq.edu.au
%
% Primary Business: Queensland, Australia
% Licensed under the Open Software License version 3.0
% http://www.opensource.org/licenses/osl-3.0.php
%
% Development until 2012 by Earth Systems Science Computational Center (ESSCC)
% Development 2012-2013 by School of Earth Sciences
% Development from 2014 by Centre for Geoscience Computing (GeoComp)
%
%%%%%%%%%%%%%%%%%%%%%%%%%%%%%%%%%%%%%%%%%%%%%%%%%%%%%%%%%%%%%%%%%%%%%%%%%%%%%%

\vbox{}
\vfill
\begin{center}
\textbf{\Large Researchers and Developers}\pdfbookmark[0]{Researchers}{researchers}
\vspace{0.5cm}

Escript is the product of years of work by many people.
The active researchers for the current release series (3.X) are listed here in alphabetical order.
While development is collaborative, each person is listed with some of their major contributions --- this list is not exhaustive.
Personel for previous release series are listed in an appendix of the user guide.

\vspace{1cm}
\hrule
\vspace{1cm}
\begin{description}
\item[Cihan Altinay] \texttt{esys.weipa} visualisation package, SCons build system rework, CUDA implementations.
\item[Vince Boros] Magnetism.
\item[Joel Fenwick] Lazy evaluation, maintenance of escript module, release wrangler.
\item[Lutz Gross] Patriarch, technical lead, solvers, large chunks of the original code.
\item[Jaco du Plessis] Symbolic toolbox, GMSH reader MPI implementation, DC resistivity.
\item[Simon Shaw] \texttt{esys.speckley} module, release help, large cluster improvements.
\end{description}
\end{center}
\vfill
\pagebreak

