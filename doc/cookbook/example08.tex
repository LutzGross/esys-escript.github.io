
%%%%%%%%%%%%%%%%%%%%%%%%%%%%%%%%%%%%%%%%%%%%%%%%%%%%%%%%
%
% Copyright (c) 2003-2010 by University of Queensland
% Earth Systems Science Computational Center (ESSCC)
% http://www.uq.edu.au/esscc
%
% Primary Business: Queensland, Australia
% Licensed under the Open Software License version 3.0
% http://www.opensource.org/licenses/osl-3.0.php
%
%%%%%%%%%%%%%%%%%%%%%%%%%%%%%%%%%%%%%%%%%%%%%%%%%%%%%%%%

\section{Seismic Wave Propagation in Two Dimensions}
\editor{This chapter aims to expand the readers understanding of escript by
modelling the wave equations.
Challenges will include a second order differential (multiple initial
conditions). A new PDE to fit to the general form. Movement to a 3D problem
(maybe)??? }

\sslist{example08a.py}

We will now expand upon the previous chapter by introducing a vector form of
the wave equation. This means that the waves will have both a scalar magnitude,
but also a direction. This type of scenario is apparent in wave forms that
exhibit compressional and transverse particle motion. A common type of wave
that obeys this principle are seismic waves.

Wave propagation in the earth can be described by the elastic wave equation:
\begin{equation} \label{eqn:wav} \index{wave equation}
\rho \frac{\partial^{2}u\hackscore{i}}{\partial t^2} - \frac{\partial
\sigma\hackscore{ij}}{\partial x\hackscore{j}} = 0
\end{equation}
where $\sigma$ is the stress given by:
\begin{equation} \label{eqn:sigw}
 \sigma \hackscore{ij} = \lambda u\hackscore{k,k} \delta\hackscore{ij} + \mu (
u\hackscore{i,j} + u\hackscore{j,i})
\end{equation}
where $\lambda$ and $\mu$ are the Lame Coefficients. Specifically $\mu$ is the
bulk modulus. The \refEq{eqn:wav} can be written with the Einstein summation
convention as:
\begin{equation}
\rho u\hackscore{i,tt} = \sigma\hackscore{ij,j}
\end{equation}

In a similar process to the previous chapter, we will use the acceleration
solution to solve this PDE. By substituting $a$ directly for
$\frac{\partial^{2}u\hackscore{i}}{\partial t^2}$ we can derive the
displacement solution. Using $a$ \refEq{eqn:wav} becomes;
\begin{equation} \label{eqn:wava} 
\rho a\hackscore{i} - \frac{\partial
\sigma\hackscore{ij}}{\partial x\hackscore{j}} = 0
\end{equation}

\section{Vector source on the boundary}
For this particular example, we will introduce the source by applying a
displacment to the boundary during the initial time steps. The source will again
be
a radially propagating wave but due to the vector nature of the PDE used, a
direction will need to be applied to the source.

The first step is to choose an amplitude and create the source as in the
previous chapter. 
\begin{python}
 src_length = 20; print "src_length = ",src_length
# set initial values for first two time steps with source terms
y=U0*(cos(length(x-xc)*3.1415/src_length)+1)*whereNegative(length(x-xc)-src_leng
th)
\end{python}
where \verb xc  is the source point on the boundary of the model. The source
direction is then defined as an $(x,y)$ array and multiplied by the source
function. The directional array must have a magnitude of $1$ otherwise the
amplitude of the source will become modified. For this example, the source is
directed in the $-y$ direction.
\begin{python}
src_dir=numpy.array([0.,-1.]) # defines direction of point source as down
y=y*src_dir
\end{python}
The function can then be applied as a boundary condition by setting it equal to
$y$ in the general form.
\begin{python}
mypde.setValue(y=y) #set the source as a function on the boundary
\end{python}
Because we are no longer using the source to define our initial condition to
the model, we must set the model state to zero for the first two time steps.
\begin{python}
# initial value of displacement at point source is constant (U0=0.01)
# for first two time steps
u=[0.0,0.0]*whereNegative(x)
u_m1=u
\end{python}

If the source will introduce energy to the system over a period longer than one
or two time steps (ie the initial conditions), $y$ can be updated during the
iteration stage. 