
%%%%%%%%%%%%%%%%%%%%%%%%%%%%%%%%%%%%%%%%%%%%%%%%%%%%%%%%
%
% Copyright (c) 2003-2009 by University of Queensland
% Earth Systems Science Computational Center (ESSCC)
% http://www.uq.edu.au/esscc
%
% Primary Business: Queensland, Australia
% Licensed under the Open Software License version 3.0
% http://www.opensource.org/licenses/osl-3.0.php
%
%%%%%%%%%%%%%%%%%%%%%%%%%%%%%%%%%%%%%%%%%%%%%%%%%%%%%%%%

flexibility on modeling
\chapter{Introduction}
\label{CHAP INTRO}
\section{Why \esc?}
\esc is an environment for mathematical modeling based on partial differential equations (PDEs).
By using the mathematical language of PDEs
it provides a high-level of abstraction from the underlying numerical schemes (e.g. finite elements (FEM)) and their implementations (eg. from aspects of parallelization) so in their work user can concentrate on the modelling aspects.
Modelling codes are portable from desktop computers upto supercomputers\footnote{\esc suports distributed memory architectures with mult-core processors through MPI and threading.} with no modifications to the code. \esc is 
integrated into the interpretive programming languge python\footnote{see \url{www.python.org} } which gives users direct access a large number
of software tool for linear algebra, visualization, image processing, data plotting, and many others.

If you start a mathematical modelling project, what are the benefits for you using \esc as a software 
platform for your project? Using an existing environment such as \esc rather than starting from scratch saves software development time as fundamental problems eg. to find appropriate data structures and numerical algorithms have already been solved and their implemenation has heavily tested. Even if the environment may not provide the user with the fastest algorithms for his/her problem, in most cases the overall time of implementing and testing an optimal algorithm is exceeding the time for using tested software. This is a particular advantage if a simulation code has relatively short lifetime, for instance when used for a publication, in a thesis. When it comes to solving partial differential equations \esc provides the ad vantage that is especially design for PDEs rather than beeing an add-on to a linear algebra focued system (e.g. MATLAB). The \esc approach gives the user a cleaner 
environment to work with and provides better efficiency when dealing with PDE coefficients such as permeabilty. 
For instance, the user does not need to worry about the data type of permeabilty while implementing the model:
If a model has been tested with constant permeabilty the unchanged code can be run with variable permeabilty set from a data base or as a spatically dependent variable eg. through temperature dependence. This capability of \esc bases on the fact that \esc is 
using the language of PDEs (rather than linear algebra) to describe a model. As it turned out the \esc approach can efficiently be applied even in bigger software projects as it leads to clearer structure for the code by separting modelling issues from low-level numerical and code performance issues but at the same time allows implementing complex model coupling on a higher-level. Moreover,
the usage of python as development platform for \esc greatly simplifies the development of models from a user prospectiv as 
python is intuitive and easy to learn even for users with little experience in program and on the other hand provides direct 
access to a very large number of tools making python an attractive tool for experienced programmers.    

\section{How to use the Cookbook}
This manual is written with the intention to give new users a practical introduction for using \esc to solve a given problem. It is recommended that a new user works through 
the introduction chapter~\ref{CHAP INTRO} and the first chapter~\ref{CHAP HEAT DIFF} which present the neccessary basic knowledge and a set of simple examples which cover basic expertise for using \esc Other chapters
will cover aspects users hit when solving problems with intention to provide ready to use solutions.   
All of the \esc scripts associated with this document are available from the examples folder.


%%%%%%%%%%%%%%%%%%%%%%%%%%%%%%%%%%%%%%%%%%%%%%%%%%%%%%%%
%
% Copyright (c) 2003-2009 by University of Queensland
% Earth Systems Science Computational Center (ESSCC)
% http://www.uq.edu.au/esscc
%
% Primary Business: Queensland, Australia
% Licensed under the Open Software License version 3.0
% http://www.opensource.org/licenses/osl-3.0.php
%
%%%%%%%%%%%%%%%%%%%%%%%%%%%%%%%%%%%%%%%%%%%%%%%%%%%%%%%%

\section{Quickstart}
For information on how to install and run \esc please look at the installation and users guides which are available for download from launchpad at  \url{https://launchpad.net/escript-finley/+download}.

%%%%%%%%%%%%%%%%%%%%%%%%%%%%%%%%%%%%%%%%%%%%%%%%%%%%%%%%%%%%%%%%%%%%%%%%%%%%%%
% Copyright (c) 2003-2016 by The University of Queensland
% http://www.uq.edu.au
%
% Primary Business: Queensland, Australia
% Licensed under the Apache License, version 2.0
% http://www.apache.org/licenses/LICENSE-2.0
%
% Development until 2012 by Earth Systems Science Computational Center (ESSCC)
% Development 2012-2013 by School of Earth Sciences
% Development from 2014 by Centre for Geoscience Computing (GeoComp)
%
%%%%%%%%%%%%%%%%%%%%%%%%%%%%%%%%%%%%%%%%%%%%%%%%%%%%%%%%%%%%%%%%%%%%%%%%%%%%%%

\section{Escript and Python Basics} \label{sec:escpybas}

The \pyt scripting language is a powerful and easy to learn environment with a wide variety of applications. \esc has been developed as a packaged module for \pyt specifically to solve complex partial differential equations. As a result, all the conventions and programming syntax associated with \pyt are coherent with \esc. If you are unfamiliar with \pyt, there are a large number of simple to advanced guides and tutorials available online. These texts should provide an introduction that is comprehensive enough to use \esc. A handful of \pyt tutorials are listed below.
\begin{itemize}
\item \url{http://hetland.org/writing/instant-python.html} is a very crisp introduction. It covers everything you need to get started with \esc.
\item A nice and easy to follow introduction: \url{http://www.sthurlow.com/python/}
\item Another crisp tutorial: \url{http://www.zetcode.com/tutorials/pythontutorial/}. 
 \item A very comprehensive tutorial from the \pyt authors: \url{http://www.python.org/doc/2.5.2/tut/tut.html}. It covers much more than what you will ever need for \esc.
\item Another comprehensive tutorial: \url{http://www.tutorialspoint.com/python/index.htm}
\end{itemize} 

\subsection{The \modesys Modules}
\esc is part of the \esys package. 
Apart from the particle simulation library
\verb|ESyS-Particle|\footnote{see \url{https://launchpad.net/esys-particle}} which is not covered
in this tutorial \esys also includes the following modules
\begin{enumerate}
\item \modescript is the PDE solving module.
\item \modfinley is the discretisation tool and finite element package.
\item \modpycad  is a package for creating irregular shaped domains.
\end{enumerate}
Further explanations of each of these are available in the \esc user guide or in the API documentation\footnote{Available from \url{https://launchpad.net/escript-finley/+download}}. 
\esc is also dependent on a few other open-source packages which are not maintained by the \esc development team. These are \modnumpy (an array and matrix handling package), \modmpl \footnote{\modnumpy and \modmpl are part of the SciPy package, see \url{http://www.scipy.org/}} (a simple plotting tool) and \verb gmsh \footnote{See \url{http://www.geuz.org/gmsh/}} (which is required by \modpycad). These packages (\textbf{except} for \verb gmsh ) are included with the support bundles. 

 

%%%%%%%%%%%%%%%%%%%%%%%%%%%%%%%%%%%%%%%%%%%%%%%%%%%%%%%%%%%%%%%%%%%%%%%%%%%%%%
% Copyright (c) 2003-2012 by University of Queensland
% http://www.uq.edu.au
%
% Primary Business: Queensland, Australia
% Licensed under the Open Software License version 3.0
% http://www.opensource.org/licenses/osl-3.0.php
%
% Development until 2012 by Earth Systems Science Computational Center (ESSCC)
% Development since 2012 by School of Earth Sciences
%
%%%%%%%%%%%%%%%%%%%%%%%%%%%%%%%%%%%%%%%%%%%%%%%%%%%%%%%%%%%%%%%%%%%%%%%%%%%%%%

\chapter{The Einstein Summation Convention}

The Einstein Summation Convention (ESC) is a notational convention that is prefered by the \esc developers. It is a condensed and practical way to deal with multi-dimensional and convoluted PDEs. By suppressing the need to write out the many terms of each problem it is possible to increase efficiency and reduce the number of errors created through poor working. According to the convention, when an index variable appears twice in a single term, it implies that we are summing over all of its possible values.
So we have;
\begin{equation}
a_{1}\frac{\partial^2 f}{\partial x_{1}^2} + a_{2}\frac{\partial^2 f}{\partial x_{2}^2} = a_{i}\frac{\partial^2 f}{\partial x_{i}^2}
\end{equation}

For a scalar function $f(x_{1},x_{2},..x_{i})$ and a vector $\mathbf{u}(u_{1},u_{2},..u_{i})$ with $u_{i}(x_{1},x_{2},..x_{i})$, we have the following notation:
\begin{equation}
\mathbf{u}=\sum_{i}u_{i}e^i = u_{i}e^i
\end{equation}
\begin{equation}
\mathbf{grad}(f) = \mathbf{\nabla}(f) = \sum_{i}\frac{\partial f}{\partial x_{i}}e^i = (\partial_{i} f)e^i = f_{,i}e^i
\end{equation}
\begin{equation}
div(\mathbf{u}) = \mathbf{\nabla}.\mathbf{u} = \sum_{i}\frac{\partial u_{i}}{\partial x_{i}} = \partial_{i} u_{i} = u_{i,i}
\end{equation}
\begin{equation}
div(\mathbf{grad}(f)) = \nabla^2 f = \Delta f = \sum_{i}\frac{\partial^2 f}{\partial x_{i}^2} = f_{,ii}
\end{equation}


%%%%%%%%%%%%%%%%%%%%%%%%%%%%%%%%%%%%%%%%%%%%%%%%%%%%%%%%
%
% Copyright (c) 2003-2009 by University of Queensland
% Earth Systems Science Computational Center (ESSCC)
% http://www.uq.edu.au/esscc
%
% Primary Business: Queensland, Australia
% Licensed under the Open Software License version 3.0
% http://www.opensource.org/licenses/osl-3.0.php
%
%%%%%%%%%%%%%%%%%%%%%%%%%%%%%%%%%%%%%%%%%%%%%%%%%%%%%%%%

\section{An introduction to Partial Differential Equations}
Maybe not assuming enough prior knowledge from the reader here, this section may be redundant as a result.
