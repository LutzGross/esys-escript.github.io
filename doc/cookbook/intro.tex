
%%%%%%%%%%%%%%%%%%%%%%%%%%%%%%%%%%%%%%%%%%%%%%%%%%%%%%%%
%
% Copyright (c) 2003-2009 by University of Queensland
% Earth Systems Science Computational Center (ESSCC)
% http://www.uq.edu.au/esscc
%
% Primary Business: Queensland, Australia
% Licensed under the Open Software License version 3.0
% http://www.opensource.org/licenses/osl-3.0.php
%
%%%%%%%%%%%%%%%%%%%%%%%%%%%%%%%%%%%%%%%%%%%%%%%%%%%%%%%%

\chapter{Introduction}
\label{CHAP INTRO}
\section{Why \esc?}
\esc is an environment for mathematical modelling based on partial differential equations (PDEs). It provides a high-level of abstraction from the underlying numerical schemes (e.g. finite elements (FEM)) and their implementations (e.g.. from aspects of parallelization) so the user can concentrate more on the modelling aspects of their problem while still properly utilising the powerful mathematical capabilities of PDEs. \esc is built upon the interpretive programming language python\footnote{see \url{www.python.org} }. Python is a basic scripting language with many underlying functions. There are also  a large number of software tools for python which can be run in conjunction with \esc; these include packages for linear algebra, visualization, image processing, data plotting, and many others. Any scripts written for \esc is scalable with the ability to run on desktop computers right through to supercomputers\footnote{\esc supports distributed memory architectures with multi-core processors through MPI and threading.} with no modifications to the scripts. 

There are many benefits to using a software platform like \esc for your mathematical modelling project. Using an existing environment such as \esc rather than starting from scratch saves software development time and solves fundamental numerical problems such as selecting appropriate data structures and establishing numerical algorithms. \esc has already solved these problems and their implementation has been heavily tested for bugs. While a developed environment may not provide the user with the fastest algorithms for their problem, it is generally the case that, the overall time of implementing and testing an optimal algorithm will exceed the time needed to use already proven software. This is particularly true if a simulation does not need to be executed repetitively or has relatively short lifetime. A model for publication or thesis would be one such instance. 

When it comes to solving partial differential equations \esc provides an advantage as it is especially design for PDEs rather than being an add-on to a linear algebra focused system (e.g. MATLAB). The \esc approach gives the user a cleaner environment to work with and provides better efficiency when dealing with PDE coefficients such as permeability. Data structure in \esc allow the user to abstract them selves from problems like the data type of these coefficients. If a model has been tested with constant permeability the unchanged code can be run with variable permeability set from a data base or as a spatially dependent variable like temperature dependence. This capability of \esc is possible because \esc uses the language of PDEs (as opposed to linear algebra) to describe a model. As it turns out, the \esc approach can efficiently be applied in very large software projects as it leads to a clearer structure for the code by separating modelling issues from low-level numerical and code performance issues but at the same time allows implementing complex model coupling on a higher-level. Moreover, the usage of python as development platform for \esc greatly simplifies the development of models from a user prospective as python is intuitive and easy to learn even for users with little experience in programming and on the other hand provides direct access to a very large number of tools making python an attractive tool for experienced programmers.    

Best of all, \esc is released under an open software license and is available freely for download.

\section{How to use this Cookbook}
This manual is written with the intention of giving new users a practical introduction to \esc by solving a variety of simple to advanced problems. \esc has many modules and dependencies along with some subtleties that need to be massaged to get you where you want to go. It is recommended that new users work through the \textit{Introduction} (\refCh{CHAP INTRO}) and the \textit{first two sets of examples} (\refCh{CHAP HEAT DIFF} and \refCh{CHAP HEAT 2}) which present the necessary basic knowledge and explain some of the more common aspects and modules of \esc. The examples are simple but they give reference to implementation of PDEs, data structures, how to create models and visualising the solution.

Future chapters will cover more advanced topics and more complex models.

The examples covered in this cookbook have all been scripted and are ready to run. They are available from the \exf folder in the escript directory. The scripts provide a basis for users to develop their own models while at the same time demonstrating the steps required to completely solve and visualise a PDE model.


%%%%%%%%%%%%%%%%%%%%%%%%%%%%%%%%%%%%%%%%%%%%%%%%%%%%%%%%
%
% Copyright (c) 2003-2009 by University of Queensland
% Earth Systems Science Computational Center (ESSCC)
% http://www.uq.edu.au/esscc
%
% Primary Business: Queensland, Australia
% Licensed under the Open Software License version 3.0
% http://www.opensource.org/licenses/osl-3.0.php
%
%%%%%%%%%%%%%%%%%%%%%%%%%%%%%%%%%%%%%%%%%%%%%%%%%%%%%%%%

\section{Quickstart}
For information on how to install and run \esc please look at the installation and users guides which are available for download from launchpad at  \url{https://launchpad.net/escript-finley/+download}.

%%%%%%%%%%%%%%%%%%%%%%%%%%%%%%%%%%%%%%%%%%%%%%%%%%%%%%%%
%
% Copyright (c) 2003-2009 by University of Queensland
% Earth Systems Science Computational Center (ESSCC)
% http://www.uq.edu.au/esscc
%
% Primary Business: Queensland, Australia
% Licensed under the Open Software License version 3.0
% http://www.opensource.org/licenses/osl-3.0.php
%
%%%%%%%%%%%%%%%%%%%%%%%%%%%%%%%%%%%%%%%%%%%%%%%%%%%%%%%%

\section{Escript and Python Basics}

\begin{enumerate}
 \item Basic Commands for both.
 \item Important functions.
 \item Library tree!!! (Where to find stuff!!)
 \item os commands for file operations load save etc
 \item information on plotting depending on path chosen,
\end{enumerate}


