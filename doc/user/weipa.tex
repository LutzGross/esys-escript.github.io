
%%%%%%%%%%%%%%%%%%%%%%%%%%%%%%%%%%%%%%%%%%%%%%%%%%%%%%%%
%
% Copyright (c) 2003-2010 by University of Queensland
% Earth Systems Science Computational Center (ESSCC)
% http://www.uq.edu.au/esscc
%
% Primary Business: Queensland, Australia
% Licensed under the Open Software License version 3.0
% http://www.opensource.org/licenses/osl-3.0.php
%
%%%%%%%%%%%%%%%%%%%%%%%%%%%%%%%%%%%%%%%%%%%%%%%%%%%%%%%%

\chapter{The \weipa Module and Data Visualization}\label{chap:weipa}
%\declaremodule{extension}{weipa}
%\modulesynopsis{Exporting escript data and domains for post-processing}

The {\it weipa} C++ library and accompanying \PYTHON module allow exporting
\escript \Data objects and their domain in a format suitable for visualization.
Besides creating output files, {\it weipa} can also interface with the \VisIt
visualization software. This allows accessing the latest simulation data while
the simulation is still running without the need to save any files.

\section{The \class{EscriptDataset} class}
\begin{classdesc}{EscriptDataset}{}
    holds an {\it escript} dataset including a domain and data variables
    for a single time step and offers methods to export the data in various
    formats.
    It is not recommended to create a dataset object directly in \PYTHON using
    the constructor (it is not actually exposed) but use the
    \var{createDataset} function from \weipa instead, see \Sec{sec:weipafuncs}.
\end{classdesc}

\noindent The following methods are available:
\begin{methoddesc}[EscriptDataset]{setDomain}{domain}
    sets the \Domain for this dataset. Note that the domain can only be set
    once and all \Data objects added to this dataset must be defined on the
    same domain.
\end{methoddesc}

\begin{methoddesc}[EscriptDataset]{addData}{data, name \optional{, units=""}}
    adds the \Data object \var{data} to this dataset which will be exported by
    the given \var{name}. Some export formats support data units which can be
    set through the \var{units} parameter, e.g. \code{"km/h"}.
    Before calling this method a domain must be set with \member{setDomain}
    and all \Data objects added must be defined on the same domain.
    There is no restriction, however, on the \FunctionSpace used.
\end{methoddesc}

\begin{methoddesc}[EscriptDataset]{setCycleAndTime}{cycle, time}
    sets the cycle and time values for this dataset.
    The cycle is an integer value which usually corresponds with the loop
    counter of the simulation script. That is, every time a new data file is
    created this counter is incremented.
    The value of \var{time} on the other hand is a floating point number that
    encodes some form of simulation time.
    Both, cycle and time may be read by analysis tools and shown alongside
    other metadata to the user.
\end{methoddesc}

\begin{methoddesc}[EscriptDataset]{setMeshLabels}{x, y \optional{, z=""}}
    sets the labels of the X, Y, and Z axis. By default, visualization tools
    display default strings such as "X-Axis" or "X" along the axes. Some export
    formats allow overriding these with more specific strings such as "Width",
    "Horizontal Distance", etc.
\end{methoddesc}

\begin{methoddesc}[EscriptDataset]{setMeshUnits}{x, y \optional{, z=""}}
    sets the units to be displayed along the X, Y, and Z axis in visualization
    tools (if supported). Not all export formats will use these values.
\end{methoddesc}

\begin{methoddesc}[EscriptDataset]{setMetadataSchemaString}{\optional{schema="" \optional{, metadata=""}}}
    adds custom metadata and/or XML schema strings to VTK files.
    The content of \var{schema} is added to the top-level \emph{VTKFile}
    element so care must be taken to keep the resulting file valid.
    As an example, \var{schema} may contain the string
    \code{xmlns:gml="http://www.opengis.net/gml"}. The content of \var{metadata}
    will be written enclosed in \code{<MetaData>} tags. Thus, a valid example
    would be \code{<dataSource>something</dataSource>}.
    Note that these values are ignored by other exporters.
\end{methoddesc}

\begin{methoddesc}[EscriptDataset]{saveSilo}{filename}
    saves the dataset in the \SILO file format to a file named \var{filename}.
    The file extension \code{.silo} will be automatically added if not present.
\end{methoddesc}

\begin{methoddesc}[EscriptDataset]{saveVTK}{filename}
    saves the dataset in the \VTK file format to a file named \var{filename}.
    The file extension \code{.vtu} will be automatically added if not present.
    Certain combinations of function spaces cannot be written to a single \VTK
    file due to format restrictions. In these cases this method will save
    separate files where each file contains compatible data.
    The function space name is appended to the filename to distinguish them.
\end{methoddesc}

\section{Functions}\label{sec:weipafuncs}
\begin{funcdesc}{createDataset}{domain, **data}
    creates an \class{EscriptDataset} object, sets its domain, populates it
    with the given \Data objects and returns it.
    Note that it is not possible to set units for the data variables added with
    this function. If this is required, it is recommended to call this function
    with a domain only and use the \member{addData} method subsequently.
\end{funcdesc}

\begin{funcdesc}{saveVTK}{filename \optional{, domain=None \optional{, metadata="" \optional{, metadata_schema=None}}}, **data}
    convenience function that creates a dataset with the given domain and \Data
    objects and saves it to a file in the \VTK file format.
    If \var{domain} is \code{None} the domain will be determined by the \Data
    objects.
    See the \member{setDomain}, \member{addData}, \member{saveVTK}, and
    \member{setMetadataSchemaString} methods of the \class{EscriptDataset}
    class for details.
    Unlike the class method, the \var{metadata_schema} parameter should be a
    dictionary that maps namespace name to URI, e.g.\\
    \code{\{"gml":"http://www.opengis.net/gml"\}}.
\end{funcdesc}

\begin{funcdesc}{saveSilo}{filename \optional{, domain=None}, **data}
    convenience function that creates a dataset with the given domain and \Data
    objects and saves it to a file in the \SILO file format.
    If \var{domain} is \code{None} the domain will be determined by the \Data
    objects.
    See the \member{setDomain}, \member{addData}, and \member{saveSilo}
    methods of the \class{EscriptDataset} class for details.
\end{funcdesc}

\begin{funcdesc}{visitInitialize}{simFile \optional{, comment=""}}
    initializes the \VisIt simulation interface which is responsible for the
    communication with a \VisIt client.
    This function will create a file by the name given via \var{simFile}
    (extension \code{.sim2}) which can be loaded by a compatible \VisIt client
    in order to connect to the simulation. The optional \var{comment} string
    is forwarded to the client.
    Note that this function only succeeds if {\it escript} was compiled with
    support for \VisIt and the appropriate libraries are found in the runtime
    environment. Clients wanting to connect can only do so if the version
    number matches the version number used to compile \weipa.
    Calling this function does not make any data available yet, see the
    \var{visitPublishData} function.
\end{funcdesc}

\begin{funcdesc}{visitPublishData}{dataset}
    publishes an \class{EscriptDataset} object through the \VisIt simulation
    interface, checks for client requests and handles any outstanding ones.
    Before publishing any data, the \var{visitInitialize} function must be
    called to set up the interface.
    Since this function not only publishes new data but polls for incoming
    connections and handles requests, it should be called as often as practical
    (even with the same dataset) to avoid timeout errors from clients.
    On the other hand it should be noted that the same process(es) deal with
    visualization requests that run your simulation. So a request for an
    expensive task by a \VisIt client will pause the simulation code while it
    is being processed.
\end{funcdesc}

