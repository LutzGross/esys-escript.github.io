
%%%%%%%%%%%%%%%%%%%%%%%%%%%%%%%%%%%%%%%%%%%%%%%%%%%%%%%%%%%%%%%%%%%%%%%%%%%%%%
% Copyright (c) 2003-2014 by University of Queensland
% http://www.uq.edu.au
%
% Primary Business: Queensland, Australia
% Licensed under the Open Software License version 3.0
% http://www.opensource.org/licenses/osl-3.0.php
%
% Development until 2012 by Earth Systems Science Computational Center (ESSCC)
% Development 2012-2013 by School of Earth Sciences
% Development from 2014 by Centre for Geoscience Computing (GeoComp)
%
%%%%%%%%%%%%%%%%%%%%%%%%%%%%%%%%%%%%%%%%%%%%%%%%%%%%%%%%%%%%%%%%%%%%%%%%%%%%%%

\chapter{The \ripley Module}\label{CHAPTER ON RIPLEY}
%\declaremodule{extension}{ripley}
%\modulesynopsis{Solving linear, steady partial differential equations using finite elements}

\ripley is a specialised form of \finley, supporting structured, 
uniform meshes in 2D and 3D. Uniform meshes allow a more regular 
division of elements among compute nodes. \ripley also supports fast
assemblers for certain types of PDE (specifically Lame and Wave PDEs).
These assemblers are optimisations of the stiffness matrix assembly process
for these specific problems.

\ripley domains cannot be created by reading from a mesh file.

The family of domain that will result from a 
\class{Rectangle} or \class{Brick} call depends on which is imported in the
specific script. The following line is an example of importing \ripley domains:

\begin{python}
 from esys.ripley import Rectangle, Brick
\end{python}

\section{Formulation}
For a single PDE that has a solution with a single component the linear PDE is
defined in the following form:
\begin{equation}\label{RIPLEY.SINGLE.1}
\begin{array}{cl} &
\displaystyle{
\int_{\Omega}
A_{jl} \cdot v_{,j}u_{,l}+ B_{j} \cdot v_{,j} u+ C_{l} \cdot v u_{,l}+D \cdot vu \; d\Omega }
+ \displaystyle{\int_{\Gamma} d \cdot vu \; d{\Gamma} }\\
= & \displaystyle{\int_{\Omega}  X_{j} \cdot v_{,j}+ Y \cdot v \; d\Omega }
+ \displaystyle{\int_{\Gamma} y \cdot v \; d{\Gamma}}
\end{array}
\end{equation}

\section{Meshes}
\label{RIPLEY MESHES}
An example 2D mesh from \ripley is shown in Figure~\ref{FIG:RIPLEY:MESH}. 
Meshfiles cannot be used to generate \ripley domains.

\begin{figure}
\centerline{\includegraphics{ripley}}
\caption{3x3 ripley Rectangle}
\label{FIG:RIPLEY:MESH}
\end{figure}

\section{Functions}
\begin{funcdesc}{Brick}{n0,n1,n2,l0=1.,l1=1.,l2=1.,d0=-1,d1=-1,d2=-1,
diracPoints=list(), diracTags=list()}
generates a \Domain object representing a three-dimensional brick between
$(0,0,0)$ and $(l0,l1,l2)$ with orthogonal faces. All elements will be regular.
The brick is filled with
\var{n0} elements along the $x_0$-axis,
\var{n1} elements along the $x_1$-axis and
\var{n2} elements along the $x_2$-axis.
If built with \MPI support, the domain will be subdivided 
\var{d0} times along the $x_0$-axis,
\var{d1} times along the $x_1$-axis, and
\var{d2} times along the $x_2$-axis.
\var{d0}, \var{d1}, and \var{d2} must be factors of the number of
\MPI processes requested.
If axial subdivisions are not specified, automatic domain subdivision will take
place. This may not be the most efficient construction and will likely result in
extra elements being added to ensure proper distribution of work. Any extra
elements added in this way will change the length of the domain proportionately.
\var{diracPoints} is a list of coordinate-tuples of points within the mesh,
each point tagged with the respective string within \var{diracTags}.
\end{funcdesc}

\begin{funcdesc}{Rectangle}{n0,n1,n2,l0=1.,l1=1.,l2=1.,d0=-1,d1=-1,d2=-1,
diracPoints=list(), diracTags=list()}
generates a \Domain object representing a two-dimensional rectangle between
$(0,0)$ and $(l0,l1)$ with orthogonal faces. All elements will be regular.
The rectangle is filled with
\var{n0} elements along the $x_0$-axis and
\var{n1} elements along the $x_1$-axis.
If built with \MPI support, the domain will be subdivided 
\var{d0} times along the $x_0$-axis and
\var{d1} times along the $x_1$-axis.
\var{d0} and \var{d1} must be factors of the number of \MPI processes requested.
If axial subdivisions are not specified, automatic domain subdivision will take
place. This may not be the most efficient construction and will likely result in
extra elements being added to ensure proper distribution of work. Any extra
elements added in this way will change the length of the domain proportionately.
\var{diracPoints} is a list of coordinate-tuples of points within the mesh,
each point tagged with the respective string within \var{diracTags}.
\end{funcdesc}

\section{Linear Solvers in \SolverOptions}
Currently direct solvers are not supported under \MPI.
By default, \ripley uses the iterative solvers \PCG for symmetric and \BiCGStab
for non-symmetric problems.
If the direct solver is selected, which can be useful when solving very
ill-posed equations, \ripley uses the \MKL\footnote{If the stiffness matrix is
non-regular \MKL may return without a proper error code. If you observe
suspicious solutions when using \MKL, this may be caused by a non-invertible
operator.} solver package. If \MKL is not available \UMFPACK is used.
If \UMFPACK is not available a suitable iterative solver from \PASO is used.



