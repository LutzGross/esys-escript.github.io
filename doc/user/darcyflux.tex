\section{Darcy Flux}
\label{DARCY FLUX}
We want to calculate the velocity $u$ and pressure $p$ on a domain $\Omega$ solving 
the Darcy flux problem \index{Darcy flux}\index{Darcy flow}
\begin{equation}\label{DARCY PROBLEM}
\begin{array}{rcl}
u_{i} + \kappa_{ij} p_{,j} & = & g_{i} \\
u_{k,k} & = & f
\end{array}
\end{equation} 
with the boundary conditions
\begin{equation}\label{DARCY BOUNDARY}
\begin{array}{rcl}
u_{i} \; n_{i}  = u^{N}_{i}  \; n_{i} & \mbox{ on } & \Gamma_{N} \\
p = p^{D} &  \mbox{ on } & \Gamma_{D} \\ 
\end{array}
\end{equation} 
where $\Gamma_{N}$ and $\Gamma_{D}$ are a partition of the boundary of $\Omega$ with $\Gamma_{D}$ non empty, $n_{i}$ is the outer normal field of the boundary of $\Omega$, $u^{N}_{i}$ and $p^{D}$ are given functions on $\Omega$, $g_{i}$ and $f$ are given source terms and $\kappa_{ij}$ is the given permeability. We assume that $\kappa_{ij}$ is symmetric (which is not really required) and positive definite, i.e there are positive constants $\alpha_{0}$ and $\alpha_{1}$ which are independent from the location in $\Omega$ such that
\begin{equation}
\alpha_{0} \; x_{i} x_{i} \le \kappa_{ij} x_{i} x_{j} \le \alpha_{1} \; x_{i} x_{i}
\end{equation}
for all $x_{i}$. 

\subsection{Solution Method \label{DARCY SOLVE}}
It is useful to write equation~\ref{DARCY PROBLEM} in operator form. For any pressure $p$
we set 
\begin{equation}
(Gp)_{i} =  p_{,j}
\end{equation}
and a velocity $v$ we set
\begin{equation}
Dv = v_{k,k}
\end{equation}
We $K=(\kappa_{ij})$ we can write the Darcy problem~\ref{DARCY PROBLEM} as
\begin{equation}
\begin{array}{rcl}
u + K \, Gp & = & g \\
Du & = & f 
\end{array}
\end{equation} 
We solve this equation by minimising the functional
\begin{equation}\label{DARCY COST}
J(u,p):=\|K^{-\frac{1}{2}}(u + K \, G p - g)\|^2_{0} +  \|\lambda (Du-f) \|_{0}^2 
\end{equation} 
over all suitable $u$ and $p$. In this equation we set $\|p\|^2_{0}=(p,p)_{0}$ with
\begin{equation}
(p,q)_{0} = \int_{\Omega } p\cdot q \, dx
\end{equation} 
The factor $\lambda>0$ is a weighting factor.
A simple calculation shows that
one has to solve
\begin{equation}
( K^{-1} (v + K \, Gq) , u +K \, G p - g)_{0} +  (\lambda Dv,\lambda (Du-f) )_{0} =0 
\end{equation} 
for all velocities $v$ and pressure $q$ which fulfill the homogeneous boundary conditions~\ref{DARCY BOUNDARY}.
This so-called variational equation can be translated back into operator notation
\begin{equation}
\begin{array}{rcl}
(K^{-1}+ D^*\lambda^2 D)u + Gp & = &  D^*\lambda f + K^{-1} g \\
G^*u  + G^*K \, G p & = & G^*g \\ 
\end{array}
\end{equation} 
where $D^*$ and $Q^*$ denote the  adjoint operators with respect to $(.,.)_{0}$. 
In~\cite{LEASTSQUARESFEM1994} it has been shown that this problem is continuous and coercive and therefore has a unique solution. Also standard FEM methods can be used for discretization. It is also possible 
to solve the problem in coupled form, however this approach leads in some cases to a very ill-conditioned stiffness matrix in particular in the case of a very small or large permeability ($\alpha_{1} \ll 1$ or $\alpha_{0} \gg 1$)  

The approach we are taking is to eliminate the velocity $v$ from the problem. Assuming that $p$ is known we have
\begin{equation}\label{DARCY V FORM}
v= (K^{-1}+ D^*\lambda^2 D)^{-1} ( D^*\lambda f + K^{-1} g - Gp)
\end{equation} 
(notice that $K^{-1}+\lambda D^*D$ is coercive) which is inserted into the second equation
\begin{equation}
G^* (K^{-1}+\lambda D^*D)^{-1} (\lambda D^*f + K^{-1} g - Gp) + G^* KG p = G^*g 
\end{equation} 
which is 
\begin{equation}
G^* ( K - (K^{-1}+ D^*\lambda^2 D)^{-1} ) G p = G^* (g-(K^{-1}+D^*\lambda^2 D)^{-1} ( D^*\lambda f + K^{-1} g) ) 
\end{equation} 
We use the PCG \index{linear solver!PCG}\index{PCG} method to solve this.
The residual $r$  is given as
\begin{equation}
\begin{array}{rcl}
r & =&  G^* \left( g - K\, G p - v \right)
\end{array}
\end{equation} 
for the current pressure approximation $p$ and current velocity $v$ defined by  
equation~\ref{DARCY V FORM}.
So in a particular implementation we use $\hat{r}=g-K\, Gp-v$ to represent the residual. 
The evaluation of the iteration operator for a given $p$ is then 
returning $Qp+v$ where $v$ is the solution of 
\begin{equation}\label{UPDATE W}
(K^{-1}+ D^*\lambda^2 D)v = Gp
\end{equation}
To derive a preconditioner we use the identity
\begin{equation}
 \begin{array}{rcl}

G^* ( K - (K^{-1}+ D^*\lambda^2 D)^{-1} ) G  & = & G^* (I - (I + K D^*\lambda^2 D)^{-1}) K G \\
                               & \approx &  G^* (K D^*\lambda^2 D) K G  \\
			       & =  & G^*  K ( D^*  \lambda^2 D) K G \\
			       & \approx  & G^* \frac{\lambda^2}{dx^2} K^2 G
\end{array}
\end{equation} 
where $dx$ is the local mesh size and we use the approximation
\begin{equation}
D^*  \lambda^2 D \approx \frac{\lambda^2}{dx^2}
\end{equation} 
Therefore we use $G^* \frac{\lambda^2}{dx^2} K^2 G$ as a preconditioner.To evalute the preconditioner
we need to solve the equation
\begin{equation}\label{UPDATE P}
 G^* \frac{\lambda^2}{dx^2} K^2 G p =  G^* \hat{r}
\end{equation}
It remains to answer the question how to choose $\lambda$. We need to balance the first and second
term in $J(u,p)$ in equation~\ref{DARCY COST}. We inspect $J$ for 
$(\hat{u}, \hat{p})$ which is a perturbed exact solution $(u,p)$.   
Assuming $\hat{u}=u+u_{0}e^{ik^tx}$
and $\hat{p}=p+p_{0}e^{ik^tx}$ and constant $K$ we get 
\begin{equation}
J(\hat{u},\hat{p}) = C \left[ ( \|K^{-1}\|_{2} |u_{0}|^2 + \|K\|_{2} \|k\|_{2}^2 |p_{0}|^2| )
+ \lambda^2 \|k\|_{2}^2 |u_{0}|^2  \right]
\end{equation} 
with some constant $C>0$. The first two terms and the third term correspond to the first term and
second term in the definition of $J(u,p)$ in equation~\ref{DARCY COST}. For small $\|k\|_{2}$
(i.e. for a smooth perturbation) $J(\hat{u},\hat{p})$ is dominated by $\|K^{-1}\|_{2} |u_{0}|^2$.
To scale the second term which is corresponds to the incompressibility condition for the velocity
we need to meet the condition $\|K^{-1}\|_{2} =  \lambda^2 \|k\|_{2}^2$. 
Taking the boundary conditions into consideration the smallest possible value for $\|k\|_{2} = \frac{\pi}{l}$ where $l$ is the longest edge of the domain. This leads to the 
\begin{equation}\label{DARCY LAMBDA}
\lambda = \|K^{-1}\|_{2}^{\frac{1}{2}} \frac{l}{\pi}
\end{equation} 
Notice that with this setting the preconditioner $G^* \frac{\lambda^2}{dx^2} K^2 G$ becomes 
equivalent to $G^* K G$ if $K$ is a diagonal matrix and the mesh has a constant mesh size.

The residual norm used in the PCG is given as
\begin{equation}\label{DARCY R NORM}
\|r\|_{PCG}^2 = \int r (G^* \frac{\lambda^2}{dx^2} K^2 G)^{-1} r \; dx =\int \hat{r}  G ( G^* \frac{\lambda^2}{dx^2} K^2 G)^{-1}  G^* \hat{r} \; dx \approx
\int \hat{r} K^{-1}  \hat{r} \; dx 
\end{equation}
The iteration is terminated if 
\begin{equation}\label{DARCY STOP}
\|r\|_{PCG} \le \mbox{ATOL}
\end{equation}
where we set 
\begin{equation}\label{DARCY ATOL DEF}
\mbox{ATOL} = \mbox{atol} + \mbox{rtol} \cdot \left(\frac{1}{\|K^{-\frac{1}{2}}v\|_{0}} + \frac{1}{\|K^{\frac{1}{2}} G p\|_{0}} \right)^{-1} 
\end{equation}
where rtol is a given relative tolerance and $\mbox{atol}$ is a given absolute tolerance (typically $=0$).  
Notice that if $Gp$ and $v$ both are zero, the pair $(0,p)$ is a solution.
The problem is that ATOL is depending on the solution $p$ and $v$ calculated form~\ref{DARCY V FORM}. 
In practice one use the initial guess for $p$ 
to get a first value for ATOL. If the stopping criterion is met in the PCG iteration, a new $v$ is calculated from the current pressure approximation and ATOL is recalculated. If \ref{DARCY STOP} is still fulfilled the calculation is terminated and $(v,p)$ is returned. Otherwise PCG is restarted with a new ATOL.

\subsection{Functions}
\begin{classdesc}{DarcyFlow}{domain}
opens the Darcy flux problem\index{Darcy flux} on the \Domain domain.
\end{classdesc}

\begin{methoddesc}[DarcyFlow]{setValue}{\optional{f=None, \optional{g=None, \optional{location_of_fixed_pressure=None, \optional{location_of_fixed_flux=None, 
\\\optional{permeability=None}}}}}}
assigns values to the model parameters. Values can be assigned using various calls - in particular 
in a time dependent problem only values that change over time needs to be reset. The permeability can be defined as scalar (isotropic), a vector (orthotropic) or a matrix (anisotropic). 
\var{f} and \var{g} are the corresponding parameters in~\ref{DARCY PROBLEM}.
The locations and components where the flux is prescribed are set by positive values in
\var{location_of_fixed_flux}. 
The locations where the pressure is prescribed are set by 
by positive values of \var{location_of_fixed_pressure}. 
The values of the pressure and flux are defined by the initial guess.
Notice that at any point on the boundary of the domain the pressure or the normal component of
the flux must be defined. There must be at least one point where the pressure is prescribed. 
The method will try to cast the given values to appropriate 
\Data class objects.
\end{methoddesc}

\begin{methoddesc}[DarcyFlow]{setTolerance}{\optional{rtol=1e-4}}
sets the relative tolerance \mbox{rtol} in \ref{DARCY ATOL DEF}.
\end{methoddesc}

\begin{methoddesc}[DarcyFlow]{setAbsoluteTolerance}{\optional{atol=0.}}
sets the absolute tolerance \mbox{atol} in \ref{DARCY ATOL DEF}.
\end{methoddesc}

\begin{methoddesc}[DarcyFlow]{getSolverOptionsFlux}{}
Returns the solver options used to solve the flux problems~(\ref{DARCY V FORM}) and~(\ref{UPDATE W}). Use the returned \SolverOptions object to control the solution algorithms. 
\end{methoddesc}

\begin{methoddesc}[DarcyFlow]{getSolverOptionsPressure}{}
Returns the solver options used to solve the pressure problems~(\ref{UPDATE P}) as a preconditioner. 
Use the returned \SolverOptions object to control the solution algorithms. 
\end{methoddesc}

\begin{methoddesc}[DarcyFlow]{solve}{u0,p0, \optional{max_iter=100, \optional{verbose=False}}}
solves the problem. and returns approximations for the flux $v$ and the pressure $p$. 
\var{u0} and \var{p0} define initial guess for flux and pressure. Values marked
by positive values \var{location_of_fixed_flux} and \var{location_of_fixed_pressure}, respectively, are kept unchanged. \var{max_iter} sets the maximum number of iterations steps allowed for solving the coupled problem.
\end{methoddesc}


\subsection{Example: Gravity Flow}
later