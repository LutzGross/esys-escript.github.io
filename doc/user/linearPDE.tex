% $Id$

\chapter{The module \linearPDEs}

\declaremodule{extension}{linearPDEs} \modulesynopsis{Linear partial pifferential equation handler}
The module \linearPDEs provides an interface to define and solve linear partial 
differential equations within \escript. \linearPDEs does not provide any 
solver capabilities in itself but hands the PDE over to 
the PDE solver library defined through the \Domain of the PDE.
The general interface is provided through the \LinearPDE class. The
\AdvectivePDE which is derived from the \LinearPDE class
provides an interface to PDE dominated by its advective terms. The \Poisson
class which is also derived form the \LinearPDE class should be used
to define the Poisson equation \index{Poisson}.  

\section{\LinearPDE Class}
\label{SEC LinearPDE}

The \LinearPDE class is used to define a general linear, steady, second order PDE
for an unknown function $u$ on a given $\Omega$ defined through a \Domain object.
In the following $\Gamma$ denotes the boundary of the domain $\Omega$. $n$ denotes
the outer normal field on $\Gamma$. 

For a single PDE with a solution with a single component the linear PDE is defined in the 
following form:
\begin{equation}\label{LINEARPDE.SINGLE.1}
-(A\hackscore{jl} u\hackscore{,l}){,j}+(B\hackscore{j} u)\hackscore{,j}+C\hackscore{l} u\hackscore{,l}+D u =-X\hackscore{j,j}+Y \; .
\end{equation}
$u_{,j}$ denotes the derivative of $u$ with respect to the $j$-th spatial direction. Einstein's summation convention, ie. summation over indexes appearing twice in a term of a sum is performed, is used. 
The coefficients $A$, $B$, $C$, $D$, $X$ and $Y$ have to be specified through \Data objects in the 
\Function on the PDE or objects that can be converted into such \Data objects. 
$A$ is a \RankTwo, $B$, $C$ and $X$ are \RankOne and $D$ and $Y$ are scalar. 
The following natural
boundary conditions are considered \index{boundary condition!natural} on $\Gamma$:
\begin{equation}\label{LINEARPDE.SINGLE.2}
n\hackscore{j}(A\hackscore{jl} u\hackscore{,l}+B\hackscore{j} u)+d u=n\hackscore{j}X\hackscore{j} + y  \;.
\end{equation}
Notice that the coefficients $A$, $B$ and $X$ are defined in the PDE. The coefficients $d$ and $y$ are  
each a \Scalar in the \FunctionOnBoundary.  Constraints \index{constraint} for the solution prescribing the value of the 
solution at certain locations in the domain. They have the form
\begin{equation}\label{LINEARPDE.SINGLE.3}
u=r \mbox{ where } q>0
\end{equation}
$r$ and $q$ are each \Scalar where $q$ is the characteristic function
\index{characteristic function} defining where the constraint is applied.
The constraints defined by \eqn{LINEARPDE.SINGLE.3} override any other condition set by \eqn{LINEARPDE.SINGLE.1}
or \eqn{LINEARPDE.SINGLE.2}. The PDE is symmetrical \index{symmetrical} if
\begin{equation}\label{LINEARPDE.SINGLE.4}
A\hackscore{jl}=A\hackscore{lj} \mbox{ and } B\hackscore{j}=C\hackscore{j}
\end{equation}
For a system of PDEs and a solution with several components the PDE has the form
\begin{equation}\label{LINEARPDE.SYSTEM.1}
-(A\hackscore{ijkl} u\hackscore{k,l}){,j}+(B\hackscore{ijk} u_k)\hackscore{,j}+C\hackscore{ikl} u\hackscore{k,l}+D\hackscore{ik} u_k =-X\hackscore{ij,j}+Y\hackscore{i} \; .
\end{equation}
$A$ is a \RankFour, $B$ and $C$ are each a \RankThree, $D$ and $X$ are each a \RankTwo and $Y$ is a \RankOne. 
The natural boundary conditions \index{boundary condition!natural} take the form:
\begin{equation}\label{LINEARPDE.SYSTEM.2}
n\hackscore{j}(A\hackscore{ijkl} u\hackscore{k,l}){,j}+(B\hackscore{ijk} u_k)+d\hackscore{ik} u_k=n\hackscore{j}-X\hackscore{ij}+y\hackscore{i}  \;.
\end{equation}
The coefficient $d$ is a \RankTwo and $y$ is a  
\RankOne both in the \FunctionOnBoundary. Constraints \index{constraint} take the form
\begin{equation}\label{LINEARPDE.SYSTEM.3}
u\hackscore{i}=r\hackscore{i} \mbox{ where } q\hackscore{i}>0
\end{equation}
$r$ and $q$ are each \RankOne. Notice that at some locations not necessarily all components must 
have a constraint. The system of PDEs is symmetrical \index{symmetrical} if
\begin{eqnarray}\label{LINEARPDE.SYSTEM.4}
A\hackscore{ijkl}=A\hackscore{klij} \\
B\hackscore{ijk}=C\hackscore{kij} \\
D\hackscore{ik}=D\hackscore{ki} \\
d\hackscore{ik}=d\hackscore{ki} \
\end{eqnarray}
\LinearPDE also supports solution discontinuities \index{discontinuity} over contact region $\Gamma^{contact}$
in the domain $\Omega$. To specify the conditions across the discontinuity we are using the
generalised flux $J$ which is in the case of a systems of PDEs and several components of the solution
defined as 
\begin{equation}\label{LINEARPDE.SYSTEM.5}
J\hackscore{ij}=A\hackscore{ijkl}u\hackscore{k,l}+B\hackscore{ijk}u\hackscore{k}-X\hackscore{ij}
\end{equation}
For the case of single solution component and single PDE $J$ is defined
\begin{equation}\label{LINEARPDE.SINGLE.5}
J\hackscore{j}=A\hackscore{jl}u\hackscore{,l}+B\hackscore{j}u\hackscore{k}-X\hackscore{j}
\end{equation}
In the context of discontinuities \index{discontinuity} $n$ denotes the normal on the 
discontinuity pointing from side 0 towards side 1. For a system of PDEs
the contact condition takes the form
\begin{equation}\label{LINEARPDE.SYSTEM.6}
n\hackscore{j} J^{0}\hackscore{ij}=n\hackscore{j} J^{1}\hackscore{ij}=y^{contact}\hackscore{i} - d^{contact}\hackscore{ik} [u]\hackscore{k} \; .
\end{equation}
where $J^{0}$ and $J^{1}$ are the fluxes on side $0$ and side $1$ of the
discontinuity $\Gamma^{contact}$, respectively. $[u]$, which is the difference
of the solution at side 1 and at side 0, denotes the jump of $u$ across $\Gamma^{contact}$.
The coefficient $d^{contact}$ is a \RankTwo and $y^{contact}$ is a  
\RankOne both in the \FunctionOnContactZero or \FunctionOnContactOne.
In case of a single PDE and a single component solution the contact condition takes the form
\begin{equation}\label{LINEARPDE.SINGLE.6}
n\hackscore{j} J^{0}\hackscore{j}=n\hackscore{j} J^{1}\hackscore{j}=y^{contact} - d^{contact}[u]
\end{equation}
In this case the the coefficient $d^{contact}$ and $y^{contact}$ are eaach \Scalar
both in the \FunctionOnContactZero or \FunctionOnContactOne.
 
\begin{classdesc}{LinearPDE}{domain,numEquations=0,numSolutions=0}
opens a linear, steady, second order PDE on the \Domain \var{domain}. \var{numEquations}
and \var{numSolutions} gives the number of equations and the number of solutiopn components.
If \var{numEquations} and \var{numSolutions} is non-positive, the number of equations 
and the number solutions, respctively, stay undefined until a coefficient is
defined. 
\end{classdesc}

\begin{methoddesc}[LinearPDE]{setValues}{arg1,arg2,...,argN}
assigns new values to coefficients \var{arg1}, \var{arg2}, $\ldots$ \var{argN}. 
The coefficients must be one of the valid coefficient names
\var{A},
\var{B},
\var{C},
\var{D},
\var{X},
\var{Y},
\var{d},
\var{y},
\var{dcontact},
\var{ycontact},
\var{r},
or \var{q}.
If the coefficient is not a \Data object, it is converted into a \Data object in the
appropriate \FunctionSpace.
\end{methoddesc}

\begin{methoddesc}[LinearPDE]{createCoefficient}{name}
facilitates the creation of a \Data object corresponding to coefficient of name
\var{name}. If \var{name} is not a valid name an exception is raised. The
returned \Data object is initialised to zero and is not set as a value of the
LinearPDE.
\end{methoddesc}

\begin{methoddesc}[LinearPDE]{getCoefficient}{name}
return the value assigned to coefficient \var{name}. If \var{name} is not a valid name 
an exception is raised. 
\end{methoddesc}

\begin{methoddesc}[LinearPDE]{cleanCoefficients}{}
resets all coefficients to their initialization values. This method is useful to call when trying to save memory
as is releaces eqnerences to coefficients but keeping the differential operator.
\end{methoddesc}
\begin{methoddesc}[LinearPDE]{createNewCoefficient}{name}
returns a \Data object which has the \FunctionSpace and \Shape of coefficient \var{name}. The initial value is $0$.
\end{methoddesc}

\begin{methoddesc}[LinearPDE]{getShapeOfCoefficient}{name}
returns the shape of coefficient \var{name} even if no value has been assigned to it.
\end{methoddesc}


\begin{methoddesc}[LinearPDE]{getFunctionSpaceOfCoefficient}{name}
returns the \FunctionSpace of coefficient \var{name} even if no value has been assigned to it.
\end{methoddesc}

\begin{methoddesc}[LinearPDE]{hasCoefficient}{name}
returns \True if \var{name} is valid name of a coefficient
\end{methoddesc}


\begin{methoddesc}[LinearPDE]{getFunctionSpaceForEquation}{}
returns \FunctionSpace of the right hand side which is identical to \FunctionSpace of 
the result of the PDE operator.
\end{methoddesc}

\begin{methoddesc}[LinearPDE]{getFunctionSpaceForSolution}{}
returns \FunctionSpace of the solution of the PDE hich is identical to \FunctionSpace of 
the result of argument of the PDE operator.
\end{methoddesc}

\begin{methoddesc}[LinearPDE]{setDebugOn}{}
switches the debug mode to on.
\end{methoddesc}

\begin{methoddesc}[LinearPDE]{setDebugOff}{}
switches the debug mode to on.
\end{methoddesc}

\begin{methoddesc}[LinearPDE]{debug}{}
returns \True if the debug mode is switched on. Otherwise it return \False.
\end{methoddesc}

\begin{methoddesc}[LinearPDE]{setLumpingOn}{}
switches on lumping \index{lumping}. If lumping is switched on 
the operator is 
\end{methoddesc}


\begin{methoddesc}[LinearPDE]{setLumpingOff}
switches lumping off \index{lumping}.
\end{methoddesc}

\begin{methoddesc}[LinearPDE]{setLumping}{flag=\False}
switches on lumping if \var{flag} is \True. Otherwise lumping is swiched off.
\end{methoddesc}

\begin{methoddesc}[LinearPDE]{isUsingLumping}{}
returns \True if lumping is switched on. Otherwise \False is returned.
\end{methoddesc}

\begin{memberdesc}[LinearPDE]{DEFAULT_METHOD}
default method to be used to solve the PDE. An appropriate method should be 
chosen by the used PDE solver library.
\end{memberdesc}

\begin{memberdesc}[LinearPDE]{DIRECT}
direct linear solver~\Ref{SAAD} 
\end{memberdesc}

\begin{memberdesc}[LinearPDE]{CHOLEVSKY}
direct solver based on Cholevsky factorization (or similar), see~\Ref{SAAD}. The solver will require a symmetric PDE. 
\end{memberdesc}

\begin{memberdesc}[LinearPDE]{PCG}
preconditioned conjugate gradient method, see~\Ref{WEISS}. The solver will require a symmetric PDE.
\end{memberdesc}

\begin{memberdesc}[LinearPDE]{GMRES}
the GMRES method, see~\Ref{WEISS}. Truncation and restart ar econtrolled by the parameters
\var{truncation} and \var{restart} of \method{getSolution}.
\end{memberdesc}

\begin{memberdesc}[LinearPDE]{PRES20}
the GMRES method with trunction after five residuals and
restart after 20 steps, see~\Ref{WEISS}.
\end{memberdesc}

\begin{memberdesc}[LinearPDE]{CR}
conjugate residual method, see~\Ref{WEISS}. 
\end{memberdesc}

\begin{memberdesc}[LinearPDE]{CGS}
conjugate gradient squared method, see~\Ref{WEISS}.
\end{memberdesc}

\begin{memberdesc}[LinearPDE]{BICGSTAB}
stabilzed bi-conjugate gradients methods, see~\Ref{WEISS}. 
\end{memberdesc}

\begin{memberdesc}[LinearPDE]{SSOR}
symmetric successive overrelaxtion method, see~\Ref{WEISS}.
\end{memberdesc}

\begin{methoddesc}[LinearPDE]{setSolverMethod}{solver=linearPDE.DEFAULT_METHOD}
sets the solver method to be used. It is pointed out that the PDE solver library
does not know the specified solver method but may choose a similar method. 
\end{methoddesc}

\begin{methoddesc}[LinearPDE]{setTolerance}{tol=1.e-8}
resets the tolerance for solution. The actually meaning of tolerance is
depending on the underlying PDE library. In most cases, the tolerance 
will only consider the error from solving the discerete problem but will
not consider any discretization error.
\end{methoddesc}

\begin{methoddesc}[LinearPDE]{getTolerance}{}
returns the current tolerance of the solution
\end{methoddesc}

\begin{methoddesc}[LinearPDE]{isSymmetric}{}
returns \True if the PDE has been indicated to be symmetric.
Otherwise \False is returned.
\end{methoddesc}

\begin{methoddesc}[LinearPDE]{setSymmetryOn}{}
indicates that the PDE is symmetric.
\end{methoddesc}

\begin{methoddesc}[LinearPDE]{setSymmetryOff}{}
indicates that the PDE is not symmetric.
\end{methoddesc}

\begin{methoddesc}[LinearPDE]{setSymmetryTo}{flag=\False}
indicates that the PDE is symmetric if \var{flag}=\True
and indicates a non-symmetric PDE is \var{flag}=\False. 
\end{methoddesc}

\begin{methoddesc}[LinearPDE]{setReducedOrderOn}{}
switches on the reduction of polynomial order for the solution and 
equation evaluation.
\end{methoddesc}

\begin{methoddesc}[LinearPDE]{setReducedOrderOff}{}
switches off the reduction of polynomial order for the solution and 
equation evaluation.
\end{methoddesc}

\begin{methoddesc}[LinearPDE]{setReducedOrderTo}{flag=\False}
switches on the reduction of polynomial order for the solution and 
equation evaluation if \var{flag}=\True. Otherwise
the order reduction is switched off.
\end{methoddesc}

\begin{methoddesc}[LinearPDE]{setReducedOrderForSolutionOn}{}
switches on reduction of polynomial order for the solution.
\end{methoddesc}

\begin{methoddesc}[LinearPDE]{setReducedOrderForSolutionOff}{}
switches off reduction of polynomial order for the solution.
\end{methoddesc}

\begin{methoddesc}[LinearPDE]{setReducedOrderForSolutionTo}{flag=\False}
switches on the reduction of polynomial order for the solution 
if \var{flag}=\True. Otherwise
the order reduction is switched off.
\end{methoddesc}

\begin{methoddesc}[LinearPDE]{setReducedOrderForEquationOn}{}
switches on reduction of polynomial order for the equation evaluation.
\end{methoddesc}

\begin{methoddesc}[LinearPDE]{setReducedOrderForEquationOff}{}
switches off reduction of polynomial order for the equation evaluation.
\end{methoddesc}

\begin{methoddesc}[LinearPDE]{setReducedOrderForEquationTo}{flag=\False}
switches on the reduction of polynomial order for the equation 
evaluation if \var{flag}=\True. Otherwise
the order reduction is switched off.
\end{methoddesc}

\begin{methoddesc}[LinearPDE]{getOperator}{}
returns the \Operator of the PDE.
\end{methoddesc}

\begin{methoddesc}[LinearPDE]{getRightHandSide}{ignoreConstraint=\False}
returns the right hand side of the PDE as a \Data object. If
\var{ignoreConstraint}=\True the constraints are not considered
when building up the right hand side.
\end{methoddesc}

\begin{methoddesc}[LinearPDE]{getSystem}{}
returns the \Operator and right hand side of the PDE.
\end{methoddesc}

\begin{methoddesc}[LinearPDE]{getSolution}{option1,option2,...,optionN}
returns (an approximation of) the solution of the PDE. \var{options1},  \var{options2} 
$\ldots$ \var{optionsN} are options handed over to the underlying PDE solver library.  
\end{methoddesc}

\begin{methoddesc}[LinearPDE]{getDomain}{}
returns the \Domain of the PDE.
\end{methoddesc}

\begin{methoddesc}[LinearPDE]{getDim}{}
returns the spatial dimension of the PDE.
\end{methoddesc}

\begin{methoddesc}[LinearPDE]{getNumEquations}{}
returns the number of equations.
\end{methoddesc}

\begin{methoddesc}[LinearPDE]{getNumSolutions}{}
returns the number of components of the solution.
\end{methoddesc}

\begin{methoddesc}[LinearPDE]{checkSymmetry}{verbose=\False}
returns \True if the PDE is symmetric and \False otherwise. 
The method is very computational expensive and should only be 
called for testing purposes. The symmetry flag is not altered.
If \var{verbose}=\True information about where symmetry is violated
are printed.
\end{methoddesc}

\begin{methoddesc}[LinearPDE]{getFlux}{u}
returns the flux $J\hackscore{ij}$ \index{flux} for given solution \var{u}
defined by \eqn{LINEARPDE.SYSTEM.5} and \eqn{LINEARPDE.SINGLE.5}, respectively.
\end{methoddesc}

\begin{methoddesc}[LinearPDE]{applyOperator}{u}
applies the PDE operator to \var{u}
\end{methoddesc}

\begin{methoddesc}[LinearPDE]{getResidual}{u}
returns the residual when insering \var{u} into the PDE
\end{methoddesc}

\section{\AdvectivePDE Class}
In cases of PDEs dominated by the advection terms $B$ and $C$ against the diffusion term $A$
up-winding has been used. 
The \AdvectivePDE class applies upwinding to the advective terms, see \Ref{SUPG}. 

In the following we set
\begin{eqnarray}\label{LINEARPDE.UPWIND.Z}
Z\hackscore{j}=C\hackscore{j}-B\hackscore{j}
\mbox{ or } \\
Z\hackscore{ikl}=C\hackscore{ikl}-B\hackscore{ilk}
\end{eqnarray}
To measure the dominance of the advective terms over the diffusive term $A$ the 
Pelclet number is used \index{Pelclet number}. The are defined as
\begin{eqnarray}\label{LINEARPDE.Peclet.single}
P=\frac{h\|Z\hackscore{:}\|}{2\|A\hackscore{::}\|}
\end{eqnarray}
\begin{eqnarray}\label{LINEARPDE.Peclet.system}
P\hackscore{ik}=\frac{h\|Z\hackscore{i:k}\|}{2\|A\hackscore{i:k:}\|}
\end{eqnarray}
where $h$ is the local cell size and
\begin{eqnarray}\label{LINEARPDE.ADVECTIVE.1b}
\|Z\hackscore{:}\|^2=Z\hackscore{j}Z\hackscore{j} \\
\|A\hackscore{::}\|^2=A\hackscore{jl}A\hackscore{jl} \\
\|Z\hackscore{i:k}\|^2=\delta\hackscore{in} \delta\hackscore{km} Z\hackscore{njm}Z\hackscore{njm} \\
\|A\hackscore{i:k:}\|^2=\delta\hackscore{in} \delta\hackscore{km} A\hackscore{njml}A\hackscore{njml} \; .
\end{eqnarray}
In the case that it is $\|A\hackscore{::}\|$ we set $P=0$ if $\|Z\hackscore{:}\|=0$ and
$P=\infinity$ if $\|Z\hackscore{:}\|=0$. Analogously for $P$ in the case of a systems of PDEs.

From the Pelclet number the stabilization parameters $\Xi$ and $\Xi$ are calculated:
\begin{eqnarray}\label{LINEARPDE.Peclet.2}
\Xi=\xi(P) \frac{h}{\|Z\hackscore{:}\|}  \\
\mbox{ or } \\
\Xi\hackscore{ik}=\xi(P\hackscore{ik}) \frac{h}{\|Z\hackscore{i:k}\|}
\end{eqnarray}
where $\xi$ is a suitable function of the Peclet number.
In the case of a single PDE the coefficient are up-dated in the following way:
\begin{eqnarray}\label{LINEARPDE.ADVECTIVE.1}
A\hackscore{jl} \leftarrow A\hackscore{jl} + \Xi Z\hackscore{j} Z\hackscore{l} \\
B\hackscore{j} \leftarrow B\hackscore{j} + \Xi C\hackscore{j} D \\
C\hackscore{j} \leftarrow C\hackscore{j} + \Xi B\hackscore{j} D \\
X\hackscore{j} \leftarrow X\hackscore{j} + \Xi Z\hackscore{j} Y \\
\end{eqnarray}
Similar for the case of a systems of PDEs:
\begin{eqnarray}\label{LINEARPDE.ADVECTIVE.SYSTEM}
A\hackscore{ijkl} \leftarrow A\hackscore{ijl} + \delta\hackscore{pm} \Xi\hackscore{mi} Z\hackscore{pij} Z\hackscore{mkl} \\
B\hackscore{ijk} \leftarrow B\hackscore{ijk} +  \delta\hackscore{pm} \Xi\hackscore{mi} D\hackscore{pk} C\hackscore{mij} \\
C\hackscore{ikl} \leftarrow C\hackscore{ikl} +  \delta\hackscore{pm} \Xi\hackscore{mi} D\hackscore{pk} B\hackscore{mli} \\
X\hackscore{ij} \leftarrow X\hackscore{ij} + \delta\hackscore{pm} \Xi\hackscore{mi}  Y\hackscore{p} Z\hackscore{mij}\\
\end{eqnarray}
Using upwinding in this form, introduces an additonal error which is proprtional to the cell size $h$
but with the intension to stabilize the solution.

\begin{classdesc}{AdvectivePDE}{domain,numEquations=0,numSolutions=0,xi=AdvectivePDE.ELMAN_RAMAGE}
opens a linear, steady, second order PDE on the \Domain \var{domain}. \var{numEquations}
and \var{numSolutions} gives the number of equations and the number of solutiopn components.
If \var{numEquations} and \var{numSolutions} is non-positive, the number of equations 
and the number solutions, respectively, stay undefined until a coefficient is
defined. \var{xi} defines a function which returns for any given  Preclet number $P\ge 0$ the 
$\xi$-value used to define the stabilization parameters $\Xi$ and $\Xi$.
\AdvectivePDE is derived from \LinearPDE.
\end{classdesc}

\begin{memberdesc}[AdvectivePDE]{SIMPLIFIED_BROOKS_HUGHES}{}
Predefined function to set a values for $\xi$ from a Preclet number $P$. 
This function uses the method suggested in \Ref{SUPG1}
where $\xi(P)$ is given by 
\begin{equation}\label{LINEARPDE.ADVECTIVE.1d}
\xi(P)=coth(P)-\frac{1}{P} \;.
\end{equation}
As the evaluation of $coth$ is expensive we are using the approximation:
The function $\xi$ is approximated by
\begin{equation}\label{LINEARPDE.ADVECTIVE.23}
\xi(P)=
\left\{ 
\begin{array}{lc}
\frac{P}{6} & P<3 \\
\frac{1}{2} & \mbox{otherwise}
\end{array}
\right.
\end{equation}
\end{memberdesc}

\begin{memberdesc}[AdvectivePDE]{ELMAN_RAMAGE}{}
Predefined function to set a values for $\xi$ from a Preclet number $P$. 
This function uses the method suggested in \Ref{SUPG2}
where $\xi(P)$ is given by 
\begin{equation}\label{LINEARPDE.ADVECTIVE.23b}
\xi(P)=
\left\{ 
\begin{array}{lc}
0  & P<1 \\
\frac{1}{2}(1-\frac{1}{P}) & \mbox{otherwise}
\end{array}
\right.
\end{equation}
\end{memberdesc}


\section{The \Poisson Class}

The \Poisson class provides an easy way to define and solve the Poisson
equation
\begin{equation}\label{POISSON.1}
-u\hackscore{,ii}=f\; .
\end{equation}
with homogeneous boundary conditions
\begin{equation}\label{POISSON.2}
n\hackscore{i}u\hackscore{,i}=0
\end{equation}
and homogeneous constraints
\begin{equation}\label{POISSON.3}
u=0 \mbox{ where } q>0
\end{equation}
$f$ has to be a \Scalar in the \Function and $q$ must be
a \Scalar in  the \SolutionFS. 

\begin{classdesc}{Poisson}{domain}
opens a Poisson equation on the \Domain domain. \Poisson is derived from \LinearPDE.
\end{classdesc}
\begin{methoddesc}[Poisson]{setValue}{f=escript.Data(),q=escript.Data()}
assigns new values to \var{f} and \var{q}.
\end{methoddesc}
