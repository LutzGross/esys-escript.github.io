
%%%%%%%%%%%%%%%%%%%%%%%%%%%%%%%%%%%%%%%%%%%%%%%%%%%%%%%%
%
% Copyright (c) 2009-2012 by University of Queensland
% Earth Systems Science Computational Center (ESSCC)
% http://www.uq.edu.au/esscc
%
% Primary Business: Queensland, Australia
% Licensed under the Open Software License version 3.0
% http://www.opensource.org/licenses/osl-3.0.php
%
%%%%%%%%%%%%%%%%%%%%%%%%%%%%%%%%%%%%%%%%%%%%%%%%%%%%%%%%

\chapter{Changes from previous releases}
\label{app:changes}

\subsection*{3.2.1 to 3.3}
\begin{itemize}
 \item Experimental support for \pythonthree.
 \item Parameter order for the table interpolation methods has changed to be consistant. 
 Please test your scripts if you use these functions.
 \item setX on \finley and \dudley domains will now only accept coordinates from ContinuousFunction spaces.
 This is to avoid some potentially nasty behaviour when use periodic boundary conditions.
 You can still use setX, just make sure that you interpolate first.
\end{itemize}


\subsection*{3.1 to 3.2}
\begin{itemize}
\item The deprecated name for the launcher has been removed. To run scripts use
    \emph{run-escript} not \emph{escript}.
\item \escript is no longer automatically imported by importing \finley.
    You will need to import escript explicitly. (All of our example scripts do this anyway.)
\item An experimental version of the new Dudley domain is now available.
\item Various bug fixes and optimisations.
\item New algorithms for gmsh support.
\item Improvements to the AMG solver. AMG is the recommended solver for symmetric problems.
\item Fixed compilation issues using netcdf.
\item Redesigned configuration files to make it easier to compile from source
    without finding the locations of all your libraries.
\item Faster rendering of documentation.
\item Documentation is now hyperlinked.
\item New data export module \weipa. The \texttt{saveVTK} functionality has
    been moved into this module, and while calling \texttt{saveVTK} from the
    \escript module still works it is discouraged and will be removed in a
    future release.
\item New \texttt{esys.escript.DataManager} class for convenient checkpointing
    and exporting of escript data.
\item \VisIt simulation interface for online data access and visualization.
\item Simpler interpolation and support for interpolation from 3D vectors.
\item HRZ lumping has been added and some claification on how to use it.
\item Data objects populated with ``random'' values can be created.
\end{itemize}

\subsection*{3.0 to 3.1}
\begin{itemize}
\item The \emph{escript} launcher has been renamed to \emph{run-escript}. The
    old name will still work in this version but will be removed in the future.
\item Lazy evaluation features have been improved and documented (see \Sec{sec:lazy}).
\item The \emph{escript} documentation now includes a new Cookbook which
    demonstrates how to solve sample problems using escript.
\item Macro elements have been introduced.
\item The \var{saveDataCSV} method allows one or more \Data objects to be
    exported in CSV format (see \Sec{sec:savedatacsv}).
\item \Data objects can be populated by interpolating from values in a table.
\item The new \var{getInfLocator} and \var{getSupLocator} functions in
    \pdetools return Locators to a minimal/maximal point over the
    data.
\item There is a new class to model fault systems (\var{esys.escript.faultsystems.FaultSystem}).
\item A beta version of an Algebraic Multigrid (AMG) solver is included.
\item Inverting square matrices larger than 3x3 is now permitted if escript is
    compiled with Lapack support.
\item If escript is compiled with a modern compiler, then inf/sup/Lsup will now
    report NaN, +/-inf as appropriate if those values appear in the data.
\item Data.setTags will take tag names as well as tag numbers.
\item The Scalar, Vector, Tensor, Tensor3, Tensor4 factory methods can now take
    arrays/nested sequence like objects as their initial values.
\item escript.util.mkDir can now take a list of directories to create.
\item Behind the scenes, \PYTHON docstrings have been rewritten from epydoc to
    restructured text.
\item Various other bug fixes and performance tweaks.
\end{itemize}

\subsection*{2.0 to 3.0}
\begin{itemize}
\item The major change here was replacing \module{numarray} with \numpy.
For general instructions on converting scripts to use numpy see \url{http://www.stsci.edu/resources/software_hardware/numarray/numarray2numpy.pdf}.
The specific changes to \escript are:
 \begin{itemize}
  \item getValueOfDataPoint() which returned a \module{numarray}.array has been
      replaced by getTupleForDataPoint() which returns a \PYTHON tuple
      containing the components of the data point. In the case of matrices or
      higher ranked data, the tuples will be nested.
      Use \numpy.array(data.getTupleForDataPoint()) if a \numpyNDA object is
      required.
  \item getValueOfGlobalDataPoint() has similarly been replaced by
      getTupleForGlobalDataPoint().
  \item \code{integrate(data)} now returns a \numpyNDA instead of a
      \module{numarray}.array.
 \end{itemize}
Any python methods which previously accepted \module{numarray} objects now
accept \numpy objects instead.

\item The way to define solver options for \LinearPDE objects has changed.
    There is now a \SolverOptions object attached to the \LinearPDE
    object which handles the options of solvers used to solve the PDE.
    The following changes apply:  
 \begin{itemize}
  \item The \method{setTolerance} and \method{setAbsoluteTolerance} methods
      have been removed. Instead use \method{setTolerance} and
      \method{setAbsoluteTolerance} on the \SolverOptions object. For example:
      \method{getSolverOptions().setTolerance(\ldots)}
  \item The \method{setSolverPackage} and \method{setSolverMethod} methods have
      been removed. Instead use the methods \method{setPackage},
      \method{setSolverMethod} and \method{setPreconditioner}. For example:
      getSolverOptions().setPackage(\ldots).
  \item The static class variables defining packages, solvers and
      preconditioners have been removed and are now accessed via the
      corresponding static class variables in \SolverOptions.
      For instance use \method{SolverOptions.PCG} instead of
      \method{LinearPDE.PCG} to select the preconditioned conjugate gradient
      method.
  \item The \method{getSolution} now takes no argument. Use the corresponding
      methods of the \SolverOptions object returned by
      \method{getSolverOptions()} to set values,\\ e.g. use
      \method{getSolverOptions().setVerbosityOn()} instead of argument
      \code{verbose=True} and \method{getSolverOptions().setIterMax(1000)}
      instead of argument \code{iter_max=1000}.
 \end{itemize}

\item The \pyvisi module from previous releases has been deprecated and will no
    longer be supported. It is still present in the source code and can be used
    if you compile \escript from source. It will not be available in binary
    releases and its use is discouraged.
    \emph{The documentation for \pyvisi is not available in this release.}
% The documentation for \pyvisi can be found in Appendix~\ref{PYVISI CHAP}.
\end{itemize}

