
%%%%%%%%%%%%%%%%%%%%%%%%%%%%%%%%%%%%%%%%%%%%%%%%%%%%%%%%
%
% Copyright (c) 2009 by University of Queensland
% Earth Systems Science Computational Center (ESSCC)
% http://www.uq.edu.au/esscc
%
% Primary Business: Queensland, Australia
% Licensed under the Open Software License version 3.0
% http://www.opensource.org/licenses/osl-3.0.php
%
%%%%%%%%%%%%%%%%%%%%%%%%%%%%%%%%%%%%%%%%%%%%%%%%%%%%%%%%

\section{Changes from previous releases}
\label{app:changes}

\subsection*{2.0 to 3.0}
The major change here was replacing \module{numarray} with \numpy.
For general instructions on converting scripts to use numpy see \url{http://www.stsci.edu/resources/software_hardware/numarray/numarray2numpy.pdf}.
The specific changes to \escript are:
\begin{itemize}
  \item getValueOfDataPoint() which returned a \module{numarray}.array has been replaced by
 getTupleForDataPoint() which returns a \PYTHON tuple containing
the components of the data point. In the case of matricies or higher ranked data, the tuples will be nested.
 \item getValueOfGlobalDataPoint has similarly been replaced by getTupleForGlobalDataPoint().
 \item integrate(data) now returns a \numpyNDA instead of a \module{numarray}.array.
\end{itemize}
Any python methods which previously accepted \module{numarray} objects will accept \numpy objects instead.

