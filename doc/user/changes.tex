
%%%%%%%%%%%%%%%%%%%%%%%%%%%%%%%%%%%%%%%%%%%%%%%%%%%%%%%%%%%%%%%%%%%%%%%%%%%%%%
% Copyright (c) 2003-2015 by The University of Queensland
% http://www.uq.edu.au
%
% Primary Business: Queensland, Australia
% Licensed under the Open Software License version 3.0
% http://www.opensource.org/licenses/osl-3.0.php
%
% Development until 2012 by Earth Systems Science Computational Center (ESSCC)
% Development 2012-2013 by School of Earth Sciences
% Development from 2014 by Centre for Geoscience Computing (GeoComp)
%
%%%%%%%%%%%%%%%%%%%%%%%%%%%%%%%%%%%%%%%%%%%%%%%%%%%%%%%%%%%%%%%%%%%%%%%%%%%%%%

\chapter{Changes from previous releases}
\label{app:changes}

\subsection*{4.0 to 4.1}
\begin{itemize}
  \item The gmshReader now supports reading with multiple processes
  \item Using the help() function on some domains is now more informative
  \item User guide updated with information on use of Dirac points
  \item Minimizer misfit now available via the callback function
  \item Specifying use of a direct solver without a direct solver being available now raises an exception rather than silently default to a non-direct solver
  \item Synthetic seismic examples for various wave types now included in the examples distributed
  \item Reading NetCDF files when the default value in the file is nan no longer causes issues
  \item Various documentation updates and fixes in user guide, install guide, and code documentation
  \item Various compatibility fixes with Python3
  \item HTI and VTIWave models now support setting a q value for custom boundary conditions
  \item Using a MPI-enabled gmsh with an MPI build of \escript should now reliably wait for gmsh to end
\end{itemize}

\subsection*{3.4.2 to 4.0}
\begin{itemize}
  \item New spectral element domain, \speckley
  \item CUDA-based solvers and block-diagonal system matrices in \ripley
  \item DC Resistivity forward modeling
  \item New build options:
   \begin{itemize}
     \item 'domains': allows only the specified domains to be built
     \item 'prelaunch', 'launcher', 'postlaunch': allows build-time
           customization of escript launcher
     \item 'cuda', 'nvcc', 'nvccflags', 'thrust_prefix': path and flags for
           CUDA compiler and thrust headers
   \end{itemize}
  \item Gmsh reader now \MPI parallelised
  \item cycle and timestamp support for saveSilo and saveVTK
  \item PDEs using \ripley now have per-PDE assemblers to allow domains to have
        multiple PDEs of different types without conflict
  \item General documentation updates, both guides and code commenting
  \item Test framework compatible with older versions of Python
  \item Removed BASHisms from \escript launcher
  \item Data objects now have hasNaN() and replaceNaN() methods
  \item readBinaryGrid() now supports scaling and 64-bit floats in non-native
        byte order
  \item Now compiles with Cray compiler 
  \item Wave models corrected
  \item Equality checks on Data objects now throw exceptions
  \item General changes to help compilers with \OPENMP parallelisation
  \item Intel-based \OPENMP flavours now have proper thread binding
  \item \ripley dirac points no longer get lost in domains with enormous
        elements
  \item lots of improvements in inversion toolbox \downunder and symbolic
        toolbox
\end{itemize}

\subsection*{3.4.1 to 3.4.2}
\begin{itemize}
 \item Changes to SolverOptions: Most of this is under the hood at this point
  but any code which accesses parameters from a SolverOptions object like this:
 \begin{python}
   variable.getSolverOptions().DIRECT
 \end{python}
 will need to be modified to:
 \begin{python}
   SolverOptions.DIRECT
 \end{python}

 \item Data objects can be created using smoothed random data.
 \item Support for OSX10.9
 \item Support for additional problem types in downunder.
 \item More efficient ripley domains for large compute clusters
 \item Added symbolic toolbox and associated documentation
 \item Ripley now supports Dirac delta functions.
 \item Improved reading of netCDF files
 \item Paso is now in its own namespace
 \item Examples have been updated to use Python 3 behaviour (with future imports for Python 2).
 \item Ripley supports reading from gzipped files.
 \item Fixed fault in documentation generation
 \item Fixes to keep clang++ happy
\end{itemize}


\subsection*{3.4 to 3.4.1}
\begin{itemize}
 \item Renamed design.Design to design.AbstractDesign as a more explicit/descriptive name, {\bf this will break any existing custom implementation} until changed to match.

 \item Efficiency improvements in esys.downunder inversions.
 \item Implemented more CF conventions for the netCDF datasource and added GDAL support.
 \item Strongly coupled joint inversion added.
 \item Preliminary work on seismic modelling.
 \item Added support for more input data types in ER Mapper files.
 \item Improved stability for long MPI runs.
 \item Added timeStepFormat option to \texttt{esys.escript.saveESD}.
 \item Document generation process more stable.
 \item Added support for smoothed random 2D data.
 \item Documentation updated.
 \item Improved read support for grid-reading.
 \item Removed last usages of C compiler.
 \item Efficiency improvements in \texttt{esys.downunder}.
 \item Removed support for VSL random (\texttt{boost::random} is sufficient).
 \item Fixes related to compatibility with Intel compiler.
 \item Simplified module structure.
 \item Various code cleanup and bug fixes.
\end{itemize}

\subsection*{3.3.1 to 3.4}
\begin{itemize}
 \item This release does not use the support bundles from $3.0$ so building from source will be required in more cases.
 \item In Debian and Ubuntu, there are now two packages. One containing escript itself and the other containing documentation.
This fits better with the way Debian does things.
 \item esys.weipa can now write Voxet files for data on ripley.
 \item All support for OpenDX has been dropped.
 \item Improved documentation HTML API documentation.
 \item Improvements to inversion module. 
\end{itemize}


\subsection*{3.3 to 3.3.1}
\begin{itemize}
 \item Source packages for Debian and Ubuntu can now be made directly by dpkg-source from the source tree\footnote{For this to work, you will need to place a file called \texttt{svn_version} containing the revision number in the root of the source.}.
 \item non-uniform 1D interpolation (see Section~\ref{sec:interpolation}).
 \item The minimum version of python required for escript is now $2.6$.
 \item saveVTK and saveDX (previously deprecated) have been removed from the main escript module.
Please see the \module{weipa} documentation for export functionality.
 \item The \module{downunder} inversion module is included.
\begin{itemize}
 \item The documentation includes an inversion cookbook to get you started.
 \item Most operations are possible with \escript's existing dependencies but some \emph{may} require \module{pyproj} or \module{gdal}
  but \escript will let you know if that happens.
\end{itemize}
 
\end{itemize}


\subsection*{3.2.1 to 3.3}
\begin{itemize}
 \item Experimental support for \pythonthree.
 \item Parameter order for the table interpolation methods has changed to be consistent. 
 Please test your scripts if you use these functions.
 \item setX on \finley and \dudley domains will now only accept coordinates from ContinuousFunction spaces.
 This is to avoid some potentially nasty behaviour when using periodic boundary conditions.
 You can still use setX, just make sure that you interpolate first.
\end{itemize}


\subsection*{3.1 to 3.2}
\begin{itemize}
\item The deprecated name for the launcher has been removed. To run scripts use
    \emph{run-escript} not \emph{escript}.
\item \escript is no longer automatically imported by importing \finley.
    You will need to import escript explicitly. (All of our example scripts do this anyway.)
\item An experimental version of the new Dudley domain is now available.
\item Various bug fixes and optimisations.
\item New algorithms for gmsh support.
\item Improvements to the AMG solver. AMG is the recommended solver for symmetric problems.
\item Fixed compilation issues using netcdf.
\item Redesigned configuration files to make it easier to compile from source
    without finding the locations of all your libraries.
\item Faster rendering of documentation.
\item Documentation is now hyperlinked.
\item New data export module \weipa. The \texttt{saveVTK} functionality has
    been moved into this module, and while calling \texttt{saveVTK} from the
    \escript module still works it is discouraged and will be removed in a
    future release.
\item New \texttt{esys.escript.DataManager} class for convenient checkpointing
    and exporting of escript data.
\item \VisIt simulation interface for online data access and visualization.
\item Simpler interpolation and support for interpolation from 3D vectors.
\item HRZ lumping has been added and some clarification on how to use it.
\item Data objects populated with ``random'' values can be created.
\end{itemize}

\subsection*{3.0 to 3.1}
\begin{itemize}
\item The \emph{escript} launcher has been renamed to \emph{run-escript}. The
    old name will still work in this version but will be removed in the future.
\item Lazy evaluation features have been improved and documented (see \Sec{sec:lazy}).
\item The \emph{escript} documentation now includes a new Cookbook which
    demonstrates how to solve sample problems using escript.
\item Macro elements have been introduced.
\item The \var{saveDataCSV} method allows one or more \Data objects to be
    exported in CSV format (see \Sec{sec:savedatacsv}).
\item \Data objects can be populated by interpolating from values in a table.
\item The new \var{getInfLocator} and \var{getSupLocator} functions in
    \pdetools return Locators to a minimal/maximal point over the
    data.
\item There is a new class to model fault systems (\var{esys.escript.faultsystems.FaultSystem}).
\item A beta version of an Algebraic Multigrid (AMG) solver is included.
\item Inverting square matrices larger than 3x3 is now permitted if escript is
    compiled with Lapack support.
\item If escript is compiled with a modern compiler, then inf/sup/Lsup will now
    report NaN, +/-inf as appropriate if those values appear in the data.
\item Data.setTags will take tag names as well as tag numbers.
\item The Scalar, Vector, Tensor, Tensor3, Tensor4 factory methods can now take
    arrays/nested sequence like objects as their initial values.
\item escript.util.mkDir can now take a list of directories to create.
\item Behind the scenes, \PYTHON docstrings have been rewritten from epydoc to
    restructured text.
\item Various other bug fixes and performance tweaks.
\end{itemize}

\subsection*{2.0 to 3.0}
\begin{itemize}
\item The major change here was replacing \module{numarray} with \numpy.
For general instructions on converting scripts to use numpy see \url{http://www.stsci.edu/resources/software_hardware/numarray/numarray2numpy.pdf}.
The specific changes to \escript are:
 \begin{itemize}
  \item getValueOfDataPoint() which returned a \module{numarray}.array has been
      replaced by getTupleForDataPoint() which returns a \PYTHON tuple
      containing the components of the data point. In the case of matrices or
      higher ranked data, the tuples will be nested.
      Use \numpy.array(data.getTupleForDataPoint()) if a \numpyNDA object is
      required.
  \item getValueOfGlobalDataPoint() has similarly been replaced by
      getTupleForGlobalDataPoint().
  \item \code{integrate(data)} now returns a \numpyNDA instead of a
      \module{numarray}.array.
 \end{itemize}
Any python methods which previously accepted \module{numarray} objects now
accept \numpy objects instead.

\item The way to define solver options for \LinearPDE objects has changed.
    There is now a \SolverOptions object attached to the \LinearPDE
    object which handles the options of solvers used to solve the PDE.
    The following changes apply:  
 \begin{itemize}
  \item The \method{setTolerance} and \method{setAbsoluteTolerance} methods
      have been removed. Instead use \method{setTolerance} and
      \method{setAbsoluteTolerance} on the \SolverOptions object. For example:
      \method{getSolverOptions().setTolerance(\ldots)}
  \item The \method{setSolverPackage} and \method{setSolverMethod} methods have
      been removed. Instead use the methods \method{setPackage},
      \method{setSolverMethod} and \method{setPreconditioner}. For example:
      getSolverOptions().setPackage(\ldots).
  \item The static class variables defining packages, solvers and
      preconditioners have been removed and are now accessed via the
      corresponding static class variables in \SolverOptions.
      For instance use \method{SolverOptions.PCG} instead of
      \method{LinearPDE.PCG} to select the preconditioned conjugate gradient
      method.
  \item The \method{getSolution} now takes no argument. Use the corresponding
      methods of the \SolverOptions object returned by
      \method{getSolverOptions()} to set values,\\ e.g. use
      \method{getSolverOptions().setVerbosityOn()} instead of argument
      \code{verbose=True} and \method{getSolverOptions().setIterMax(1000)}
      instead of argument \code{iter_max=1000}.
 \end{itemize}
\end{itemize}

