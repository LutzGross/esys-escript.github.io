% $Id: admin.tex,v 1.4 2005/06/24 00:27:34 paultcochrane Exp $

\documentclass[12pt,a4paper]{article}

\begin{document}

\section{Epydoc}

Epydoc is used to build the API documentation of pyvisi.  It strips out
relevant information from the python code and the docstrings in the python
code to generate a set of web pages documenting the different classes and
functions defined.

The command used to run epydoc is:
\begin{verbatim}
epydoc --html -o doc/api_epydoc -n pyvisi pyvisi
\end{verbatim}

\section{A-A-P}

The tool, \texttt{aap} is used to build the website html files from their base
(.part) files, and to publish them to the website.  To build or refresh the
html source from the .part files, use:
\begin{verbatim}
aap
\end{verbatim}
To send the files to the website (known as ``publishing'' in aap parlance)
use:
\begin{verbatim}
aap publish
\end{verbatim}

\section{Todo list}

To add a new todo item, use the \texttt{devtodo} (\texttt{todo}) command.
Then, to produce the TODO list, use \texttt{todo -T}, and commit the changes
to the repository.

\section{Pylint}

Pylint is a handy tool to check the quality of the code.  It is like the old
lint program for C, and gives a lot of checking and helpful output.  To
generate html output from pylint use the following command from the pyvisi
main directory:
\begin{verbatim}
pylint --html=y pyvisi > pylint.pyvisi.html
\end{verbatim}

\end{document}
