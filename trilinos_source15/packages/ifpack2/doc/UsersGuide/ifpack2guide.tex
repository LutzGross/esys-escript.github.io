%
% $Id: SANDExampleReportNotstrict.tex,v 1.26 2009-05-01 20:59:19 rolf Exp $
%
% This is an example LaTeX file which uses the SANDreport class file.
% It shows how a SAND report should be formatted, what sections and
% elements it should contain, and how to use the SANDreport class.
% It uses the LaTeX report class, but not the strict option.
%
% Get the latest version of the class file and more at
%    http://www.cs.sandia.gov/~rolf/SANDreport
%
% This file and the SANDreport.cls file are based on information
% contained in "Guide to Preparing {SAND} Reports", Sand98-0730, edited
% by Tamara K. Locke, and the newer "Guide to Preparing SAND Reports and
% Other Communication Products", SAND2002-2068P.
% Please send corrections and suggestions for improvements to
% Rolf Riesen, Org. 9223, MS 1110, rolf@cs.sandia.gov
%
\documentclass[pdf,12pt,report]{SANDreport}
\usepackage{algpseudocode}
\usepackage{amsthm}
\usepackage{booktabs}
\usepackage{calc}
\usepackage{eso-pic}
\usepackage{fancyhdr}
\usepackage{ifthen}
\usepackage{indentfirst}
\usepackage{geometry}
\usepackage{graphicx}
\usepackage[colorlinks, bookmarksopen, %pagebackref=true, backref=page,
             linkcolor={blue},
             anchorcolor={black},
             citecolor={blue},
             filecolor={magenta},
             menucolor={blue},
             pagecolor={red},
             plainpages=false,pdfpagelabels,
             pdfauthor={Andrey Prokopenko, Chris Siefert, Jonathan J. Hu, Mark
             Hoemmen, Alicia Klinvex},
             pdftitle={Ifpack2 User's Guide},
             pdfkeywords={Ifpack2,preconditioners,guide,user},
             urlcolor={blue}]{hyperref}
\usepackage{listings}
\usepackage{mathptmx}	% Use the Postscript Times font
\usepackage{multirow}
\usepackage{pifont}
\usepackage[FIGBOTCAP,normal,bf,tight]{subfigure}
\usepackage{tabularx}
\usepackage{verbatim}
\usepackage{xspace}
\usepackage{flowchart} % also loads tikz
\usepackage{algorithm}
\usetikzlibrary{arrows}

%\usepackage{draftwatermark}
%\SetWatermarkScale{.5}

\algrenewcommand{\algorithmiccomment}[1]{\hskip3em // #1}

%%%%%%%%%%%%%%%%%%%%%%%%%%%%%%%%%%%%%%%%%%%%%%%%%%%%%%%%%%%%%%%%%%%%%%%%%%%%%%%%%%%%%%%%%%%%%%%%%%%%%%%%%%%%%%%%%%%%%
% Want larger todonotes on margins?
% First, use package showframes to show the frames
% Then, adjust the geometry
% NOTE: this must be removed in the final version
% \usepackage{showframe}
% \setlength{\marginparwidth}{3.5cm}

% Add disable to todonotes options to disable all TODO notes without removing them
% \usepackage[colorinlistoftodos,prependcaption,textsize=small]{todonotes}

% \usepackage{xargs}
% \usepackage{soul}
% \newcommandx{\fix}     [3][1=]{\todo[linecolor=red,backgroundcolor=red!25,bordercolor=red,#1]{\textbf{#2: }#3}}
% \newcommandx{\unsure}  [3][1=]{\todo[linecolor=green,backgroundcolor=green!25,bordercolor=green,#1]{\textbf{#2: }#3}}
% \newcommandx{\improve} [3][1=]{\todo[linecolor=blue,backgroundcolor=blue!25,bordercolor=blue,#1]{\textbf{#2: }#3}}
% \newcommandx{\info}    [3][1=]{\todo[linecolor=gray,backgroundcolor=gray!25,bordercolor=gray,#1]{\textbf{#2: }#3}}
% \newcommandx{\fixhl}   [2]    {\texthl{#1}\fix{#2}}
%%%%%%%%%%%%%%%%%%%%%%%%%%%%%%%%%%%%%%%%%%%%%%%%%%%%%%%%%%%%%%%%%%%%%%%%%%%%%%%%%%%%%%%%%%%%%%%%%%%%%%%%%%%%%%%%%%%%%


% If you want to relax some of the SAND98-0730 requirements, use the "relax"
% option. It adds spaces and boldface in the table of contents, and does not
% force the page layout sizes.
% e.g. \documentclass[relax,12pt]{SANDreport}
%
% You can also use the "strict" option, which applies even more of the
% SAND98-0730 guidelines. It gets rid of section numbers which are often
% useful; e.g. \documentclass[strict]{SANDreport}



% ---------------------------------------------------------------------------- %
%
% Set the title, author, and date
%
\title{Ifpack2 User's Guide 1.0 \\
(Trilinos version 12.6)}

\author{
  Andrey Prokopenko \\
  Scalable Algorithms \\
  Sandia National Laboratories\\
  Mailstop 1318 \\
  P.O.~Box 5800 \\
  Albuquerque, NM 87185-1318\\
  aprokop@sandia.gov\\
  \and
  Christopher Siefert \\
  Computational Math \& Algorithms \\
  Sandia National Laboratories\\
  Mailstop 1318 \\
  P.O.~Box 5800 \\
  Albuquerque, NM 87185-1318 \\
  \and
  Jonathan J. Hu \\
  Scalable Algorithms \\
  Sandia National Laboratories\\
  Mailstop 9159 \\
  P.O.~Box 0969 \\
  Livermore, CA 94551-0969\\
  jhu@sandia.gov \\
  \and
  Mark Hoemmen \\
  Scalable Algorithms \\
  Sandia National Laboratories\\
  Mailstop 1320 \\
  P.O.~Box 5800 \\
  Albuquerque, NM 87185-1318\\
  mhoemme@sandia.gov\\
  \and
  Alicia Klinvex \\
  Scalable Algorithms \\
  Sandia National Laboratories\\
  Mailstop 1320 \\
  P.O.~Box 5800 \\
  Albuquerque, NM 87185-1318\\
  amklinv@sandia.gov\\
}

% There is a "Printed" date on the title page of a SAND report, so
% the generic \date should generally be empty.
\date{}

\newcommand{\amesos}       {\textsc{Amesos}\xspace}
\newcommand{\amesostwo}    {\textsc{Amesos2}\xspace}
\newcommand{\anasazi}      {\textsc{Anasazi}\xspace}
\newcommand{\aztecoo}      {\textsc{AztecOO}\xspace}
\newcommand{\belos}        {\textsc{Belos}\xspace}
\newcommand{\epetra}       {\textsc{Epetra}\xspace}
\newcommand{\epetraext}    {\textsc{EpetraExt}\xspace}
\newcommand{\galeri}       {\textsc{Galeri}\xspace}
\newcommand{\ifpack}       {\textsc{Ifpack}\xspace}
\newcommand{\ifpacktwo}    {\textsc{Ifpack2}\xspace}
\newcommand{\isorropia}    {\textsc{Isorropia}\xspace}
\newcommand{\loca}         {\textsc{Loca}\xspace}
\newcommand{\ml}           {\textsc{ML}\xspace}
\newcommand{\muelu}        {\textsc{MueLu}\xspace}
\newcommand{\nox}          {\textsc{NOX}\xspace}
\newcommand{\stratimikos}  {\textsc{Stratimikos}\xspace}
\newcommand{\teuchos}      {\textsc{Teuchos}\xspace}
\newcommand{\teko}         {\textsc{Teko}\xspace}
\newcommand{\thyra}        {\textsc{Thyra}\xspace}
\newcommand{\tpetra}       {\textsc{Tpetra}\xspace}
\newcommand{\trilinos}     {\textsc{Trilinos}\xspace}
\newcommand{\xpetra}       {\textsc{Xpetra}\xspace}
\newcommand{\zoltan}       {\textsc{Zoltan}\xspace}
\newcommand{\zoltantwo}    {\textsc{Zoltan2}\xspace}


\newcommand{\klu}          {\textsc{Klu}\xspace}
\newcommand{\metis}        {\textsc{Metis}\xspace}
\newcommand{\mumps}        {\textsc{Mumps}\xspace}
\newcommand{\umfpack}      {\textsc{Umfpack}\xspace}
\newcommand{\superlu}      {\textsc{SuperLU}\xspace}
\newcommand{\superludist}  {\textsc{SuperLU\_dist}\xspace}
\newcommand{\parmetis}     {\textsc{ParMetis}\xspace}
\newcommand{\paraview}     {\textsc{ParaView}\xspace}

\newcommand{\parameterlist}{\texttt{ParameterList}\xspace}

\newcommand \trilinosWeb   {trilinos.sandia.gov\xspace}

%\newcommand{\be}  {\begin{enumerate}}
%\newcommand{\ee}  {\end{enumerate}}
%\newcommand{\cba}[3]{\choicebox{\texttt{#1}}{[{\texttt #2}] #3}}
%\newcommand{\cbb}[4]{\choicebox{\texttt{#1}}{[{\texttt #2}] #4 {\bf Default:~}#3.}}
%\newcommand{\cbc}[4]{\choicebox{\texttt{\color{red}#1}}{[{\texttt #2}] #4 {\bf Default:~}#3.}}
%
%\newcommand{\comm}[2]{\bigskip
%                      \begin{tabular}{|p{4.5in}|}\hline
%                      \multicolumn{1}{|c|}{{\bf Comment by #1}}\\ \hline
%                      #2\\ \hline
%                      \end{tabular}\\
%                      \bigskip
%                     }


\newtheorem*{mycomment}{\ding{42}}
\newtheoremstyle{plain}
  {\topsep}   % ABOVESPACE
  {\topsep}   % BELOWSPACE
  {\normalfont}  % BODYFONT
  {0pt}       % INDENT (empty value is the same as 0pt)
  {\bfseries} % HEADFONT
  {}         % HEADPUNCT
  {5pt plus 1pt minus 1pt} % HEADSPACE
  {}          % CUSTOM-HEAD-SPEC

% further declarations and additional commands
\definecolor{hellgelb}{rgb}{1,1,0.8}   % background color for C++ listings
\definecolor{darkgreen}{rgb}{0.0, 0.2, 0.13}
%\definecolor{hellrot}{HTML}{FFA4C2}    % background color for xml files

% settings for listings package
\lstset{
  backgroundcolor=\color{hellgelb},
  basicstyle=\ttfamily\small,
  breakautoindent=true,
  breaklines=true,
  captionpos=b,
  columns=flexible,
  commentstyle=\color{darkgreen},
  extendedchars=true,
  float=hbp,
  frame=single,
  identifierstyle=\color{black},
  keywordstyle=\color{blue},
  numbers=none,
  numberstyle=\tiny,
  showspaces=false,
  showstringspaces=false,
  stringstyle=\color{purple},
  tabsize=2,
}


% ---------------------------------------------------------------------------- %
% Set some things we need for SAND reports. These are mandatory
%
\SANDnum{SAND2016-5338}
\SANDprintDate{June 2016}
\SANDauthor{Andrey Prokopenko, Christopher M. Siefert, Jonathan J. Hu, \\Mark
Hoemmen, Alicia Klinvex}


% ---------------------------------------------------------------------------- %
% Include the markings required for your SAND report. The default is "Unlimited
% Release". You may have to edit the file included here, or create your own
% (see the examples provided).
%
% \include{MarkUR} % Not needed for unlimted release reports


% ---------------------------------------------------------------------------- %
% The following definition does not have a default value and will not
% print anything, if not defined
%
%\SANDsupersed{SAND1901-0001}{January 1901}
%\input{MarkOUO}


% ---------------------------------------------------------------------------- %
%
% Start the document
%
\begin{document}

    \maketitle

    % ------------------------------------------------------------------------ %
    % An Abstract is required for SAND reports
    %
    \begin{abstract}
	%This is the definitive user guide for the \muelu{} library in Trilinos version XX.YY.
%\muelu{} is a C++ multigrid framework that can work with either the \epetra or \tpetra linear
%algebra libraries.
%\muelu{} provides a variety of aggregation-based multigrid algorithms,
%including smoothed aggregation algebraic multigrid (AMG), Petrov-Galerkin AMG, and AMG for
%Maxwell's equations, as well as many different types of smoothers.
%\muelu{} is templated on the index, scalar, and compute node types.
%Thus it is possible to use \muelu{} on problems with scalar types other than double, on very
%large problems, and to exploit node-level parallelism.

This is the official user guide for \muelu{} multigrid library in Trilinos
version~\input{version}. This guide provides an overview of \muelu, its capabilities, and
instructions for new users who want to start using \muelu{} with a minimum of
effort. Detailed information is given on how to drive \muelu{} through its XML
interface. Links to more advanced use cases are given. This guide gives
information on how to achieve good parallel performance, as well as how to
introduce new algorithms. Finally, readers will find a comprehensive listing of
available \muelu{} options.  {\em Any options not documented in this manual
should be considered strictly experimental.}

%%% Local Variables:
%%% mode: latex
%%% TeX-master: "mueluguide"
%%% End:

    \end{abstract}


    % ------------------------------------------------------------------------ %
    % An Acknowledgement section is optional but important, if someone made
    % contributions or helped beyond the normal part of a work assignment.
    % Use \section* since we don't want it in the table of context
    %
    \clearpage
    \chapter*{Acknowledgment}
	Many people have helped develop \ifpacktwo{}, and we would like to acknowledge
their contributions here: Ross Bartlett, Tom Benson, Erik Boman, Joshua Booth,
Julian Cortial, Kevin Deweese, Jeremie Gaidamour, Paul Lin, Travis Fisher, Sarah
Osborn, Eric Phipps, and Paul Tsuji. Finally, Alan Williams did the original
port from \ifpack{} and was the original lead developer of \ifpacktwo{}.


    % ------------------------------------------------------------------------ %
    % The table of contents and list of figures and tables
    % Comment out \listoffigures and \listoftables if there are no
    % figures or tables. Make sure this starts on an odd numbered page
    %
    \cleardoublepage		% TOC needs to start on an odd page
    \tableofcontents
    \listoffigures
    \listoftables


    % ---------------------------------------------------------------------- %
    % An optional preface or Foreword
    %\clearpage
    %\chapter*{Preface}
    %\addcontentsline{toc}{chapter}{Preface}
	%\input{CommonPreface}


    % ---------------------------------------------------------------------- %
    % An optional executive summary
    %\clearpage
    %\chapter*{Summary}
    %\addcontentsline{toc}{chapter}{Summary}
	%\input{CommonSummary}


    % ---------------------------------------------------------------------- %
    % An optional glossary. We don't want it to be numbered
    %\clearpage
    %\chapter*{Nomenclature}
    %\addcontentsline{toc}{chapter}{Nomenclature}
    %\begin{description}
	%\item[dry spell]
	%    using a dry erase marker to spell words
	%\item[dry wall]
	%    the writing on the wall
	%\item[dry humor]
	%    when people just do not understand
	%\item[DRY]
	%    Don't Repeat Yourself
    %\end{description}


    % ---------------------------------------------------------------------- %
    % This is where the body of the report begins; usually with an Introduction
    %
    \SANDmain		% Start the main part of the report

    %-----------------------------%
    % \chapter{Introduction}\label{sec:introduction}
    %-----------------------------%
    % % $Id$


\chapter{Introduction}
\label{INTRO}

\subsection{Getting the software}


\escript, \ESyS, all freely available.  Where do people get \finley from?



\begin{enumerate}
 \item general structure 
 \item how to get the software
 \item a few words about the general structure
\item installation
\end{enumerate}

\subsection{Acknowlegements}
\begin{itemize}
\item Margeret Kahn Australian Nationional Unversity, Canberra.
\end{itemize}


    %-----------------------------%
    \chapter{Getting Started}\label{sec:getting started}
    This section is meant to get you using \muelu{} as quickly as possible.  \S\ref{sec:overview} gives a
summary of \muelu's design.  \S\ref{sec:configuration and build} lists \muelu's dependencies on other
\trilinos libraries and provides a sample cmake configuration line.  Finally, code examples using the XML
interface are given in \S\ref{sec:examples in code}.

\section{Overview of \muelu}
\label{sec:overview}
%algorithm types
%problems types
\muelu{} is an extensible algebraic multigrid (AMG) library that is part of the
\trilinos{} project. \muelu{} works with \epetra (32-bit version) and
\tpetra matrix types. The library is written in C++ and allows for different
ordinal (index) and scalar types.  \muelu{} is designed to be efficient on many
different computer architectures, from workstations to supercomputers, relying
on ``MPI+X" principle, where ``X" can be threading, CUDA, or any other back-end provided by the \kokkos package.

\muelu{} provides a number of different multigrid algorithms:
\be
  \item smoothed aggregation AMG (for Poisson-like and elasticity problems);
  \item Petrov-Galerkin aggregation AMG (for convection-diffusion problems);
  \item energy-minimizing AMG;
  \item aggregation-based AMG for problems arising from the eddy current
    formulation of Maxwell's equations.
\ee
\muelu's software design allows for the rapid introduction of new multigrid algorithms.
The most important features of \muelu{} can be summarized as:
\begin{description}
  \item \textbf{Easy-to-use interface}

    \muelu{} has a user-friendly parameter input deck which covers
    most important use cases.  Reasonable defaults are provided for common problem types
    (see Table \ref{t:problem_types}).

  \item \textbf{Modern object-oriented software architecture}

    \muelu{} is written completely in C++ as a modular object-oriented multigrid
    framework, which provides flexibility to combine and reuse existing
    components to develop novel multigrid methods.

  \item \textbf{Extensibility}

    Due to its flexible design, \muelu{} is an excellent toolkit for
    research on novel multigrid concepts. Experienced multigrid users have full
    access to the underlying framework through an advanced XML based interface.
    Expert users may use and extend the C++ API directly.

  \item \textbf{Integration with \trilinos{} library}

    As a package of \trilinos, \muelu{} is well integrated into the \trilinos
    environment. \muelu{} can be used with either the \tpetra{} or \epetra{}
    (32-bit) linear algebra stack. It is templated on the local index, global
    index, scalar, and compute node types. This makes \muelu{} ready for
    future developments.

  \item \textbf{Broad range of supported platforms}

    \muelu{} runs on wide variety of architectures, from desktop workstations to
    parallel Linux clusters and supercomputers (\cite{lin2014}).

  \item \textbf{Open source}

    \muelu{} is freely available through a simplified BSD license (see Appendix~\ref{sec:license}).
\end{description}

\section{Configuration and Build}
\label{sec:configuration and build}

\muelu{} has been compiled successfully under Linux with the following C++
compilers: GNU versions 4.1 and later, Intel versions 12.1/13.1, and clang versions 3.2 and later.
In the future, we recommend using compilers supporting C++11 standard.

\subsection{Dependencies}

\noindent{\bf Required Dependencies}

\muelu{} requires that \teuchos{} and either \epetra/\ifpack or \tpetra/\ifpacktwo
are enabled.

\noindent{\bf Recommended Dependencies}

We strongly recommend that you enable the following \trilinos libraries along with \muelu:

\begin{itemize}
  \item \epetra stack: \aztecoo, \epetra, \amesos, \ifpack, \isorropia, \galeri,
    \zoltan;
  \item \tpetra stack: \amesostwo, \belos, \galeri, \ifpacktwo, \tpetra,
    \zoltantwo.
\end{itemize}

\noindent{\bf Tutorial Dependencies}

In order to run the \muelu{} Tutorial \cite{MueLuTutorial} located in \verb!muelu/doc/Tutorial!, \muelu{} must be configured with the following
dependencies enabled:

  \aztecoo, \amesos, \amesostwo, \belos, \epetra, \ifpack, \ifpacktwo, \isorropia,
  \galeri, \tpetra, \zoltan, \zoltantwo.

\begin{mycomment}
Note that the \muelu{} tutorial \cite{MueLuTutorial} comes with a VirtualBox image with a pre-installed
Linux and \trilinos{}.   In this way, a user can immediately begin experimenting with \muelu{} without
having to install the \trilinos{} libraries. Therefore, it is an ideal starting point to get in touch with \muelu{}.
\end{mycomment}

\noindent{\bf Complete List of Direct Dependencies}

\begin{table}[ht]
  \centering
  \begin{tabular}{p{3.5cm} c c c c}
    \toprule
    \multirow{2}{*}{Dependency} & \multicolumn{2}{c}{Required} & \multicolumn{2}{c}{Optional} \\
    \cmidrule(r){2-3} \cmidrule(l){4-5}
                   & Library  & Testing  & Library  & Testing        \\
    \hline
    \amesos        &          &          & $\times$ & $\times$  \\
    \amesostwo     &          &          & $\times$ & $\times$  \\
    \aztecoo       &          &          &          & $\times$  \\
    \belos         &          &          &          & $\times$  \\
    \epetra        &          &          & $\times$ & $\times$  \\
    \ifpack        &          &          & $\times$ & $\times$  \\
    \ifpacktwo     &          &          & $\times$ & $\times$  \\
    \isorropia     &          &          & $\times$ & $\times$  \\
    \galeri        &          &          &          & $\times$  \\
    \kokkosclassic &          &          & $\times$ & \\
    \teuchos{}     & $\times$ & $\times$ &          & \\
    \tpetra        &          &          & $\times$ & $\times$  \\
    \xpetra        & $\times$ & $\times$ &          & \\
    \zoltan        &          &          & $\times$ & $\times$  \\
    \zoltantwo     &          &          & $\times$ & $\times$  \\
    \midrule
    Boost          &          &          & $\times$ & \\
    BLAS           & $\times$ & $\times$ &          & \\
    LAPACK         & $\times$ & $\times$ &          & \\
    MPI            &          &          & $\times$ & $\times$  \\
    \bottomrule
  \end{tabular}
  \caption{\label{tab:dependencies}\muelu's required and optional dependencies,
    subdivided by whether a dependency is that of the \muelu{} library itself
    (\textit{Library}), or of some \muelu{} test (\textit{Testing}). }
\end{table}

Table~\ref{tab:dependencies} lists the dependencies of \muelu, both required and
optional. If an optional dependency is not present, the tests requiring that
dependency are not built.

\begin{mycomment}
\amesos{}/\amesostwo{} are necessary if one wants to use a sparse direct solve on the coarsest level.
\zoltan{}/\zoltantwo{} are necessary if one wants to use matrix rebalancing in parallel runs (see~\S\ref{sec:performance}).
\aztecoo{}/\belos{} are necessary if one wants to test \muelu{} as a preconditioner instead of a solver.
\end{mycomment}

\begin{mycomment}
\muelu{} has also been successfully tested with SuperLU 4.1 and SuperLU 4.2.
\end{mycomment}
\begin{mycomment}
Some packages that \muelu{} depends on may come with additional requirements for
third party libraries, which are not listed here as explicit dependencies of \muelu{}.
It is highly recommended to install ParMetis 3.1.1 or newer for \zoltan{},
ParMetis 4.0.x for \zoltantwo{}, and SuperLU 4.1 or newer for
\amesos{}/\amesostwo{}.
\end{mycomment}

\subsection{Configuration}
The preferred way to configure and build \muelu{} is to do that outside of the source directory.
Here we provide a sample configure script that will enable \muelu{} and all of its optional dependencies:
\begin{lstlisting}
  export TRILINOS_HOME=/path/to/your/Trilinos/source/directory
  cmake \
      -D BUILD_SHARED_LIBS:BOOL=ON \
      -D CMAKE_BUILD_TYPE:STRING="RELEASE" \
      -D CMAKE_CXX_FLAGS:STRING="-g" \
      -D Trilinos_ENABLE_EXPLICIT_INSTANTIATION:BOOL=ON \
      -D Trilinos_ENABLE_TESTS:BOOL=OFF \
      -D Trilinos_ENABLE_EXAMPLES:BOOL=OFF \
      -D Trilinos_ENABLE_MueLu:BOOL=ON \
      -D   MueLu_ENABLE_TESTS:STRING=ON \
      -D   MueLu_ENABLE_EXAMPLES:STRING=ON \
      -D TPL_ENABLE_BLAS:BOOL=ON \
      -D TPL_ENABLE_MPI:BOOL=ON \
  ${TRILINOS_HOME}
\end{lstlisting}

\noindent
More configure examples can be found in \texttt{Trilinos/sampleScripts}.
For more information on configuring, see the \trilinos CMake Quickstart guide \cite{TrilinosCmakeQuickStart}.

\section{Examples in code}
\label{sec:examples in code}
% simple scaling test
%   galeri
%   XML input
%   belos/aztecoo or stand-alone solver
%   look @ tutorial or elsewhere for more advanced usage

The most commonly used scenario involving \muelu{} is using a multigrid
preconditioner inside an iterative linear solver. In \trilinos{}, a user has a
choice between \epetra and \tpetra for the underlying linear algebra library.
Important Krylov subspace methods (such as preconditioned CG and GMRES) are
provided in \trilinos{} packages \aztecoo (\epetra{}) and \belos
(\epetra{}/\tpetra{}).

At this point, we assume that the reader is comfortable with \teuchos{} referenced-counted
pointers (RCPs) for memory management (an introduction to RCPs can be found
in~\cite{RCP2010}) and the \texttt{Teuchos::ParameterList} class~\cite{TeuchosURL}.

\subsection{\muelu{} as a preconditioner within \belos}
\label{sec:tpetraexample}
The following code shows the basic steps of how to use a \muelu{}
multigrid preconditioner with \tpetra{} linear algebra library and with a linear
solver from \belos{}. To keep the example short and clear, we skip the template
parameters and focus on the algorithmic outline of setting up
a linear solver. For further details, a user may refer to the \texttt{examples} and
\texttt{test} directories.

First, we create the \muelu{} multigrid preconditioner. It can be done in a few
ways. For instance, multigrid parameters can be read from an XML file
(e.g., \textit{mueluOptions.xml} in the example below).
\begin{lstlisting}[language=C++]
    Teuchos::RCP<Tpetra::CrsMatrix<> > A;
    // create A here ...
    std::string optionsFile = "mueluOptions.xml";
    Teuchos::RCP<MueLu::TpetraOperator> mueLuPreconditioner =
       MueLu::CreateTpetraPreconditioner(A, optionsFile);
\end{lstlisting}
The XML file contains multigrid options. A typical file with \muelu{} parameters
looks like the following.
\begin{lstlisting}[language=XML]
<ParameterList name="MueLu">

  <Parameter name="verbosity" type="string" value="low"/>

  <Parameter name="max levels" type="int" value="3"/>
  <Parameter name="coarse: max size" type="int" value="10"/>

  <Parameter name="multigrid algorithm" type="string" value="sa"/>

  <!-- Damped Jacobi smoothing -->
  <Parameter name="smoother: type" type="string" value="RELAXATION"/>
  <ParameterList name="smoother: params">
    <Parameter name="relaxation: type"  type="string" value="Jacobi"/>
    <Parameter name="relaxation: sweeps" type="int" value="1"/>
    <Parameter name="relaxation: damping factor" type="double" value="0.9"/>
  </ParameterList>

  <!-- Aggregation -->
  <Parameter name="aggregation: type" type="string" value="uncoupled"/>
  <Parameter name="aggregation: min agg size" type="int" value="3"/>
  <Parameter name="aggregation: max agg size" type="int" value="9"/>

</ParameterList>
\end{lstlisting}
It defines a three level smoothed aggregation multigrid algorithm. The
aggregation size is between three and nine(2D)/27(3D) nodes.  One sweep with a
damped Jacobi method is used as a level smoother. By default, a direct solver is
applied on the coarsest level. A complete list of available parameters and valid
parameter choices is given in \S\ref{sec:muelu_options} of this User's Guide.

Users can also construct a multigrid preconditioner using a provided \parameterlist
without accessing any files in the following manner.
\begin{lstlisting}[language=C++]
  Teuchos::RCP<Tpetra::CrsMatrix<> > A;
  // create A here ...
  Teuchos::ParameterList paramList;
  paramList.set("verbosity", "low");
  paramList.set("max levels", 3);
  paramList.set("coarse: max size", 10);
  paramList.set("multigrid algorithm", "sa");
  // ...
  Teuchos::RCP<MueLu::TpetraOperator> mueLuPreconditioner =
     MueLu::CreateTpetraPreconditioner(A, paramList);
\end{lstlisting}

Besides the linear operator $A$, we also need an initial guess vector for the
solution $X$ and a right hand side vector $B$ for solving a linear system.
\begin{lstlisting}[language=C++]
    Teuchos::RCP<const Tpetra::Map<> > map = A->getDomainMap();

    // Create initial vectors
    Teuchos::RCP<Tpetra::MultiVector<> > B, X;
    X = Teuchos::rcp( new Tpetra::MultiVector<>(map,numrhs) );
    Belos::MultiVecTraits<>::MvRandom( *X );
    B = Teuchos::rcp( new Tpetra::MultiVector<>(map,numrhs) );
    Belos::OperatorTraits<>::Apply( *A, *X, *B );
    Belos::MultiVecTraits<>::MvInit( *X, 0.0 );
\end{lstlisting}
To generate a dummy example, the above code first declares two vectors. Then, a
right hand side vector is calculated as the matrix-vector product of a random vector
with the operator $A$. Finally, an initial guess is initialized with zeros.

Then, one can define a \texttt{Belos::LinearProblem} object where the
\texttt{mueLuPreconditioner} is used for left preconditioning
\begin{lstlisting}[language=C++]
    Belos::LinearProblem<> problem( A, X, B );
    problem->setLeftPrec(mueLuPreconditioner);
    bool set = problem.setProblem();
\end{lstlisting}

Next, we set up a \belos{} solver using some basic parameters
\begin{lstlisting}[language=C++]
    Teuchos::ParameterList belosList;
    belosList.set( "Block Size", 1 );
    belosList.set( "Use Single Reduction", true );
    belosList.set( "Maximum Iterations", 100 );
    belosList.set( "Convergence Tolerance", 1e-10 );
    belosList.set( "Output Frequency", 1 );
    belosList.set( "Verbosity", Belos::TimingDetails + Belos::FinalSummary );

    Belos::BlockCGSolMgr<> solver( rcp(&problem,false), rcp(&belosList,false) );
\end{lstlisting}

Finally, we solve the system.
\begin{lstlisting}[language=C++]
    Belos::ReturnType ret = solver.solve();
\end{lstlisting}

\subsection{\muelu{} as a preconditioner for \aztecoo}

For \epetra, users have two library options: \belos{} (recommended) and \aztecoo{}.
\aztecoo{} and \belos both provide fast and mature implementations of common iterative Krylov linear solvers.
\belos has additional capabilities, such as Krylov subspace recycling and ``tall skinny QR".

Constructing a \muelu{} preconditioner for Epetra operators is done in a similar
manner to Tpetra.
\begin{lstlisting}[language=C++]
    Teuchos::RCP<Epetra_CrsMatrix> A;
    // create A here ...
    Teuchos::RCP<MueLu::EpetraOperator> mueLuPreconditioner;
    std::string optionsFile = "mueluOptions.xml";
    mueLuPreconditioner = MueLu::CreateEpetraPreconditioner(A, optionsFile);
\end{lstlisting}
\muelu{} parameters are generally Epetra/Tpetra agnostic, thus the XML parameter file
could be the same as~\S\ref{sec:tpetraexample}.

Furthermore, we assume that a right hand side vector and a solution vector with
the initial guess are defined.
\begin{lstlisting}[language=C++]
    Teuchos::RCP<const Epetra_Map> map = A->DomainMap();
    Teuchos::RCP<Epetra_Vector> B = Teuchos::rcp(new Epetra_Vector(map));
    Teuchos::RCP<Epetra_Vector> X = Teuchos::rcp(new Epetra_Vector(map));
    X->PutScalar(0.0);
\end{lstlisting}

Then, an \texttt{Epetra\_LinearProblem} can be defined.
\begin{lstlisting}[language=C++]
    Epetra_LinearProblem epetraProblem(A.get(), X.get(), B.get());
\end{lstlisting}

The following code constructs an \aztecoo{} CG solver.
\begin{lstlisting}[language=C++]
    AztecOO aztecSolver(epetraProblem);
    aztecSolver.SetAztecOption(AZ_solver, AZ_cg);
    aztecSolver.SetPrecOperator(mueLuPreconditioner.get());
\end{lstlisting}

Finally, the linear system is solved.
\begin{lstlisting}[language=C++]
    int maxIts = 100;
    double tol = 1e-10;
    aztecSolver.Iterate(maxIts, tol);
\end{lstlisting}


\subsection{\muelu's structured algorithms}

Some users might use structured meshes to discretize their problems. In such cases it can be advantageous to use the structured grid algorithms provided in \muelu. To use these algorithms the user has to provide extra information to \muelu such as the number of spatial dimensions in the problem and the number of nodes in each direction on the local rank. As demonstrated in the code bellow \muelu expect these additional inputs to be stored in a sublist called ``user data".
\begin{lstlisting}[language=C++]
  const std::string userName = "user data";
  Teuchos::ParameterList& userParamList = paramList.sublist(userName);
  userParamList.set<int>("int numDimensions", numDimensions);
  userParamList.set<Teuchos::Array<LO> >("Array<LO> lNodesPerDim", lNodesPerDim);
  userParamList.set<RCP<RealValuedMultiVector> >("Coordinates", coordinates);
  H = MueLu::CreateXpetraPreconditioner(A, paramList, paramList);
\end{lstlisting}
Full examples demonstrating the structured capabilities of \muelu can be found in the \trilinos source directories
\begin{itemize}
  \setlength{\itemsep}{-10pt}
\item \texttt{packages/muelu/test/structured},
\item \texttt{packages/trilinoscouplings/examples/scaling}.
\end{itemize}


\subsection{\muelu's Maxwell solver}

\muelu can be used to solve Maxwell's equations in eddy current formulation which can be written as
\begin{equation}
  \nabla\times \left(\alpha \nabla  \times \vec{E}\right) + \beta \vec{E} = \vec{f}, \label{eq:maxwell}
\end{equation}
where \(\vec{E}\) is the unknown electric field, \(\alpha\) and \(\beta\) are material parameters,
and \(\vec{f}\) is the known right-hand side.
In order to deal with the large nullspace of the curl-curl operator a specialized multigrid approach
is required.
For a detailed description of the solver see ~\cite{RefMaxwell2008}.

A preconditioner for equation~\ref{eq:maxwell} can be constructed as follows:
\begin{lstlisting}[language=C++]
  RCP<Matrix> SM_Matrix = ... ;    \\ Edge-mass + curl-curl
  RCP<Matrix> D0_Matrix = ... ;    \\ Discrete gradient matrix
  RCP<Matrix> M0inv_Matrix = ... ; \\ Approximate inverse of node-mass matrix with weight 1/alpha
  RCP<Matrix> M1_Matrix = ... ;    \\ Edge-mass matrix with constant weight 1
  RCP<MultiVector> coords = ...;   \\ Nodal coordinates
  Teuchos::ParameterList params = ...; \\ Parameters

  RCP<MueLu::RefMaxwell> preconditioner
  = rcp( new MueLu::RefMaxwell(SM_Matrix, D0_Matrix, M0inv_Matrix,
         M1_Matrix, Teuchos::null, coords, params) );
\end{lstlisting}
An example set of parameters is given below:
\begin{lstlisting}[language=XML]
  <ParameterList name="MueLu">

  <Parameter name="refmaxwell: mode"  type="string" value="additive"/>

  <Parameter name="smoother: type" type="string" value="RELAXATION"/>
  <ParameterList name="smoother: params">
    <Parameter name="relaxation: type" type="string" value="Symmetric Gauss-Seidel"/>
    <Parameter name="relaxation: sweeps" type="int" value="2"/>
  </ParameterList>

  <ParameterList name="refmaxwell: 11list">
    <Parameter name="number of equations"   type="int"    value="3"/>
    <Parameter name="aggregation: type"     type="string" value="uncoupled"/>
    <Parameter name="coarse: max size"      type="int"    value="2500"/>
    <Parameter name="smoother: type" type="string" value="RELAXATION"/>
    <ParameterList name="smoother: params">
      <Parameter name="relaxation: type" type="string" value="Symmetric Gauss-Seidel"/>
      <Parameter name="relaxation: sweeps" type="int" value="2"/>
    </ParameterList>
  </ParameterList>

  <ParameterList name="refmaxwell: 22list">
    <Parameter name="aggregation: type"     type="string" value="uncoupled"/>
    <Parameter name="coarse: max size"      type="int"    value="2500"/>
    <Parameter name="smoother: type" type="string" value="RELAXATION"/>
    <ParameterList name="smoother: params">
      <Parameter name="relaxation: type" type="string" value="Symmetric Gauss-Seidel"/>
      <Parameter name="relaxation: sweeps" type="int" value="2"/>
    </ParameterList>
  </ParameterList>

</ParameterList>
\end{lstlisting}
Further examples of how to use \muelu to solve Maxwell's equations can be found in the \trilinos source directories
\begin{itemize}
  \setlength{\itemsep}{-10pt}
\item \texttt{packages/muelu/test/maxwell},
\item \texttt{packages/panzer/mini-em/example/BlockPrec} and
\item \texttt{packages/trilinoscouplings/examples/scaling}.
\end{itemize}




\subsection{Further remarks}

This section is only meant to give a brief introduction on how to use \muelu{}
as a preconditioner within the \trilinos{} packages for iterative solvers. There
are other, more complicated, ways to use \muelu{} as a preconditioner for \belos
and \aztecoo through the \xpetra interface. Of course, \muelu{} can also work as
standalone multigrid solver. For more information on these topics, the reader
may refer to the examples and tests in the \muelu{} source directory
(\texttt{packages/muelu/example} and \texttt{packages/muelu/test}) and in the trilinosCouplings source directory
(\texttt{packages/trilinosCouplings}), as well as to the \muelu{}
tutorial~\cite{MueLuTutorial}.
For in-depth details of \muelu applied to multiphysics problems, please see~\cite{Wiesner2014}.

%%% Local Variables:
%%% mode: latex
%%% TeX-master: "mueluguide"
%%% End:


    %-----------------------------%
    \chapter{\ifpacktwo options}
    \label{sec:options}
In this section, we report the complete list of input parameters. Input
parameters are passed to \ifpacktwo in a single \parameterlist.

In some cases, the parameter types may depend on runtime template parameters.
In such cases, we will follow the conventions in Table~\ref{tab:conventions}.

\begin{table}[htbp]
  \centering
  \begin{tabular}{p{13.3cm} p{2.5cm}}
    \toprule
    \verb!MatrixType::local_ordinal_type!                                  & \verb!local_ordinal! \\
    \verb!MatrixType::global_ordinal_type!                                 & \verb!global_ordinal! \\
    \verb!MatrixType::scalar_type!                                         & \verb!scalar! \\
    \verb!MatrixType::node_type!                                           & \verb!node! \\
    \verb!Tpetra::Vector<scalar,local_ordinal,global_ordinal,node>!        & \verb!vector!\\
    \verb!Tpetra::MultiVector<scalar,local_ordinal,global_ordinal,node>!   & \verb!multi_vector!\\
    \verb!vector::mag_type!                                                & \verb!magnitude! \\
    \bottomrule
  \end{tabular}
  \caption{\label{tab:conventions}Conventions for option types that depend on templates.}
\end{table}

\noindent\textbf{Note:} if \verb!scalar! is \texttt{double}, then \verb!magnitude! is also \texttt{double}.

\section{Point relaxation}\label{s:relaxation}

\textbf{Preconditioner type:} ``RELAXATION''.

\ifpacktwo{} implements the following classical relaxation methods: Jacobi (with
optional damping), Gauss-Seidel, Successive Over-Relaxation (SOR), symmetric
version of Gauss-Seidel and SOR. \ifpacktwo{} calls both Gauss-Seidel and SOR
"Gauss-Seidel". The algorithmic details can be found in~\cite{Saad2003}.

Besides the classical relaxation methods, \ifpacktwo{} also implements $l_1$
variants of Jacobi and Gauss-Seidel methods proposed in~\cite{Baker2011}, which
lead to a better performance in parallel applications.

\noindent{\bf Note:} if a user provides a \texttt{Tpetra::BlockCrsMatrix}, the point relaxation
methods become block relaxation methods, such as block Jacobi or block
Gauss-Seidel.

The following parameters are used in the point relaxation methods:

\ccc{relaxation: type}
    {string}
    {``Jacobi''}
    {Relaxation method to use. Accepted values: ``Jacobi'',
     ``Gauss-Seidel'', ``Symmetric Gauss-Seidel''.}
\ccc{relaxation: sweeps}
    {int}
    {1}
    {Number of sweeps of the relaxation.}
\ccc{relaxation: damping factor}
    {scalar}
    {1.0}
    {The value of the damping factor $\omega$ for the relaxation.}
\ccc{relaxation: backward mode}
    {bool}
    {\false}
    {Governs whether Gauss-Seidel is done in forward-mode (\false) or
     backward-mode (\true). Only valid for ``Gauss-Seidel'' type.}
\ccc{relaxation: use l1}
    {bool}
    {\false}
    {Use the $l_1$ variant of Jacobi or Gauss-Seidel.}
\ccc{relaxation: l1 eta}
    {magnitude}
    {1.5}
    {$\eta$ parameter for $l_1$ variant of Gauss-Seidel. Only used if
     {\tt "relaxation: use l1"} is \true.}
\ccc{relaxation: zero starting solution}
    {bool}
    {\true}
    {Governs whether or not \ifpacktwo{} uses existing values in the left hand
     side vector. If true, \ifpacktwo{} fill it with zeros before applying
     relaxation sweeps which may make the first sweep more efficient.}
\ccc{relaxation: fix tiny diagonal entries}
    {bool}
    {\false}
    {If true, the compute() method will do extra work (computation only, no MPI
     communication) to fix diagonal entries. Specifically, the diagonal values
     with a magnitude smaller than the magnitude of the threshold \texttt{relaxation: min
     diagonal value} are increased to threshold for the diagonal inversion. The
     matrix is not modified, instead the updated diagonal values are stored. If the
     threshold is zero, only the diagonal entries that are exactly zero are replaced
     with a small nonzero value (machine precision).}
\ccc{relaxation: min diagonal value}
    {scalar}
    {0.0}
    {The threshold value used in {\tt "relaxation: fix tiny diagonal entries"}.
     Only used if {\tt "relaxation: fix tiny diagonal entries"} is \true.}
\ccc{relaxation: check diagonal entries}
    {bool}
    {\false}
    {If true, the \texttt{compute()} method will do extra work (both computation
     and communication) to count diagonal entries that are zero, have negative
     real part, or are small in magnitude. This information can be later shown
     in the description.}
\ccc{relaxation: mtgs cluster size}
    {int}
    {1}
    {Only has an effect if {\tt "relaxation: type"} is {\tt "MT Gauss-Seidel"}
     or {\tt "MT Symmetric Gauss-Seidel"}. If equal to 1 (default), point
     coloring parallel Gauss-Seidel is used. This has a faster \texttt{compute()}
     but may cause the preconditioned solver
     to converge more slowly. If set to $k > 1$, then multicolor block Gauss-Seidel
     is used with blocks of size $k$ (see \cite{Saad1999}).
     In the \texttt{apply()} there is significantly less
     error due to parallel updates of the LHS vector.}
\ccc{relaxation: local smoothing indices}
    {Teuchos::ArrayRCP<local\_ordinal>}
    {empty}
    {}

%Teuchos::ArrayRCP MatrixType::local_ordinal_type}{\texttt{Teuchos::null}}
    {A given method will only relax on the local indices listed in the
     \texttt{ArrayRCP}, in the order that they are listed. This can be used to
     reorder the relaxation, or to only relax on a subset of ids.}

\section{Block relaxation}\label{s:block_relaxation}

\textbf{Preconditioner type:} ``BLOCK\_RELAXATION''.

% \info[inline]{AP}{ILUTP cannot be constructed through {\tt Ifpack2::Factory},
% only through additive Schwarz}

\ifpacktwo{} supports block relaxation methods. Each block corresponds to a set
of degrees of freedom within a local subdomain. The blocks can be
non-overlapping or overlapping. Block relaxation can be considered as domain
decomposition within an MPI process, and should not be confused with additive
Schwarz preconditioners (see~\ref{s:schwarz}) which implement domain
decomposition across MPI processes.

There are several ways the blocks are constructed:
\begin{itemize}
  \item Linear partitioning of unknowns

    The unknowns are divided equally among a specified number of
    partitions $L$ defined by {\tt "partitioner: local parts"}. In other words,
    assuming number of unknowns $n$ is divisible by $L$, unknown $i$ will belong
    to block number $\lfloor iL/n \rfloor$.

  \item Line partitioning of unknowns

    The unknowns are grouped based on a geometric criteria which tries to
    identify degrees of freedom that form an approximate geometric line.
    Current approach uses a local line detection inspired by the work of
    Mavriplis~\cite{Mavriplis1999} for convection-diffusion. \ifpacktwo uses
    coordinate information provided by {\tt "partitioner: coordinates"} to pick
    "close" points if they are sufficiently far away from the "far" points. It
    also makes sure the line can never double back on itself.

    These "line" partitions were found to be very beneficent to problems on
    highly anisotropic geometries such as ice-sheet simulations.

  \item User partitioning of unknowns

    The unknowns are grouped according to a user provided partition. A user
    may provide a non-overlapping partition {\tt "partitioner: map"} or an
    overlapping one {\tt "partitioner: parts"}.

    A particular example of a smoother using this approach is a Vanka
    smoother~\cite{Vanka1986}, where a user may in {\tt "partition: parts"} pressure
    degrees of freedom, and request a overlap of one thus constructing Vanka
    blocks.
\end{itemize}
The original partitioning may be further modified with {\tt "partitioner: overlap"}
parameter which will use the local matrix graph to construct overlapping
partitions.

The blocks are applied in the order they were constructed. This means that in
the case of overlap the entries in the solution vector relaxed by one block may
later be overwritten by relaxing another block.

The following parameters are used in the block relaxation methods:

\cccc{relaxation: type}
    {See~\ref{s:relaxation}.}
\cccc{relaxation: container}
    {string}
    {``TriDi''}
    {Containers are used to store and solve block matrices. These container
     types are always available: ``Dense'', ``TriDi''
     (equivalent to ``Tridiagonal''), ``Banded'' and ``SparseILUT''.
     ``Dense'', ``TriDi'' and ``Banded'' block matrices are
     solved exactly LAPACK routines, and ``SparseILUT'' blocks are solved approximately
     using an incomplete LU factorization with thresholding.

     If Amesos2 is enabled, ``SparseAmesos'' (equivalent to ``SparseAmesos2'') is available.
     The default Amesos2 sparse solver is KLU2, but this can be configured by setting
     ``Amesos2 solver name'' (see the Amesos2 documentation for all available solvers).

     If experimental kokkos-kernels features are enabled (true by default), the ``BlockTriDi''
     container (equivalent to ``Block Tridiagonal'') is available. This container's solver is the damped Jacobi method, using
     block tridiagonal matrices as the diagonal D.
     For a block size of 1, this is equivalent to standard damped Jacobi.
     This container is designed for high performance on KNL and GPU.}
\cccc{relaxation: sweeps}
    {See~\ref{s:relaxation}.}
\cccc{relaxation: damping factor}
    {See~\ref{s:relaxation}.}
\cccc{relaxation: zero starting solution}
    {See~\ref{s:relaxation}.}
\cccc{relaxation: backward mode}
    {See~\ref{s:relaxation}. Currently has no effect. }
\ccc{block relaxation: decouple dofs}
    {bool}
    {false}
    {Whether to separate blocks according to the different degrees of
     freedom (PDEs) at each node. This assumes that dofs/node is constant
     throughout the matrix. Each block will have the same sparsity
     pattern as the mesh graph's corresponding diagonal block.
     For example, when using a line partitioner this
     enables the use of the tridiagonal container even if the matrix's
     bandwidth is greater than 3.
     Decoupling matches the behavior of line smoothing in ML.}
\ccc{partitioner: type}
    {string}
    {``linear''}
    {The partitioner to use for defining the blocks.  This can be either
     ``linear'', ``line'' or ``user''.}
\ccc{partitioner: overlap}
    {int}
    {0}
    {The amount of overlap between partitions (0 corresponds to no overlap).
     Only valid for ``Jacobi'' relaxation.}
\ccc{partitioner: local parts}
    {int}
    {1}
    {Number of local partitions (1 corresponds to one local partition, which
     means "do not partition locally"). Only valid for ``linear'' partitioner
     type.}
\ccc{partitioner: map}
    {Teuchos::ArrayRCP<local\_ordinal>}
    {empty}
    {An array containing the partition number for each element.
     The $i$th entry in the \texttt{ArrayRCP} is the part (block) number that
     row $i$ belongs to. Use this option if the parts (blocks) do not
     overlap. Only valid for ``user'' partitioner type.}
\ccc{partitioner: parts}
    {Teuchos::Array<Teuchos::ArrayRCP\\<local\_ordinal>>}
    {empty}
    {Use this option if the parts (blocks) overlap. The $i$th entry in the
     \texttt{Array} is an \texttt{ArrayRCP} that contains all the rows in part
     (block) $i$. Only valid for ``user'' partitioner type.}
\ccc{partitioner: line detection threshold}
    {magnitude}
    {0.0}
    {Threshold used in line detection. If the distance between two connected
     points $i$ and $j$ is within the threshold times maximum distance of all
     points connected to $i$, then point $j$ is considered close enough to line
     smooth. Only valid for ``line'' partition type.}
\ccc{partitioner: PDE equations}
    {int}
    {1}
    {Number of equations per node. Only used for ``line'' partitioning, and
     decoupled BlockRelaxation.}
\ccc{partitioner: coordinates}
    {Teuchos::RCP<multi\_vector>}
    {null}
    {Coordinates of local nodes. Only valid for ``line'' partitioner type.}
\ccc{partitioner: maintain sparsity}
    {bool}
    {\false}
    {For OverlappingPartitioner, whether to sort the entries in each partition.}

\section{Chebyshev}\label{s:Chebyshev}

\textbf{Preconditioner type:} ``CHEBYSHEV''.

% Mark Hoemmen (2016/05/31):
%   The "textbook version" of Chebyshev doesn't really
%   work; we need to get rid of it.

\ifpacktwo{} implements a variant of Chebyshev iterative method following
\ifpack{}'s implementation.  \ifpack{} has a special-case modification of the
eigenvalue bounds for the case where the maximum eigenvalue estimate is close to
one. Experiments show that the \ifpack{} imitation is much less sensitive to the
eigenvalue bounds than the textbook version.

\ifpacktwo{} uses the diagonal of the matrix to precondition the linear system on the
left. Diagonal elements less than machine precision are replaced with machine
precision.

\ifpacktwo{} requires can take any matrix $A$ but can only guarantee convergence
for real valued symmetric positive definite matrices.
\iffalse
If users could provide the ellipse parameters ($d$ and $c$ in the literature,
where $d$ is the real-valued center of the ellipse, and $d-c$ and $d+c$ the two
foci), the iteration itself would work fine with nonsymmetric real-valued $A$,
as long as the eigenvalues of $A$ can be bounded in an ellipse that is entirely
to the right of the origin.
\unsure[inline]{AP}{Really unsure about Chebyshev nonsymmetric matrices. There does not
seem anything in the code to work with ellipse. I need to ask Mark Hoemmen
about this.}
\fi

The following parameters are used in the Chebyshev method:

\ccc{chebyshev: degree}
    {int}
    {1}
    {Degree of the Chebyshev polynomial, or the number of iterations. This
     overrides parameters {\tt "relaxation: sweeps"} and {\tt "smoother: sweeps"}.}
\cccc{relaxation: sweeps}
    {Same as {\tt "chebyshev: degree"}, for compatibility with \ifpack{}.}
\cccc{smoother: sweeps}
    {Same as {\tt "chebyshev: degree"}, for compatibility with \ml{}.}
\ccc{chebyshev: max eigenvalue}
    {scalar|double}
    {computed}
    {An upper bound of the matrix eigenvalues. If not provided, the value will
     be computed by power method (see parameters {\tt "eigen-analysis: type"} and
     {\tt "chebyshev: eigenvalue max iterations"}).}
\ccc{chebyshev: min eigenvalue}
    {scalar|double}
    {computed}
    {A lower bound of the matrix eigenvalues.  If not provided, \ifpacktwo{}
     will provide an estimate based on the maximum eigenvalue and the ratio.}
\ccc{chebyshev: ratio eigenvalue}
    {scalar|double}
    {30.0}
    {The ratio of the maximum and minimum estimates of the matrix
     eigenvalues.}
\cccc{smoother: Chebyshev alpha}
    {Same as {\tt "chebyshev: ratio eigenvalue"}, for compatibility with \ml{}.}
% \ccc{chebyshev: textbook algorithm}
    % {bool}
    % {\false}
    % {If true, use the textbook variant; otherwise, use the \ifpack{} variant.}
\ccc{chebyshev: compute max residual norm}
    {bool}
    {\false}
    {The \texttt{apply} call will optionally return the norm of the residual.}
\ccc{eigen-analysis: type}
    {string}
    {"power-method"}
    {The algorithm for estimating the max eigenvalue. Currently only supports
     power method ("power-method" or "power method"). The cost of the procedure is
     roughly equal to several matrix-vector multiplications.}
\ccc{chebyshev: eigenvalue max iterations}
    {int}
    {10}
    {Number of iterations to be used in calculating the estimate for the maximum
     eigenvalue, if it is not provided by the user.}
\cccc{eigen-analysis: iterations}
    {Same as {\tt "chebyshev: eigenvalue max iterations"}, for compatibility with \ml{}.}
\ccc{chebyshev: min diagonal value}
    {scalar}
    {0.0}
    {Values on the diagonal smaller than this value are increased to this value
     for the diagonal inversion.}
\ccc{chebyshev: boost factor}
    {double}
    {1.1}
    {Factor used to increase the estimate of matrix maximum eigenvalue to ensure
    the high-energy modes are not magnified by a smoother.}
\ccc{chebyshev: assume matrix does not change}
    {bool}
    {\false}
    {Whether \texttt{compute()} should assume that the matrix has not changed
     since the last call to \texttt{compute()}. If true, \texttt{compute()}
     will not recompute inverse diagonal or eigenvalue estimates.}
\ccc{chebyshev: operator inv diagonal}
    {Teuchos::RCP<const vector>|\\Teuchos::RCP<vector>|const vector*|\\vector}
    {Teuchos::null}
    {If nonnull, a deep copy of this vector will be used as the inverse
     diagonal of the matrix, instead of computing it. Expert use only.}
\ccc{chebyshev: min diagonal value}
    {scalar}
    {machine precision}
    {If any entry of the matrix diagonal is less that this in magnitude, it will
     be replaced with this value in the inverse diagonal used for left scaling.}
\cccc{chebyshev: zero starting solution}
    {See {\tt "relaxation: zero starting solution"}.}

\section{Incomplete factorizations}

\subsection{ILU($k$)}\label{s:ILU}

\textbf{Preconditioner type:} ``RILUK''.

\ifpacktwo{} implements a standard and modified (MILU) variants of the
ILU($k$) factorization~\cite{Saad2003}. In addition, it also provides an
optional \textit{a priori} modification of the diagonal entries of a matrix to
improve the stability of the factorization.

The following parameters are used in the ILU($k$) method:

\ccc{fact: iluk level-of-fill}
    {int|global\_ordinal|magnitude|double}
    {0}
    {Level-of-fill of the factorization.}
\ccc{fact: relax value}
    {magnitude|double}
    {0.0}
    {MILU diagonal compensation value. Entries dropped during factorization
     times this factor are added to diagonal entries.}
\ccc{fact: absolute threshold}
    {magnitude|double}
    {0.0}
    {Prior to the factorization, each diagonal entry is updated by adding
     this value (with the sign of the actual diagonal entry). Can be combined
     with {\tt "fact: relative threshold"}. The matrix remains unchanged.}
\ccc{fact: relative threshold}
    {magnitude|double}
    {1.0}
    {Prior to the factorization, each diagonal element is scaled by this factor
     (not including contribution specified by {\tt "fact: absolute
     threshold"}). Can be combined with {\tt "fact: absolute threshold"}.
     The matrix remains unchanged.}
% All overlap-related code was removed by M. Hoemmen in
%
% commit 162f64572fbf93e2cac73e3034d76a3db918a494
% Author: Mark Hoemmen <mhoemme@sandia.gov>
% Date:   Fri Jan 24 17:16:19 2014 -0700
%
%     Ifpack2: RILUK: Removed all overlap-related code.
%
%     Overlap never had a correct implementation in RILUK.  Furthermore,
%     AdditiveSchwarz is the proper place for overlap to be implemented, not
%     RILUK.  Ifpack2's incomplete factorizations are local (per MPI
%     process) solvers and don't need to know anything about overlap across
%     processes.  Thus, this commit removes all overlap-related code from
%     RILUK.
%
% So, older parameter "fact: iluk level-of-overlap" is no longer valid and is ignored.

\subsection{ILUT}\label{s:ILUT}

\textbf{Preconditioner type:} ``ILUT''.

\ifpacktwo{} implements a slightly modified variant of the standard ILU factorization with specified fill and
drop tolerance ILUT($p,\tau$)~\cite{Saad1994}. The modifications follow the \aztecoo implementation.
The main difference between the \ifpacktwo implementation and the algorithm in \cite{Saad1994} is the definition of
\texttt{fact: ilut level-of-fill}.

The following parameters are used in the ILUT method:

\ccc{fact: ilut level-of-fill}
    {int|magnitude|double}
    {1}
    {Maximum number of entries to keep in each row of $L$ and $U$. Each row of
     $L$ ($U$) will have at most $\lceil\frac{(\mbox{\small\tt
     level-of-fill}-1)nnz(A)}{2n}\rceil$ nonzero entries, where $nnz(A)$ is the
     number of nonzero entries in the matrix, and $n$ is the number of rows.
     ILUT always keeps the diagonal entry in the current row, regardless of the
     drop tolerance or fill level. \textbf{Note:} \textit{This is
     different from the $p$ in the classic algorithm in~\cite{Saad1994}.}}
\ccc{fact: drop tolerance}
    {magnitude|double}
    {0.0}
    {A threshold for dropping entries ($\tau$ in~\cite{Saad1994}).}
\cccc{fact: absolute threshold}
    {See~\ref{s:ILU}.}
\cccc{fact: relative threshold}
    {See~\ref{s:ILU}.}
\cccc{fact: relax value}
    {Currently has no effect. For backwards compatibility only.}

\subsection{ILUTP}\label{s:ILUTP}

\textbf{Preconditioner type:} ``AMESOS2''.

% \info[inline]{AP}{ILUTP cannot be constructed through {\tt Ifpack2::Factory},
% only through additive Schwarz}

\ifpacktwo{} implements a standard ILUTP factorization~\cite{Saad2003}. This is
done through is through the \amesostwo interface to SuperLU~\cite{Li2011}. We
reproduce the \amesostwo options here for convenience. {\em You should consider
the \href{http://trilinos.org/docs/dev/packages/amesos2/doc/html/group__amesos2__solver__parameters.html#superlu_parameters}{\amesostwo
documentation} to be the final authority.}

The following parameters are used in the ILUTP method:

\ccc{ILU\_DropTol}
    {double}
    {1e-4}
    {ILUT drop tolerance.}
\ccc{ILU\_FillFactor}
    {double}
    {10.0}
    {ILUT fill factor.}
\ccc{ILU\_Norm}
    {string}
    {``INF\_NORM''}
    {Norm to be used in factorization. Accepted values: ``ONE\_NORM'', ``TWO\_NORM'', or ``INF\_NORM''.}
\ccc{ILU\_MILU}
    {string}
    {``SILU''}
    {Type of modified ILU to use. Accepted values: ``SILU'', ``SMILU\_1'', ``SMILU\_2'', or ``SMILU\_3''.}

\subsection{ShyLU FastILU}\label{s:FastILU}
\ifpacktwo{} provides an interface to the FastILU family of factorizations provided by ShyLU.
They are available if Trilinos was configured with the

\texttt{-D Trilinos\_ENABLE\_ShyLU\_Node=ON}

option. There are three values of ``Preconditioner type:'' that use the FastILU subpackage:

\begin{table}[h!]
\centering
\begin{tabular}{|l|l|}
\hline
``Preconditioner type:'' & Factorization   \\ \hline \hline
FAST\_ILU                & Incomplete LU       \\ \hline
FAST\_IC                 & Incomplete Cholesky \\ \hline
FAST\_ILDL               & Incomplete LDL*     \\ \hline
\end{tabular}
\end{table}

FAST\_ILU, FAST\_IC, and FAST\_ILDL all use iterative factorization algorithms in compute(). \texttt{"sweeps"} controls
this iteration count. A higher number of sweeps improves the quality of the factorization. All three
preconditioners also use an triangular block Jacobi solver in apply().
The Jacobi iteration count is controlled by \texttt{"triangular solve iterations"}.
The valid set of parameters is the same for FAST\_ILU, FAST\_IC, and FAST\_ILDL:

\ccc{sweeps}
    {int}
    {5}
    {Number of iterations of ILU/IC/ILDL factorization algorithm.}
\ccc{triangular solve iterations}
    {int}
    {1}
    {Number of iterations of the block Jacobi triangular solver.}
\ccc{level}
    {int}
    {0}
    {Level of fill.}
\ccc{damping factor}
    {double}
    {0.5}
    {Damping factor $\omega$ for the Jacobi triangular solver. $0 < \omega \leq 1$. A lower $\omega$ slows convergence but improves stability.}
\ccc{shift}
    {double}
    {0}
    {Manteuffel shifting parameter $\alpha$.}
\ccc{guess}
    {bool}
    {true}
    {Whether to run another instance of FastILU/IC/ILDL (but with a lower level of fill) to compute the initial guess (only has an effect if level of fill $> 0$).} 
\ccc{block size}
    {int}
    {1}
    {Block size for the block Jacobi solver.}

\section{Additive Schwarz}\label{s:schwarz}

\textbf{Preconditioner type:} ``SCHWARZ''.

\ifpacktwo{} implements additive Schwarz domain decomposition with optional
overlap. Each subdomain corresponds to exactly one MPI process in the given
matrix's MPI communication. For domain decomposition within an
MPI process see~\ref{s:block_relaxation}.

One-level overlapping domain decomposition preconditioners use local solvers of
Dirichlet type. This means that the inverse of the local matrix (possibly with
overlap) is applied to the residual to be preconditioned. The preconditioner can
be written as:
$$ P_{AS}^{-1} = \sum_{i=1}^M P_i A_i^{-1} R_i, $$
where $M$ is the number of subdomains (in this case, the number of (MPI)
processes in the computation), $R_i$ is an operator that restricts the global
vector to the vector lying on subdomain $i$, $P_i$ is the prolongator
operator, and $A_i = R_i A P_i$.

Constructing a Schwarz preconditioner requires defining two components.

{\bf Definition of the restriction and prolongation operators.}
Users may control how the data is combined with existing data by setting {\tt
"combine mode"} parameter. Table~\ref{t:combine_mode} contains a list of modes to
combine overlapped entries. The default mode is ``ZERO'' which is equivalent to
using a restricted additive Schwarz~\cite{Cai1999} method.

\begin{table}[htbp]
  \centering
  \begin{tabular}{p{3.5cm} p{12.0cm}}
    \toprule
    Combine mode name & Description \\
    \midrule
    ``ADD''           & Sum values into existing values \\
    ``ZERO''          & Replace old values with zero \\
    ``INSERT''        & Insert new values that don't currently exist \\
    ``REPLACE''       & Replace existing values with new values \\
    ``ABSMAX''        & Replace old values with maximum of magnitudes of old and new values \\
    \bottomrule
  \end{tabular}
  \caption{\label{t:combine_mode}Combine mode descriptions.}
\end{table}

{\bf Definition of a solver for subdomain linear system.}
Some preconditioners may benefit from local modifications to the subdomain
matrix. It can be filtered to eliminate singletons and/or reordered.
Reordering will often improve performance during incomplete factorization setup,
and improve the convergence. The matrix reordering algorithms specified in {\tt
"schwarz: reordering list"} are provided by \zoltantwo.  At the present time,
the only available reordering algorithm is RCM (reverse Cuthill-McKee). Other
orderings will be supported by the Zoltan2 package in the future.

To solve linear systems involving $A_i$ on each subdomain, a user can specify
the inner solver by setting {\tt "inner preconditioner name"} parameter (or any
of its aliases) which allows to use any \ifpacktwo preconditioner. These include
but are not necessarily limited to the preconditioners in
Table~\ref{t:schwarz_inner}.

\begin{table}[htbp]
  \centering
  \begin{tabular}{p{5.0cm} p{10.5cm}}
    \toprule
    Inner solver type       & Description \\
    \midrule
    ``DIAGONAL''            & Diagonal scaling \\
    ``RELAXATION''          & Point relaxation (see~\ref{s:relaxation}) \\
    ``BLOCK\_RELAXATION''   & Block relaxation (see~\ref{s:block_relaxation}) \\
    ``CHEBYSHEV''           & Chebyshev iteration (see~\ref{s:Chebyshev}) \\
    ``RILUK''               & ILU($k$) (see~\ref{s:ILU}) \\
    ``ILUT''                & ILUT (see~\ref{s:ILUT}) \\
    ``FAST\_ILU''             & FastILU (see~\ref{s:FastILU}) \\
    ``FAST\_IC''              & FastIC (see~\ref{s:FastILU}) \\
    ``FAST\_ILDL''            & FastILDL(see~\ref{s:FastILU}) \\
    ``AMESOS2''             & \amesostwo's interface to sparse direct solvers \\
    ``DENSE'' or ``LAPACK'' & LAPACK's LU factorization for a dense representation of a subdomain matrix \\
    ``CUSTOM''              & User provided inner solver \\
    % ``RBILUK''
    \bottomrule
  \end{tabular}
  \caption{\label{t:schwarz_inner}Additive Schwarz solver preconditioner types.}
\end{table}

The following parameters are used in the Schwarz method:

\ccc{schwarz: inner preconditioner name}
    {string}
    {none}
    {The name of the subdomain solver.}
\cccc{inner preconditioner name}
    {Same as {\tt "schwarz: inner preconditioner name"}.}
\cccc{schwarz: subdomain solver name}
    {Same as {\tt "schwarz: inner preconditioner name"}.}
\cccc{subdomain solver name}
    {Same as {\tt "schwarz: inner preconditioner name"}.}
\ccc{schwarz: inner preconditioner parameters}
    {\parameterlist}
    {empty}
    {Parameters for the subdomain solver. If not provided, the subdomain solver
     will use its specific default parameters.}
\cccc{inner preconditioner parameters}
    {Same as {\tt "schwarz: inner preconditioner parameters"}.}
\cccc{schwarz: subdomain solver parameters}
    {Same as {\tt "schwarz: inner preconditioner parameters"}.}
\cccc{subdomain solver parameters}
    {Same as {\tt "schwarz: inner preconditioner parameters"}.}
\ccc{schwarz: combine mode}
    {string}
    {``ZERO''}
    {The rule for combining incoming data with existing data in overlap regions.
     Accepted values: see Table~\ref{t:combine_mode}.}
\ccc{schwarz: overlap level}
    {int}
    {0}
    {The level of overlap (0 corresponds to no overlap).}
\ccc{schwarz: num iterations}
    {int}
    {1}
    {Number of iterations to perform.}
\ccc{schwarz: use reordering}
    {bool}
    {\false}
    {If true, local matrix is reordered before computing subdomain solver. \trilinos must have been built with
     \zoltantwo and \xpetra enabled.}
\ccc{schwarz: reordering list}
    {\parameterlist}
    {empty}
    {Specify options for a \zoltantwo reordering algorithm to use. See {\tt
     "order\_method"}. {\em You should consider the
     \href{http://trilinos.org/docs/dev/packages/zoltan2/doc/html/z2_parameters.html}{\zoltantwo
     documentation} to be the final authority.}}
\ccc{order\_method}
    {string}
    {``rcm''}
    {Reordering algorithm. Accepted values: ``rcm'', ``minimum\_degree'',
     ``natural'', ``random'', or ``sorted\_degree''. Only used in {\tt
     "schwarz: reordering list"} sublist.}
\cccc{schwarz: zero starting solution}
    {See {\tt "relaxation: zero starting solution"}.}
\ccc{schwarz: filter singletons}
    {bool}
    {\false}
    {If true, exclude rows with just a single entry on the calling process.}
\cccc{schwarz: subdomain id}
    {Currently has no effect.}
\cccc{schwarz: compute condest}
    {Currently has no effect. For backwards compatibility only.}
\ccc{schwarz: update damping}
    {double}
    {1.0}
    {The amount by which to damp the updates from the Schwarz solve
      (1.0 is no damping).}
\section{Hiptmair}

\ifpacktwo{} implements Hiptmair algorithm of~\cite{Hiptmair1997}. The method
operates on two spaces: a primary space and an auxiliary space. This situation
arises, for instance,  when preconditioning Maxwell's equations discretized by
edge elements. It is used in \muelu~\cite{MueLu} ``RefMaxwell''
solver~\cite{RefMaxwell}.

Hiptmair's algorithm does not use \texttt{Ifpack2::Factory} interface for
construction.  Instead, a user must explicitly call the constructor
\begin{lstlisting}[language=C++]
  Teuchos::RCP<Tpetra::CrsMatrix<> > A, Aaux, P;
  // create A, Aaux, P here ...
  Teuchos::ParameterList paramList;
  paramList.set("hiptmair: smoother type 1", "CHEBYSHEV");
  // ...
  RCP<Ifpack2::Ifpack2Preconditioner<> > ifpack2Preconditioner =
    Teuchos::rcp(new Ifpack2::Hiptmair(A, Aaux, P);
  ifpack2Preconditioner->setParameters(paramList);
\end{lstlisting}
\noindent Here, $A$ is a matrix in the primary space, $Aaux$ is a matrix in
auxiliary space, and $P$ is a prolongator/restrictor between the two spaces.

The following parameters are used in the Hiptmair method:

\ccc{hiptmair: smoother type 1}
    {string}
    {"CHEBYSHEV"}
    {Smoother type for smoothing the primary space.}
\ccc{hiptmair: smoother list 1}
    {\parameterlist}
    {empty}
    {Smoother parameters for smoothing the primary space.}
\ccc{hiptmair: smoother type 2}
    {string}
    {"CHEBYSHEV"}
    {Smoother type for smoothing the auxiliary space.}
\ccc{hiptmair: smoother list 2}
    {\parameterlist}
    {empty}
    {Smoother parameters for smoothing the auxiliary space.}
\ccc{hiptmair: pre or post}
    {string}
    {``both''}
    {\ifpacktwo{} always relaxes on the auxiliary space. ``pre'' (``post'') means
     that it relaxes on the primary space before (after) the relaxation on the
     auxiliary space. ``both'' means that we do both ``pre'' and ``post''.}
\cccc{hiptmair: zero starting solution}
    {See {\tt "relaxation: zero starting solution"}.}


    %-----------------------------%
    % \chapter{Performance}
    % \section{How to wring the last bit of performance out of Ifpack2 (jhu,csiefer)}
\section{Published results}
Cite the PPL paper \cite{Lin2014}.


    %\nocite{*}

    % ---------------------------------------------------------------------- %
    % References
    %
    \clearpage
    % If hyperref is included, then \phantomsection is already defined.
    % If not, we need to define it.
    \providecommand*{\phantomsection}{}
    \phantomsection
    \addcontentsline{toc}{chapter}{References}
    \bibliographystyle{plain}
    \bibliography{ifpack2guide}


    % ---------------------------------------------------------------------- %
    %
    \appendix
    
%%%%%%%%%%%%%%%%%%%%%%%%%%%%%%%%%%%%%%%%%%%%%%%%%%%%%%%%
%
% Copyright (c) 2003-2012 by University of Queensland
% Earth Systems Science Computational Center (ESSCC)
% http://www.uq.edu.au/esscc
%
% Primary Business: Queensland, Australia
% Licensed under the Open Software License version 3.0
% http://www.opensource.org/licenses/osl-3.0.php
%
%%%%%%%%%%%%%%%%%%%%%%%%%%%%%%%%%%%%%%%%%%%%%%%%%%%%%%%%


% \chapter{Notation}

%
% $Id: notation.tex 1318 2007-09-26 04:39:14Z ksteube $
%
%%%%%%%%%%%%%%%%%%%%%%%%%%%%%%%%%%%%%%%%%%%%%%%%%%%%%%%
%
%           Copyright 2003-2007 by ACceSS MNRF
%       Copyright 2007 by University of Queensland
%
%                http://esscc.uq.edu.au
%        Primary Business: Queensland, Australia
%  Licensed under the Open Software License version 3.0
%     http://www.opensource.org/licenses/osl-3.0.php
%
%%%%%%%%%%%%%%%%%%%%%%%%%%%%%%%%%%%%%%%%%%%%%%%%%%%%%%%
%

\section{Einstein Notation}
\label{EINSTEIN NOTATION}

Compact notation is used in equations such continuum mechanics and linear algebra; it is known as Einstein notation or the Einstein summation convention. It makes the conventional notation of equations involing tensors more compact, by shortening and simplifying them.

There are two rules which make up the convention:

firstly, the rank of the tensor is represented by an index. For example, $a$ is a scalar; $b\hackscore{i}$ represents a vector; and $c\hackscore{ij}$ represents a matrix.

Secondly, if an expression contains subscripted variables, they are assumed to be summed over all possible values, from $0$ to $n$. For example, for the following expression:



\begin{equation}
y = a\hackscore{0}b\hackscore{0} + a\hackscore{1}b\hackscore{1} + \ldots + a\hackscore{n}b\hackscore{n}
\label{NOTATION1}
\end{equation}

can be represented as:

\begin{equation}
y = \sum\hackscore{i=0}^n  a\hackscore{i}b\hackscore{i}
\label{NOTATION2}
\end{equation}

then in Einstein notion:

\begin{equation}
y = a\hackscore{i}b\hackscore{i}
\label{NOTATION3}
\end{equation}

Another example:

\begin{equation}
\nabla p = \frac{\partial p}{\partial x\hackscore{0}}\textbf{i} + \frac{\partial p}{\partial x\hackscore{1}}\textbf{j} + \frac{\partial p}{\partial x\hackscore{2}}\textbf{k}
\label{NOTATION4}
\end{equation}

can be expressed in Einstein notation as:

\begin{equation}
\nabla p = p,\hackscore{i}
\label{NOTATION5}
\end{equation}

where the comma ',' indicates the partial derivative.

For a tensor:

\begin{equation}
\sigma \hackscore{ij}= 
\left[ \begin{array}{ccc}
\sigma\hackscore{00} & \sigma\hackscore{01} & \sigma\hackscore{02} \\
\sigma\hackscore{10} & \sigma\hackscore{11} & \sigma\hackscore{12} \\
\sigma\hackscore{20} & \sigma\hackscore{21} & \sigma\hackscore{22} \\
\end{array} \right]
\label{NOTATION6}
\end{equation}


The $\delta\hackscore{ij}$ is the Kronecker $\delta$-symbol, which is a matrix with ones for its diagonal entries ($i = j$) and zeros for the remaining entries ($i \neq j$).

\begin{equation}
\delta \hackscore{ij} = 
\left \{ \begin{array}{cc}
1, & \mbox{if $i = j$} \\
0, & \mbox{if $i \neq j$} \\
\end{array}
\right.
\label{KRONECKER}
\end{equation}

\chapter{Non-Linear Partial Differential Equations}
\label{APP: NEWTON}

The solution $u_i$ is given as a solution of
the nonlinear equation
\begin{equation} \label{APP NEWTON EQU 40}
\int_{\Omega} v_{i,j} \cdot X_{ij} + v_{i} \cdot Y_{i} \; dx
+ \int_{\partial \Omega}  v_{i} \cdot y_{i} \; ds  = 0 
\end{equation}
for all smooth $v_i$ with $v_i=0$ where $q_i>0$ and
\begin{equation} \label{APP NEWTON EQU 40b}
u_i=r_i \mbox{ where } q_i>0
\end{equation}
where $X_{ij}$ and $Y_i$ are non-linear functions of the solution $u_k$ and its gradient $u_{k,l}$
and $y_i$ is a function of solution $u_k$. For further convenience we will use the 
notation  
\begin{equation} \label{APP NEWTON EQU 40c}
<F(u),v> :=\int_{\Omega} v_{i,j} \cdot X_{ij} + v_{i} \cdot Y_{i} \; dx
+ \int_{\partial \Omega}  v_{i} \cdot y_{i} \; ds
\end{equation}
for all smooth $v$ on $\Omega$. If one interprets $F(u)$ as defined above as a functional 
over the set of admissible functions $v$  
equation~(\ref{APP NEWTON EQU 40}) can be written in compact formulation
\begin{equation} \label{APP NEWTON EQU 40d}
F(u)= 0 
\end{equation}

\section{Newton-Raphson Scheme}
This equation is iteratively solved by the Newton-Raphson method\index{Newton-Raphson method}, see \cite{lit171}.
Starting with the initial guess
$u^{(0)}$ the sequence
\begin{equation} \label{APP NEWTON EQU 43}
  u^{(\nu)}= u^{(\nu-1)} - \delta^{(\nu-1)}
\end{equation}
for $\nu=1,2,\ldots \;$ generates the (general) Newton-Raphson iteration for the
solution $u$. The correction $\delta^{(\nu-1)}$ is the solution of the linear problem
\begin{equation} \label{APP NEWTON EQU 43b0}
< \fracp{F}{u^{(\nu-1)}} \delta^{(\nu-1)} ,v > = <F(u^{(\nu-1)}),v>
\end{equation}
for all smooth $v$ on $\Omega$ with $v_i=0$ where $q_i>0$. 
where
\begin{equation} \label{APP NEWTON EQU 43b}
< \fracp{F}{u} \cdot \delta ,v > = 
\int_{\Omega} \left( \fracp{X_{ij}}{u_{k,l}} v_{i,j}\delta_{k,l} + 
\fracp{X_{ij}}{u_{k}} v_{i,j}\delta_{k} + \fracp{Y_{i}}{u_{k,l}} v_{i}\delta_{k,l} + 
\fracp{Y_{i}}{u_{k}} v_{i}\delta_{k} \right) \; dx 
+ \int_{\partial \Omega} 
\fracp{y_{i}}{u_{k}} v_{i}\delta_{k} \; ds 
\end{equation}
It is assumed that the initial guess $u^{(0)}$ fulfills the constraint~(\ref{APP NEWTON EQU 40b}). 
The $\delta^{(\nu-1)}$ has to fullfill the homogeneous constraint. 
Notice that the calculation of $\delta^{(\nu-1)}$ requires the solution of a linear PDE
as presented in section~\ref{SEC LinearPDE}.

The Newton iteration should be stopped in the $k$-th step if for
all components of the solution the error of the Newton
approximation is lower than the given relative tolerance {rtol}:
\begin{equation}\label{APP NEWTON EQU 61}
    \| u_{i} - u_{i}^{(\nu)} \|_{\infty} \le \mbox{rtol} \cdot \|u_{i} \|_{\infty}  \; ,
\end{equation}
for all components $i$ 
where $\|. \|_{\infty}$ denoted the $L^{sup}$ norm. To measure the quality of the solution approximation
on the level of the equation we introduce the weak norm
\begin{equation}\label{APP NEWTON EQU 62}
  \| F(u) \|_{i} := \sup_{v , v=0 \mbox{ where } q_{i}>0 } \frac{<F(u), ve_{i}>}{\|v\|_1}
\end{equation}
where $(e_{i})_{j}=\delta_{ij}$ and $\|v\|_1=\int_{\Omega} |v| \,dx$ is the $L^1$ norm of $v$
\footnote{In practice a discretization method is applied to solve the update $\delta^{(\nu-1)}$.
In this case also an approximation of $\| F(u) \|_{i}$ is calculated taking the maximum over all
base function used to represent the solution $u$.}.

The stopping criterion (\ref{APP NEWTON EQU 61}) is changed to the level of
equation.  We use the reasonable heuristic but mathematically 
incorrect argument that the change on the level of the solution and
the change on the level of the equation are proportional:
\begin{equation}\label{APP NEWTON EQU 64}
 \frac{ \| u_i - u^{(\nu)}_i \|_{\infty} }{ \| 0 - F(u^{(\nu)}) \|_{i} } =
 \frac{ \| \delta^{(\nu)}_i \|_{\infty} }{ \| F(u^{(\nu)}) - F(u^{(\nu-1)}) \|_{i} }  \; .
\end{equation}
where we assume that that component $\nu$ $F(u)$ is mainly controlled by component $\nu$ of the solution.


We assume that the term $F(u^{(\nu)})$ can be neglected versus
$F(u^{(\nu-1)})$ since $u^{(\nu)}$ is a better approximation,
and use the stopping criterion in the formulation:
\begin{equation} \label{APP NEWTON EQU 65}
        \| F(u^{(\nu)}) \|_i \le
  \frac{ \| F(u^{(\nu-1)})\|_{i} \cdot  \|u_{i} \|_{\infty} }  { \| \delta^{(\nu)}_i \| _{\infty} }
   \,\mbox{\it rtol}\, =:\, \mbox{\it qtol}_i \; ,
\end{equation}
which has to hold for all components $\nu$.
Now {\it qtol} defines a tolerance for the level of equation.  This stopping criterion is not free of problems, because a
decrease of the defect $F(u^{(\nu)})$ coupled with a constant
correction $\delta^{(\nu)}$ suggests a good approximation.
But the quality of the approximation $u^{(\nu)}$ is ensured if the
Newton iteration converges quadratically.  This convergence behavior
is given by the error estimation:
\begin{equation}
    \| u - u^{(\nu)} \|_{\infty} \le C \;
    \| u - u^{(\nu-1)} \|_{\infty}^2
\end{equation}
with a positive value $C$ \cite{lit171,lit159}.  Therefore a quadratic
convergence of the Newton iteration can be assumed if the corrections
of the current and the last step fulfill the following condition:
\begin{equation} \label{APP NEWTON EQU 66}
  \max_{i}
  \frac{\| \delta^{(\nu)}_i \|_{\infty} }{\| \delta^{(\nu-1)}_i \|_{\infty} } 
                           < \frac{1}{5} \;,
\end{equation}
where the limit $\frac{1}{5}$ was found by a large number of experiments.
The approximation $u^{(\nu)}$ is accepted if the conditions
(\ref{APP NEWTON EQU 65}) and (\ref{APP NEWTON EQU 66}) hold.  Consequently a
safe approximation requires at least two Newton steps.

To stop a divergent iteration, which occurs for a bad initial solution,
the norms of the defects for the $(k-1)$-th and $k$-th
Newton step are compared.  Here we use the estimation
\begin{equation} \label{APP NEWTON EQU 67}
  \| F(u^{(\nu)}) \|_i \le
  \gamma \| F(u^{(\nu-1)}) \|_i
\end{equation}
for the defects.  The value $0<\gamma<1$ depends on $F$
and the distance of the initial guess $u^{(0)}$ to the true solution $u$.
Since the constant $\gamma$ is unknown and can be close to one,
convergence is assumed if the following (weaker) condition holds
for all components $i$:
\begin{equation} \label{APP NEWTON EQU 68}
  \| F(u^{(\nu)}) \|_i
  < \| F(u^{(\nu-1)}) \|_i \; .
\end{equation}
If condition (\ref{APP NEWTON EQU 68}) fails, divergence may begin. Therefore
under-relaxation is started.  Beginning with $\omega=1$
the Newton-Raphson iteration is computed by
\begin{equation} \label{APP NEWTON EQU 69}
  u^{(\nu)} = u^{(\nu-1)} - \omega \, \delta^{(\nu)}
\end{equation}
instead of (\ref{APP NEWTON EQU 43}).  If this new iteration fulfills the
condition (\ref{APP NEWTON EQU 68}) of decreasing defect, we accept it.  Otherwise
we put $\omega \rightarrow \frac{\omega}{2}$, recompute $u^{(\nu)}$ from equation
(\ref{APP NEWTON EQU 69}) and retry condition (\ref{APP NEWTON EQU 68}) for the
new $u^{(\nu)}$, and so on until either condition (\ref{APP NEWTON EQU 68})
holds or $\omega$ becomes to small ($\omega < \omega_{lim}=0.01$).  In the latter case the
iteration gives up. The under-relaxation
converges only linearly for $\omega<1$. it is a rather robust
procedure.
 which switches back to $\omega=1$ as soon as possible.
The price for the robustness is the additional computation of the
defects.

Due to the quadratic convergence near the solution the error decreases
rapidly.  Then the solution will not change much and it will not be
necessary to mount a new coefficient matrix in each iteration step.
This algorithm is called the simplified Newton method.  It converges
linearly by
\begin{equation} 
    \| u_i - u^{(\nu)}_i \|_{\infty} \le \gamma^{\nu}
    \| u_i - u ^{(0)}_i \|_{\infty} \; ,
\end{equation}
where $\gamma$ equals the $\gamma$ in the estimation (\ref{APP NEWTON EQU 67}).
If the general iteration converges quadratically
(condition (\ref{APP NEWTON EQU 66}) holds) and we have
\begin{equation}
   \| F(u^{(\nu)}) \|_i < 0.1
              \| F(u^{(\nu-1)}) \|_i
\end{equation}
for all components $\nu$,
we can expect $\gamma \le 0.1$.  Then the simplified iteration produces
one digit in every step and so we change from the general to the
simplified method.  The `slow' convergence requires more iteration steps,
but the expensive mounting of the coefficient matrix is saved.

The lienar PDE~(\ref{APP NEWTON EQU 43b}) is solved with with a certain tolerance, namely when
the defect of the current
approximation of $\delta^{(\nu)}$ relative to the defect of
the current Newton iteration is lower than {LINTOL}.  To avoid
wasting CPU time the FEMLIN iteration must be controlled by an
efficient stopping criterion.  We set
\begin{equation}
  \mbox{LINTOL} = 0.1 \cdot \max ( \left( \frac{\|\delta^{(\nu)}_i\|_{\infty}}{\|u_i^{(\nu)}\|_{\infty}} \right)^2 
,\min_{i} \frac{\mbox{qtol}_i}{\|F(u^{(\nu)})\|_{i}} )
\end{equation}
but restrict {LINTOL} by
\begin{equation}
       10^{-4} \le \mbox{LINTOL} \le 0.1 \quad \mbox{ and } \quad
             \mbox{LINTOL}=0.1 \; \mbox{ for }\nu=0  \;.
\end{equation}
The first term means that it would be useless to compute digits by the
linear solver, which are overwritten by the next Newton step.  In the
region of quadratic convergence the number of significant digits is
doubled in each Newton step, i.e.~later digits are overwritten by the
following Newton-Raphson correction \cite{nfs28}.  The second 
mean that no digits should be computed which are in significance
below the prescribed tolerance {\it rtol}.  The number $0.1$ is a `safety factor' to take care
of the coarse norm estimations. Figure~\ref{APP NEWTON PIC 61} shows the workflow of
the Newton-Raphson update aglorithm.

\begin{figure}
\begin{center}
{
\unitlength0.92mm
\begin{picture}(200,235) \thicklines

\put(-10,-065){\begin{picture}(170,280) \thicklines

\newsavebox{\BZar}
\savebox{\BZar}(0,0) {
   \thicklines
   \put(0,10){\line(-3,-1){30}}
   \put(0,10){\line(3,-1){30}}
   \put(0,-10){\line(-3,1){30}}
   \put(0,-10){\line(3,1){30}}
   \put(0,20){\vector(0,-1){10}}
   \put(0,-10){\vector(0,-1){10}}
   \put(30,0){\vector(1,0){25}} }
\newsavebox{\BZal}
\savebox{\BZal}(0,0) {
   \thicklines
   \put(0,10){\line(-3,-1){30}}
   \put(0,10){\line(3,-1){30}}
   \put(0,-10){\line(-3,1){30}}
   \put(0,-10){\line(3,1){30}}
   \put(0,20){\vector(0,-1){10}}
   \put(0,-10){\vector(0,-1){10}}
   \put(-30,0){\vector(-1,0){10}} }

\put(20,285){\framebox(60,20){\parbox{48mm}
           {Start: \\ $\nu=0$ , $\omega=1$ \\
            calculate $F(u^{(0)})$ }} }
\put(50,285){\vector(0,-1){10}}
\put(20,265){\framebox(60,10){\parbox{48mm}
           {next iteration: $\nu \leftarrow \nu+1$}} }
\put(50,265){\vector(0,-1){10}}
\put(20,240){\framebox(60,15){\parbox{54mm}
           {\hspace*{2.00mm} Solve \\
            $\fracp{F}{ u^{(\nu-1)}} \delta^{(\nu)} =
                                 F(u^{(\nu-1)})$}} }
\put(50,240){\vector(0,-1){10}}
\put(20,220){\framebox(60,10){\parbox{48mm}
           {$\omega = \min(2\,\omega,1)$}} }
\put(50,220){\vector(0,-1){10}}
  \put(120,225){\line(-3,-1){30}}
  \put(120,225){\line(3,-1){30}}
  \put(120,205){\line(-3,1){30}}
  \put(120,205){\line(3,1){30}}
  \put(110,214){$\omega \le \omega_{lim}$ ?}
  \put(83,219){no}
  \put(153,219){yes}
  \put(090,215){\vector(-1,0){40}}
  \put(150,215){\line(1,0){05}}
\put(20,200){\framebox(60,10){\parbox{48mm}
{$u^{(\nu)} = u^{(k-1)} - \omega\delta^{(\nu)}$}} }
\put(50,180){\usebox{\BZar}}
\put(30,179){$F(u^{(\nu)}) < F(u^{(\nu-1)})$ ?}
\put(83,184){no}
\put(55,165){yes}
  \put(105,175){\framebox(30,10){\parbox{23mm}
             {$\omega=\omega/2$}} }
  \put(120,185){\vector(0,1){20}}

\put(20,145){\framebox(60,15){\parbox{48mm}
           {evaluate stopping criterions~(\ref{APP NEWTON EQU 65})  \\
           and if not simplified mode~(\ref{APP NEWTON EQU 66}) } } }
\put(50,145){\vector(0,-1){10}}

\put(50,125){\usebox{\BZar}}
\put(34,120){\shortstack{stopping criterion \\ satisfied ?}}
\put(83,129){yes}
\put(55,110){no}
  \put(105,117.5){\framebox(30,15){\parbox{23mm}
             {Newton ends \\ successfully!}} }
  \put(135,125){\vector(1,0){20}}
\put(50,095){\usebox{\BZal}}
\put(26,093.5){$F(u^{(\nu)}) < 0.1 F(u^{(\nu-1)})$ ?}
\put(12,099){no}
\put(55,080){yes}
\put(20,065){\framebox(60,10){\parbox{48mm}
           {switch to simplified Newton!}} }
\put(20,070){\vector(-1,0){10}}

    \put(155,215){\vector(0,-1){60}}
    \put(140,145){\framebox(30,10){\parbox{23mm}
               {Newton fails!}} }
    \put(155,145){\vector(0,-1){070}}
    \put(155,070){\oval(40,10)\makebox(0,0){END}}

\put(010,280){\line(0,-1){210}}
\put(010,280){\vector(1,0){40}}


\end{picture} }
\end{picture} }
\end{center}

\caption{\label{APP NEWTON PIC 61}Flow diagram of the Newton-Raphson algorithm}
\end{figure}

\section{Local Sensitivity Analysis}
If the coefficients 
$X_{ij}$, $Y_i$ and $y_{i}$ in equation~(\ref{APP NEWTON EQU 40})
depend on a vector of input factors $f_i$ index{input factor} and its gradient one is interested in how the solution $u_i$
is changing if the input factors are changed. This problem is called a local sensitivity 
analysis \index{local sensitivity analysis}. If $u(f)$ denotes the 
solution of equation~(\ref{APP NEWTON EQU 40}) for the input factor $p$ 
and $u(f+\alpha \cdot g)$ denotes the solution 
for a perturbed value  $f+\alpha \cdot g$ for input factor $f$ where
$q$ denotes the direction of perturbation and $\alpha$ is the small scaling factor.
The derivative of the solution in the direction $g$ is defined as
\begin{equation}
\fracp{u}{g} : = \lim_{\alpha \rightarrow 0} \frac{ u(f+\alpha \cdot g) - u(f)}{\alpha}
\end{equation}
In practice one needs to distinguish between the cases of a spatially constant
spatially variable. In the first case $g$ is set to a unit vector while
in a second case an appropriate function needs to be given for $g$. 

The function  $\fracp{u}{g}$ is calculated from solving the equation
\begin{align} \label{APP NEWTON EQU 100} 
\int_{\Omega} \left( \fracp{X_{ij}}{u_{k,l}} v_{i,j} \left(\fracp{u_k}{g}\right)_{,l} + 
\fracp{X_{ij}}{u_{k}} v_{i,j}\fracp{u_k}{g} + \fracp{Y_{i}}{u_{k,l}} v_{i}\left(\fracp{u_k}{g}\right)_{,l}  + 
\fracp{Y_{i}}{u_{k}} v_{i}\fracp{u_k}{g}\right) \; dx 
+ \int_{\partial \Omega}  
\fracp{y_{i}}{u_{k}} v_{i}\fracp{u_k}{g}\; ds \\
+ \int_{\Omega} v_{i,j}  \left( \fracp{X_{ij}}{f_{k,l}} g_{k,l} + \fracp{X_{ij}}{f_{k}} g_k \right) 
+ v_{i} \left( \fracp{Y_{i}}{f_{k,l}} g_{k,l} +  \fracp{Y_{i}}{f_{k}} g_k \right) \; dx 
+ \int_{\partial \Omega}  v_{i}
\fracp{y_{i}}{f_{k}}  g_k\; ds
\end{align}
for all smooth $v$ on $\Omega$ with $v_i=0$ where $q_i>0$
for the unknown sensitivity $\fracp{u}{g}$. 
Notice that this equation is similar to the equation which needs to be solved for
the Newton-Raphson correction $\delta_k$, see equation~(\ref{APP NEWTON EQU 43b}).
\chapter{The Data Inversion Problem}
\label{APP: FITTER}

We solve the following nonlinear PDE for the unknown solution $u_i$:
\begin{equation} \label{APP FIT EQU 1a}
\int_{\Omega} v_{i,j} \cdot X_{ij} + v_{i} \cdot Y_{i} \; dx
+ \int_{\partial \Omega}  v_{i} \cdot y_{i} \; ds  = 0 
\end{equation}
for all smooth $v_i$ with $v_i=0$ where $q_i>0$ and
\begin{equation} \label{APP FIT EQU 1b}
u_i=r_i \mbox{ where } q_i>0
\end{equation}
where $X_{ij}$ and $Y_i$ are non-linear functions of the solution $u_k$ and its gradient $u_{k,l}$
and $y_i$ is a function of solution $u_k$. 

The equation may depend on set of parameters $p_i$. It is the task to 
give values of the parameters $p_i$ such that the solution gives the best 
approximation of given measurements. In mathematical terms we need to minimize
a so-called cost function $J$ which measures the distance of the solution
to the data over the set of valid parameters. A typical example 
for a cost function measuring the gradient of a scalar solution $u$ 
in a given direction $d_i$ against measured data $\hat{g}$ is given as
\begin{equation}\label{APP FIT EQU 2a}
J_{data}(u) = \frac{1}{2}\int_{\Omega}  \chi \cdot ( d_i u_{,i} - \hat{g})^2 dx
\end{equation} 
where $\chi$ is a weighting function which has a non-negative value where data are available. 
Typically, $\chi$ is to be the inverse of the square of the deviation of the measurements or zero.
If the parameter $p_i$ is spatially variable a regularization term needs to be
added into the cost function in order to get a unique solution. Typically,
for a scalar parameter $p$ the regularization term takes the from
\begin{equation}\label{APP FIT EQU 3}
J_{reg}(p) =  \int_{\Omega} \frac{1}{2} \cdot (p_{,i}p_{,i})^{2} dx
\end{equation} 
The cost function to be minimized then takes the from
\begin{equation}\label{APP FIT EQU 2a}
J(p,u) = \frac{1}{2}\int_{\Omega}  \chi \cdot ( d_i u_{,i} - \hat{g})^2
+ (p_{,i}p_{,i})^{2} \;
dx
\end{equation} 
A more general from is given as 
\begin{equation}\label{APP FIT EQU 4}
J(u,p) = \int_{\Omega} H\; dx +  \int_{\partial \Omega} h \; ds
\end{equation} 
where $H$ is a scalar, possibly spatially variable  function 
of the solution $u_i$ and the parameter $p_i$ and their gradients
and  $H$ is a scalar, possibly spatially variable  function 
of the solution $u_i$ and the parameter $p_i$. For example~(\ref{APP FIT EQU 2a})
one has
\begin{equation}\label{APP FIT EQU 2a}
H = \frac{1}{2} \chi \cdot (d_i u_{,i} - \hat{g})^2 +  \frac{1}{2} (p_{,i}p_{,i})^{2} 
\end{equation} 
So task is to minimize the cost function $J$ over $u$ and $p$  
subject to the PDE~(\ref{APP FIT EQU 1a} connection $p$ and $u$. The secondary condition
is mixed into the cost function using a Lagrangean multiplier before the variation is calculated:
\begin{equation}\label{APP FIT EQU 5}
J(u,p,\lambda) = \int_{\Omega} H \; dx + \int_{\partial \Omega} h \; ds
+ \int_{\Omega} \lambda_{i,j} \cdot X_{ij} + \lambda_{i} \cdot Y_{i} \; dx
+ \int_{\partial \Omega}  \lambda_{i} \cdot y_{i} \; ds
\end{equation}
Notice that the Lagrangean multiplier needs to fullfull the constraint
\begin{equation} \label{APP FIT EQU 1b}
\lambda_{i}=0 \mbox{ where } q_i>0
\end{equation}

We can rearrange $J$ to 
\begin{equation}\label{APP FIT EQU 5}
J(u,p,\lambda) = \int_{\Omega} Z \; dx 
+  \int_{\partial \Omega} z \; ds
\end{equation}
with 
\begin{align}\label{APP FIT EQU 6}
 Z =  H+ \lambda_{i,j} \cdot X_{ij} +  \lambda_{i} \cdot Y_{i} \\
z= h  + \lambda_{i} \cdot y_{i} 
\end{align}

We are taking variation along $p$:
\begin{equation}\label{APP FIT EQU 10}
\int_{\Omega} \fracp{Z}{p_{i,j}}  \cdot  (\delta p)_{i,j} + \fracp{Z}{p_{i}} \cdot  (\delta p)_{i}
\; dx +  \int_{\partial \Omega} \fracp{z}{p_{i}}  \cdot  (\delta p)_{i} \; ds =0
\end{equation}
along $u$:
\begin{align}\label{APP FIT EQU 11}
\int_{\Omega} \fracp{Z}{u_{i,j}}  \cdot (\delta u)_{i,j} + \fracp{Z}{u_{i}} \cdot  (\delta u)_{i}
\; dx +  \int_{\partial \Omega} \fracp{z}{u_{i}}  \cdot  (\delta u)_{i} \; ds =0
\end{align}
and $\lambda$:
\begin{equation}\label{APP FIT EQU 12}
\int_{\Omega} X_{ij}  \cdot (\delta \lambda)_{i,j}  +  Y_{i}  \cdot (\delta \lambda)_{i}  \; dx
+ \int_{\partial \Omega}   y_{i}  \cdot  (\delta \lambda)_{i}  \; ds = 0
\end{equation}
This defines a system of non-linear PDEs for the unknown solution $\widehat{u} = (p,u,\lambda)$. With 
$\widehat{v} = (\delta p,\delta \lambda, \delta u)$\footnote{Notice that in comparison to the solution
the corresponding components for $u$ and $\lambda$ are swapped in order to bring strong couplings into the main-diagonal.}
we can write equations~(\ref{APP FIT EQU 10})-(\ref{APP FIT EQU 12}) in the form:
\begin{equation} \label{APP FIT EQU 13}
\int_{\Omega} \widehat{v}_{i,j} \cdot \widehat{X}_{ij} + \widehat{v}_{i} \cdot \widehat{Y}_{i} \; dx
+ \int_{\partial \Omega}  \widehat{v}_{i} \cdot \widehat{y}_{i} \; ds  = 0 
\end{equation}
with
\begin{align}\label{APP FIT EQU 15}
\widehat{X}_{:j} = \left[ \fracp{Z}{p_{:,j}}, X_{:j}, \fracp{Z}{u_{:,j}} \right] \\
\widehat{Y}_{:} = \left[ \fracp{Z}{p_{:}}, Y_{:}, \fracp{Z}{u_{:}} \right] \\
\widehat{y}_{:} = \left[ \fracp{z}{p_{:}}, y_{:}, \fracp{z}{u_{:}} \right] 
\end{align}
In some cases values for the parameter are known. So similar to the constraint~(\ref{APP FIT EQU 1b}) fro the solution
we need to observe a constraint for the parameter $p_i$:
\begin{equation} \label{APP FIT EQU 12a}
p_i=rp_i \mbox{ where } qp_i>0
\end{equation}
So for the composed solution $\widehat{u} = (p,u,\lambda)$ we need to observe the constraint
\begin{equation} \label{APP FIT EQU 12b}
\widehat{u}_i=\widehat{r}_i \mbox{ where } \widehat{q}_i>0
\end{equation}
with 
\begin{align}\label{APP FIT EQU 12c}
\widehat{q}_{:j} = \left[ qp_{:}, q_{:}, q_{:} \right] \\
\widehat{r}_{:} = \left[ rp_{:}, r_{:}, 0\right] \\
\end{align}



%%%%%%%%%%%%%%%%%%%%%%%%%%%%%%%%%%%%%%%%%%%%%%%%%%%%%%%%
%
% Copyright (c) 2009 by University of Queensland
% Earth Systems Science Computational Center (ESSCC)
% http://www.uq.edu.au/esscc
%
% Primary Business: Queensland, Australia
% Licensed under the Open Software License version 3.0
% http://www.opensource.org/licenses/osl-3.0.php
%
%%%%%%%%%%%%%%%%%%%%%%%%%%%%%%%%%%%%%%%%%%%%%%%%%%%%%%%%

\section{Changes from previous releases}
\label{app:changes}

\subsection*{2.0 to 3.0}
The major change here was replacing \module{numarray} with \numpy.
For general instructions on converting scripts to use numpy see \url{http://www.stsci.edu/resources/software_hardware/numarray/numarray2numpy.pdf}.
The specific changes to \escript are:
\begin{itemize}
  \item getValueOfDataPoint() which returned a \module{numarray}.array has been replaced by
 getTupleForDataPoint() which returns a \PYTHON tuple containing
the components of the data point. In the case of matricies or higher ranked data, the tuples will be nested.
 \item getValueOfGlobalDataPoint has similarly been replaced by getTupleForGlobalDataPoint().
 \item integrate(data) now returns a \numpyNDA instead of a \module{numarray}.array.
\end{itemize}
Any python methods which previously accepted \module{numarray} objects will accept \numpy objects instead.



%%%%%%%%%%%%%%%%%%%%%%%%%%%%%%%%%%%%%%%%%%%%%%%%%%%%%%%%%%%%%%%%%%%%%%%%%%%%%%
% Copyright (c) 2003-2015 by The University of Queensland
% http://www.uq.edu.au
%
% Primary Business: Queensland, Australia
% Licensed under the Open Software License version 3.0
% http://www.opensource.org/licenses/osl-3.0.php
%
% Development until 2012 by Earth Systems Science Computational Center (ESSCC)
% Development 2012-2013 by School of Earth Sciences
% Development from 2014 by Centre for Geoscience Computing (GeoComp)
%
%%%%%%%%%%%%%%%%%%%%%%%%%%%%%%%%%%%%%%%%%%%%%%%%%%%%%%%%%%%%%%%%%%%%%%%%%%%%%%

\chapter{Escript researchers and developers by release}

\begin{center}
\begin{tabular}{r|lr|}
& Releases\\ \cline{2-3}
Cihan Altinay & 2.0 & Current \\
Joel Fenwick & 2.0 & Current \\
Lutz Gross & 1.0 & Current \\
Jaco du Plessis & 3.4.2 & Current \\
Simon Shaw & 3.4.1 & Current \\
	\cline{1-3}  	
Artak Amirbekyan & 2.0 & 3.2.1 \\
Imran Syed Azeezullah & 1.0 & 1.0 \\
Vince Boros & 3.2.1 & 4.0 \\
Paul Cochrane & 1.0 & 1.0 \\
Matt Davies & 1.0 & 2.0 \\
Lin Gao & 2.0 & 3.3 \\
Jon Gui & 2.0 & 2.0 \\
Derek Hawcroft & 1.0 & 2.0 \\
Peter Hornby & 1.0 & 2.0 \\
Azadeh Salehi & 3.2.1 & 3.4 \\
John Smilie & 1.0 & 1.0 \\
Ken Steube & 1.0 & 2.0 \\
Elspeth Thorne & 1.0 & 2.0 \\
Brett Tully & 2.0 & 2.0 \\
Rob Woodcock & 1.0 & 2.0 \\
\cline{2-3}
\end{tabular}
\end{center}

\raggedbottom

%%%%%%%%%%%%%%%%%%%%%%%%%%%%%%%%%%%%%%%%%%%%%%%%%%%%%%%%%%%%%%%%%%%%%%%%%%%%%%
% Copyright (c) 2003-2016 by The University of Queensland
% http://www.uq.edu.au
%
% Primary Business: Queensland, Australia
% Licensed under the Open Software License version 3.0
% http://www.opensource.org/licenses/osl-3.0.php
%
% Development until 2012 by Earth Systems Science Computational Center (ESSCC)
% Development 2012-2013 by School of Earth Sciences
% Development from 2014 by Centre for Geoscience Computing (GeoComp)
%
%%%%%%%%%%%%%%%%%%%%%%%%%%%%%%%%%%%%%%%%%%%%%%%%%%%%%%%%%%%%%%%%%%%%%%%%%%%%%%

\chapter{Escript references}
\label{app:ourrefs}

If you use escript in your research we would appreciate a citation (of course
we do not require this). Possible references include:

\begin{shellCode}
@InProceedings{GROSS2010,
	author = {L. Gross and A. Amirbekyan and J. Fenwick and L. Gao 
	    and A. Mohajeri and H. M\"uhlhaus},
	title = {On lazy evaluation as a tool to optimize the 
	    efficiency of large scale numerical simulations in Python},
	booktitle = {ICCS 2010: Proceedings of the International 
	    Conference on Computational Science},
	pages = {2145--2153},
	year = {2010},
	editor = {Michael Blackman},
	publisher = {Elsevier}
	series = {Procedia Computer Science},
	month = {May},
	issn = {1877--0509},
	doi={doi:10.1016/j.procs.2010.04.240}
}
\end{shellCode}


\begin{shellCode}
@InProceedings{lazyauspdc,
	author = {Joel Fenwick and Lutz Gross},
	title = {Lazy Evaluation of PDE Coefficients in the EScript System},
	booktitle = {Parallel and Distributed Computing 2010 (AusPDC2010)},
	pages = {71--76},
	year = {2010},
	editor = {Jinjun Chen and Rajiv Ranjan},
	volume = {107},
	series = {Conferences in Research and Practice in Information Technology},
	month = {January},
	issn = {1445--1336}
}
\end{shellCode}

%This should really be a separate style and verbatim indents too far
\begin{shellCode}
@article{GROSS2006,
        author = {L. Gross and L. Bourgouin and A. J. Hale and H.-B Muhlhaus},
        title = {Interface Modeling in Incompressible Media 
	using Level Sets in Escript},
        journal = {Physics of the Earth and Planetary Interiors},
        year = 2007,
        volume = {163},
        pages = {23--34},
        month = {August},
        doi = {doi:10.1016/j.pepi.2007.04.004},
}
\end{shellCode}

\begin{shellCode}
@article{GROSS2007,
	author = {L. Gross and B. Cumming and K. Steube and D. Weatherley},
	title = {A Python Module for PDE-Based Numerical Modelling},
	journal = {PARA},
	year = {2007},
	volume = {4699},
	pages = {270--279},
	doi = {doi:10.1007/978-3-540-75755-9},
	publisher = {Springer}
}
\end{shellCode}


\chapter{Python3 Support}\label{app:py3}

\textbf{We are not dropping support for recent \pythontwo releases.} ($2.6$ or later is still supported) \\
All we are doing is preparing for the time when \pythonthree is used more widely.

\escript compiles and passes tests under \pythonthree.
However, it depends on a number of other packages for its operation.
At the moment, there are no precompiled versions of these packages available for \pythonthree.
This can be because the changes needed to support \pythonthree have not made it into the 
release branch yet. 
In the case of some Linux distributions, some packages are not built for \pythonthree yet.

Regardless, if you wish to use \escript with \pythonthree, you will need to compile it (and some of
its dependencies) yourself.
See the install guide for more details. 

\section{Impact on scripts}
We have attempted to minimise disruption and caused by supporting both \pythontwo and \pythonthree.
As long as your scripts work under the \pythontwo you don't \emph{need} to change anything.
However, you might consider the following:
\begin{itemize}
 \item Use \texttt{//} for division where you expect an integer answer. 
In \pythonthree, \texttt{/} always produces a floating point answer\footnote{Division involving escript types (eg \texttt{Data}) has always produced floating point answers. 
.}.
To use this behaviour now, add the following to the top of your script:\\
\texttt{from __future__ import division}
\item Use \texttt{print} as a function rather than a statement.
That is:  \texttt{print("x", x)}  instead of \texttt{print "x",x}.
To enable this in \pythontwo add \texttt{from __future__ import print_function} to the top of your script.
\item{Don't use \verb|<tabs>| for indentation.  The \texttt{expand} utility can help here.}
\end{itemize}

In our experience, many (but not all) changes required to get simple scripts working under \pythonthree will also
work under \pythontwo.
For more information about the differences in the languages see \url{http://wiki.python.org/moin/Python2orPython3} 
or \url{http://docs.python.org/py3k/whatsnew/3.0.html}.


    %\chapter{Historical Perspective}
	%\input{CommonHistory}


    %\chapter{Some Other Appendix}
	%\input{CommonAppendix}

    % \printindex

    %
% This is an example of how to create the distribution page. Some
% distributions are required by Sandia; e.g. the housekeeping copies.
% Depending on the type of report; e.g. CRADA, Patent Caution, etc.
% additional distribution lines may have to be added. See the
% "Guide for Preparing SAND Reports"
%
% SANDdistribution takes CA or NM as an optional argument. If given,
% the approrpiate housekeeping copies are inserted autmatically.
% Inside the SANDdistribution environment, several commands can be used
% insert the distributions for CRADA, LDRD, etc. See example below.
%
% You can leave the CA or NM option off and not use any of the SANDdist*
% commands. This will allow you to create a distribution list manually.
%
\begin{SANDdistribution}[NM]
    % Housekeeping copies necessary for every unclassified report:
    % \SANDdistCRADA	% If this report is about CRADA work
    % \SANDdistPatent	% If this report has a Patent Caution or Patent Interest
    % \SANDdistLDRD	% If this report is about LDRD work

    % Some external Addresses
    %\SANDdistExternal{1}{An Address\\ 99 $99^{th}$ street NW\\City, State}
    %\SANDdistExternal{3}{Some Address\\ and street\\City, State}
    %\SANDdistExternal{12}{Another Address\\ On a street\\City, State\\U.S.A.}
    \bigskip


    % The following MUST BE between the external and internal distributions!
    % \SANDdistClassified % If this report is classified


    % Internal Addresses
    \SANDdistInternal{1}{1320}{Michael Heroux}{1426}
    \SANDdistInternal{1}{1318}{Robert Hoekstra}{1423}
    \SANDdistInternal{1}{1318}{Erik Strack}{1426}
    \SANDdistInternal{1}{1320}{Mark Hoemmen}{1426}
    \SANDdistInternal{1}{1320}{Alicia Klinvex}{1426}
    \SANDdistInternal{1}{1320}{Paul Lin}{1426}
    \SANDdistInternal{1}{1318}{Andrey Prokopenko}{1426}
    \SANDdistInternal{1}{1322}{Christopher Siefert}{1443}

\end{SANDdistribution}

\begin{SANDdistribution}[CA]
    \SANDdistInternal{1}{9159}{Jonathan Hu}{1426}
    \SANDdistInternal{1}{9159}{Raymond Tuminaro}{1442}
\end{SANDdistribution}


\end{document}
