\chapter{Cost Function}\label{chapter:ref:inversion cost function}
The general form of the cost function minimized in the inversion is given in the form (see also Chapter~\ref{chapter:ref:Drivers})
\begin{equation}\label{REF:EQU:DRIVE:10}
J(m) = J^{reg}(m) + \sum_{f} \mu^{data}_{f} \cdot J^{f}(p^f)
\end{equation} 
where $m$ represents the level set function, $J^{reg}$ is the regularization term, see Chapter~\ref{Chp:ref:regularization},
and $J^{f}$ are a set of forward problems, see Chapter~\ref{Chp:ref:forward models} depending of 
physical parameters $p^f$.  The physical parameters $p^f$ are known functions 
of the  level set function $m$ which is the unknown to be calculated by the optimization process. 
$\mu^{data}_{f}$ are trade-off factors. It is pointed out that the regularization term includes additional trade-off factors 
The \class{InversionCostFunction} is class to define cost functions of an inversion. It is pointed out that
the \class{InversionCostFunction} class implements the \class{CostFunction} template class, see Chapter~\ref{chapter:ref:Minimization}.

In the simplest case there is a single forward model using a single physical parameter which is 
derived form single-values level set function. The following script snippet shows the creation of the
\class{InversionCostFunction} for the case of a gravity inversion:
\begin{verbatim}
p=DensityMapping(...)
f=GravityModel(...)
J=InversionCostFunction(Regularization(...), \
                        mappings=p, \
                        forward_models=f)
\end{verbatim}
The argument \verb|...| refers to an appropriate argument list.

If two forward models are coming into play using two different physical parameters 
the \member{mappings} and \member{forward_models} are defined as lists in the following form:
\begin{verbatim}
p_rho=DensityMapping(...)
p_k=SusceptibilityMapping(...)
f_mag=MagneticModel(...)
f_grav=GravityModel(...)

J=InversionCostFunction(Regularization(...), \
                        mappings=[p_rho, p_k], \
                        forward_models=[(f_mag, 1), (f_grav,0)])
\end{verbatim}
Here we define a joint inversion of gravity and magnetic data. \member{forward_models} is given as a list of
a tuple of a forward model and an index which referring to parameter in the \member{mappings} list to be used as an input.
The magnetic forward model \member{f_mag} is using the the second parameter (=\member{p_k}) in \member{mappings} list.
In this case the physical parameters are defined by a single-valued level set function. It is also possible
to link physical parameters to components of a level set function:
\begin{verbatim}
p_rho=DensityMapping(...)
p_k=SusceptibilityMapping(...)
f_mag=MagneticModel(...)
f_grav=GravityModel(...)

J=InversionCostFunction(Regularization(numLevelSets=2,...), \
                        mappings=[(p_rho,0), (p_k,1)], \
                        forward_models=[[(f_mag, 1), (f_grav,0)])   
\end{verbatim}
The \member{mappings} argument is now a list of pairs where the first pair entry specifies the parameter mapping and
the second pair entry specifies the index of the component of the level set function to be used to evaluate the parameter.
In this case the level set function has two components, where the density mapping uses the first component of the level set function 
while the susceptibility mapping uses the second component.

\section{\class{InversionCostFunction} API}\label{chapter:ref:inversion cost function:api}

The \class{InversionCostFunction} implements a \class{CostFunction} class used
to run optimization solvers, see Section~\ref{chapter:ref:Minimization: costfunction class}.
Its API is defined as follows:

\begin{classdesc}{InversionCostFunction}{regularization, mappings, forward_models}
Constructor for the inversion cost function. \member{regularization} sets the regularization to be used, see Chapter~\ref{Chp:ref:regularization}.
\member{mappings} is a list of pairs where each pair comprises of a 
physical parameter mapping (see Chapter~\ref{Chp:ref:mapping}) and an index which refers to the component of level set function
defined by the \member{regularization} to be used to calculate the corresponding physical parameter.
If the level set function has a single component the index can be omitted.
If in addition there is a single physical parameter the mapping can be given instead of a list.
\member{forward_models} is a list of pairs where the first pair component is a
forward model (see Chapter~\ref{Chp:ref:forward models}) and the second pair
component refers to the physical parameter in the \member{mappings} list
providing the physical parameter for the model.
If a single physical parameter is present the index can be omitted.
If in addition a single forward model is used this forward model can be
assigned to \member{forward_models} in replacement of a list.
\end{classdesc}

\begin{methoddesc}[InversionCostFunction]{getDomain}{}
returns the \escript domain of the inversion, see~\cite{ESCRIPT}.
\end{methoddesc}
        
\begin{methoddesc}[InversionCostFunction]{getNumTradeOffFactors}{}
returns the total number of trade-off factors.
The count includes the trade-off factors $\mu^{data}_{f}$ for the forward
models and (hidden) trade-off factors in the regularization term,
see Definition~\ref{REF:EQU:DRIVE:10}.
\end{methoddesc}

\begin{methoddesc}[InversionCostFunction]{getForwardModel}{\optional{idx=\None}}
returns the forward model with index \member{idx}.
If the cost function contains one model only argument \member{idx} can be omitted.
\end{methoddesc}
        
\begin{methoddesc}[InversionCostFunction]{getRegularization}{}
returns the regularization component of the cost function, see \class{regularization} in Chapter~\ref{Chp:ref:regularization}.
\end{methoddesc}

\begin{methoddesc}[InversionCostFunction]{setTradeOffFactorsModels}{\optional{mu=\None}}
sets the trade-off factors $\mu^{data}_{f}$ for the forward model components.
If a single model is present \member{mu} must be a floating point number.
Otherwise \member{mu} must be a list of floating point numbers.
It is assumed that all numbers are positive.
The default value for all trade-off factors is one.
\end{methoddesc}

\begin{methoddesc}[InversionCostFunction]{getTradeOffFactorsModels}{}
returns the values of the trade-off factors $\mu^{data}_{f}$ for the forward model components.
\end{methoddesc}

\begin{methoddesc}[InversionCostFunction]{setTradeOffFactorsRegularization}{\optional{mu=\None}, \optional{mu_c=\None}}
sets the trade-off factors for the regularization component of the cost function.
\member{mu} defines the trade-off factors for the level-set variation part and
\member{mu_c} sets the trade-off factors for the cross-gradient variation part.
This method is a shortcut for calling \member{setTradeOffFactorsForVariation}
and \member{setTradeOffFactorsForCrossGradient} for the underlying the
regularization.
Please see \class{Regularization} in Chapter~\ref{Chp:ref:regularization} for
more details on the arguments \member{mu} and \member{mu_c}.
\end{methoddesc}
        
\begin{methoddesc}[InversionCostFunction]{setTradeOffFactors}{\optional{mu=\None}}
sets the trade-off factors for the forward model and regularization terms.
\member{mu} is a list of positive floats. The length of the list is the total
number of trade-off factors given by the method \method{getNumTradeOffFactors}.
The first part of \member{mu} defines the trade-off factors $\mu^{data}_{f}$
for the forward model components while the remaining entries define the
trade-off factors for the regularization components of the cost function.
By default all values are set to one.
\end{methoddesc}

\begin{methoddesc}[InversionCostFunction]{getProperties}{m}
returns the physical properties from a given level set function \member{m}
using the mappings of the cost function. The physical properties are
returned in the order in which they are given in the \member{mappings} argument
in the class constructor.
\end{methoddesc}

\begin{methoddesc}[InversionCostFunction]{createLevelSetFunction}{*props}
returns the level set function corresponding to set of given physical properties.
This method is the inverse of the \method{getProperties} method.
The arguments \member{props} define a tuple of values for the physical
properties where the order needs to correspond to the order in which the
physical property mappings are given in the \member{mappings} argument in the
class constructor. If a value for a physical property is given as \None the
corresponding component of the returned level set function is set to zero.
If no physical properties are given all components of the level set function
are set to zero.
\end{methoddesc}
    
\begin{methoddesc}[InversionCostFunction]{getNorm}{m}
returns the norm of a level set function \member{m} as a floating point number.
\end{methoddesc}

\begin{methoddesc}[InversionCostFunction]{getArguments}{m}
returns pre-computed values for the evaluation of the cost function and its
gradient for a given value \member{m} of the level set function.
In essence the method collects pre-computed values for the underlying
regularization and forward models\footnote{Using pre-computed values can
significantly speed up the optimization process when the value of the cost
function and its gradient are needed for the same level set function.}.
\end{methoddesc}

\begin{methoddesc}[InversionCostFunction]{getValue}{m\optional{, *args}}
returns the value of the cost function for a given level set function \member{m}
and corresponding pre-computed values \member{args}.
If the pre-computed values are not supplied \member{getArguments} is called.
\end{methoddesc}

\begin{methoddesc}[InversionCostFunction]{getGradient}{m\optional{, *args}}
returns the gradient of the cost function at level set function \member{m}
using the corresponding pre-computed values \member{args}.
If the pre-computed values are not supplied \member{getArguments} is called.
The gradient is represented as a tuple $(Y,X)$ where in essence $Y$ represents
the derivative of the cost function kernel with respect to the level set
function and $X$ represents the derivative of the cost function kernel with
respect to the gradient of the level set function, see
Section~\ref{chapter:ref:inversion cost function:gradient} for more details.
\end{methoddesc}
       
\begin{methoddesc}[InversionCostFunction]{getDualProduct}{m, g}
returns the dual product of a level set function \member{m} with a gradient
\member{g}, see Section~\ref{chapter:ref:inversion cost function:gradient} for more details.
This method uses the dual product of the regularization.
\end{methoddesc}

\begin{methoddesc}[InversionCostFunction]{getInverseHessianApproximation}{m, g \optional{, *args}}
returns an approximative evaluation of the inverse of the Hessian operator of
the cost function for a given gradient \member{g} at a given level set function
\member{m} using the corresponding pre-computed values \member{args}.
If no pre-computed values are present \member{getArguments} is called.
In the current implementation contributions to the Hessian operator from the
forward models are ignored and only contributions from the regularization and
cross-gradient term are used.
\end{methoddesc}


\section{Gradient calculation}\label{chapter:ref:inversion cost function:gradient}
In this section we briefly discuss the calculation of the gradient and the Hessian operator.
If $\nabla$ denotes the gradient operator (with respect to the level set function $m$) 
the gradient of  $J$ is given as
\begin{equation}\label{REF:EQU:DRIVE:10b}
\nabla J(m) = \nabla J^{reg}(m) + \sum_{f} \mu^{data}_{f} \cdot \nabla J^{f}(p^f) \; .
\end{equation} 
We first focus on the calculation of $\nabla J^{reg}$. In fact the 
regularization cost function $J^{reg}$ is given through a cost function
kernel\index{cost function!kernel} $K^{reg}$ in the form
\begin{equation}\label{REF:EQU:INTRO 2a}
J^{reg}(m) = \int_{\Omega} K^{reg} \; dx
\end{equation} 
where $K^{reg}$ is a given function of the
level set function $m_k$ and its spatial derivative $m_{k,i}$. If $n$ is an increment to the level set function 
then the directional derivative of $J^{ref}$ in the direction of $n$ is given as
\begin{equation}\label{REF:EQU:INTRO 2aa}
<n, \nabla J^{reg}(m)> = \int_{\Omega} \frac{ \partial K^{reg}}{\partial m_k} n_k + \frac{ \partial K^{reg}}{\partial m_{k,i}} n_{k,i} \; dx
\end{equation} 
where $<.,.>$ denotes the dual product, see Chapter~\ref{chapter:ref:Minimization}. Consequently, the gradient $\nabla J^{reg}$
can be represented by a pair of values $Y$ and $X$ 
\begin{equation}\label{ref:EQU:CS:101}
\begin{array}{rcl}
  Y_k & = & \displaystyle{\frac{\partial K^{reg}}{\partial m_k}} \\
   X_{ki} & = & \displaystyle{\frac{\partial K^{reg}}{\partial m_{k,i}}} 
\end{array}
\end{equation} 
while the dual product $<.,.>$ of a level set increment $n$ and a gradient increment $g=(Y,X)$ is given as
\begin{equation}\label{REF:EQU:INTRO 2aaa}
<n,g> = \int_{\Omega} Y_k n_k + X_{ki} n_{k,i} \; dx
\end{equation} 
We also need to provide (an approximation of) the value $p$ of the inverse of the Hessian operator $\nabla \nabla J$
for a given gradient increment $g=(Y,X)$. This means we need to (approximatively) solve the variational problem 
\begin{equation}\label{REF:EQU:INTRO 2b}
<n,\nabla \nabla J p > = \int_{\Omega} Y_k n_k + X_{ki} n_{k,i} \; dx
\end{equation} 
for all increments $n$ of the level set function. If we ignore contributions 
from the forward models the left hand side takes the form
\begin{equation}\label{REF:EQU:INTRO 2c}
<n,\nabla \nabla J^{reg} p > = \int_{\Omega} 
\displaystyle{\frac{\partial Y_k}{\partial m_l}} p_l n_k +
\displaystyle{\frac{\partial Y_k}{\partial m_{l,j}}} p_{l,j} n_k +
\displaystyle{\frac{\partial X_{ki}}{\partial m_l}} p_l n_{k,i} +
\displaystyle{\frac{\partial X_{ki}}{\partial m_{l,j}}} p_{l,j} n_{k,i} 
\; dx
\end{equation}  We follow the concept as outlined in section~\ref{chapter:ref:inversion cost function:gradient}.
Notice that equation~\ref{REF:EQU:INTRO 2b} defines a system of linear PDEs
which is solved using \escript \class{LinearPDE} class. In the \escript notation we need to provide 
\begin{equation}\label{ref:EQU:REG:600}
\begin{array}{rcl}
 A_{kilj} & = &  \displaystyle{\frac{\partial X_{ki}}{\partial m_{l,j}}}  \\
 B_{kil} & = &  \displaystyle{\frac{\partial X_{ki}}{\partial m_l}}   \\
 C_{klj} & = &  \displaystyle{\frac{\partial Y_k}{\partial m_{l,j}}}    \\
  D_{kl} & = & \displaystyle{\frac{\partial Y_k}{\partial m_l}}    \\ 
\end{array}
\end{equation} 
The calculation of the gradient of the forward model component is more complicated:
the data defect $J^{f}$ for forward model $f$ is expressed use a cost function kernel $K^{f}$ 
\begin{equation}\label{REF:EQU:INTRO 2bb}
J^{f}(p^f) = \int_{\Omega} K^{f} \; dx
\end{equation} 
In this case the cost function kernel $K^{f}$ is a function of the 
physical parameter $p^f$, which again is a function of the level-set function,
and the state variable $u^f_{k}$ and its gradient $u^f_{k,i}$. For the sake of a simpler 
presentation the upper index $f$ is dropped.
 
The gradient $\nabla_{p} J$ of the  $J$ with respect to
the physical property $p$ is given as 
\begin{equation}\label{REF:EQU:costfunction 100b}
<q, \nabla_{p} J(p)> =    \int_{\Omega}
\displaystyle{\frac{\partial K }{\partial u_k }  } \displaystyle{\frac{\partial u_k }{\partial q }  } +
\displaystyle{\frac{\partial K }{\partial u_{k,i} }  } \left( \displaystyle{\frac{\partial u_k }{\partial q }  } \right)_{,i}+
\displaystyle{\frac{\partial K }{\partial p }  }  q \; dx
\end{equation}
for any $q$ as an increment to the physical parameter $p$. If the change 
  of the state variable
$u_f$ for physical parameter $p$ in the direction of $q$ is denoted as
\begin{equation}\label{REF:EQU:costfunction 100c}
d_k =\displaystyle{\frac{\partial u_k }{\partial q }  }
\end{equation} 
equation~\ref{REF:EQU:costfunction 100b} can be written as
\begin{equation}\label{REF:EQU:costfunction 100d}
<q, \nabla_{p} J(p)> =    \int_{\Omega}
\displaystyle{\frac{\partial K }{\partial u_k }  } d_k +
\displaystyle{\frac{\partial K }{\partial u_{k,i} }  } d_{k,i}+
\displaystyle{\frac{\partial K }{\partial p }  }  q \; dx
\end{equation}
The  state variable are the solution of PDE which in variational from is given
\begin{equation}\label{REF:EQU:costfunction 100}
\int_{\Omega} F_k \cdot r_k +  G_{li} \cdot r_{k,i} \; dx = 0 
\end{equation} 
for all increments $r$ to the stat $u$. The functions $F$ and $G$ are given and describe the physical
model. They depend of the state variable $u_{k}$ and its gradient $u_{k,i}$ and the physical parameter $p$. The change 
$d_k$  of the state 
$u_f$ for physical parameter $p$ in the direction of $q$ is given from the equation
\begin{equation}\label{REF:EQU:costfunction 100bb}
 \int_{\Omega} 
\displaystyle{\frac{\partial F_k }{\partial u_l }  } d_l r_k +
\displaystyle{\frac{\partial F_k }{\partial u_{l,j}} } d_{l,j} r_k +
\displaystyle{\frac{\partial F_k }{\partial p} }q r_k +
\displaystyle{\frac{\partial G_{ki}}{\partial u_l} } d_l r_{k,i} +
\displaystyle{\frac{\partial G_{ki}}{\partial u_{l,j}} } d_{l,j} r_{k,i}+
\displaystyle{\frac{\partial G_{ki}}{\partial p} } q r_{k,i}  
\; dx = 0  
\end{equation}
to be fulfilled for all functions $r$. Now let $d^*_k$ be the solution of the 
variational equation
\begin{equation}\label{REF:EQU:costfunction 100dd}
 \int_{\Omega} 
\displaystyle{\frac{\partial F_k }{\partial u_l }  } h_l d^*_k +
\displaystyle{\frac{\partial F_k }{\partial u_{l,j}} } h_{l,j} d^*_k +
\displaystyle{\frac{\partial G_{ki}}{\partial u_l} } h_l d^*_{k,i} +
\displaystyle{\frac{\partial G_{ki}}{\partial u_{l,j}} } h_{l,j} d^*_{k,i}
\; dx
= \int_{\Omega} 
\displaystyle{\frac{\partial K }{\partial u_k }  } h_k +
\displaystyle{\frac{\partial K }{\partial u_{k,i} }  } h_{k,i}  \; dx
\end{equation}
for all increments $h_k$ to the physical property $p$. This problem
is solved using \escript \class{LinearPDE} class. In the \escript notation we need to provide 
\begin{equation}\label{ref:EQU:REG:600b}
\begin{array}{rcl}
 A_{kilj} & = &  \displaystyle{\frac{\partial G_{lj}}{\partial u_{k,i}} }  \\
 B_{kil} & = &  \displaystyle{\frac{\partial F_l }{\partial u_{k,i}} }   \\
 C_{klj} & = & \displaystyle{\frac{\partial G_{lj}}{\partial u_k} }    \\
  D_{kl} & = & \displaystyle{\frac{\partial F_l }{\partial u_k }  }   \\ 
  Y_{k} & = & \displaystyle{\frac{\partial K }{\partial u_k }  }     \\ 
  X_{ki} & = & \displaystyle{\frac{\partial K }{\partial u_{k,i} }  }    \\ 
\end{array}
\end{equation} 
Notice that these coefficient are transposed to the coefficients used to solve for the 
state variables in equation~\ref{REF:EQU:costfunction 100}. 

Setting $h_l=d_l$ in equation~\ref{REF:EQU:costfunction 100d} and
$r_k=d^*_k$ in equation~\ref{REF:EQU:costfunction 100b} one gets
\begin{equation}\label{ref:EQU:costfunction:601}
\int_{\Omega}
\displaystyle{\frac{\partial K }{\partial u_k }  } d_k +
\displaystyle{\frac{\partial K }{\partial u_{k,i} }  } d_{k,i}+
\displaystyle{\frac{\partial F_k }{\partial p} } q d^*_k +
\displaystyle{\frac{\partial G_{ki}}{\partial p} } q d^*_{k,i}  
\; dx = 0  
\end{equation} 
which is inserted into equation~\ref{REF:EQU:costfunction 100d} to get
\begin{equation}\label{REF:EQU:costfunction 602}
<q, \nabla_{p} J(p)> =    \int_{\Omega} \left(
\displaystyle{\frac{\partial K }{\partial p }  } - \displaystyle{\frac{\partial F_k }{\partial p} } d^*_k
- \displaystyle{\frac{\partial G_{ki}}{\partial p} }  d^*_{k,i} \right) q \; dx
\end{equation}
We need in fact the gradient of $J^f$ with respect to the level set function which is given as
\begin{equation}\label{REF:EQU:costfunction 603}
<n, \nabla J^f> =    \int_{\Omega} \left(
\displaystyle{\frac{\partial K^f}{\partial p^f}  } - \displaystyle{\frac{\partial F^f_k }{\partial p^f} } d^{f*}_k
- \displaystyle{\frac{\partial G^f_{ki}}{\partial p^f} }  d^{f*}_{k,i} \right) 
\cdot \displaystyle{\frac{\partial p^f }{\partial m_l} } n_l \; dx
\end{equation}
for any increment $n$ to the level set function. So in summary we get  
\begin{equation}\label{ref:EQU:CS:101b}
\begin{array}{rcl}
  Y_k & = & \displaystyle{\frac{\partial K^{reg}}{\partial m_k}} + 
 \sum_{f} \mu^{data}_{f} \left(
\displaystyle{\frac{\partial K^f}{\partial p^f}  } - \displaystyle{\frac{\partial F^f_l }{\partial p^f} } d^{f*}_l
- \displaystyle{\frac{\partial G^f_{li}}{\partial p^f} }  d^{f*}_{l,i} \right) 
\cdot \displaystyle{\frac{\partial p^f }{\partial m_k} } 

\\
   X_{ki} & = & \displaystyle{\frac{\partial K^{reg}}{\partial m_{k,i}}} 
\end{array}
\end{equation} 
to represent $\nabla J$ as the tuple $(Y,X)$. Contributions of the forward model to the 
Hessian operator are ignored.

