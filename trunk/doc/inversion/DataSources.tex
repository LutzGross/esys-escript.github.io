\chapter{Data Sources}\label{Chp:ref:data sources}

At the source of every inversion is data in the form of gravity anomaly or
magnetic flux density values for at least a part of the region of interest.
These usually come from surveys and are preprocessed to correct for various
factors and distortions.
This chapter provides an overview of the classes related to data input for
inversions.

\section{Overview}
The inversion module comes with a number of classes that can read gridded
(raster) data on a 2-dimensional plane from file or provide artificial values
for testing purposes. These classes all derive from the abstract
\class{DataSource} class and override methods that return information about
the data and the values themselves.
The \class{DomainBuilder} class is responsible for creating an \escript domain
with a suitable grid spacing and spatial extents that include all data sources
attached to it (see Figure~\ref{fig:domainBuilder}).
%
\begin{figure}[ht]
    \centering\includegraphics{domainbuilder}
    \caption{\class{DataSource} instances are added to a \class{DomainBuilder}
        which creates a suitable domain and \Data objects for the inversion}
    \label{fig:domainBuilder}
\end{figure}
%
Notice that in the figure there are cells in the region of interest that are
not covered by any data source instance.
Ideally, all data sources used for an inversion have the same spatial resolution
and are spatially adjacent so that all cells have a value but this is not a
requirement.


\section{Domain Builder}\label{Chp:ref:domain builder}
Every inversion requires one \class{DomainBuilder} instance which creates and
holds a reference to the \escript domain as well as associated \Data objects for
the input data used for the inversion.
The class has the following public methods:

\begin{classdesc}{DomainBuilder}{\optional{dim=3}}
Constructor for the domain builder. \member{dim} sets the dimensionality of the
target domain and must be 2 or 3. By default a 3-dimensional domain is created.
\end{classdesc}

\begin{methoddesc}[DomainBuilder]{addSource}{source}
adds survey data \member{source} (a \class{DataSource} object) to the domain
builder. The dimensionality of the data must be less than or equal to the
domain dimensionality.
\end{methoddesc}

\begin{methoddesc}[DomainBuilder]{setVerticalExtents}{%
\optional{depth=40000.}%
\optional{, air_layer=10000.}%
\optional{, num_cells=25}}
sets the parameters for the vertical dimension of the domain. The parameter
\member{depth} specifies the thickness in meters of the subsurface layer
($-x_2^{min}$ in Figure~\ref{fig:cartesianDomain}).
The default value of $40$ km is usually appropriate. Similarly, the
\member{air_layer} parameter defines the buffer zone thickness above the surface
($x_2^{max}$ in Figure~\ref{fig:cartesianDomain}) which should be a few
kilometres to avoid artefacts in the inversion.
The number of elements (or cells) in the vertical dimension is set with the
\member{num_cells} parameter. Consider the size and resolution of your datasets,
the total vertical length (=\member{depth}+\member{air_layer}) and available
compute resources when setting this value.
\end{methoddesc}

\begin{methoddesc}[DomainBuilder]{setFractionalPadding}{%
\optional{pad_x=\None}%
\optional{, pad_y=\None}}
sets the amount of padding around the dataset as a fraction of the dataset side
lengths.  
For example, calling \member{setFractionalPadding(0.2, 0.1)} with a data source
of size $10 \times 20$ will result in the padded data set size $14 \times 24$
(that is $10 \times (1+2 \times 0.2)$ and $20 \times (1+2 \times 0.1)$).
By default no padding is applied and \member{pad_y} is ignored for 2-dimensional
domains.
\end{methoddesc}

\begin{methoddesc}[DomainBuilder]{setPadding}{%
\optional{pad_x=\None}%
\optional{, pad_y=\None}}
sets the amount of padding around the dataset in absolute length units.
The final domain size will be the length in x (in y) of the dataset plus twice
the value of \member{pad_x} (\member{pad_y}). The arguments must be non-negative.
By default no padding is applied and \member{pad_y} is ignored for 2-dimensional
domains.
\end{methoddesc}

\begin{methoddesc}[DomainBuilder]{setElementPadding}{%
\optional{pad_x=\None}%
\optional{, pad_y=\None}}
sets the amount of padding around the dataset in number of elements (cells).
When the domain is constructed \member{pad_x} (\member{pad_y}) elements are
added on each side of the x- (y-) dimension. The arguments must be non-negative.
By default no padding is applied and \member{pad_y} is ignored for 2-dimensional
domains.
\end{methoddesc}

\begin{methoddesc}[DomainBuilder]{fixDensityBelow}{%
\optional{depth=\None}}
defines the depth below which the density anomaly is fixed to zero.
This method is only useful for inversions that involve gravity data.
\end{methoddesc}

\begin{methoddesc}[DomainBuilder]{fixSusceptibilityBelow}{%
\optional{depth=\None}}
defines the depth below which the susceptibility anomaly is fixed to zero.
This method is only useful for inversions that involve magnetic data.
\end{methoddesc}

\begin{methoddesc}[DomainBuilder]{getGravitySurveys}{}
returns a list of all gravity surveys added to the domain builder. See
\member{getSurveys()} for more details.
\end{methoddesc}

\begin{methoddesc}[DomainBuilder]{getMagneticSurveys}{}
returns a list of all magnetic surveys added to the domain builder. See
\member{getSurveys()} for more details.
\end{methoddesc}

\begin{methoddesc}[DomainBuilder]{getSurveys}{datatype}
returns a list of surveys of type \member{datatype} available to this domain
builder. In the current implementation each survey is a tuple of two \Data
objects, the first containing anomaly values and the second standard error
values for the survey.
\end{methoddesc}

\begin{methoddesc}[DomainBuilder]{getDomain}{}
returns an \escript domain (see~\cite{ESCRIPT}) suitable for running inversions
on the attached data sources.
The first time this method is called the target parameters (such as resolution,
extents and number of elements) are computed, and the domain is created.
Subsequent calls return the same domain instance so calls to
\member{setPadding()}, \member{addSource()} and other methods that influence
the domain will fail once \member{getDomain()} is called the first time.
\end{methoddesc}

\begin{methoddesc}[DomainBuilder]{setBackgroundMagneticFluxDensity}{B}
sets the background magnetic flux density $B=(B_r,B_\theta,B_\phi)$ which is
required for magnetic inversions.
A implementation of the dipole approximation as described in
Equation~\ref{ref:MAG:EQU:5} is provided through the function
\member{simpleGeoMagneticFluxDensity} (see Section~\ref{sec:ref:DataSource}).
$B_\theta$ is ignored for 2-dimensional magnetic inversions.
\end{methoddesc}

\begin{methoddesc}[DomainBuilder]{getBackgroundMagneticFluxDensity}{}
returns the background magnetic flux density $B$ set via
\member{setBackgroundMagneticFluxDensity()} in a form suitable for the inversion.
There should be no need to call this method directly.
\end{methoddesc}

\begin{methoddesc}[DomainBuilder]{getSetDensityMask}{}
returns the density mask \Data object which is non-zero for cells that have a
fixed density value, zero otherwise.
There should be no need to call this method directly.
\end{methoddesc}

\begin{methoddesc}[DomainBuilder]{getSetSusceptibilityMask}{}
returns the susceptibility mask \Data object which is non-zero for cells that
have a fixed susceptibility value, zero otherwise.
There should be no need to call this method directly.
\end{methoddesc}

\section{\class{DataSource} Class}\label{sec:ref:DataSource}

Data sources added to a \class{DomainBuilder} must provide an implementation for
a few methods as described in the class template \class{DataSource} from
the \module{esys.downunder.datasources} module:

\begin{classdesc}{DataSource}{}
Base constructor which initializes members and should therefore be invoked by
subclasses. Subclasses may then use the member \member{logger} to print any
output.
\end{classdesc}

\begin{methoddesc}[DataSource]{getDataExtents}{}
This method should be implemented to return a tuple of tuples
( (x0, y0), (nx, ny), (dx, dy) ), where (x0, y0) denote the UTM coordinates of
the data origin, (nx, ny) the number of data points, and (dx, dy) the spacing
of data points.
\end{methoddesc}

\begin{methoddesc}[DataSource]{getDataType}{}
Subclasses must return \class{DataSource}.\member{GRAVITY} or
\class{DataSource}.\member{MAGNETIC} depending on the type of data they provide.
\end{methoddesc}

\begin{methoddesc}[DataSource]{getSurveyData}{domain, origin, NE, spacing}
This method is called by the \class{DomainBuilder} to retrieve the actual survey
data in the form of \Data objects on the \member{domain}.
Data sources are responsible to map or interpolate their data onto the domain
which has been constructed to fit the data.
The domain \member{origin}, number of elements \member{NE} and element
\member{spacing} are provided as tuples or lists to aid with interpolation.
\end{methoddesc}

\begin{methoddesc}[DataSource]{setSubsamplingFactor}{factor}
Notifies the data source that data should be subsampled by \member{factor}.
This method does not need to be overwritten.
See \member{getSubsamplingFactor()} for an explanation.
\end{methoddesc}

\begin{methoddesc}[DataSource]{getSubsamplingFactor}{}
Returns the subsampling factor which was set by \member{setSubsamplingFactor()}
or $1$ which indicates that no subsampling is requested.
Data sources that support subsampling (or interleaving) of their data should use
this method to query the subsampling factor before returning surveys via
\member{getSurveyData}. If supported, the factor should be applied in all
dimensions. For example, a 2-dimensional dataset with 300 x 150 data points
should be reduced to 150 x 75 data points when the subsampling factor equals $2$.
Subsampling becomes important when the survey data resolution is too fine or
when using data with varying resolution in one inversion.
Note that data sources may choose to ignore the subsampling factor if they
don't support it.
\end{methoddesc}

\vspace{1em}\noindent The \module{esys.downunder.datasources} module contains the following helper
functions:

\begin{funcdesc}{simpleGeoMagneticFluxDensity}{latitude%
\optional{, longitude=0.}}
returns an approximation of the geomagnetic flux density $B$ as described in
Equation~\ref{ref:MAG:EQU:5} for the given \member{latitude}.
The \member{longitude} parameter is currently ignored and the return value is
the tuple $(B_r, B_{\theta}, 0)$.
\end{funcdesc}

\begin{funcdesc}{LatLonToUTM}{longitude, latitude%
\optional{, wkt_string=\None}}
converts one or more (longitude,latitude) pairs to the corresponding (x,y)
coordinates in the \emph{Universal Transverse Mercator} (UTM) projection.
This function requires the \module{pyproj} module for conversion and the
\module{gdal} module to parse the \member{wkt_string} parameter if supplied.
\end{funcdesc}

\subsection{ER Mapper Raster Data}

\subsection{NetCDF Data}
An example script how to create a data input file for both gravity and magnetic
data using the \netcdf file format~\cite{netcdf} is available in the script
\examplefile{create_ncinput.py}.
To plot an input file using matplotlib~\cite{matplotlib} see the script \examplefile{show_ncinput.py}.

\subsection{Synthetic Data}

