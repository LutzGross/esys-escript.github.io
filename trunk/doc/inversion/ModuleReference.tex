
%%%%%%%%%%%%%%%%%%%%%%%%%%%%%%%%%%%%%%%%%%%%%%%%%%%%%%%%%%%%%%%%%%%%%%%%%%%%%%
% Copyright (c) 2003-2012 by University of Queensland
% http://www.uq.edu.au
%
% Primary Business: Queensland, Australia
% Licensed under the Open Software License version 3.0
% http://www.opensource.org/licenses/osl-3.0.php
%
% Development until 2012 by Earth Systems Science Computational Center (ESSCC)
% Development since 2012 by School of Earth Sciences
%
%%%%%%%%%%%%%%%%%%%%%%%%%%%%%%%%%%%%%%%%%%%%%%%%%%%%%%%%%%%%%%%%%%%%%%%%%%%%%%

\chapter{The \downunder Module}\label{chap:ModuleRef}

\section{Concepts}
\downunder is a \python module that allows running inversions within the
\escript framework as described in the previous chapters.

\section{\downunder classes}

\subsection{Cost Functions}

%%%%%%%%
\begin{classdesc}{CostFunction}{}
    This is the base class for functions $f(x)$ that can be minimized.
    The function calls update statistical information.
    The actual work is done by the methods with corresponding name and a
    leading underscore. These functions need to be overwritten for a particular
    cost function implementation.
\end{classdesc}

\noindent The following methods are implemented:
%
\begin{methoddesc}[CostFunction]{resetCounters}{}
    resets all statistical counters.
\end{methoddesc}
%
\begin{methoddesc}[CostFunction]{getInner}{f0, f1}
    returns the inner product of \member{f0} and \member{f1}.
\end{methoddesc}
%
\begin{methoddesc}[CostFunction]{getValue}{x, *args}
    returns the value $f(x)$ using precalculated values for $x$.
\end{methoddesc}
%
\begin{methoddesc}[CostFunction]{getGradient}{x, *args}
    returns the gradient of $f$ at $x$ using precalculated values for $x$.
\end{methoddesc}
%
\begin{methoddesc}[CostFunction]{getDirectionalDerivative}{x, d, *args}
    returns \texttt{inner(grad(f(x)), d)} using precalculated values for $x$.
\end{methoddesc}
%
\begin{methoddesc}[CostFunction]{getArguments}{x}
    returns precalculated values that are shared in the calculation of $f(x)$
    and \texttt{grad(f(x))}.
\end{methoddesc}

%%%%%%%%
\begin{classdesc}{SimpleCostFunction}{regularization, mapping, forwardmodel}
    This is a simple cost function with a single continuous (mapped) variable.
    It is the sum of two weighted terms, a single forward model and a single
    regularization term. This cost function is used by the provided gravity
    and magnetic inversion implementations.
\end{classdesc}


\subsection{Forward Models}

\subsection{Mappings}

\subsection{Regularizations}

\subsection{Solvers}

\subsection{Data Sources}

\subsection{The \class{DomainBuilder} class}

\subsection{Inversion classes}

