
%%%%%%%%%%%%%%%%%%%%%%%%%%%%%%%%%%%%%%%%%%%%%%%%%%%%%%%%
%
% Copyright (c) 2003-2010 by University of Queensland
% Earth Systems Science Computational Center (ESSCC)
% http://www.uq.edu.au/esscc
%
% Primary Business: Queensland, Australia
% Licensed under the Open Software License version 3.0
% http://www.opensource.org/licenses/osl-3.0.php
%
%%%%%%%%%%%%%%%%%%%%%%%%%%%%%%%%%%%%%%%%%%%%%%%%%%%%%%%%

\section{3D pycad}
\sslist{example09n.py}

This example explains how to build a 3D layered model using pycad. There are a
few import concepts that one must remember when using pycad to build a layered
model that will function correctly.
\begin{itemize}
  \item There must be no duplication of any geometric features whether they be
  points, lines, loops, surfaces or volumes.
  \item All objects with dimensions greater then a line have a normal defined by
  the right hand rull (RHR). It is important to consider which direction a
  normal is oriented when combining primatives to form higher order shapes.
\end{itemize}

The first step as always is to import the external modules. To build a 3D model
and mesh we will need pycad, some GMesh interfaces, the finley domain builder
and some additional tools.
\begin{python}
#######################################################EXTERNAL MODULES
from esys.pycad import * #domain constructor
from esys.pycad.gmsh import Design #Finite Element meshing package
from esys.finley import MakeDomain #Converter for escript
from esys.escript import mkDir, getMPISizeWorld
import os
\end{python}
After carrying out some routine checks and setting the \verb!save_path! we then
specify the parameters of the model. This model will be 2000 by 2000 meters on
the surface and extend to a depth of 500 meters. An interface of boundary
between two layers will be created at half the total depth or 250 meters. This
type of model is known as a horizontally layered model or a layer cake model. 
\begin{python}
################################################ESTABLISHING PARAMETERS
#Model Parameters
xwidth=2000.0*m   #x width of model
ywidth=2000.0*m   #y width of model
depth=500.0*m   #depth of model
intf=depth/2.   #Depth of the interface.
\end{python}
We now start to specify the components of our model starting with the vertexes
using the \verb!Point! primative. These are then joined by lines in a regular
manner taking note of the right hand rule. Finally, the lines are turned into
loops and then planar surfaces.
\footnote{Some code has been emmitted here for
simlpicity. For the full script please refer to the script referenced at the beginning of
this section.}
\begin{python}
####################################################DOMAIN CONSTRUCTION
# Domain Corners
p0=Point(0.0,    0.0,      0.0)
#..etc..
l45=Line(p4, p5)

# Join line segments to create domain boundaries and then surfaces
ctop=CurveLoop(l01, l12, l23, l30);     stop=PlaneSurface(ctop)
cbot=CurveLoop(-l67, -l56, -l45, -l74); sbot=PlaneSurface(cbot)
\end{python}
With the top and bottom of the domain taken care of, it is now time to focus on
the interface. Again the vertexes of the planar interface are created. With
these, vertical and horizontal lines (edges) are created joining the interface
with itself and the top and bottom surfaces. 
\begin{python}
# for each side
ip0=Point(0.0,    0.0,      intf)
#..etc..
linte_ar=[]; #lines for vertical edges
linhe_ar=[]; #lines for horizontal edges
linte_ar.append(Line(p0,ip0))
#..etc..
linhe_ar.append(Line(ip3,ip0))
\end{python}
Consider now the sides of the domain. One could specify the whole side using the
points first defined for the top and bottom layer. This would specify the whole
domain as one volume. However, there is an interface and we wish to define each
layer individually. Therefore there will be 8 surfaces on the sides of our
domain. We can do this operation quite simply using the points and lines that we
had defined previously. First loops are created and then surfaces making sure to
keep a normal for each layer which is consistent with upper and lower surfaces
of the layer. For example all normals must face outwards from or inwards towards
 the centre of the volume.
\begin{python}
cintfa_ar=[]; cintfb_ar=[] #curveloops for above and below interface on sides
cintfa_ar.append(CurveLoop(linte_ar[0],linhe_ar[0],-linte_ar[2],-l01))
#..etc..
cintfb_ar.append(CurveLoop(linte_ar[7],l45,-linte_ar[1],-linhe_ar[3]))

sintfa_ar=[PlaneSurface(cintfa_ar[i]) for i in range(0,4)]
sintfb_ar=[PlaneSurface(cintfb_ar[i]) for i in range(0,4)]

sintf=PlaneSurface(CurveLoop(*tuple(linhe_ar)))
\end{python}
Assuming all is well with the normals, the volumes can be created from our
surface arrays. Note the use here of the \verb!*tuple*! function. This allows us
to pass an list array as an argument list to a function. It must be placed at
the end of the function arguments and there cannot be more than one per function
call.
\begin{python}
vintfa=Volume(SurfaceLoop(stop,-sintf,*tuple(sintfa_ar)))
vintfb=Volume(SurfaceLoop(sbot,sintf,*tuple(sintfb_ar)))

# Create the volume.
#sloop=SurfaceLoop(stop,sbot,*tuple(sintfa_ar+sintfb_ar))
#model=Volume(sloop)
\end{python}
The final steps are designing the mesh, tagging the volumes and the interface
and outputting the data to file so it can be imported by an \esc sollution
script.
\begin{python}
#############################################EXPORTING MESH FOR ESCRIPT
# Create a Design which can make the mesh
d=Design(dim=3, element_size=5.0*m)
d.addItems(PropertySet('vintfa',vintfa))
d.addItems(PropertySet('vintfb',vintfb))
d.addItems(sintf)

d.setScriptFileName(os.path.join(save_path,"example09m.geo"))

d.setMeshFileName(os.path.join(save_path,"example09m.msh"))
#
#  make the finley domain:
#
domain=MakeDomain(d)
# Create a file that can be read back in to python with
# mesh=ReadMesh(fileName)
domain.write(os.path.join(save_path,"example09m.fly"))
\end{python}


\section{Layer Cake Models}
Whilst this type of model seems simple enough to construct for two layers,
specifying multiple layers can become combersome. A function exists to generate
layer cake models called \verb!layer_cake!. A detailed description of its
arguments and returns is available in the API and the function can be imported
from pycad.
\begin{python}
from esys.pycad import layer_cake
\end{python}

\section{Troubleshooting Pycad}
There are some techniques which can be useful when trying to trouble shoot
problems with pycad.
