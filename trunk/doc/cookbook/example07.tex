
%%%%%%%%%%%%%%%%%%%%%%%%%%%%%%%%%%%%%%%%%%%%%%%%%%%%%%%%
%
% Copyright (c) 2003-2010 by University of Queensland
% Earth Systems Science Computational Center (ESSCC)
% http://www.uq.edu.au/esscc
%
% Primary Business: Queensland, Australia
% Licensed under the Open Software License version 3.0
% http://www.opensource.org/licenses/osl-3.0.php
%
%%%%%%%%%%%%%%%%%%%%%%%%%%%%%%%%%%%%%%%%%%%%%%%%%%%%%%%%



The acoustic wave equation governs the propagation of pressure waves. Wave
types that obey this law tend to travel in liquids or gases where shear waves
or longitudinal style wave motion is not possible. The obvious example is sound
waves.

The acoustic wave equation is;
\begin{equation}
 \nabla ^2 p - \frac{1}{c^2} \frac{\partial ^2 p}{\partial t^2} = 0
\label{eqn:acswave}
\end{equation}
where $p$ is the pressure, $t$ is the time and $c$ is the wave velocity.



\section{Numerical Solution Stability}
Unfortunately, the wave equation is belongs to a class of equations called
\textbf{stiff} PDEs. This types of equations can be difficult to solve
numerically as they tend to oscilate about the exact solution and can
eventually blow up. To counter this problem, explicitly stable schemes like the
backwards Euler method are required. In terms of the wave equation, the
analytical wave must not propagate faster than the numerical wave is able to,
and in general, needs to be much slower than the numerical wave.

For example, a line 100m long is discretised into 1m intervals or 101 nodes. If
a wave enters with a propagation velocity of 100m/s then the travel time for
the wave between each node will be 0.01 seconds. The time step, must therefore
be significantly less than this. Of the order $10E-4$ would be appropriate. 

This requirement for very small step sizes makes stiff equations difficult to
solve numerically due to the large number of time iterations required in each
solution. Models with very high velocities and fine meshes will be the worst
affected by this problem.


\section{Displacement Solution}
\sslist{example07a.py}

We begin the solution to this PDE with the centred difference formula for the
second derivative;
\begin{equation}
 f''(x) \approx \frac{f(x+h - 2f(x) + f(x-h)}{h^2}
\label{eqn:centdiff}
\end{equation}
substituting \refEq{eqn:centdiff} for $\frac{\partial ^2 p }{\partial t ^2}$
in \refEq{eqn:acswave};
\begin{equation}
 \nabla ^2 p - \frac{1}{c^2h^2} \left[p_{(t+1)} - 2p_{(t)} + p_{(t-1)} \right]
= 0
\label{eqn:waveu}
\end{equation}
Rearranging for $p_{(t+1)}$;
\begin{equation}
 p_{(t+1)} = c^2 h^2 \nabla ^2 p_{(t)} +2p_{(t)} - p_{(t-1)}
\end{equation}
this can be compared with the general form of the \modLPDE module and it
becomes clear that $D=1$, $X=-c^2 h^2 \nabla ^2 p_{(t)}$ and $Y=2p_{(t)} -
p_{(t-1)}$.

\section{Acceleration Solution}
\sslist{example07b.py}

An alternative method is to solve for the acceleration $\frac{\partial ^2
p}{\partial t^2}$ directly, and derive the the displacement solution from the
PDE solution. \refEq{eqn:waveu} is thus modified;
\begin{equation}
  \nabla ^2 p - \frac{1}{c^2} a = 0
\label{eqn:wavea}
\end{equation}
and can be solved directly with $Y=0$ and $X=-c^2 \nabla ^2 p_{(t)}$.
After each iteration the displacement is re-evaluated via;
\begin{equation}
 p_{(t+1)}=2p_{(t)} - p_{(t-1)} + h^2a
\end{equation}


