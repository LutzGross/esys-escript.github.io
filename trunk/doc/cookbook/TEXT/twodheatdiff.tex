
%%%%%%%%%%%%%%%%%%%%%%%%%%%%%%%%%%%%%%%%%%%%%%%%%%%%%%%%
%
% Copyright (c) 2003-2009 by University of Queensland
% Earth Systems Science Computational Center (ESSCC)
% http://www.uq.edu.au/esscc
%
% Primary Business: Queensland, Australia
% Licensed under the Open Software License version 3.0
% http://www.opensource.org/licenses/osl-3.0.php
%
%%%%%%%%%%%%%%%%%%%%%%%%%%%%%%%%%%%%%%%%%%%%%%%%%%%%%%%%

\documentclass{manual}
\title{One Dimensional Heat Equation Var 001}
\newcommand{\editor}[1] {\textcolor{red}{#1}}

\begin{document}
 Building upon our success from the 1D models it is now prudent to expand our domain by another dimension. For this example we will again be using an intrusion as the basis for our model. Our intrusion will be a single event where some molten granite has formed a semi-circle at the base of some cold granite country rock. \editor{the aim of this example is to introduce the second dimension, as well as some more complicated boundary and initial conditions }
\end{document}
