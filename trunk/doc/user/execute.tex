\chapter{Execution of an {\it escript} Script}
\label{EXECUTION}

\section{Overview}
A typical way of starting your {\it escript} script \file{myscript.py} is with the \program{escript} command\footnote{The \program{escript} launcher is not supported under \WINDOWS yet.}:
\begin{verbatim}
escript myscript.py
\end{verbatim}
as already shown in section~\ref{FirstSteps}\footnote{For this discussion, it is assumed that \program{escript} is included in your \env{PATH} environment. See installation guide for details.}
. In some cases 
it can be useful to work interactively e.g. when debugging a script, with the command 
\begin{verbatim}
escript -i myscript.py
\end{verbatim}
This will execute \var{myscript.py} and when it completes (or an error occurs), a \PYTHON prompt will be provided.
To leave the prompt press \kbd{Control-d}.

To start 
\program{escript} using four threads (eg. if you use a multi-core processor) you can use
\begin{verbatim}
escript -t 4 myscript.py
\end{verbatim}
This will require {\it escript} to be compiled for \OPENMP\cite{OPENMP}. 

To start \program{escript} using \MPI\cite{MPI} with $8$ processes you use
\begin{verbatim}
escript -p 8 myscript.py
\end{verbatim}
If the processors which are used are multi--core processors or multi--processor shared memory archtiectures you can use threading in addition to \MPI. For instance to run $8$ \MPI processes with using $4$ threads each, you use the command
\begin{verbatim}
escript -p 8 -t 4 myscript.py
\end{verbatim}
In the case of a super computer or a cluster, you may wish to distribute the workload over a number of nodes\footnote{For simplicity, we will use the term node to refer to either a node in a super computer or an individual machine in a cluster}.
For example, to use $8$ nodes, with $4$ \MPI processes per node, write
\begin{verbatim}
escript -n 8 -p 4 myscript.py
\end{verbatim}
Since threading has some performance advantages over processes, you may specify a number of threads as well.
\begin{verbatim}
escript -n 8 -p 4 -t 2 myscript.py
\end{verbatim}
This runs the script on $8$ nodes, with $4$ processes per node and $2$ threads per process.

\section{Options}
The general form of the \program{escript} launcher is as follows:

\program{escript} 
\optional{\programopt{-n \var{nn}}} 
\optional{\programopt{-p \var{np}}}
\optional{\programopt{-t \var{nt}}}
\optional{\programopt{-f \var{hostfile}}}
\optional{\programopt{-x}}
\optional{\programopt{-V}}
\optional{\programopt{-e}}
\optional{\programopt{-h}}
\optional{\programopt{-v}}
\optional{\programopt{-o}}
\optional{\programopt{-c}}
\optional{\programopt{-i}}
\optional{\var{file}}
\optional{\var{ARGS}}

where \var{file} is the name of a script, \var{ARGS} are arguments for the script.
The \program{escript} program will import your current environment variables. 
If no \var{file} is given, then you will be given a \PYTHON prompt (see \programopt{-i} for restrictions).


The options are used as follows:
\begin{itemize}

 \item[\programopt{-n} \var{nn}] the number of compute nodes \var{nn} to be used. The total number of process being used is 
$\var{nn} \cdot \var{ns}$. This option overwrites the value of the \env{ESCRIPT_NUM_NODES} environment variable. If $\var{nn}>1$ but {\it escript}  is not compiled for \MPI a warning is pronted but execution is continued with $\var{nn}=1$. The default value is 1.
 
\item[\programopt{-p} \var{np}] the number of processor slots per node. This options specifies how many \MPI processes are used on each node. The total number of process being used is 
$\var{nn} \cdot \var{np}$. This option overwrites the value of the \env{ESCRIPT_NUM_PROC} environment variable. If $\var{np}>1$ but {\it escript}  is not compiled for \MPI a warning is printed but execution is continued with $\var{np}=1$. The default value is 1.

 \item[\programopt{-t} \var{nt}] the number of threads used per processes.
The option overwrites the value of the \env{ESCRIPT_NUM_THREADS} environment variable.
If $\var{nt}>1$ but {\it escript} is not compiled for \OPENMP a warning is printed but execution is continued with $\var{nt}=1$. The default value is 1.

 \item[\programopt{-f} \var{hostfile}] the name of a file with a list of host names. Some systems require to specify the addresses or names of the compute nodes where \MPI process should be spawned. The list of addresses or names of the compute nodes is listed in the file with the name \var{hostfile}. In order to avoid oversubsription the number of differnt
hosts defined in \var{hostfile} must be greater or equal to the number of requsted compute nodes \var{nn}. The option overwrites the value of the \env{ESCRIPT_HOSTFILE} environment variable. By default value no host file is used.
 \item[\programopt{-c}] prints the information about the settings used to compile {\it escript} and stops execution..
 \item[\programopt{-V}] prints the version of {\it escript} and stops execution.
 \item[\programopt{-h}] prints a help message and stops execution.
 \item[\programopt{-i}] excutes the script \var{file} and switches to interactive mode after the execution is finished or an exception has occured. This option is useful for debugging a script. The option cannot be used if more then one process ($\var{nn} \cdot \var{np}>1$) is used.

 \item[\programopt{-e}] shows additional environment variables and commands used during \program{escript} execution. This option is useful if users wish to excute scripts without using the \program{escript} command.

 \item[\programopt{-o}] switches on the redirection of output of processors with \MPI rank greater than zero to the files \file{stdout_\var{r}.out} and \file{stderr_\var{r}.out} where \var{r} is the rank of the processor. The option overwrites the value of the \env{ESCRIPT_STDFILES} environment variable 

%  \item[\programopt{-x}] interpret \var{file} as an \esysxml \footnote{{\it esysxml} has not been released yet.} task.
% This option is still expermental.

 \item[\programopt{-v}] prints some diagonstic information.
\end{itemize}

\section{Input and Output}
When \MPI is used on more than one process ($\var{nn} \cdot \var{np} >1$) no input from the standard input is accepted. Standard output on any process other the the master process (\var{rank}=0) will not be available.
Error output from any processor will be redirected to the node where \program{escript} has been envoked.
If the \programopt{-o} or \env{ESCRIPT_STDFILES} is set\footnote{That is, it has a non-empty value.}, then the standard and error output from any process other than the master process will be written to files of the names \file{stdout_\var{r}.out} and \file{stderr_\var{r}.out} (where 
\var{r} is the rank of the process).

If files are created or read by individual \MPI processes with information local to the process (e.g in the \function{dump} function)  and more than one process is used ($\var{nn} \cdot \var{np} >1$), the \MPI process rank is appended to the file names.
This will avoid problems if processes are using a shared file system.
Files which collect data which are global for all \MPI processors will created by the process with \MPI rank 0 only.
Users should keep in mind that if the file system is not shared, then a file containing global information
which is read by all processors needs to be copied to the local file system before \program{escript} is invoked.
 

\section{Hints for MPI programming}
Later