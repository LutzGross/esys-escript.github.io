\chapter{The module \pyvisi}
\label{PYVISI CHAP}
\declaremodule{extension}{esys.pyvisi}
\modulesynopsis{Python Visualization Interface}

\section{Introduction}
\pyvisi is a Python module that is used to generate 2D and 3D visualization 
for escript and its PDE solvers: finley and bruce. This module provides 
an easy to use interface to the \VTK library (\VTKUrl).  

The general rule of thumb when using \pyvisi is to perform the following 
in sequence:

\begin{enumerate}
\item Create a scene instance, in which objects are to be rendered on.
\item Create an input instance, which deals with the source of data for 
the visualization.
\item Create a data visualization instance (i.e. Map, Velocity, Ellipsoid, 
etc), which extracts and manipulates the data accordingly.
\item Create a camera instance, which controls the lighting 
source and view angle.
\item Finally, render the object.
\end{enumerate}
\begin{center}
\begin{math}
scene \rightarrow input \rightarrow visualization \rightarrow 
camera \rightarrow render
\end{math}
\end{center}

The sequence in which instances are created is very important due to
to the dependencies among them. For example, an input instance must 
always be created BEFORE a data visualisation instance is created. 
If the sequence is switched, the program will throw an error because a 
source data needs to be specified before the data can be 
manipulated. Similarly, a camera instance must always be created
AFTER an input instance has been created. Otherwise, the program will throw 
an error because the camera instance needs to calculate its 
default position (automatically carried out in the background) based on 
the source data. 

\section{\pyvisi Classes}
This section gives a brief overview of the important classes and their 
corresponding methods. Please refer to \ReferenceGuide for full details.
%=====================================================================================
\subsection{Scene Classes}
\begin{classdesc}{Scene}{renderer = Renderer.ONLINE, num_viewport = 1, 
x_size = 1152, y_size = 864}
Displays a scene in which objects are to be rendered on.
\end{classdesc}

\begin{classdesc}{Camera}{}
 Controls the camera manipulation. 
\end{classdesc}

\begin{classdesc}{Light}{}
 Controls the light manipulation.
\end{classdesc}

%============================================================================================================
\subsection{Input Classes}

\begin{classdesc}{Image}{}
 Displays an image.
\end{classdesc}

\begin{classdesc}{Text}{}
 Shows some 2D text.
\end{classdesc}

\begin{classdesc}{DataCollector}{}
Deals with the source of data for visualization.
\end{classdesc}

%============================================================================================================
\subsection{Data Visualization}
\begin{classdesc}{Map}{}
 Displays a scalar field using a domain surface.
\end{classdesc}

\begin{classdesc}{MapOnPlaneCut}{}
 Displays a scalar field using a domain surface cut on a plane. 
\end{classdesc}

\begin{classdesc}{MapOnPlaneClip}{}
 Displays a scalar field using a domain surface clipped 
		on a plane.
\end{classdesc}

\begin{classdesc}{MapOnScalarClip}{}
 Displays a scalar field using a domain surface clipped 
		using a scalar value.
\end{classdesc}

\begin{classdesc}{Velocity}{}
 Displays a vector field using arrows.
\end{classdesc}

\begin{classdesc}{VelocityOnPlaneCut}{}
 Displays a vector field using arrows cut on a plane.
\end{classdesc}

\begin{classdesc}{VelocityOnPlaneClip}{}
 Displays a vector field using arrows clipped on a 
		plane.
\end{classdesc}

\begin{classdesc}{Ellipsoid}{}
 Displays a tensor field using spheres.
\end{classdesc}

\begin{classdesc}{EllipsoidOnPlaneCut}{}
 Displays a tensor field using spheres cut on a
        plane.
\end{classdesc}

\begin{classdesc}{EllipsoidOnPlaneClip}{}
 Displays a tensor field using spheres clipped 
        on a plane.
\end{classdesc}

        
\begin{classdesc}{Contour}{}
 Shows a scalar field by contour surfaces. 
\end{classdesc}

\begin{classdesc}{ContourOnPlane}{}
 Shows a scalar field by contour surfaces on 
a given plane.
\end{classdesc}

\begin{classdesc}{ContourOnClip}{}
 Shows a scalar field by contour surfaces on 
a given clip.
\end{classdesc}

\begin{classdesc}{IsoSurface}{}
 Shows a scalar field for a given value by 
an isosurface.
\end{classdesc}

\begin{classdesc}{IsoSurfaceOnPlane}{}
 Shows a scalar field for a given value by 
an isosurfaceon a given plane.
\end{classdesc}

\begin{classdesc}{IsoSurfaceOnClip}{}
 Shows a scalar field for a given vlaue by 
an isosurface on a given clip.
\end{classdesc}

\begin{classdesc}{StreamLines}{}
 Shows the path of particles in a vector field.
\end{classdesc}

\begin{classdesc}{Carpet}{}
 Shows a scalar field as plane deformated along 
the plane normal.
\end{classdesc}

\section{Geometry}
\begin{classdesc}{Position}{}
 Defines the x,y and z coordinates rendered object.
\end{classdesc}

\begin{classdesc}{Transform}{}
Defines the orientation of rendered object.
\end{classdesc}

\begin{classdesc}{Plane}{}
Defines the cutting/clipping of rendered objects.
\end{classdesc}


\subsection{Beautification}
\begin{classdesc}{Style}{}
Defines the style of text.
\end{classdesc}

\begin{classdesc}{BlueToRed}{}
 Defines a map spectrum from blue to red.
\end{classdesc}

\begin{classdesc}{RedToBlue}{}
 Defines a map spectrum from red to blue.
\end{classdesc}
%===========================================

\section{Rendering}
same word on rendering, off-line, on-line, how to rotate, zoom, close the window, ...

%==============================================
\section{How to Make a Movie}
