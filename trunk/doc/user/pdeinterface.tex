% $Id$
\section{\LinearPDE Class}
\label{SEC LinearPDE}

The \LinearPDE class is used to define a general linear, steady, second order PDE
for an unknown function $u$ on a given $\Omega$ defined through a \Domain object.
In the following $\Gamma$ denotes the boundary of the domain $\Omega$. $n$ denotes
the outer normal field on $\Gamma$. 

For a single PDE with a solution with a single component the linear PDE is defined in the 
following form:
\begin{equation}\label{LINEARPDE.SINGLE.1}
-(A\hackscore{jl} u\hackscore{,l}){,j}+(B\hackscore{j} u)\hackscore{,j}+C\hackscore{l} u\hackscore{,l}+D u =-X\hackscore{j,j}+Y \; .
\end{equation}
$u_{,j}$ denotes the derivative of $u$ with respect to the $j$-th spatial direction. Einstein's summation convention, ie. summation over indexes appearing twice in a term of a sum is performed, is used. 
The coefficients $A$, $B$, $C$, $D$, $X$ and $Y$ have to be specified through \Data objects in the 
\Function on the PDE or objects that can be converted into such \Data objects. 
$A$ is a \RankTwo, $B$, $C$ and $X$ are \RankOne and $D$ and $Y$ are scalar. 
The following natural
boundary conditions are considered \index{boundary condition!natural} on $\Gamma$:
\begin{equation}\label{LINEARPDE.SINGLE.2}
n\hackscore{j}(A\hackscore{jl} u\hackscore{,l}+B\hackscore{j} u)+d u=n\hackscore{j}X\hackscore{j} + y  \;.
\end{equation}
Notice that the coefficients $A$, $B$ and $X$ are defined in the PDE. The coefficients $d$ and $y$ are  
each a \Scalar in the \FunctionOnBoundary.  Constraints \index{constraint} for the solution prescribing the value of the 
solution at certain locations in the domain. They have the form
\begin{equation}\label{LINEARPDE.SINGLE.3}
u=r \mbox{ where } q>0
\end{equation}
$r$ and $q$ are each \Scalar where $q$ is the characteristic function
\index{characteristic function} defining where the constraint is applied.
The constraints defined by \eqn{LINEARPDE.SINGLE.3} override any other condition set by \eqn{LINEARPDE.SINGLE.1}
or \eqn{LINEARPDE.SINGLE.2}. The PDE is symmetrical \index{symmetrical} if
\begin{equation}\label{LINEARPDE.SINGLE.4}
A\hackscore{jl}=A\hackscore{lj} \mbox{ and } B\hackscore{j}=C\hackscore{j}
\end{equation}
For a system of PDEs and a solution with several components the PDE has the form
\begin{equation}\label{LINEARPDE.SYSTEM.1}
-(A\hackscore{ijkl} u\hackscore{k,l}){,j}+(B\hackscore{ijk} u_k)\hackscore{,j}+C\hackscore{ikl} u\hackscore{k,l}+D\hackscore{ik} u_k =-X\hackscore{ij,j}+Y\hackscore{i} \; .
\end{equation}
$A$ is a \RankFour, $B$ and $C$ are each a \RankThree, $D$ and $X$ are each a \RankTwo and $Y$ is a \RankOne. 
The natural boundary conditions \index{boundary condition!natural} take the form:
\begin{equation}\label{LINEARPDE.SYSTEM.2}
n\hackscore{j}(A\hackscore{ijkl} u\hackscore{k,l}){,j}+(B\hackscore{ijk} u_k)+d\hackscore{ik} u_k=n\hackscore{j}-X\hackscore{ij}+y\hackscore{i}  \;.
\end{equation}
The coefficient $d$ is a \RankTwo and $y$ is a  
\RankOne both in the \FunctionOnBoundary. Constraints \index{constraint} take the form
\begin{equation}\label{LINEARPDE.SYSTEM.3}
u\hackscore{i}=r\hackscore{i} \mbox{ where } q\hackscore{i}>0
\end{equation}
$r$ and $q$ are each \RankOne. Notice that at some locations not necessarily all components must 
have a constraint. The system of PDEs is symmetrical \index{symmetrical} if
\begin{eqnarray}\label{LINEARPDE.SYSTEM.4}
A\hackscore{ijkl}=A\hackscore{klij} \\
B\hackscore{ijk}=C\hackscore{kij} \\
D\hackscore{ik}=D\hackscore{ki} \\
d\hackscore{ik}=d\hackscore{ki} \
\end{eqnarray}
\LinearPDE also supports solution discontinuities \index{discontinuity} over contact region $\Gamma^{contact}$
in the domain $\Omega$. To specify the conditions across the discontinuity we are using the
generalised flux $J$ which is in the case of a systems of PDEs and several components of the solution
defined as 
\begin{equation}\label{LINEARPDE.SYSTEM.5}
J\hackscore{ij}=A\hackscore{ijkl}u\hackscore{k,l}+B\hackscore{ijk}u\hackscore{k}-X\hackscore{ij}
\end{equation}
For the case of single solution component and single PDE $J$ is defined
\begin{equation}\label{LINEARPDE.SINGLE.5}
J\hackscore{j}=A\hackscore{jl}u\hackscore{,l}+B\hackscore{j}u\hackscore{k}-X\hackscore{j}
\end{equation}
In the context of discontinuities \index{discontinuity} $n$ denotes the normal on the 
discontinuity pointing from side 0 towards side 1. For a system of PDEs
the contact condition takes the form
\begin{equation}\label{LINEARPDE.SYSTEM.6}
n\hackscore{j} J^{0}\hackscore{ij}=n\hackscore{j} J^{1}\hackscore{ij}=y^{contact}\hackscore{i} - d^{contact}\hackscore{ik} [u]\hackscore{k} \; .
\end{equation}
where $J^{0}$ and $J^{1}$ are the fluxes on side $0$ and side $1$ of the
discontinuity $\Gamma^{contact}$, respectively. $[u]$, which is the difference
of the solution at side 1 and at side 0, denotes the jump of $u$ across $\Gamma^{contact}$.
The coefficient $d^{contact}$ is a \RankTwo and $y^{contact}$ is a  
\RankOne both in the \FunctionOnContactZero or \FunctionOnContactOne.
In case of a single PDE and a single component solution the contact condition takes the form
\begin{equation}\label{LINEARPDE.SINGLE.6}
n\hackscore{j} J^{0}\hackscore{j}=n\hackscore{j} J^{1}\hackscore{j}=y^{contact} - d^{contact}[u]
\end{equation}
In this case the the coefficient $d^{contact}$ and $y^{contact}$ are eaach \Scalar
both in the \FunctionOnContactZero or \FunctionOnContactOne.
