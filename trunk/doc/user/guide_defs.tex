
%%%%%%%%%%%%%%%%%%%%%%%%%%%%%%%%%%%%%%%%%%%%%%%%%%%%%%%%
%
% Copyright (c) 2003-2008 by University of Queensland
% Earth Systems Science Computational Center (ESSCC)
% http://www.uq.edu.au/esscc
%
% Primary Business: Queensland, Australia
% Licensed under the Open Software License version 3.0
% http://www.opensource.org/licenses/osl-3.0.php
%
%%%%%%%%%%%%%%%%%%%%%%%%%%%%%%%%%%%%%%%%%%%%%%%%%%%%%%%%


\usepackage{subfigure}
\usepackage{epsfig}
\usepackage{graphicx,color}
\usepackage{makeidx}  % handle the index properly
\usepackage{xspace}   % handle spaces after commands more nicely
% use the ams math stuff, as it makes the maths easier to code, and
% nicer output than the standard LaTeX stuff
\usepackage{amsmath,amsfonts,amssymb} % this is handy for mathematicians and physicists
			              % see http://www.ams.org/tex/amslatex.html
\usepackage{alltt}   % handy verbatim stuff


% define some handy commands for escript stuff
\newcommand{\LINUX}{{\it Linux}\xspace}
\newcommand{\WINDOWS}{{\it MS Windows}\xspace}
\newcommand{\PYTHON}{{\it python}\xspace}
% \newcommand{\netCDF}{{\it netCDF}\cite{NETCDF}\index{netCDF} \xspace}
\newcommand{\netCDF}{{\it netCDF}\index{netCDF}\xspace}
\newcommand{\escript}{\module{esys.escript}\xspace}
\newcommand{\finley}{\module{esys.finley}\xspace}
\newcommand{\esys}{\module{esys}\xspace}
\newcommand{\pyvisi}{\module{esys.pyvisi}\xspace}
\newcommand{\pycad}{\module{esys.pycad}\xspace}
\newcommand{\gmsh}{\module{esys.pycad.gmsh}\xspace}
\newcommand{\gmshextern}{{\it Gmsh}\cite{GMSH}\index{Gmsh} \xspace}
\newcommand{\env}[1]{\textbf{\mbox{#1}}\index{Environment!#1}}
% \newcommand{\MPI}{{\it MPI}\xspace\index{Message Passing Interface!MPI}\cite{MPI}\xspace}
% \newcommand{\OPENMP}{{\it OpenMP}\xspace\index{OpenMP!threading}\cite{OPENMP}\xspace}
\newcommand{\MPI}{{\it MPI}\xspace\index{Message Passing Interface!MPI}\xspace}
\newcommand{\OPENMP}{{\it OpenMP}\xspace\index{OpenMP!threading}\xspace}
\newcommand{\linearPDEs}{\module{esys.escript.linearPDEs}\xspace}
\newcommand{\LinearPDE}{\class{LinearPDE}\xspace}
\newcommand{\timeseries}{\module{esys.timeseries}\xspace}
\newcommand{\modelframe}{\module{esys.modelframe}\xspace}
\newcommand{\pdetools}{\module{esys.pdetools}\xspace}
\newcommand{\esysxml}{{\it esysxml}}

\newcommand{\AdvectivePDE}{\class{AdvectivePDE}\xspace}
\newcommand{\Poisson}{\class{Poisson}\xspace}
\newcommand{\Helmholtz}{\class{Helmholtz}\xspace}
\newcommand{\Lame}{\class{Lame}\xspace}
\newcommand{\Data}{\class{Data}\xspace}
\newcommand{\EmptyData}{empty \class{Data}\index{empty Data}\xspace}
\newcommand{\Domain}{\class{Domain}\xspace}
%\newcommand{\VTK}{{\it vtk} \cite{VTK}\index{visualization!vtk} \xspace}
\newcommand{\GnuPlot}{{\it gnuplot} \cite{GNUPLOT}\index{visualization!gnuplot}\index{gnuplot}}
\newcommand{\mayavi}{{\it mayavi}\index{visualization!mayavi}\index{mayavi}}
\newcommand{\OpenDX}{{\it OpenDX} \cite{OPENDX}\xspace}
\newcommand{\finleyelement}[1]{{\it #1}\index{finley!#1}}
\newcommand{\False}{\constant{False}\xspace}
\newcommand{\True}{\constant{True}\xspace}
\newcommand{\PCG}{\constant{LinearPDE.PCG}\xspace\index{linear solver!PCG}\index{PCG}}
\newcommand{\BiCGStab}{\constant{LinearPDE.BICGSTAB}\xspace\index{linear solver!BiCGStab}\index{BiCGStab}}
\newcommand{\Direct}{\constant{LinearPDE.DIRECT}\xspace\index{linear solver!Direct}\index{Direct solver}}
\newcommand{\GMRES}{\constant{LinearPDE.GMRES}\xspace\index{linear solver!GMRES}\index{GMRES}}
\newcommand{\PRESTWENTY}{\constant{LinearPDE.PRES20}\xspace\index{linear solver!PRES20}\index{PRES20}}
\newcommand{\JACOBI}{\constant{LinearPDE.JACOBI}\xspace\index{preconditioner!Jacobi}\index{Jacobi}}
\newcommand{\ILU}{\constant{LinearPDE.ILU0}\xspace\index{preconditioner!ILU0}\index{ILU0}}
\newcommand{\ILUT}{\constant{LinearPDE.ILUT}\xspace\index{preconditioner!ILUT}\index{ILUT}}
\newcommand{\LUMPING}{\constant{LinearPDE.LUMPING}\xspace\index{linear solver!lumping}\index{lumping}}
\newcommand{\NOREORDERING}{\constant{LinearPDE.NO\hackscore REORDERING}\xspace}
\newcommand{\MINIMUMFILLIN}{\constant{LinearPDE.MINIMUM\hackscore FILL\hackscore IN}\xspace\index{linear solver!minimum fill-in ordering}\index{minimum fill-in ordering}}
\newcommand{\NESTEDDESCTION}{\constant{LinearPDE.NESTED\hackscore DISSECTION}\xspace\index{linear solver!nested dissection ordering}\index{nested dissection}}

\newcommand{\FunctionSpace}{\class{FunctionSpace}\xspace}
\newcommand{\Operator}{\class{Operator}\xspace}
\newcommand{\SolutionFS}{solution \class{FunctionSpace}\index{solution}\xspace}
\newcommand{\ReducedSolutionFS}{reduced solution \class{FunctionSpace}\index{solution!reduced}\xspace}
\newcommand{\FunctionOnBoundary}{boundary \class{FunctionSpace}\xspace}
\newcommand{\Function}{general \class{FunctionSpace}\xspace}
\newcommand{\FunctionOnContactZero}{contact \class{FunctionSpace} on side 0\xspace}
\newcommand{\FunctionOnContactOne}{contact \class{FunctionSpace} on side 1\xspace}
\newcommand{\ContinuousFunction}{continuous \class{FunctionSpace}\xspace}
\newcommand{\RankOne}{{rank-1 \Data object}\xspace}
\newcommand{\RankTwo}{{rank-2 \Data object}\xspace}
\newcommand{\RankThree}{{rank-3 \Data object}\xspace}
\newcommand{\RankFour}{{rank-4 \Data object}\xspace}
\newcommand{\Tensor}{{tensor \Data object}\xspace}
\newcommand{\Vector}{{vector \Data object}\xspace}
\newcommand{\Scalar}{{scalar \Data object}\xspace}
\newcommand{\DataSample}{{data sample}\index{data sample}\xspace}
\newcommand{\DataSamplePoints}{{data sample points}\index{data sample!points}\xspace}
\newcommand{\numarray}{\module{numarray}\xspace}
\newcommand{\numarrayNA}{\module{numarray}.\class{NumArray}\xspace}
\newcommand{\Shape}{shape\xspace\index{shape}}
\newcommand{\Rank}{rank\xspace\index{shape}}
\newcommand{\ExampleDirectory}{example directory\xspace}
\newcommand{\ReferenceGuide}{\url{http://shake200.esscc.uq.edu.au/esys/esys13/release/epydoc/index.html}\xspace}
\newcommand{\Point}{\class{Point} \xspace}
\newcommand{\PropertySet}{\class{PropertySet} \xspace}
\newcommand{\Design}{\class{Design} \xspace}
\newcommand{\TagMap}{\class{TagMap} \xspace}
\newcommand{\ManifoldOneD}{\class{Manifold1D} \xspace}
\newcommand{\ManifoldTwoD}{\class{Manifold2D} \xspace}
\newcommand{\ManifoldThreeD}{\class{Manifold3D} \xspace}

% handy commands for pyvisi
\newcommand{\VTK}{{\it VTK} \xspace}
\newcommand{\VTKUrl}{\url{http://www.vtk.org/}\xspace}
\newcommand{\Scene}{\class{Scene}\xspace}
\newcommand{\Camera}{\class{Camera}\xspace}
\newcommand{\Light}{\class{Light}\xspace}
\newcommand{\DataCollector}{\class{DataCollector}\xspace}
\newcommand{\ImageReader}{\class{ImageReader}\xspace}
\newcommand{\TextTwoD}{\class{Text2D}\xspace}
\newcommand{\ActorTwoD}{\class{Actor2D}\xspace}
\newcommand{\ActorThreeD}{\class{Actor3D}\xspace}
\newcommand{\Transform}{\class{Transform}\xspace}
\newcommand{\Clipper}{\class{Clipper}\xspace}
\newcommand{\Map}{\class{Map}\xspace}
\newcommand{\MapOnPlaneCut}{\class{MapOnPlaneCut}\xspace}
\newcommand{\MapOnPlaneClip}{\class{MapOnPlaneClip}\xspace}
\newcommand{\MapOnScalarClip}{\class{MapOnScalarClip}\xspace}
\newcommand{\MapOnScalarClipWithRotation}{\class{MapOnScalarClipWithRotation}\xspace}
\newcommand{\GlyphThreeD}{\class{Glyph3D}\xspace}
\newcommand{\MaskPoints}{\class{MaskPoints}\xspace}
\newcommand{\Velocity}{\class{Velocity}\xspace}
\newcommand{\VelocityOnPlaneCut}{\class{VelocityOnPlaneCut}\xspace}
\newcommand{\VelocityOnPlaneClip}{\class{VelocityOnPlaneClip}\xspace}
\newcommand{\Sphere}{\class{Sphere}\xspace}
\newcommand{\TensorGlyph}{\class{TensorGlyph}\xspace}
\newcommand{\Ellipsoid}{\class{Ellipsoid}\xspace}
\newcommand{\EllipsoidOnPlaneCut}{\class{EllipsoidOnPlaneCut}\xspace}
\newcommand{\EllipsoidOnPlaneClip}{\class{EllipsoidOnPlaneClip}\xspace}
\newcommand{\ContourModule}{\class{ContourModule}\xspace}
\newcommand{\Contour}{\class{Contour}\xspace}
\newcommand{\ContourOnPlaneCut}{\class{ContourOnPlaneCut}\xspace}
\newcommand{\ContourOnPlaneClip}{\class{ContourOnPlaneClip}\xspace}
\newcommand{\PointSource}{\class{PointSource}\xspace}
\newcommand{\StreamLineModule}{\class{StreamLineModule}\xspace}
\newcommand{\Tube}{\class{Tube}\xspace}
\newcommand{\StreamLine}{\class{StreamLine}\xspace}
\newcommand{\Warp}{\class{Warp}\xspace}
\newcommand{\Carpet}{\class{Carpet}\xspace}
\newcommand{\PlaneSource}{\class{PlaneSource}\xspace}
\newcommand{\Image}{\class{Image}\xspace}
\newcommand{\Logo}{\class{Logo}\xspace}
\newcommand{\ImageReslice}{\class{ImageReslice}\xspace}
\newcommand{\Position}{\class{Position}\xspace}
\newcommand{\DataSetMapper}{\class{DataSetMapper}\xspace}
\newcommand{\Legend}{\class{Legend}\xspace}
\newcommand{\ScalarBar}{\class{ScalarBar}\xspace}
\newcommand{\Movie}{\class{Movie}\xspace}
\newcommand{\CubeSource}{\class{CubeSource}\xspace}
\newcommand{\Rectangle}{\class{Rectangle}\xspace}
\newcommand{\LocalPosition}{\class{LocalPosition}\xspace}
\newcommand{\GlobalPosition}{\class{GlobalPosition}\xspace}
\newcommand{\Rotation}{\class{Rotation}\xspace}
\newcommand{\thumbnailwidth}{50mm}

% default width for figures
\newcommand{\figwidth}{100mm}
% commands useful in cross-referencing
\newcommand {\Ref}[1] {Reference~\cite{#1}}
\newcommand {\Sec}[1] {Section~\ref{#1}}
\newcommand {\App}[1] {Appendix~\ref{#1}}
\newcommand {\Chap}[1] {Chapter~\ref{#1}}
\newcommand {\etal} {\emph{~et~al.}}
\newcommand {\fig}[1] {Figure~\ref{#1}}
\newcommand {\eqn}[1] {Equation~(\ref{#1})} 
\newcommand {\tab}[1] {Table~\ref{#1}}

% this stops one figure taking up a whole page and lets more text onto
% the one page when a figure exists
\renewcommand{\floatpagefraction}{0.8} %   Default = 0.5

% improved version of caption handling
\usepackage{ccaption}
\captionnamefont{\scshape}
\captionstyle{}
\makeatletter
\renewcommand{\fnum@figure}[1]{\quad\small\textsc{\figurename~\thefigure}:}
\renewcommand{\@makecaption}[2]{%
\vskip\abovecaptionskip
\sbox\@tempboxa{#1: #2}%
\ifdim \wd\@tempboxa >\hsize
  \def\baselinestretch{1}\@normalsize
  #1: #2\par
  \def\baselinestretch{1.5}\@normalsize
\else
  \global \@minipagefalse
  \hb@xt@\hsize{\hfil\box\@tempboxa\hfil}%
\fi
\vskip\belowcaptionskip}
\makeatother

% \usepackage{fancyvrb}  % fancy verbatim stuff.  Needed so code below goes
%%% this code grabbed from the PyScript docs
%%% pyscript.sourceforge.net

% --------------------------------------------------------------
% Code format within \Verb
% --------------------------------------------------------------

% \definecolor{pycolor}{rgb}{0,0.4,0}

%% \DefineVerbatimEnvironment{python}{Verbatim}
%% {frame=leftline,framerule=.5mm,rulecolor=\color{pycolor},
%% formatcom=\color{pycolor}\small,fontshape=rm}

%\DefineShortVerb[formatcom=\color{dgreen}\small,fontshape=sl]{\|}

% \RecustomVerbatimCommand{\Verb}{Verb}{formatcom=\color{pycolor}\small,fontshape=rm}

%%% end of grabbed code

% this is for when one uses pdflatex and therefore needs to load pdf
% figures into \includegraphics
\ifpdf
	\DeclareGraphicsExtensions{.pdf}  % this command defined in graphicx
	\pdfcompresslevel=9  % 0: no compression, 9: highest compression
			     % or, set compress_level 9 in file pdftex.cfg
\else
	\DeclareGraphicsExtensions{.eps}
\fi

% defines the colour for the background of code examples
\definecolor{LightGrey}{gray}{0.9}

% add the listings package to pretty print the code output
\usepackage{listings}


%Some colour definitions added to keep pdflatex happy
%I make no claim that these values are particularly good
\definecolor{Purple}{rgb}{0.7, 0, 0.6}
\definecolor{Tan}{rgb}{0.5,0.5,0.5}
\definecolor{BrickRed}{rgb}{0.7, 0.2, 0.2}

% All the \color{x} used to be \color[named]{x}
%end color defs

\lstdefinestyle{myC++}{%
%\lstset{%
language=C++,
showstringspaces=false,
basicstyle=\small\ttfamily,
commentstyle=\color{BrickRed}\ttfamily,
keywordstyle=\color{Purple}\ttfamily,
%identifierstyle=\color{Blue}\ttfamily,
%functionstyle=\color{Blue}\ttfamily,
%typestyle=\color{ForestGreen}\ttfamily,
stringstyle=\color{Tan}\ttfamily,%
morekeywords={,complex,}%
frame=none,%
backgroundcolor=\color{LightGrey}%
}

\lstdefinestyle{myMatlab}{%
%\lstset{%
language=Matlab,
showstringspaces=false,
basicstyle=\small\ttfamily,
commentstyle=\color{BrickRed}\ttfamily,
keywordstyle=\color{Purple}\ttfamily,
%identifierstyle=\color{Blue}\ttfamily,
%functionstyle=\color{Blue}\ttfamily,
%typestyle=\color{ForestGreen}\ttfamily,
stringstyle=\color{Tan}\ttfamily,%
frame=none,%
backgroundcolor=\color{LightGrey}%
}

\lstdefinestyle{myScilab}{%
%\lstset{%
language=Scilab,
showstringspaces=false,
basicstyle=\small\ttfamily,
commentstyle=\color{BrickRed}\ttfamily,
keywordstyle=\color{Purple}\ttfamily,
%identifierstyle=\color{Blue}\ttfamily,
%functionstyle=\color{Blue}\ttfamily,
%typestyle=\color{ForestGreen}\ttfamily,
stringstyle=\color{Tan}\ttfamily,%
frame=none,%
backgroundcolor=\color{LightGrey}%
}

\lstdefinestyle{myShell}{%
%\lstset{%
language=ksh,
showstringspaces=false,
basicstyle=\small\ttfamily,
commentstyle=\color{Black}\ttfamily,
keywordstyle=\color{Black}\ttfamily,
%identifierstyle=\color{Blue}\ttfamily,
%functionstyle=\color{Blue}\ttfamily,
%typestyle=\color{ForestGreen}\ttfamily,
stringstyle=\color{Black}\ttfamily,%
frame=none,%
backgroundcolor=\color{LightGrey}%
}

\lstdefinestyle{myPython}{%
%\lstset{%
language=python,
showstringspaces=false,
basicstyle=\small\ttfamily,
commentstyle=\color{BrickRed}\ttfamily,
keywordstyle=\color{Purple}\ttfamily,
%identifierstyle=\color{Blue}\ttfamily,
%functionstyle=\color{Blue}\ttfamily,
%typestyle=\color{ForestGreen}\ttfamily,
stringstyle=\color{Tan}\ttfamily,%
frame=none,%
%backgroundcolor=\color{LightGrey}%
}

\lstdefinestyle{myhtml}{%
%\lstset{%
language=xml,
showstringspaces=false,
basicstyle=\small\ttfamily,
commentstyle=\color{BrickRed}\ttfamily,
keywordstyle=\color{Purple}\ttfamily,
%identifierstyle=\color{Blue}\ttfamily,
%functionstyle=\color{Blue}\ttfamily,
%typestyle=\color{ForestGreen}\ttfamily,
stringstyle=\color{Tan}\ttfamily,
morekeywords={,simulation,prop_dim,error_check,stochastic,%
  globals,field,dimensions,lattice,domains,samples,vector,%
  components,fourier_space,sequence,integrate,algorithm,%
  interval,k_operators,constant,operator_names,vectors,%
  output,filename,group,sampling,moments,benchmark,use_double,%
  use_wisdom,use_prefs,binary_output,cycles,filter,post_propagation,%
  default_value,argv,arg,iterations,cross_propagation,%
  use_mpi,paths,seed,noises,author,description,name,type,%
}
frame=none,%
%framerule=2pt,%
backgroundcolor=\color{LightGrey}%
}

% this implements producing nice code blocks
% it also saves time, typing and
% *should* reduce errors in the text by removing doubling up of code
\lstnewenvironment{xmdsCode}[1][]{\lstset{style=myhtml}\lstset{#1}}{}

% this implements nicely formatted shell code
\lstnewenvironment{shellCode}[1][]{\lstset{style=myShell}\lstset{#1}}{}

% this implements nicely formatted Perl code
\lstnewenvironment{perlCode}[1][]{\lstset{style=myPerl}\lstset{#1}}{}

% this implements nicely formatted Python code
\lstnewenvironment{python}[1][]{\lstset{style=myPython}\lstset{#1}}{}

% this implements nicely formatted C++ code
\lstnewenvironment{CCode}{\lstset{style=myC++}}{}

% this implements nicely formatted matlab code
\lstnewenvironment{matlabCode}{\lstset{style=myMatlab}}{}

% this implements nicely formatted scilab code
\lstnewenvironment{scilabCode}{\lstset{style=myScilab}}{}

