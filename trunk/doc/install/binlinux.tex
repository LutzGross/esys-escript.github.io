
%%%%%%%%%%%%%%%%%%%%%%%%%%%%%%%%%%%%%%%%%%%%%%%%%%%%%%%%
%
% Copyright (c) 2003-2008 by University of Queensland
% Earth Systems Science Computational Center (ESSCC)
% http://www.uq.edu.au/esscc
%
% Primary Business: Queensland, Australia
% Licensed under the Open Software License version 3.0
% http://www.opensource.org/licenses/osl-3.0.php
%
%%%%%%%%%%%%%%%%%%%%%%%%%%%%%%%%%%%%%%%%%%%%%%%%%%%%%%%%

\section{Linux binary installation}
\label{sec:binlinux}

\esfinley can be installed as a stand-alone bundle, containing all the required tools.
Alternatively, if we have a package for your distribution you can use the standard tools to install.

Please note, the current packages do not support \openmp\footnote{This is due to a bug related to gcc 4.3.2.} or \mpi\footnote{Producing packages for MPI requires knowing something about your computer's configuration.}.
If you need these features you may need to compile \esfinley from source (see Sections~\ref{sec:compilesrc} and \ref{sec:compileescriptlinux}.)

For more information on using the \filename{escript} command please see the User Guide.

If you are using Debian then see Section~\ref{sec:lenny}.
If you are using Debian (5.0 - ``Lenny'') or Ubuntu (8.10-``Intrepid Ibex'', 9.04-``Jaunty Jakalope'') then see Section~\ref{sec:lenny}
If you are using some other distribution of Linux, see Section~\ref{sec:standalonelinux} 


\subsection{Debian 5.0(``Lenny'')}\label{sec:lenny}

Download the \filename{escript.deb} file.
(At time of writing we only produce debs for the i386 architecture.)
Execute the following commands as root (you need to be in the directory containing the file).
\begin{shellCode}
 dpkg --unpack escript.deb
 aptitude install escript
\end{shellCode}

If you use sudo this would be:
\begin{shellCode}
sudo dpkg --unpack escript.deb
sudo aptitude install escript
\end{shellCode}

\subsection{Ubuntu 8.10(``Intrepid Ibex''), 9.04(``Jaunty Jakalope'')}

Since the installation process for \esfinley is pretty simple you should be able to use the Debian package for Ubuntu as well.
Please notify the development team if this is not the case.
Note that you will need to use the ``sudo'' instructions.


\subsection{Stand-alone bundle}\label{sec:standalonelinux}

Download the bundle and decompress it.
\begin{shellCode}
tar -xjf escript.tar.bz2 
\end{shellCode}
This will produce a directory called \filename{stand}. 
You can rename or move it as is convienient to you.
Test your installation by running:
\begin{shellCode}
 stand/escript.d/bin/escript
\end{shellCode}
You should get a normal python shell.
If you wish to save on typing you can add \filename{x/escript.d/bin} to your PATH variable (where x is the absolute path to your install). 




