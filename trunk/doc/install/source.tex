%!TEX root = install.tex
%%%%%%%%%%%%%%%%%%%%%%%%%%%%%%%%%%%%%%%%%%%%%%%%%%%%%%%%%%%%%%%%%%%%%%%%%%%%%%
% Copyright (c) 2012-2014 by University of Queensland
% http://www.uq.edu.au
%
% Primary Business: Queensland, Australia
% Licensed under the Open Software License version 3.0
% http://www.opensource.org/licenses/osl-3.0.php
%
% Development until 2012 by Earth Systems Science Computational Center (ESSCC)
% Development 2012-2013 by School of Earth Sciences
% Development from 2014 by Centre for Geoscience Computing (GeoComp)
%
%%%%%%%%%%%%%%%%%%%%%%%%%%%%%%%%%%%%%%%%%%%%%%%%%%%%%%%%%%%%%%%%%%%%%%%%%%%%%%

% Notes about compilers

\chapter{Installing from Source}\label{chap:source}

This chapter assumes you are using a unix/posix like system (including MacOSX).

\section{Parallel Technologies}\label{sec:par}
It is likely that the computer you run \escript on, will have more than one processor core.
\escript can make use of multiple cores [in order to solve problems more quickly] if it is told to do so,
but this functionality must be enabled at compile time.
Section~\ref{sec:needpar} gives some rough guidelines to help you determine what you need.

There are two technologies which \escript can employ here.
\begin{itemize}
 \item OpenMP -- more efficient of the two [thread level parallelism].
 \item MPI -- Uses multiple processes (less efficient), needs less help from the compiler.
\end{itemize}

Escript is primarily tested on recent versions of the GNU and Intel suites (``g++'' / ``icpc'').
However, it also passes our tests when compiled using ``clang++''.
The table below shows what methods are available with which compilers.

\begin{center}
\begin{tabular}{|l|c|c|c|}\hline
 & Serial & OpenMP & MPI \\\hline
 $\leq$ g++-4.2.1 & \checkmark & \raisebox{-0.1cm}{\footnotemark}& \checkmark \\\hline
 g++ (recent $\geq 4.3.2$)  & \checkmark& \checkmark& \checkmark \\\hline
 icpc(10) & \checkmark& \checkmark& \checkmark \\\hline
 icpc(11) & \checkmark& \raisebox{-0.1cm}{\footnotemark}  &\checkmark \\\hline
 icpc(12) & \checkmark& \checkmark&\checkmark \\\hline
 clang++ & \checkmark& & \checkmark\\\hline
\end{tabular}
\end{center}
\addtocounter{footnote}{-1}
\footnotetext{The \openmp support in g++-4.2.1 is buggy/non-functional.}
\addtocounter{footnote}{1}
\footnotetext{There is a subtle bug in icpc-11 when \openmp and c++ exception handling 
are combined.}

\noindent Where both \openmp and \mpi are marked, \escript can be compiled with either or both.
A \checkmark mark means that combination passes our tests.

\subsection{What parallel technology do I need?}\label{sec:needpar}
If you are using any version of Linux released in the past few years, then your system compiler will support 
\openmp with no extra work; so you should use it.
You will not need MPI unless your computer is some form of cluster.

If you are using BSD or MacOSX and you are just experimenting with \escript, then performance is
probably not a major issue for you at the moment so you don't need to use either \openmp or MPI. 
This also applies if you write and polish your scripts on your computer and then send them to a cluster to execute.
If in the future you find escript useful and your scripts take significant time to run, then you may want to reinstall 
\escript with more options.

In general, for a single computer, \openmp will give better performance (both time and memory) so use it if possible.
If \openmp is not an option, then use MPI.

Note that even if your version of \escript has support for \openmp or MPI, you will still need to tell the system to 
use it when you run your scripts.
If you are using the \texttt{run-escript} launcher, then this is controlled  by the \texttt{-t} and \texttt{-p} options.
If not, then consult the documentation for your MPI libraries (or the compiler documentation in the case of OpenMP
\footnote{It may be enough to set the \texttt{OMP\_NUM\_THREADS} environment variable.}).

If you are using MacOSX, then see the next section, if not, then skip to Section~\ref{sec:build}.

\section{MacOS}

To build \escript from source you need to choose and install the following (if you already have some parts installed, skip 
that section).
\begin{itemize}
 \item \texttt{XCode} and the associated command line tools. [Section~\ref{sec:initcompiler}]
 \item A package manager. [Section~\ref{sec:osxpackagemanager}]
 \item A compiler. [Section~\ref{sec:osxcompiler}] 
\end{itemize}

As noted above, different compilers have varying support for parallel processing technologies so your choice of 
compiler may limit your options of parallel technologies.
Once you have the pieces in this section, go to Section~\ref{sec:build} for build instructions.

\subsection{Initial compiler}\label{sec:initcompiler}
\emph{This information is as accurate as far as we can tell at time of writing but things may change.}

Even if you wish to install and use a different compiler later, the first step is generally 
to get some compiler onto your system.

As of OSX10.9, the command \texttt{xcodebuild} will allow you to download and install the commandline tools seemingly
without an AppleID.
For previous releases, it seems to be trickier to get the basic compiler without doing one of the following:
\begin{itemize}
 \item purchasing an iTunes gift card.
 \item giving Apple access to your credit card.
 \item signing up as an Apple developer\footnote{If you do this you can download a ``command line tools'' package
 which installs the relevant compilers without needing to install all of \texttt{XCode}.}  and giving up personal information.
\end{itemize}

If you install \texttt{XCode}, you will need to download the ``command line tools'' optional package [see \texttt{XCode} documentation for details].

There are also a number of projects on the net which aim to deliver compilers for MacOS.
Use at your own risk.
For example:
\begin{itemize}
 \item \url{http://hpc.sourceforge.net}
 \item \url{http://kennethreitz.org/experiments/xcode-gcc-and-homebrew}
\end{itemize}

\subsection{OSX Package Managers}\label{sec:osxpackagemanager}
Once you have a working compiler, you will need to consider how you will install the components \escript needs.
While it is certainly possible to compile and install these pieces manually, it can be easier to use a tool
which handles some of the details for you.
This is especially true if you will need to uninstall or upgrade them later.

On OSX, there are a number of options. 
The popular ones at time of writing seem to be \texttt{macports} and \texttt{homebrew}.
(We have not experimented with \texttt{fink}).
\escript does not assume that you are using a particular package manager\footnote{or even if you are using one at all.}.
The configuration instructions/files in this guide are just examples that we have found to work.

Note that package managers will make changes to your computer based on programs configured by other people from 
various places around the internet.
It is important to satisfy yourself as to the security of those systems.

Please consult the documentation for your chosen package manager to determine how to set it up.

\subsection{A compiler}\label{sec:osxcompiler}
While the command line tools described in \ref{sec:initcompiler}, do contain a c++ compiler, in recent versions of OSX, 
it is likely to be some version of \texttt{clang++}\footnote{even if it is 
labelled as g++ it is likely clang++ under the hood.}.
If you don't need \openmp or genuine \texttt{g++} or you have installed an \openmp supporting compiler
using some other method, then you can skip to Section~\ref{sec:build}.

If you are still reading this part, we assume that you wish to use \texttt{g++} to build escript on your system and that you are 
using a package manager.
The challenge here isn't installing the compiler itself. 
The issue is that the other dependencies that \escript needs have 
probably\footnote{or ``will probably be compiled'' in the case of \texttt{homebrew}.} been compiled by \texttt{clang} rather than the compiler 
you want to use.
A lot of the time this is fine, but in the case of \texttt{c++} libraries such as \texttt{boost}, the result can be two different 
standard c++ libraries\footnote{\texttt{libstdc++} and \texttt{libc++} on OSX 10.9.} being used in the same program and confusing matters.
Apart from giving up, there are two ways to try to solve this problem.
\begin{enumerate}
\item Changing the compiler that your package manager uses to compile its packages.
\emph{This is not for the faint hearted!} But we do provide some instructions in Appendix~\ref{chap:chcomp}.
\item Install most of the libraries using your package manager to install most things and then compile the c++ parts yourself.
(The most important one is \texttt{boost}).
We have tested this on Macports and it works well. The only trick is to make sure that your ``replacement'' components are in a separate 
directory which is ahead of you Macports directory in the various paths (\texttt{PATH, LD\_LIBRARY\_PATH, DYLD\_LIBRARY\_PATH}).
\end{enumerate}




% \subsection{Packaging System}
% The packaging system (also known as the package manager) is the tool you use to search for and install new open source software.
% For Linux, there will be one set up by default: the apt tools on Debian and Ubuntu, yast on Suse, yum on the RedHat family.
% On BSD systems this will be a combination of \texttt{pkg_add} and the \texttt{ports} tree.

% For MacOS, this is a bit more tricky.
% There are a number of possible systems including \texttt{macports} and \texttt{homebrew}\footnote{There is also \texttt{fink}, 
% but we have not experimented with that.}, but they do not come pre-installed so if you want to make use of one you will need
% you will need to install it.

% Packaging systems will make changes to your computer based on programs configured by other people from 
% various places around the internet.
% It is important to satisfy yourself as to the security of those systems.

% If you are using Linux or have decided that you don't want to use OpenMP skip to Section~\ref{sec:build}.


% \textsl{This whole section could be moved to an appendix.
% Need to make it clear somewhere near that it is not intended to be complicated but to give options.
% }


% make it clear that escript can be customised to use whatever you have

%Should include optional customisation

%Talk about options_files and the ability to specify them

%talk about -j1 and replacing it with more ops
%Talk about installation prefix

%also note that this doesn't build the doco but we do have downloads for that or you can install extra packages

\section{Building}\label{sec:build}

To simplify things for people, we have prepared \texttt{_options.py} files for a number of 
systems\footnote{These are correct a time of writing but later versions of those systems may require tweaks. 
Also, these systems represent a cross section of possible platforms rather than meaning those systems get particular support.}.
The \texttt{_options.py} files are locate in the scons/os directory. We suggest that the file most relavent to your os 
be copied from the os directory to the scons directory and renamed to the form XXXX_options.py where XXXX should be replaced with your computer's name.
If your particular system is not in the list below, or if you want a more customised 
build\footnote{for example, you want MPI functionality or you wish to use a different compiler}, 
see Section~\ref{sec:othersrc} for instructions.
\begin{itemize}
 \item Debian - \ref{sec:debsrc}
 \item Ubuntu - \ref{sec:ubsrc}
 \item OpenSuse - \ref{sec:susesrc}
 \item Centos - \ref{sec:centossrc}
 \item Fedora - \ref{sec:fedorasrc}
 \item MacOS (macports) - \ref{sec:macportsrc}
 \item MacOS (homebrew) - \ref{sec:homebrewsrc}
 \item FreeBSD - \ref{sec:freebsdsrc}
\end{itemize}

Once these are done proceed to Section~\ref{sec:cleanup} for cleanup steps.

All of these instructions assume that you have obtained the source (uncompressed it if necessary).
\subsection{Debian}\label{sec:debsrc}

\begin{shellCode}
sudo aptitude install python-dev python-numpy libboost-python-dev libnetcdf-dev
sudo aptitude install scons lsb-release
sudo aptitude install python-sympy python-matplotlib python-scipy
sudo aptitude install python-pyproj python-gdal 
\end{shellCode}


\begin{optionalstep}
If for some reason, you wish to rebuild the documentation, you would also need the following:
\begin{shellCode}
sudo aptitude install python-sphinx doxygen python-docutils texlive 
sudo aptitude install zip texlive-latex-extra latex-xcolor 
\end{shellCode}
\end{optionalstep}

\noindent In the source directory execute the following (substitute squeeze or wheezy as appropriate for XXXX):
\begin{shellCode}
scons -j1 options_file=scons/os/XXXX_options.py
\end{shellCode}

\noindent If you wish to test your build, you can use the following:
\begin{shellCode}
scons -j1 py_tests options_file=scons/os/XXXX_options.py 
\end{shellCode}

\subsection{Ubuntu}\label{sec:ubsrc}

If you have not installed \texttt{aptitude}, then substitute \texttt{apt-get} in the following.
\begin{shellCode}
sudo aptitude install python-dev python-numpy libboost-python-dev libnetcdf-dev
sudo aptitude install scons lsb-release
sudo aptitude install python-sympy python-matplotlib python-scipy
sudo aptitude install python-pyproj python-gdal 
\end{shellCode}


\begin{optionalstep}
If for some reason, you wish to rebuild the documentation, you would also need the following:
\begin{shellCode}
sudo aptitude install python-sphinx doxygen python-docutils texlive 
sudo aptitude install zip texlive-latex-extra latex-xcolor 
\end{shellCode}
\end{optionalstep}

\noindent In the source directory execute the following (substitute precise, quantal or raring as appropriate for XXXX):
\begin{shellCode}
scons -j1 options_file=scons/os/XXXX_options.py
\end{shellCode}

\noindent If you wish to test your build, you can use the following:
\begin{shellCode}
scons -j1 py_tests options_file=scons/os/XXXX_options.py 
\end{shellCode}



\subsection{OpenSuse}\label{sec:susesrc}
These instructions were prepared using release $13.1$.

\noindent Install packages from the main distribution:
\begin{shellCode}
sudo zypper install libboost_python1_53_0 python-devel python-numpy 
sudo zypper install python-scipy python-sympy python-matplotlib libnetcdf_c++-devel
sudo zypper install gcc-c++ scons boost-devel netcdf-devel
\end{shellCode}
These will allow you to use most features except some parts of the \downunder inversion library.
If you wish to use those, you will need some additional packages [python-pyproj, python-gdal].
This can be done after Escript installation.

\begin{optionalstep}
Add \url{http://ftp.suse.de/pub/opensuse/repositories/Application:/Geo/openSUSE_13.1/}
to your repositories in \texttt{zypper}.
\begin{shellCode}
sudo zypper install python-pyproj, python-gdal
\end{shellCode}
\end{optionalstep}

Now to build escript itself.
In the escript source directory:
\begin{shellCode}
scons -j1 options_file=scons/os/opensuse13.1_options.py
\end{shellCode}

\noindent If you wish to test your build, you can use the following:
\begin{shellCode}
scons -j1 py_tests options_file=scons/os/opensuse13.1_options.py 
\end{shellCode}

\noindent Now go to Section~\ref{sec:cleanup} for cleanup.

\subsection{Centos}\label{sec:centossrc}
These instructions were prepared using release $6.5$.
The core of escript works, however some functionality is not availible because the default packages for some dependencies in Centos are too old.

\noindent Install packages from the main distribution:
\begin{shellCode}
yum install python-devel numpy scipy scons boost-devel
yum install python-matplotlib gcc-c++
yum install boost-python 
\end{shellCode}

The above packages will allow you to use most features except saving and loading files in \texttt{netCDF} 
format and the \downunder inversion library.
If you wish to use those features, you will need to install some additional packages.
NetCDF needs to be installed when you compile if you wish to use it.
\begin{optionalstep}
\noindent Add the \texttt{EPEL} repository.
\begin{shellCode}
rpm -U http://download.fedoraproject.org/pub/epel/6/x86_64/epel-release-6-8.noarch.rpm
\end{shellCode}

\begin{shellCode}
yum install netcdf-devel gdal-python
\end{shellCode}
\end{optionalstep}

\noindent For some coordinate transformations, \downunder can also make use of the python interface to a tool called \texttt{proj}.
There does not seem to be an obvious centos repository for this though.
If it turns out to be necessary for your particular application, the source can be downloaded. 

\noindent Now to build escript itself.
In the escript source directory:
\begin{shellCode}
scons -j1 options_file=scons/os/centos6.5_options.py
\end{shellCode}

\noindent Now go to Section~\ref{sec:cleanup} for cleanup.

\subsection{Fedora}\label{sec:fedorasrc}
These instructions were prepared using release $20$.

\noindent Install packages
\begin{shellCode}
yum install netcdf-cxx-devel gcc-c++ scipy 
yum install sympy scons pyproj gdal python-matplotlib 
yum install boost-devel
\end{shellCode}

\noindent Now to build escript itself.
In the escript source directory:
\begin{shellCode}
scons -j1 options_file=scons/os/fedora18_options.py
\end{shellCode}

\noindent If you wish to test your build, you can use the following:
\begin{shellCode}
scons -j1 py_tests options_file=scons/os/fedora18_options.py 
\end{shellCode}

\noindent Now go to Section~\ref{sec:cleanup} for cleanup.

\subsection{MacOS (macports)}\label{sec:macportsrc}

\begin{shellCode}
port install python27
port select --set python python27
port install scons
port install openmpi
port install py27-numpy
port install boost
port install py27-sympy
port select --set py-sympy py27-sympy
install py27-scipy
install py27-pyproj
install py27-gdal
install py27-netcdf4
install netcdf-cxx
\end{shellCode}

\begin{shellCode}
scons -j1 options_file=scons/os/macports_options.py 
\end{shellCode}


\subsection{MacOS (homebrew)}\label{sec:homebrewsrc}

Note that these steps add ``non-official'' packages.
You will also want to make sure that the homebrew Python is executed in preference to the system 
Python\footnote{Putting \texttt{/usr/local/bin} at the front of your PATH is one way to do this.}.

\begin{shellCode}
brew install python
brew install scons
brew install boost
brew tap samueljohn/python
brew tap homebrew/science
pip install nose
brew install gfortran
brew install samueljohn/python/numpy
brew install scipy
brew install gdal
brew install openmpi
brew install matplotlib
brew install netcdf --enable-cxx-compat
\end{shellCode}

There do not appear to be formulae for \texttt{sympy} or \texttt{pyproj} so if you wish to use those features, then
you will need to install them separately.


\begin{shellCode}
scons -j1 options_file=scons/os/homebrew_options.py
\end{shellCode}


\subsection{FreeBSD}\label{sec:freebsdsrc}
The following was tested on the $9.1$ release of FreeBSD.
It passes the majority of tests but there is an issue related to some features in the inversion library.
The following set of installations ``works'' but is not guaranteed to be minimal\footnote{Depending on your needs you might be able to
get by with a smaller set of packages.}.

Install the following packages:
\begin{itemize}
 \item python
 \item scons
 \item boost-python-libs
 \item bash
\end{itemize}

Now install the following ports:
\begin{itemize}
 \item science/py-scipy
 \item science/py-netCDF4
 \item math/py-sympy
 \item graphics/py-pyproj
 \item graphics/py-gdal
 \item net/openmpi
\end{itemize}

You will need to add \texttt{/usr/local/mpi/openmpi/bin} to your path if you wish to build with MPI.

Next choose (or create) your options file.
In this case we have three prepared in the \texttt{scons/os} directory:
\begin{itemize}
 \item \texttt{freebsd91_options.py}
 \item \texttt{freebsd91_mpi_options.py}     If you would like to use MPI.
 \item \texttt{freebsd91_gcc46_options.py}   Use this if you have managed to change compilers to gcc4.6 (and would like to use OpenMP).
\end{itemize}

In the escript source directory (where ZZZ is your options file):
\begin{shellCode}
scons -j1 options_file=ZZZ
\end{shellCode}



\subsection{Other Systems / Custom Builds}\label{sec:othersrc}

\escript has support for a number of optional packages.
Some, like \texttt{netcdf} need to be enabled at compile time, while others, such as \texttt{sympy} and the projection packages
used in \downunder are checked at run time.
For the second type, you can install them at any time (ensuring that python can find them) and they should work.
For the first type, you need to modify the options file and recompile with scons.
The rest of this section deals with this.

To avoid having to specify the options file each time you run scons, copy an existing \texttt{_options.py} file from the 
\texttt{scons/} or \texttt{scons/os/} directories. Put the file in the \texttt{scons} directory and name 
it \textit{yourmachinename}\texttt{_options.py}.\footnote{If the name 
has - or other non-alpha characters, they must be replaced with underscores in the filename}.
For example: on a machine named toybox, the file would be \texttt{scons/toybox_options.py}.

Individual lines can be enabled/disabled, by removing or adding \# (the python comment character) to the beginning of the line.
For example, to enable OpenMP, change the line
\begin{verbatim}
#openmp = True 
\end{verbatim}
to
\begin{verbatim}
openmp = True 
\end{verbatim}.

If you are using libraries which are not installed in the standard places (or have different names) you will need to 
change the relevant lines.
A common need for this would be using a more recent version of the boost::python library.

You can also change the compiler or the options passed to it by modifying the relevant lines.

\subsubsection*{MPI}
If you wish to enable or disable MPI, or if you wish to use a different implementation of MPI, you can use the \texttt{mpi}
configuration variable.
To disable MPI use, \verb|mpi = 'none'|.
You will also need to ensure that the \texttt{mpi_prefix} and \texttt{mpi_libs} variables are uncommented and set correctly.

\subsubsection{Python3}
\escript works with \texttt{python3} but until recently, many distributions have not distributed python3 versions of their packages.
You can try it out though by modifying the following variables:

\begin{verbatim}
pythoncmd='python3'
\end{verbatim}

\begin{verbatim}
usepython3=True
\end{verbatim}

\begin{verbatim}
pythonlibname='whateveryourpython3libraryiscalled'
\end{verbatim}




\subsubsection{Testing}
As indicated earlier, you can test your build using \texttt{scons py_tests}.
Note however, that some features like \texttt{netCDF} are optional for using \escript, the tests will report a failure if
they are missing.

\section{Cleaning up}
\label{sec:cleanup}

Once the build (and optional testing) is complete, you can remove everything except:
\begin{itemize}
 \item bin
 \item esys
 \item lib
 \item doc
 \item CREDITS.TXT
 \item README_LICENSE
\end{itemize}
The last two aren't strictly required for operation.
The \texttt{doc} directory is not required either but does contain examples of escript scripts.

You can run escript using \texttt{\textit{path_to_escript_files}/bin/run-escript}.
Where \texttt{\textit{path_to_escript_files}} is replaced with the real path.

\begin{optionalstep}
You can add the escript \texttt{bin} directory to your \texttt{PATH} variable.
The launcher will then take care of the rest of the environment.
\end{optionalstep}

\section{Optional Extras}

Some other packages which might be useful include:
\begin{itemize}
 \item support for silo format (install the relevant libraries and enable them in the options file).
 \item Visit --- visualisation package. Can be used independently but our \texttt{weipa} library can make a Visit 
plug-in to allow direct visualisation of escript files.
 \item gmsh --- meshing software used by our \texttt{pycad} library.
 \item mayavi --- another visualisation tool.
\end{itemize}


%Need a better title but this is stuff like visit and silo (for non-debian distros)
%Perhaps - optional extras





