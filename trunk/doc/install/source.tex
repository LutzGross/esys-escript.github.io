%%%%%%%%%%%%%%%%%%%%%%%%%%%%%%%%%%%%%%%%%%%%%%%%%%%%%%%%%%%%%%%%%%%%%%%%%%%%%%
% Copyright (c) 2012-2013 by University of Queensland
% http://www.uq.edu.au
%
% Primary Business: Queensland, Australia
% Licensed under the Open Software License version 3.0
% http://www.opensource.org/licenses/osl-3.0.php
%
% Development until 2012 by Earth Systems Science Computational Center (ESSCC)
% Development since 2012 by School of Earth Sciences
%
%%%%%%%%%%%%%%%%%%%%%%%%%%%%%%%%%%%%%%%%%%%%%%%%%%%%%%%%%%%%%%%%%%%%%%%%%%%%%%

% Notes about compilers

\chapter{Installing from Source}\label{chap:source}

This chapter assumes you are using a unix/posix like system.

\section{Intro}

Some initial questions:
\begin{enumerate}
\item Are you using MacOS? (If not, then skip to Section~\ref{sec:par}).
\item Which parallel technologies do you wish to use?
\item Which packaging system are you using?
\end{enumerate}



\subsection{MacOS and compilers}
\emph{This information is as accurate as far as we can tell at time of writing but things may change.}

In order to install \escript from source, you need a compiler.
This is a bit harder on MacOS than on other systems which either provide a compiler as part of the base install (BSD) or allow 
people to download one as part of the install process (Linux).
In earlier releases of its operating system (``Snow Leopard'' and earlier), Apple included an optional ``developer tools'' package (\texttt{XCode}) on the install media.
For more recent releases, \texttt{XCode} is available for ``free'' from Apple's application store.
Why ``free''? 
The only way to access the application store is with an AppleID which requires either:
\begin{itemize}
 \item purchasing an iTunes gift card.
 \item giving Apple access to your credit card.
 \item signing up as an Apple developer\footnote{If you do this you can download a ``command line tools'' package
 which installs the relevant compilers without needing to install all of \texttt{XCode}.}  and giving up personal information.
\end{itemize}

If you install \texttt{XCode}, you will need to download the ``command line tools'' optional package [see \texttt{XCode} documentation for details].

There are also a number of projects on the net which aim to deliver compilers for MacOS.
Use at your own risk.
For example:
\begin{itemize}
 \item \url{http://hpc.sourceforge.net}
 \item \url{http://kennethreitz.org/experiments/xcode-gcc-and-homebrew}
\end{itemize}


\subsection{Parallel Technologies}\label{sec:par}
It is extremely likely that the computer you run \escript on, will have more than one processor core.
\escript can make use of multiple cores [in order to solve problems more quickly] if it is told to do so,
but this functionality must be enabled at compile time.

There are two technologies which \escript can employ here.
\begin{itemize}
 \item OpenMP -- more efficient of the two [thread level parallelism].
 \item MPI -- Uses multiple processes (less efficient), needs less help from the compiler.
\end{itemize}

Escript is primarily tested on recent versions of the GNU or Intel suites (``gcc, g++'' / ``icc, icpc'').
However, it also passes our tests when compiled using ``clang, clang++''.
The table below shows what methods are available with which compilers.

\begin{center}
\begin{tabular}{|l|c|c|c|}\hline
 & Serial & OpenMP & MPI \\\hline
 $\leq$ gcc-4.2.1 & \checkmark & \raisebox{-0.1cm}{\footnotemark}& \checkmark \\\hline
 gcc (recent $\geq 4.3.2$)  & \checkmark& \checkmark& \checkmark \\\hline
 icc(10) & \checkmark& \checkmark& \checkmark \\\hline
 icc(11) & \checkmark& \raisebox{-0.1cm}{\footnotemark}  &\checkmark \\\hline
 icc(12) & \checkmark& \checkmark&\checkmark \\\hline
 clang & \checkmark& & \checkmark\\\hline
\end{tabular}
\end{center}
\addtocounter{footnote}{-1}
\footnotetext{The \openmp support in gcc-4.2.1 is buggy/non-functional.}
\addtocounter{footnote}{1}
\footnotetext{There is a subtle bug in icc-11 when \openmp and c++ exception handling 
are combined.}

\noindent Where both \openmp and \mpi are marked, \escript can be compiled with either or both.
A \checkmark mark means that combination passes our tests.



\subsection{Packaging System}
The packaging system (also known as the package manager) is the tool you use to search for and install new open source software.
For Linux, there will be one set up by default: the apt tools on Debian and Ubuntu, yast on Suse, yum on the RedHat family.
On BSD systems this will be a combination of \texttt{pkg_add} and the \texttt{ports} tree.

For MacOS, this is a bit more tricky.
There are a number of possible systems including \texttt{macports} and \texttt{homebrew}\footnote{There is also \texttt{fink}, 
but we have not experimented with that.}, but they do not come pre-installed so if you want to make use of one you will need
you will need to install it.

Packaging systems will make changes to your computer based on programs configured by other people from 
various places around the internet.
It is important to satisfy yourself as to the security of those systems.

If you are using Linux or have decided that you don't want to use OpenMP skip to Section~\ref{sec:build}.

\subsubsection*{Changing Compilers on MacOS and BSD}
\emph{Not for the faint-hearted.}

In order to use OpenMP, your compiler must support it.
The compilers provided in Apple's package and BSD do not support OpenMP.
\texttt{Clang} has no support for it at all, while \texttt{GCC4.2.1} has some, 
but it does not work very well\footnote{Rejects valid code, crashes etc}.
You can use the packaging system to install a more up to date version of \texttt{gcc}.
However, you will need to install a number of other dependencies in order for escript to operate.
If these dependencies end up fighting over which versions of libraries to use, there will be problems.
\footnote{For example: Trying to use two versions of \texttt{Python} in the same program 
or two versions of the \texttt{c++} standard library.}
To avoid this, if you are intend to change the compiler or python versions, you should install the new compiler first,
then the new version of python.
After this, you can install the other dependencies.
\emph{Please note that none of the Mac based package managers nor BSD recommend changing the default $C$ compiler so
do so at your own risk.  If you want OpenMP though, there does not seem to be a choice.}

The following steps seemed to work when we tested them but we cannot make guarantees.

\subsubsection*{Changing compiler on FreeBSD(9.1)}
The following sequence passes our unit tests under OpenMP.
\begin{itemize}
 \item Install \texttt{gcc46} from ports.
 \item Modify \texttt{/etc/make.conf} to set the default compiler to be \texttt{gcc46}
 \item Install the remaining dependencies. 
 \item Configure \escript to build with \openmp.
\end{itemize}

We chose version $4.6$ rather than a later one because one of the optional dependencies (scipy) will try to install it anyway.

\subsection*{Changing compiler on MacPorts}
The following sequence passes our unit tests under OpenMP.
\begin{itemize}
 \item \texttt{port install gcc47}
 \item Set the default compiler for the command line with: \texttt{port select --set gcc mp-gcc47}
 \item Set the default compiler for macports by adding the following line to \hfill~\linebreak[4] \file{/opt/local/etc/macports/macports.conf}:
\begin{shellCode} 
#Added by user to coerce macports into using a newer compiler
default_compiler	macports-gcc-4.7 
\end{shellCode}
\item Now install the remainder of the dependencies using \texttt{port -s install X}.
This will build them from source rather than downloading precompiled versions.
(This will unfortunately mean larger downloads.).
In some cases, some dependencies will not build properly from source.
In those cases(where XYZ is the dependency that fails to build), \texttt{port clean XYZ} then \texttt{port install XYZ}.
Then \emph{try to install the rest from source}.

Installing from source via macports is necessary to convince macports to link against the libraries provided by 
your new compiler rather than the default one.
\end{itemize}




\subsection*{Changing compiler on Homebrew}
There is no configuration file that needs to be changed in order to use a different compiler.
You do need to do things in the correct order and make sure that you have made any required changes to 
your environment variables before moving on to the next step.
In particular, the compiler must be installed before anything else, followed by Python.


% make it clear that escript can be customised to use whatever you have

%Should include optional customisation

%Talk about options_files and the ability to specify them

%talk about -j1 and replacing it with more ops
%Talk about installation prefix

%also note that this doesn't build the doco but we do have downloads for that or you can install extra packages

\section{Building}\label{sec:build}

To simplify things for people, we have prepared \texttt{_options.py} files for a number of 
systems\footnote{These are correct a time of writing but later versions of those systems may require tweaks. 
Also, these systems represent a cross section of possible platforms rather than meaning those systems get particular support.}.
If your particular system is not in the above list, or if you want a more customised 
build\footnote{for example, you want MPI functionality or you wish to use a different compiler}, 
see Section~\ref{sec:othersrc} for instructions.
\begin{itemize}
 \item Debian - \ref{sec:debsrc}
 \item Ubuntu - \ref{sec:ubsrc}
 \item OpenSuse - \ref{sec:susesrc}
 \item Centos - \ref{sec:centossrc}
 \item Fedora - \ref{sec:fedorasrc}
 \item MacOS (macports) - \ref{sec:macportsrc}
 \item MacOS (homebrew) - \ref{sec:homebrewsrc}
 \item FreeBSD - \ref{sec:freebsdsrc}
\end{itemize}

Once these are done proceed to Section~\ref{sec:cleanup} for cleanup steps.

All of these instructions assume that you have obtained the source (uncompressed it if necessary).
\subsection{Debian}\label{sec:debsrc}

\begin{shellCode}
sudo aptitude install python-dev python-numpy libboost-python-dev libnetcdf-dev
sudo aptitude install scons lsb-release
sudo aptitude install python-sympy python-matplotlib python-scipy
sudo aptitude install python-pyproj python-gdal 
\end{shellCode}


\begin{optionalstep}
If for some reason, you wish to rebuild the documentation, you would also need the following:
\begin{shellCode}
sudo aptitude install python-sphinx doxygen python-docutils texlive 
sudo aptitude install zip texlive-latex-extra latex-xcolor 
\end{shellCode}
\end{optionalstep}

\noindent In the source directory execute the following (substitute squeeze or wheezy as appropriate for XXXX):
\begin{shellCode}
scons -j1 options_file=scons/os/XXXX_options.py
\end{shellCode}

\noindent If you wish to test your build, you can use the following:
\begin{shellCode}
scons -j1 py_tests options_file=scons/os/XXXX_options.py 
\end{shellCode}

\subsection{Ubuntu}\label{sec:ubsrc}

If you have not installed \texttt{aptitude}, then substitute \texttt{apt-get} in the following.
\begin{shellCode}
sudo aptitude install python-dev python-numpy libboost-python-dev libnetcdf-dev
sudo aptitude install scons lsb-release
sudo aptitude install python-sympy python-matplotlib python-scipy
sudo aptitude install python-pyproj python-gdal 
\end{shellCode}


\begin{optionalstep}
If for some reason, you wish to rebuild the documentation, you would also need the following:
\begin{shellCode}
sudo aptitude install python-sphinx doxygen python-docutils texlive 
sudo aptitude install zip texlive-latex-extra latex-xcolor 
\end{shellCode}
\end{optionalstep}

\noindent In the source directory execute the following (substitute precise, quantal or raring as appropriate for XXXX):
\begin{shellCode}
scons -j1 options_file=scons/os/XXXX_options.py
\end{shellCode}

\noindent If you wish to test your build, you can use the following:
\begin{shellCode}
scons -j1 py_tests options_file=scons/os/XXXX_options.py 
\end{shellCode}



\subsection{OpenSuse}\label{sec:susesrc}
These instructions were prepared using release $12.3$.

\noindent Install packages from the main distribution:
\begin{shellCode}
sudo yast2 --install libboost_python1_49_0 python-devel python-numpy 
sudo yast2 --install python-scipy python-sympy python-matplotlib libnetcdf_c++-devel
sudo yast2 --install gcc-c++ scons boost-devel netcdf-devel
\end{shellCode}
These will allow you to use most features except some parts of the \downunder inversion library.
If you wish to use those, you will need some additional packages [python-pyproj, python-gdal].
This can be done after Escript installation.

\begin{optionalstep}
Add \url{http://ftp.suse.de/pub/opensuse/repositories/Application:/Geo/openSUSE_12.3/}
to your repositories in \texttt{YaST}.
\begin{shellCode}
sudo yast2 --install python-pyproj, python-gdal
\end{shellCode}
\end{optionalstep}

Now to build escript itself.
In the escript source directory:
\begin{shellCode}
scons -j1 options_file=scons/os/opensuse12.3_options.py
\end{shellCode}

\noindent If you wish to test your build, you can use the following:
\begin{shellCode}
scons -j1 py_tests options_file=scons/os/opensuse12.3_options.py 
\end{shellCode}

\noindent Now go to Section~\ref{sec:cleanup} for cleanup.

\subsection{Centos}\label{sec:centossrc}
These instructions were prepared using release $6.4$.

\noindent Install packages from the main distribution:
\begin{shellCode}
yum install python-devel numpy scipy scons boost-devel
yum install python-matplotlib gcc-g++
yum install boost-python 
\end{shellCode}

The above packages will allow you to use most features except saving and loading files in \texttt{netCDF} 
format and the \downunder inversion library.
If you wish to use those features, you will need to install some additional packages.
NetCDF needs to be installed when you compile if you wish to use it.
\begin{optionalstep}
\noindent Add the \texttt{EPEL} repository.
\begin{shellCode}
rpm -U http://download.fedoraproject.org/pub/epel/6/x86_64/epel-release-6-8.noarch.rpm
\end{shellCode}

\begin{shellCode}
yum install netcdf-devel sympy gdal-python
\end{shellCode}
\end{optionalstep}

\noindent For some coordinate transformations, \downunder can also make use of the python interface to a tool called \texttt{proj}.
There does not seem to be an obvious centos repository for this though.
If it turns out to be necessary for your particular application, the source can be downloaded. 

\noindent Now to build escript itself.
In the escript source directory:
\begin{shellCode}
scons -j1 options_file=scons/os/centos6.4_options.py
\end{shellCode}

\noindent If you wish to test your build, you can use the following:
\begin{shellCode}
scons -j1 py_tests options_file=scons/os/centos6.4_options.py 
\end{shellCode}

\noindent Now go to Section~\ref{sec:cleanup} for cleanup.

\subsection{Fedora}\label{sec:fedorasrc}
These instructions were prepared using release $18$.

\noindent Install packages
\begin{shellCode}
yum install netcdf-cxx-devel gcc-c++ scipy 
yum install sympy scons pyproj gdal python-matplotlib 
yum install boost-devel
\end{shellCode}

\noindent Now to build escript itself.
In the escript source directory:
\begin{shellCode}
scons -j1 options_file=scons/os/fedora18_options.py
\end{shellCode}

\noindent If you wish to test your build, you can use the following:
\begin{shellCode}
scons -j1 py_tests options_file=scons/os/fedora18_options.py 
\end{shellCode}

\noindent Now go to Section~\ref{sec:cleanup} for cleanup.

\subsection{MacOS (macports)}\label{sec:macportsrc}

\begin{shellCode}
port install python27
port select --set python python27
port install scons
port install openmpi
port install py27-numpy
port install boost
port install py27-sympy
port select --set py-sympy py27-sympy
install py27-scipy
install py27-pyproj
install py27-gdal
install py27-netcdf4
install netcdf-cxx
\end{shellCode}

\begin{shellCode}
scons -j1 options_file=scons/os/macports_options.py 
\end{shellCode}


\subsection{MacOS (homebrew)}\label{sec:homebrewsrc}

Note that these steps add ``non-official'' packages.
You will also want to make sure that the homebrew Python is executed in preference to the system 
Python\footnote{Putting \texttt{/usr/local/bin} at the front of your PATH is one way to do this.}.

\begin{shellCode}
brew install scons
brew install python
brew install boost
brew tap samueljohn/python
brew tap homebrew/science
pip install nose
brew install gfortran
brew install samueljohn/python/numpy
brew install scipy
brew install gdal
brew install openmpi
brew install matplotlib
brew install netcdf --enable-cxx-compat
\end{shellCode}

There do not appear to be formulae for \texttt{sympy} or \texttt{pyproj} so if you wish to use those features, then
you will need to install them separately.


\begin{shellCode}
scons -j1 options_file=scons/os/homebrew_options.py
\end{shellCode}


\subsection{FreeBSD}\label{sec:freebsdsrc}
The following was tested on the $9.1$ release of FreeBSD.
It passes the majority of tests but there is an issue related to some features in the inversion library.
The following set of installations ``works'' but is not guaranteed to be minimal\footnote{Depending on your needs you might be able to
get by with a smaller set of packages.}.

Install the following packages:
\begin{itemize}
 \item subversion
 \item scons
 \item boost-python-libs
 \item bash
\end{itemize}

Now install the following ports:
\begin{itemize}
 \item science/py-scipy
 \item science/py-netCDF4
 \item math/sympy
 \item graphics/py-pyproj
 \item graphics/py-gdal
 \item net/openmpi
\end{itemize}

You will need to add \texttt{/usr/local/mpi/openmpi/bin} to your path if you wish to build with MPI.

Next choose (or create) your options file.
In this case we have three prepared in the \texttt{scons/os} directory:
\begin{itemize}
 \item \texttt{freebsd91_options.py}
 \item \texttt{freebsd91_mpi_options.py}     If you would like to use MPI.
 \item \texttt{freebsd91_gcc46_options.py}   Use this if you have managed to change compilers to gcc4.6 (and would like to use OpenMP).
\end{itemize}

In the escript source directory (where ZZZ is your options file):
\begin{shellCode}
scons -j1 options_file=ZZZ
\end{shellCode}



\subsection{Other Systems / Custom Builds}\label{sec:othersrc}

\escript has support for a number of optional packages.
Some, like \texttt{netcdf} need to be enabled at compile time, while others, such as \texttt{sympy} and the projection packages
used in \downunder are checked at run time.
For the second type, you can install them at any time (ensuring that python can find them) and they should work.
For the first type, you need to modify the options file and recompile with scons.
The rest of this section deals with this.

To avoid having to specify the options file each time you run scons, copy an existing \texttt{_options.py} file from the 
\texttt{scons/} or \texttt{scons/os/} directories. Put the file in the \texttt{scons} directory and name 
it \textit{yourmachinename}\texttt{_options.py}.\footnote{If the name 
has - or other non-alpha characters, they must be replaced with underscores in the filename}.
For example: on a machine named toybox, the file would be \texttt{scons/toybox_options.py}.

Individual lines can be enabled/disabled, by removing or adding \# (the python comment character) to the beginning of the line.
For example, to enable OpenMP, change the line
\begin{verbatim}
#openmp = True 
\end{verbatim}
to
\begin{verbatim}
openmp = True 
\end{verbatim}.

If you are using libraries which are not installed in the standard places (or have different names) you will need to 
change the relevant lines.
A common need for this would be using a more recent version of the boostpython library.

You can also change the compiler or the options passed to it by modifying the relevant lines.

\subsubsection*{MPI}
If you wish to enable or disable MPI, or if you wish to use a different implementation of MPI, you can use the \texttt{mpi}
configuration variable.
To disable MPI use, \verb|mpi = 'none'|.
You will also need to ensure that the \texttt{mpi_prefix} and \texttt{mpi_libs} variables are uncommented and set correctly.

\subsubsection{Python3}
\escript works with \texttt{python3} but until recently, many distributions have not distributed python3 versions of their packages.
You can try it out though by modifying the following variables:

\begin{verbatim}
pythoncmd='python3'
\end{verbatim}

\begin{verbatim}
usepython3=True
\end{verbatim}

\begin{verbatim}
pythonlibname='whateveryourpython3libraryiscalled'
\end{verbatim}




\subsubsection{Testing}
As indicated earlier, you can test your build using \texttt{scons py_tests}.
Note however, that some features like texttt{netCDF} are optional for using \escript, the tests will report a failure if
they are missing.

\section{Cleaning up}
\label{sec:cleanup}

Once the build (and optional testing) is complete, you can remove everything except:
\begin{itemize}
 \item bin
 \item esys
 \item lib
 \item doc
 \item CREDITS.TXT
 \item README_LICENSE
\end{itemize}
The last two aren't strictly required for operation.
The \texttt{doc} directory is not required either but does contain examples of escript scripts.

You can run escript using \texttt{\textit{path_to_escript_files}/bin/run-escript}.
Where \texttt{\textit{path_to_escript_files}} is replaced with the real path.

\begin{optionalstep}
You can add the escript \texttt{bin} directory to your \texttt{PATH} variable.
The launcher will then take care of the rest of the environment.
\end{optionalstep}

\section{Optional Extras}

Some other packages which might be useful include:
\begin{itemize}
 \item support for silo format (install the relevant libraries and enable them in the options file).
 \item Visit --- visualisation package. Can be used independently but our \texttt{weipa} library can make a Visit 
plug-in to allow direct visualisation of escript files.
 \item gmsh --- meshing software used by our \texttt{pycad} library.
 \item mayavi --- another visualisation tool.
\end{itemize}


%Need a better title but this is stuff like visit and silo (for non-debian distros)
%Perhaps - optional extras





