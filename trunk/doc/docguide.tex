\vbox{}
\vfill
\begin{center}
\textbf{\Large Guide to Documentation}\pdfbookmark[0]{Documentation guide}{documentation guide}
\vspace{0.5cm}

Documentation for \escript comes in a number of parts.
Here is a rough guide to what goes where.

\vspace{1cm}
\hrule
\vspace{1cm}

\begin{tabular}{ll}
 install.pdf & ``Installation guide for \emph{esys-Escript}'': 
Instructions for compiling \escript for your system from its source code. 
 Also briefly covers installing \texttt{.deb} packages for Debian and Ubuntu.
 
 \item[cookbook.pdf] ``The \textit{escript} COOKBOOK'':
 A introduction to \escript for new users from a geophysics perspective.
 
 \item[user.pdf]  ``\emph{esys-Escript} User's Guide: Solving Partial Differential Equations with Escript and Finley'':
 Covers main \emph{escript} concepts.
 
 \item[inversion.pdf] ``\downunder: Inversion with \escript'':
 Explanation of the inversion toolbox for \escript.
 
 \item[sphinx_api directory] Documentation for \emph{escript} Python libraries.
 
 \item[escript_examples(.tar.gz)/(.zip)] Full example scripts referred to by other parts of the documentation.
 
 \item[doxygen directory] Documentation for C++ libraries (mostly only of interest for developers).
  
 
\end{tabular}


\begin{description}
 \item[install.pdf] ``Installation guide for \emph{esys-Escript}'': 
Instructions for compiling \escript for your system from its source code. 
 Also briefly covers installing \texttt{.deb} packages for Debian and Ubuntu.
 
 \item[cookbook.pdf] ``The \textit{escript} COOKBOOK'':
 A introduction to \escript for new users from a geophysics perspective.
 
 \item[user.pdf]  ``\emph{esys-Escript} User's Guide: Solving Partial Differential Equations with Escript and Finley'':
 Covers main \emph{escript} concepts.
 
 \item[inversion.pdf] ``\downunder: Inversion with \escript'':
 Explanation of the inversion toolbox for \escript.
 
 \item[sphinx_api directory] Documentation for \emph{escript} Python libraries.
 
 \item[escript_examples(.tar.gz)/(.zip)] Full example scripts referred to by other parts of the documentation.
 
 \item[doxygen directory] Documentation for C++ libraries (mostly only of interest for developers).
  
\end{description}




\end{center}
\vfill
\pagebreak