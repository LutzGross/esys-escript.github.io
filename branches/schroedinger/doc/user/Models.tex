
%%%%%%%%%%%%%%%%%%%%%%%%%%%%%%%%%%%%%%%%%%%%%%%%%%%%%%%%
%
% Copyright (c) 2003-2008 by University of Queensland
% Earth Systems Science Computational Center (ESSCC)
% http://www.uq.edu.au/esscc
%
% Primary Business: Queensland, Australia
% Licensed under the Open Software License version 3.0
% http://www.opensource.org/licenses/osl-3.0.php
%
%%%%%%%%%%%%%%%%%%%%%%%%%%%%%%%%%%%%%%%%%%%%%%%%%%%%%%%%


\chapter{Models}

The following sections give a breif overview of the model classes and their corresponding methods.

\section{Stokes Cartesian (Saddle Point Problem)}

\subsection{Description}

Saddle point type problems emerge in a number of applications throughout physics and engineering. Finite element discretisation of the Navier-Stokes (momentum) equations for incompressible flow leads to equations of a saddle point type, which can be formulated as a solution of the following operator problem for $u \in V$ and $p \in Q$ with suitable Hilbert spaces $V$ and $Q$:

\begin{equation}
\left[ \begin{array}{cc}
A     & B \\
b^{*} & 0 \\
\end{array} \right]
\left[ \begin{array}{c}
u \\
p \\
\end{array} \right]
=\left[ \begin{array}{c}
f \\
g \\
\end{array} \right]
\label{SADDLEPOINT}
\end{equation}

where $A$ is coercive, self-adjoint linear operator in $V$, $B$ is a linear operator from $Q$ into $V$ and $B^{*}$ is the adjoint operator of $B$. $f$ and $g$ are given elements from $V$ and $Q$ respectivitly. For more details on the mathematics see references \cite{AAMIRBERKYAN2008,MBENZI2005}.

The Uzawa scheme scheme is used to solve the momentum equation with the secondary condition of incompressibility \cite{GROSS2006,AAMIRBERKYAN2008}.

\begin{classdesc}{StokesProblemCartesian}{domain,debug}
opens the stokes equations on the \Domain domain. Setting debug=True switches the debug mode to on.
\end{classdesc}

example usage:

solution=StokesProblemCartesian(mesh) \\
solution.setTolerance(TOL) \\
solution.initialize(fixed\_u\_mask=b\_c,eta=eta,f=Y) \\
velocity,pressure=solution.solve(velocity,pressure,max\_iter=max\_iter,solver=solver) \\

\subsection{Benchmark Problem}

Convection problem


\section{Temperature Cartesian}

\begin{equation}
\rho c\hackscore{p} \left (\frac{\partial T}{\partial t} + \vec{v} \cdot \nabla T \right ) = k \nabla^{2}T
\label{HEAT EQUATION}
\end{equation}

where $\vec{v}$ is the velocity vector, $T$ is the temperature, $\rho$ is the density, $\eta$ is the viscosity, $c\hackscore{p}$ is the specific heat at constant pressure and $k$ is the thermal conductivity.

\subsection{Description}

\subsection{Method}

\begin{classdesc}{TemperatureCartesian}{dom,theta=THETA,useSUPG=SUPG}
\end{classdesc}

\subsection{Benchmark Problem}


\section{Level Set Method}

\subsection{Description}

\subsection{Method}

Advection and Reinitialisation

\begin{classdesc}{LevelSet}{mesh, func\_new, reinit\_max, reinit\_each, tolerance, smooth}
\end{classdesc}

%example usage:

%levelset = LevelSet(mesh, func\_new, reinit\_max, reinit\_each, tolerance, smooth)

\begin{methoddesc}[LevelSet]{update\_parameter}{parameter}
Update the parameter.
\end{methoddesc}

\begin{methoddesc}[LevelSet]{update\_phi}{paramter}{velocity}{dt}{t\_step}
Update level set function; advection and reinitialization
\end{methoddesc}

\subsection{Benchmark Problem}

Rayleigh-Taylor instability problem


\section{Drucker Prager Model}

\section{Isotropic Kelvin Material \label{IKM}}



\begin{equation}\label{IKM-EQU-2}
D_{ij}=D_{ij}^{el}+D_{ij}^{vp}
\end{equation}
with the elastic stretching
\begin{equation}\label{IKM-EQU-3}
D_{ij}^{el}=\frac{2 \mu} \sigma'_{ij}
\end{equation}
\begin{equation}\label{IKM-EQU-4}
D_{ij}^{vp}=\sum_{q} D_{ij}^{q}
\end{equation}
\begin{equation}\label{IKM-EQU-5}
D_{ij}^{q}=\frac{1}{2 \eta^{q}} \sigma'_{ij} \mbox{ with } \eta^{q}=\eta^{q}_N \left(\frac{\tau}{\tau_t^q}\right){\frac{1}{n^{q}}-1}
\end{equation}
After inserting equation~\ref{IKM-EQU-5} into equation \ref{IKM-EQU-4} one gets:
\begin{equation}\label{IKM-EQU-4}
D_{ij}^{vp}=\frac{1}{2 \eta^{vp}} \sigma'_{ij}
\end{equation}


\begin{equation}\label{IKM-EQU-1}
-\sigma'_{ij,j}+p_j=F_j
\end{equation}

\begin{equation}\label{IKM-EQU-2}
-v_{i,i}=0
\end{equation}

\begin{equation}\label{IKM-EQU-3}
\sigma_{ij}=\sigma'_{ij,j}-\frac{1}{d} p \delta_{ij}
\end{equation}
