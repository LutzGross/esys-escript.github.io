
%%%%%%%%%%%%%%%%%%%%%%%%%%%%%%%%%%%%%%%%%%%%%%%%%%%%%%%%
%
% Copyright (c) 2003-2009 by University of Queensland
% Earth Systems Science Computational Center (ESSCC)
% http://www.uq.edu.au/esscc
%
% Primary Business: Queensland, Australia
% Licensed under the Open Software License version 3.0
% http://www.opensource.org/licenses/osl-3.0.php
%
%%%%%%%%%%%%%%%%%%%%%%%%%%%%%%%%%%%%%%%%%%%%%%%%%%%%%%%%


\section{Einstein Notation}
\label{EINSTEIN NOTATION}

Compact notation is used in equations such continuum mechanics and linear algebra; it is known as Einstein notation or the Einstein summation convention. It makes the conventional notation of equations involing tensors more compact, by shortening and simplifying them.

There are two rules which make up the convention:

firstly, the rank of the tensor is represented by an index. For example, $a$ is a scalar; $b\hackscore{i}$ represents a vector; and $c\hackscore{ij}$ represents a matrix.

Secondly, if an expression contains subscripted variables, they are assumed to be summed over all possible values, from $0$ to $n$. For example, for the following expression:



\begin{equation}
y = a\hackscore{0}b\hackscore{0} + a\hackscore{1}b\hackscore{1} + \ldots + a\hackscore{n}b\hackscore{n}
\label{NOTATION1}
\end{equation}

can be represented as:

\begin{equation}
y = \sum\hackscore{i=0}^n  a\hackscore{i}b\hackscore{i}
\label{NOTATION2}
\end{equation}

then in Einstein notion:

\begin{equation}
y = a\hackscore{i}b\hackscore{i}
\label{NOTATION3}
\end{equation}

Another example:

\begin{equation}
\nabla p = \frac{\partial p}{\partial x\hackscore{0}}\textbf{i} + \frac{\partial p}{\partial x\hackscore{1}}\textbf{j} + \frac{\partial p}{\partial x\hackscore{2}}\textbf{k}
\label{NOTATION4}
\end{equation}

can be expressed in Einstein notation as:

\begin{equation}
\nabla p = p,\hackscore{i}
\label{NOTATION5}
\end{equation}

where the comma ',' indicates the partial derivative.

For a tensor:

\begin{equation}
\sigma \hackscore{ij}= 
\left[ \begin{array}{ccc}
\sigma\hackscore{00} & \sigma\hackscore{01} & \sigma\hackscore{02} \\
\sigma\hackscore{10} & \sigma\hackscore{11} & \sigma\hackscore{12} \\
\sigma\hackscore{20} & \sigma\hackscore{21} & \sigma\hackscore{22} \\
\end{array} \right]
\label{NOTATION6}
\end{equation}


The $\delta\hackscore{ij}$ is the Kronecker $\delta$-symbol, which is a matrix with ones for its diagonal entries ($i = j$) and zeros for the remaining entries ($i \neq j$).

\begin{equation}
\delta \hackscore{ij} = 
\left \{ \begin{array}{cc}
1, & \mbox{if $i = j$} \\
0, & \mbox{if $i \neq j$} \\
\end{array}
\right.
\label{KRONECKER}
\end{equation}
