
%%%%%%%%%%%%%%%%%%%%%%%%%%%%%%%%%%%%%%%%%%%%%%%%%%%%%%%%%%%%%%%%%%%%%%%%%%%%%%
% Copyright (c) 2003-2013 by University of Queensland
% http://www.uq.edu.au
%
% Primary Business: Queensland, Australia
% Licensed under the Open Software License version 3.0
% http://www.opensource.org/licenses/osl-3.0.php
%
% Development until 2012 by Earth Systems Science Computational Center (ESSCC)
% Development since 2012 by School of Earth Sciences
%
%%%%%%%%%%%%%%%%%%%%%%%%%%%%%%%%%%%%%%%%%%%%%%%%%%%%%%%%%%%%%%%%%%%%%%%%%%%%%%


\section{Linear Magnetic Inversion}\label{sec:forward magnetic}
For the magnetic inversion we use the anomaly of the magnetic flux
density~\index{magnetic flux density} of the Earth.
The controlling material parameter is the susceptibility~\index{susceptibility}
$k$ of the rock.
With magnetization $M$ and inducing magnetic field anomaly $H^s$, the magnetic
flux density anomaly $B^s$ is given as
\begin{equation}\label{ref:MAG:EQU:1}
B_i = \mu_0 \cdot ( H^s_i  + M_i )
\end{equation}
where $\mu_0 = 4 \pi \cdot 10^{-7} \frac{Vs}{Am}$.
In this forward model we make the simplifying assumption that the magnetization
is proportional to the known geomagnetic flux density $B^b$:
\begin{equation}\label{ref:MAG:EQU:4}
\mu_0  \cdot M_i = k \cdot B^b_i \;. 
\end{equation}
Values for the magnetic flux density can be obtained by the International
Geomagnetic Reference Field (IGRF)~\cite{IGRF}
(or the Australian Geomagnetic Reference Field (AGRF)~\cite{AGRF}).
A rough approximation at latitude $\theta$ is given by 
\begin{equation}\label{ref:MAG:EQU:5}
\begin{array}{rcl}
B^b_{\theta}  & = & \displaystyle{ \frac{ \mu_0 \cdot m_{earth}}{4 \pi \cdot R_{earth}^3} sin(\theta) }  \\
B^b_r & = & \displaystyle{ \frac{\mu_0 \cdot  m_{earth}}{2 \pi \cdot R_{earth}^3} cos(\theta) }
\end{array}
\end{equation}
with the vacuum permeability\index{vacuum permeability} $\mu_0 = 4 \pi \cdot 10^{-7} \frac{Vs}{Am}$,
the magnetic dipole moment of Earth $m_{earth}=8.22 \cdot 10^{22} Am^2$ and
earth radius $R_{earth}= 6378137m$.
$B^b_r$ and $b^b_{\theta}$ denote the radial and latitudinal component of the
geomagnetic flux density.
Notice that convention~(\ref{REF:EQU:INTRO 9}) applies if Cartesian
coordinates\index{Cartesian coordinates} are used.
In most cases it is reasonable to assume that that the background field is
constant across the domain.

The magnetic field anomaly $H^s$ can be represented by the gradient of a
magnetic scalar potential\index{scalar potential!magnetic} $\psi$.
We use the form 
\begin{equation}\label{ref:MAG:EQU:6}
\mu_0  \cdot H^s_i = - \psi_{,i}
\end{equation}
With this notation one gets from Equations~(\ref{ref:MAG:EQU:1}) and~(\ref{ref:MAG:EQU:4}):
\begin{equation}\label{ref:MAG:EQU:7}
B_i = - \psi_{,i}  + k \cdot B^b_i
\end{equation}
As the $B^s$ magnetic flux density anomaly we obtain the PDE
\begin{equation}\label{ref:MAG:EQU:8}
- \psi_{,ii} = - (k B^b_i)_{,i} 
\end{equation} 
which needs to be solved for a given susceptibility $k$.
The magnetic scalar potential is set to zero at the top of the domain
$\Gamma_{0}$.
On all other faces the normal component of the magnetic flux density anomaly
$B_i$ is set to zero, i.e. $n_i \psi_{,i} = k \cdot n_i  B^b_i$ with outer
normal field $n_i$.

From the magnetic scalar potential we can calculate the magnetic flux density
anomaly via Equation~(\ref{ref:MAG:EQU:8}) to calculate the defect to the given
data.
If $B^{(s)}_i$ is a measurement of the magnetic flux density anomaly for
survey $s$ and $\omega^{(s)}_i$ is a weighting factor the data defect
$J^{mag}(k)$ in the notation of Chapter~\ref{chapter:ref:inversion cost function} is given as
\begin{equation}\label{ref:MAG:EQU:9}
J^{mag}(k) = \frac{1}{2}\sum_{s} \int_{\Omega} ( \omega^{(s)}_i \cdot (B_{i}- B^{(s)}_i) ) ^2 dx
\end{equation} 
Summation over $i$ is performed.
The cost function kernel\index{cost function!kernel} is given as
\begin{equation}\label{ref:MAG:EQU:10}
K^{mag}(\psi_{,i},k) = \frac{1}{2}\sum_{s} ( \omega^{(s)}_i \cdot (k \cdot B^b_i - \psi_{,i} - B^{(s)}_i) ) ^2
\end{equation} 
Notice that if magnetic flux density is measured in air one can ignore the
$k\cdot B^b_i$ as the susceptibility is zero.

In practice the magnetic flux density $b^{(s)}$ is measured along a certain
direction $d^{(s)}_i$ with a standard error deviation $\sigma^{(s)}$ at
certain locations in the domain.
In this case one sets $B^{(s)}_i=b^{(s)} \cdot d^{(s)}_i$ and the weighting
factors $\omega^{(s)}$ as
\begin{equation}\label{ref:MAG:EQU:11}
\omega^{(s)}_i 
= \left\{
\begin{array}{lcl}
f \cdot  \frac{d^{(s)}_i}{\sigma^{(s)}} & & \mbox{data are available} \\
& \mbox{ where } & \\
0 & & \mbox{ otherwise } \\
\end{array}
\right.
\end{equation} 
where it is assumed that $d^{(s)}_i \cdot d^{(s)}_i =1$. With the objective to control the 
gradient of the cost function the scaling factor $f$ is chosen in the way that 
\begin{equation}\label{ref:MAG:EQU:12}
\sum_{s} \int_{\Omega} ( \omega^{(s)}_i B^{(s)}_i ) 
 \cdot ( \omega^{(s)}_j \frac{1}{L_j} ) \cdot L^2 \cdot
( B^b_n \frac{1}{L_n} )
 \cdot k' \;
 dx =\alpha
\end{equation} 
where $\alpha$ defines a scaling factor which is typically set to one and $L$ is defined by equation~(\ref{ref:EQU:REG:6b}).
$k'$ is considering the 
derivative of the density with respect to the level set function. 

\subsection{Usage}

\LG{Add example}

\begin{classdesc}{MagneticModel}{domain, w, B, background_field,
        \optional{, useSphericalCoordinates=False}
        \optional{, tol=1e-8}}
opens a magnetic forward model over the \Domain \member{domain} with 
weighting factors \member{w} ($=\omega^{(s)}$) and measured magnetic flux
density anomalies \member{B} ($=B^{(s)}$).
The weighting factors and the  measured magnetic flux density anomalies must be vectors.
\member{background_field} defines the background magnetic flux density $B^b$
as a vector.
If \member{useSphericalCoordinates} is \True spherical coordinates are used. 
\member{tol} sets the tolerance for the solution of the PDE~(\ref{ref:MAG:EQU:8}).
\end{classdesc}

\begin{methoddesc}[MagneticModel]{rescaleWeights}{
        \optional{scale=1.}
 \optional{k_scale=1.}}
rescale the weighting factors such condition~(\ref{ref:MAG:EQU:12}) holds where 
\member{scale} sets the scale $\alpha$
and \member{k_scale} sets $k'$. This method should be called before any inversion is started
in order to make sure that all components of the cost function are appropriately scaled.
\end{methoddesc}


\subsection{Gradient Calculation}
This section briefly explains how the gradient
$\frac{\partial J^{mag}}{\partial k}$ of the cost function $J^{mag}$ with
respect to the susceptibility $k$ is calculated.  We follow the concept as outlined in section~\ref{chapter:ref:inversion cost function:gradient}.

The magnetic potential $\psi$ from PDE~(\ref{ref:MAG:EQU:8}) is solved in weak form:
\begin{equation}\label{ref:MAG:EQU:201}
\int_{\Omega} q_{,i} \psi_{,i} \; dx  = \int_{\Omega}  k \cdot q_{,i}  B^b_i \; dx 
\end{equation} 
for all $q$ with $q=0$ on $\Gamma_{0}$.
In the following we set $\Psi[k]=\psi$ for a given susceptibility $k$ as
solution of the variational problem~(\ref{ref:MAG:EQU:201}).
If $\Gamma_{k}$ denotes the region of the domain where the susceptibility is
known and for a given direction $p$ with $p=0$ on $\Gamma_{k}$ one has
\begin{equation}\label{ref:MAG:EQU:201aa}
\int_{\Omega}   \frac{\partial J^{mag}}{\partial k} \cdot p \; dx  = \int_{\Omega}  
\sum_{s} (\omega^{(s)}_j 
( B^{(s)}_j-B_{j}))  \cdot ( \omega^{(s)}_i ( \Psi[p]_{,i} - p  \cdot B^b_i  ) ) \; dx  
\end{equation} 
with
\begin{equation}\label{ref:MAG:EQU:202c}
Y_i[\psi]=   \sum_{s} (\omega^{(s)}_j 
(B^{(s)}_j - B_{j}) )  \cdot \omega^{(s)}_i  
\end{equation} 
This is written as 
\begin{equation}\label{ref:MAG:EQU:202cc}
\int_{\Omega}   \frac{\partial J^{mag}}{\partial k} \cdot p \;  dx  = \int_{\Omega}  
Y_i[\psi] \Psi[p]_{,i} - p \cdot Y_i[\psi]B^b_i   \; dx  
\end{equation} 
We then set $Y^*[\psi]$ as the solution of the equation 
\begin{equation}\label{ref:MAG:EQU:202d}
\int_{\Omega} r_{,i} Y^*[\psi]_{,i} \; dx  =  \int_{\Omega} r_{,i} ,Y_i[\psi]  \; dx  \mbox{ for all } p \mbox{ with } r=0 \mbox{ on } \Gamma_{0}
\end{equation} 
with $Y^*[\psi]=0$ on $\Gamma_{0}$. With $r=\Psi[p]$ we get
\begin{equation}\label{ref:MAG:EQU:202dd}
\int_{\Omega} \Psi[p]_{,i} Y^*[\psi]_{,i} \; dx  =  \int_{\Omega} \Psi[p]_{,i} ,Y_i[\psi]  \; dx
\end{equation} 
and from Equation~(\ref{ref:MAG:EQU:201}) with $q=Y^*[\psi]$ we get
\begin{equation}\label{ref:MAG:EQU:20e}
\int_{\Omega} Y^*[\psi]_{,i}  \Psi[p]_{,i} \; dx  = \int_{\Omega}  p \cdot Y^*[\psi]_{,i}  B^b_i \; dx  
\end{equation}
which leads to 
\begin{equation}\label{ref:MAG:EQU:20ee}
\int_{\Omega} \Psi[p]_{,i} ,Y_i[\psi]  \; dx  = \int_{\Omega}  p \cdot Y^*[\psi]_{,i}  B^b_i \; dx  
\end{equation}
and finally
\begin{equation}\label{ref:MAG:EQU:201a}
\int_{\Omega}   \frac{\partial J^{mag}}{\partial k} \cdot p \;  dx  = \int_{\Omega}  
p \cdot (Y_i[\psi] - Y^*[\psi]_{,i})  B^b_i \; dx  
\end{equation} 
or 
\begin{equation}\label{ref:MAG:EQU:201b}
\frac{\partial J^{mag}}{\partial k} = (Y^*[\psi]_{,i}-Y_i[\psi])  B^b_i
\end{equation}

