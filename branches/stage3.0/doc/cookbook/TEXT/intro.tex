
%%%%%%%%%%%%%%%%%%%%%%%%%%%%%%%%%%%%%%%%%%%%%%%%%%%%%%%%
%
% Copyright (c) 2003-2009 by University of Queensland
% Earth Systems Science Computational Center (ESSCC)
% http://www.uq.edu.au/esscc
%
% Primary Business: Queensland, Australia
% Licensed under the Open Software License version 3.0
% http://www.opensource.org/licenses/osl-3.0.php
%
%%%%%%%%%%%%%%%%%%%%%%%%%%%%%%%%%%%%%%%%%%%%%%%%%%%%%%%%

\section{Introduction}

\editor{This document is not meant to be a comprehensive or definitive manual for the \esc  program. It is mearly an introduction to the fundamentals of the software and a gateway to understanding its inner workings and functionality.
Presented in this document are some examples ranging from very easy to moderately difficult.
It is recommended that you start from the beggining as each example builds upon the knowledge and learned processes in the previous explained problems.
All of the escript/python scripts associated with this document are available from the examples folder.
Section on why people should use escript vs matlab vs their own code.
}

