%%%%%%%%%%%%%%%%%%%%%%%%%%%%%%%%%%%%%%%%%%%%%%%%%%%%%%%%%%%%%%%%%%%%%%%%%%%%%%
% Copyright (c) 2003-2016 by The University of Queensland
% http://www.uq.edu.au
%
% Primary Business: Queensland, Australia
% Licensed under the Apache License, version 2.0
% http://www.apache.org/licenses/LICENSE-2.0
%
% Development until 2012 by Earth Systems Science Computational Center (ESSCC)
% Development 2012-2013 by School of Earth Sciences
% Development from 2014 by Centre for Geoscience Computing (GeoComp)
%
%%%%%%%%%%%%%%%%%%%%%%%%%%%%%%%%%%%%%%%%%%%%%%%%%%%%%%%%%%%%%%%%%%%%%%%%%%%%%%


\section{Magnetic Inversion With Self-Demagnetisation}\label{sec:forward self demagnetic}
In cases of large susceptibility~\index{susceptibility} $k$ the linear magnetisation assumption
as discussed in the previous Section~\ref{sec:forward magnetic} is not valid and 
one assume that magnetisation $M$ is proportional to present magnetic field~\index{magnetic field} 
rather than the background magnetic flux
density, see~\cite{Lelievre2006a}. 

Analogous to equation~(\ref{ref:MAG:EQU:1}) it is  
\begin{equation}\label{ref:SELFREMAG:EQU:1}
B_i = \mu_0 \cdot ( H_i  + M_i )
\end{equation}
where 
$B$ is the flux density anomaly, the magnetic field anomaly $H$ and
$\mu_0 = 4 \pi \cdot 10^{-7} \frac{Vs}{Am}$. In extension to equation~(\ref{ref:MAG:EQU:4})
the magnetisation is assumed to be proportional to the 
(total) magnetic field $H_i + \frac{1}{\mu_0} B_i^{b}$ in the form 
\begin{equation}\label{ref:SELFREMAG:EQU:4}
\mu_0  \cdot M_i = k \cdot ( B_i^{b} + \mu_0  H_i ) \;. 
\end{equation}
where $B_i^{(b)}$ is known background geomagnetic flux density $B^b$
(see additional information in Section~\ref{sec:forward magnetic}).

The magnetic field anomaly $H$ can be represented by the gradient of a
magnetic scalar potential\index{scalar potential!magnetic} $\psi$.
We use the form 
\begin{equation}\label{ref:SELFREMAG:EQU:6}
\mu_0  \cdot H_i = - \psi_{,i}
\end{equation}
With this notation one gets from Equations~(\ref{ref:SELFREMAG:EQU:1}) and~(\ref{ref:SELFREMAG:EQU:4}):
\begin{equation}\label{ref:SELFREMAG:EQU:7}
B_i = - (1+k) \cdot \psi_{,i}  + k \cdot B^b_i
\end{equation}
As the $B$ magnetic flux density anomaly is divergence free ($B_{i,i}=0$) we obtain the PDE
\begin{equation}\label{ref:SELFREMAG:EQU:8}
- ( (1+k) \cdot \psi_{,i})_{,i} = - (k B^r_i)_{,i} 
\end{equation} 
with $B^r_i=B^b_i$ which needs to be solved for a given susceptibility $k$.
The magnetic scalar potential is set to zero at the top of the domain
$\Gamma_{0}$.
On all other faces the normal component of the magnetic flux density anomaly
$B_i$ is set to zero, i.e. $(1+k) \cdot  n_i \psi_{,i} = k \cdot n_i  B^b_i$ with outer
normal field $n_i$.

From the magnetic scalar potential we can calculate the magnetic flux density
anomaly via Equation~(\ref{ref:SELFREMAG:EQU:8}) to calculate the defect to the given
data.
If $B^{(s)}_i$ is a measurement of the magnetic flux density anomaly for
survey $s$ and $\omega^{(s)}_i$ is a weighting factor the data defect
$J^{mag}(k)$ in the notation of Chapter~\ref{chapter:ref:inversion cost function} is given as
\begin{equation}\label{ref:SELFREMAG:EQU:9}
J^{mag}(k) = \frac{1}{2}\sum_{s} \int_{\Omega} ( \omega^{(s)}_i \cdot (B_{i}- B^{(s)}_i) ) ^2 dx
\end{equation} 
Summation over $i$ is performed.
The cost function kernel\index{cost function!kernel} is given as
\begin{equation}\label{ref:SELFREMAG:EQU:10}
K^{mag}(\psi_{,i},k) = \frac{1}{2}\sum_{s} ( \omega^{(s)}_i \cdot (k \cdot B^b_i - (1+k) \cdot \psi_{,i} - B^{(s)}_i) ) ^2
\end{equation} 
Notice that if magnetic flux density is measured in air one can ignore the
$k\cdot B^b_i$ as the susceptibility is zero. See also the remarks on the data defect in Section~\ref{sec:forward magnetic}).


\subsection{Usage}

\begin{classdesc}{SelfDemagnetizationModel}{domain, w, B, background_field,
        \optional{, coordinates=\None}
        \optional{, fixPotentialAtBottom=False},
        \optional{, tol=1e-8},
}
opens a magnetic forward model over the \Domain \member{domain} with 
weighting factors \member{w} ($=\omega^{(s)}$) and measured magnetic flux
density anomalies \member{B} ($=B^{(s)}$).
The weighting factors and the  measured magnetic flux density anomalies must be vectors.
\member{background_field} defines the background magnetic flux density $B^b$
as a vector with north, east and vertical components. 
\member{tol} sets the tolerance for the solution of the PDE~(\ref{ref:SELFREMAG:EQU:8}).
If \member{fixPotentialAtBottom} is set to  \True, the magnetic potential 
at the bottom is set to zero in addition to the potential on the top. 
\member{coordinates} set the reference coordinate system to be used. By the default the 
Cartesian coordinate system is used.
\end{classdesc}

\begin{methoddesc}[SelfDemagnetizationModel]{rescaleWeights}{
        \optional{scale=1.}
 \optional{k_scale=1.}}
rescale the weighting factors such condition~(\ref{ref:MAG:EQU:12}) holds where 
\member{scale} sets the scale $\alpha$
and \member{k_scale} sets $k'$. This method should be called before any inversion is started
in order to make sure that all components of the cost function are appropriately scaled.
\end{methoddesc}



\subsection{Gradient Calculation}
This section briefly explains how the gradient
$\frac{\partial J^{mag}}{\partial k}$ of the cost function $J^{mag}$ with
respect to the susceptibility $k$ is calculated.  We follow the concept as outlined in section~\ref{chapter:ref:inversion cost function:gradient}.

The magnetic potential $\psi$ from PDE~(\ref{ref:SELFREMAG:EQU:8}) is solved in weak form:
\begin{equation}\label{ref:SELFREMAG:EQU:201}
\int_{\Omega} 
(1+k) \cdot q_{,i} \psi_{,i} \; dx  = \int_{\Omega}  k \cdot q_{,i}  B^r_i \; dx 
\end{equation} 
for all $q$ with $q=0$ on $\Gamma_{0}$.
If $\Gamma_{k}$ denotes the region of the domain where the susceptibility is
known and for a given direction $\hat{k}$ with $\hat{k}=0$ on $\Gamma_{k}$ one has
\begin{equation}\label{ref:SELFREMAG:EQU:201aa}
\int_{\Omega}   \frac{\partial J^{mag}}{\partial k} \cdot \hat{k} \; dx  = \int_{\Omega}  
\sum_{s} (\omega^{(s)}_j 
( B^{(s)}_j-B_{j}))  \cdot ( \omega^{(s)}_i ( (1+k) \cdot \hat{\psi}_{,i} + \hat{k} \cdot \psi_{,i} - \hat{k}  \cdot B^b_i  ) ) \; dx  
\end{equation} 
with
\begin{equation}\label{ref:SELFREMAG:EQU:201A}
\int_{\Omega} \left(
(1+k) \cdot q_{,i} \hat{\psi}_{,i} + 
\hat{k}  \cdot q_{,i} \psi_{,i} \right) 
\; dx  
= \int_{\Omega} \hat{k} \cdot q_{,i}  B^r_i \; dx 
\end{equation} 
for all $q$ with $q=0$ on $\Gamma_{0}$. This equation is obtained from equation~(\ref{ref:SELFREMAG:EQU:201})
for $k+\hat{k} \rightarrow  k$ and $\psi+\hat{\psi} \rightarrow \psi$. 
With
\begin{equation}\label{ref:SELFREMAG:EQU:202c}
Y_i =   \sum_{s} (\omega^{(s)}_j 
(B^{(s)}_j - B_{j}) )  \cdot \omega^{(s)}_i  
\end{equation} 
Equation~(\ref{ref:SELFREMAG:EQU:201aa}) is written as 
\begin{equation}\label{ref:SELFREMAG:EQU:202cc}
\int_{\Omega}   \frac{\partial J^{mag}}{\partial k} \cdot \hat{k} \;  dx  = \int_{\Omega}  
(1+k) \cdot  Y_i \hat{\psi}_{,i}  + \hat{k} \cdot  Y_i \psi_{,i} - \hat{k}  \cdot Y_i B^b_i   \; dx  
\end{equation} 

We then set adjoint function $Y^*$ as the solution of the equation 
\begin{equation}\label{ref:SELFREMAG:EQU:202d}
\int_{\Omega} (1+k) \cdot r_{,i} Y^*_{,i} \; dx  =  \int_{\Omega} (1+k) \cdot  r_{,i} Y_i  \; dx  \mbox{ for all } r \mbox{ with } r=0 \mbox{ on } \Gamma_{0}
\end{equation} 
with $Y^*=0$ on $\Gamma_{0}$. With $r=\hat{\psi}$ we get
\begin{equation}\label{ref:SELFREMAG:EQU:202dd}
\int_{\Omega} (1+k) \cdot  \hat{\psi}_{,i} Y^*_{,i} \; dx  =  \int_{\Omega}  (1+k)  \cdot \hat{\psi}_{,i} Y_i  \; dx
\end{equation} 
and from Equation~(\ref{ref:SELFREMAG:EQU:201A}) with $q=Y^*$ we get
\begin{equation}\label{ref:SELFREMAG:EQU:20e}
\int_{\Omega} (1+k) \cdot  Y^*_{,i}  \hat{\psi}_{,i} \; dx  = \int_{\Omega} \hat{k} \cdot  \left(   Y^*_{,i}  B^r_i - Y^*_{,i} \psi_{,i}  \right) \; dx  
\end{equation}
which leads to 
\begin{equation}\label{ref:SELFREMAG:EQU:20ee}
\int_{\Omega} (1+k)  \cdot \hat{\psi}_{,i} Y_i   \; dx  = 
\int_{\Omega} \hat{k} \cdot  \left(   Y^*_{,i}  B^r_i - Y^*_{,i} \psi_{,i}  \right) \; dx  
\end{equation}
and finally
\begin{equation}\label{ref:SELFREMAG:EQU:201a}
\int_{\Omega}   \frac{\partial J^{mag}}{\partial k} \cdot \hat{k} \;  dx  = \int_{\Omega}  
\hat{k} \cdot  \left(   Y^*_{,i}  B^r_i - Y^*_{,i} \psi_{,i} 
                         + Y_i \psi_{,i} - Y_i B^b_i
 \right) 
 \; dx  
\end{equation} 
or 
\begin{equation}\label{ref:SELFREMAG:EQU:201b}
\frac{\partial J^{mag}}{\partial k} =  Y^*_{,i}  B^r_i - Y^*_{,i} \psi_{,i} 
                         + Y_i \psi_{,i} - Y_i B^b_i
\end{equation}

\subsection{Geodetic Coordinates }
For geodetic coordinates $(\phi, \lambda, h)$, see Chapter~\ref{Chp:ref:coordinates}, the solution process needs to be slightly modified.
Observations are recorded along the geodetic coordinates axes $\alpha$ rather than the Cartesian axes $i$. In fact we
have in equation~\ref{ref:SELFREMAG:EQU:9}:
\begin{equation}\label{ref:SELFREMAG:EQU:300}
\omega^{(s)}_i \cdot (B_{i}- B^{(s)}_i) = \omega^{(s)}_{\alpha} \cdot (B_{{\alpha}}- B^{(s)}_{\alpha}) 
\end{equation} 
where now $B^{(s)}_{\alpha}$ are the observational data with weighting factors $\omega^{(s)}_{\alpha}$.  Using the 
fact that 
\begin{equation}
B_{{\alpha}} = 
 k \cdot B^b_{{\alpha}} -  (1+k) \cdot  d_{\alpha \alpha} \psi_{,\alpha} 
\end{equation}
equation~\ref{ref:SELFREMAG:EQU:10} translates to 
\begin{equation}\label{ref:SELFREMAG:EQU:301}
J^{mag}(k) = \frac{1}{2}\sum_{s} \int_{\widehat{\Omega}} 
( \omega^{(s)}_{\alpha} \cdot (
(1+k) \cdot  d_{\alpha \alpha} \psi_{,\alpha}  
 - k \cdot B^b_{{\alpha}}  + B^{(s)}_{\alpha} ) ) ^2 \; v \; d\widehat{x}
\end{equation} 
where $\widehat{\Omega}$ and $d\widehat{x}$ refer to integration over the geodetic coordinates axes. This can be rearranged to 
\begin{eqnarray}\label{ref:SELFREMAG:EQU:301bb}
J^{mag}(k)  & = \displaystyle{ \frac{1}{2}\sum_{s} \int_{\widehat{\Omega}} 
(  \omega^{(s)}_{\alpha} v^{\frac{1}{2}} d_{\alpha \alpha} \cdot ( (1+k) \cdot  
 \psi_{,\alpha} -  k \cdot v_{\alpha \alpha} B^b_{{\alpha}} + v_{\alpha \alpha} B^{(s)}_{\alpha} ) ) ^2 \; d\widehat{x} } \\
  & = \displaystyle{  \frac{1}{2}\sum_{s} \int_{\widehat{\Omega}} 
(  {\widehat{\omega}}^{(s)}_{\alpha}\cdot ( (1+k) \cdot  \psi_{,\alpha} -  k \cdot \widehat{B}^b_{{\alpha}}+  \widehat{B}^{(s)}_{\alpha} ) ) ^2 \; d\widehat{x} }
\end{eqnarray}
with 
\begin{equation}\label{ref:SELFREMAG:EQU:301b}
 \widehat{\omega}^{(s)}_{\alpha} = \omega^{(s)}_{\alpha} v^{\frac{1}{2}} d_{\alpha \alpha} \mbox{ , }
\widehat{B}^{(s)}_{\alpha}=
\frac{1}{d_{\alpha \alpha}} B^{(s)}_{\alpha}  \mbox{ and } \widehat{B}^b_{{\alpha}} = \frac{1}{d_{\alpha \alpha}}  B^b_{{\alpha}} 
\end{equation} 
which means one can apply the Cartesian formulation to the geodetic coordinates using modified data. 


The magnetic potential is calculated from 
\begin{equation}\label{ref:SELFREMAG:EQU:302}
\int_{\widehat{\Omega}} (1+k) \cdot v \; d_{\alpha \alpha}^2 q_{,\alpha} \psi_{,\alpha} \;  d\widehat{x}  
=  \int_{\widehat{\Omega}} v \; d_{\alpha \alpha} k \cdot q_{,\alpha}  B^r_{\alpha} \; d\widehat{x}   
=  \int_{\widehat{\Omega}}  k \cdot q_{,\alpha}  \widehat{B}^r_{\alpha} \; d\widehat{x}   
\end{equation} 
with 
\begin{equation}\label{ref:SELFREMAG:EQU:302b}
\widehat{B}^r_{\alpha}  =v \; d_{\alpha \alpha} B^r_{\alpha}
\end{equation} 
see equation~\ref{ref:SELFREMAG:EQU:201}, and the adjoint function $Y^*$ for $Y_{\alpha}$ (calculated from $\widehat{B}^b$ and $\widehat{B}^{(s)}$ is given from
\begin{equation}\label{ref:SELFREMAG:EQU:303}
\int_{\widehat{\Omega}}(1+k) \cdot  v \; d_{\alpha \alpha}^2 q_{,\alpha} Y^*_{,\alpha } \;d\widehat{x}  =
\int_{\widehat{\Omega}} (1+k) \cdot r_{,{\alpha}} ,Y_{\alpha}  \;d\widehat{x}
\end{equation} 
and finally 
\begin{equation}\label{ref:SELFREMAG:EQU:310}
\frac{\partial J^{mag}}{\partial k} = 
 Y^*_{,\alpha}   \widehat{B}^r_{\alpha} - Y^*_{,\alpha} \psi_{,\alpha} 
                         + Y_{\alpha}  \psi_{,\alpha} - Y_{\alpha}   \widehat{B}^b_{\alpha}
\end{equation} 
where integration is performed over $\widehat{\Omega}$.

