
%%%%%%%%%%%%%%%%%%%%%%%%%%%%%%%%%%%%%%%%%%%%%%%%%%%%%%%%
%
% Copyright (c) 2003-2008 by University of Queensland
% Earth Systems Science Computational Center (ESSCC)
% http://www.uq.edu.au/esscc
%
% Primary Business: Queensland, Australia
% Licensed under the Open Software License version 3.0
% http://www.opensource.org/licenses/osl-3.0.php
%
%%%%%%%%%%%%%%%%%%%%%%%%%%%%%%%%%%%%%%%%%%%%%%%%%%%%%%%%


\chapter{Models}
\label{MODELS CHAPTER}

The following sections give a breif overview of the model classes and their corresponding methods.

\section{Stokes Problem}
The velocity \index{velocity} field $v$ and pressure $p$ of an incompressible fluid \index{incompressible fluid} is given as the solution of the Stokes problem\index{Stokes problem}
\begin{equation}\label{Stokes 1}
-\left(\eta(v\hackscore{i,j}+ v\hackscore{i,j})\right)\hackscore{,j}+p\hackscore{,i}=f\hackscore{i}-\sigma\hackscore{ij,j}
\end{equation}
where $\eta$ is the viscosity, $F\hackscore{i}$ defines an internal force \index{force, internal} and $\sigma\hackscore{ij}$ is an intial stress \index{stress, initial}. We assume an incompressible media:
\begin{equation}\label{Stokes 2}
-v\hackscore{i,i}=0
\end{equation}
Natural boundary conditions are taken in the form 
\begin{equation}\label{Stokes Boundary}
\left(\eta(v\hackscore{i,j}+ v\hackscore{i,j})\right)n\hackscore{j}-n\hackscore{i}p=s\hackscore{i}+\sigma\hackscore{ij} n\hackscore{i}
\end{equation}
which can be overwritten by constraints of the form 
\begin{equation}\label{Stokes Boundary0}
v\hackscore{i}(x)=v^D\hackscore{i}(x)
\end{equation}
at some locations $x$ at the boundary of the domain. The index $i$ may depend on the location $x$ on the boundary.
$v^D$ is a given function on the domain.

\subsection{Solution Method \label{STOKES SOLVE}}
In block form equation equations~\ref{Stokes 1} and~\ref{Stokes 2} takes the form of a saddle point problem
\index{saddle point problem}
\begin{equation}
\left[ \begin{array}{cc}
A     & B^{*} \\
B & 0 \\
\end{array} \right]
\left[ \begin{array}{c}
v \\
p \\
\end{array} \right]
=\left[ \begin{array}{c}
G \\
0 \\
\end{array} \right]
\label{SADDLEPOINT}
\end{equation}
where $A$ is coercive, self-adjoint linear operator in a suitable Hilbert space, $B$ is the $(-1) \cdot$ divergence operator and $B^{*}$ is it adjoint operator (=gradient operator). For more details on the mathematics see references \cite{AAMIRBERKYAN2008,MBENZI2005}. 
We use iterative techniques to solve this problem. To make sure that the incomressibilty condition holds
with sufficient accuracy we check for 
\begin{equation}
\|v\hackscore{k,k}\| \hackscore \le  \epsilon
\|\sqrt{v\hackscore{j,k}v\hackscore{j,k}}\| 
\end{equation}
where $\epsilon$ is the desired relative accuracy and 
\begin{equation}
\|p\|^2= \int\hackscore{\Omega} p^2 \; dx
\label{PRESSURE NORM}
\end{equation}
defines the $L^2$-norm.
There are two approaches to solve this problem. The first approach, called the Uzawa scheme \index{Uzawa scheme} 
eliminates the velocity $v$ from the problem. The second approach solves the equation in coupled form after the application of a preconditioner. 

\subsubsection{Uzawa scheme} 
The first eqution in~\ref{SADDLEPOINT} gives $v=A^{-1}(G-B^{*}p)$ assuming $p$ is known. This is inserted into the 
second eqution which leads to 
\begin{equation}
S p =  B A^{-1} G
\end{equation}
with the Schur complement \index{Schur complement} $S=BA^{-1}B^{*}$. This problem can be solved iteratively using the reconditioned Conjugate Gradient Method (PCG)~\index{PCG!Preconditioned Conjugate Gradient Method} 
with the preconditioner $\hat{S}$ defined as $q=\hat{S}^{-1}p$ by solving
\begin{equation}
\frac{1}{\eta}q = p 
\end{equation}
see~\cite{ELMAN} for more details. The evaluation of $w=Sp$ is done in the form
\begin{equation}
\begin{array}{rcl}
A v & = & B^{*}p \\
w & = & Bv \\
\end{array}
\label{EVAL PCG}
\end{equation}
The residual \index{residual}  $r=B A^{-1} G - S p$ is given as 
\begin{equation}
r=B A^{-1} (G - B^* p) = Bv \mbox{ with } v = A^{-1}(G-B^{*}p)
\end{equation}
Therefore one uses the tuple $(v,Bv)$ to represent the residual of the current pressure $p$. Notice that before the iteration is started the right hand side $B A^{-1} G$ needs to be calculated. The bilinear form $(.,.)$ used is defined as
\begin{equation}
(p,(v,Bv))=\int\hackscore{\Omega} p \cdot Bv \; dx 
\end{equation}
where $p$ is the pressure increment and $(v,Bv)$ represents an increment in the residual.

\subsubsection{Coupled Solver}
An alternative approach to solve the saddle point problem~\ref{SADDLEPOINT} directly using an iterative such as 
the generalized minimal residual method (GMRES) \index{generalized minimal residual method!GMRES} with a suitable
preconditioner. Here we use the operator 
\begin{equation}
\left[ \begin{array}{cc}
A^{-1}     & 0 \\
S^{-1} B A^{-1}  & -S^{-1} \\
\end{array} \right]
\label{SADDLEPOINT PRECODITIONER}
\end{equation}
where again $S$ is the Schur complement~\cite{ELMAN}. In partice we will use an approximation $\hat{S}$ for $S$. The evaluation $(w,q)$ of the iteration operator for a given $(v,p)$ is done as 
\begin{equation}
\begin{array}{rcl}
A w & = & Av+B^{*}p \\
\hat{S} q & = & B(w-v) \\
\end{array}
\label{COUPLES SADDLEPOINT iteration}
\end{equation}
We use the inner product induced by the norm
\begin{equation}
\|(v,p)\|^2= \int\hackscore{\Omega}  v\hackscore{i,j}  v\hackscore{i,j} + \left( \frac{p}{\eta}\right)^2\; dx
\label{COUPLES NORM}
\end{equation}
In PDE form~\ref{COUPLES SADDLEPOINT iteration} takes the form
\begin{equation}
\begin{array}{rcl}
-\left(\eta(w\hackscore{i,j}+ w\hackscore{i,j})\right)\hackscore{,j} & = & -\left(\eta(v\hackscore{i,j}+ v\hackscore{i,j})\right)\hackscore{,j}+p\hackscore{,i} \\
\frac{1}{\eta}  q & = & - (w-v)\hackscore{i,i} \\
\end{array}
\label{SADDLEPOINT iteration 2}
\end{equation}


\subsection{Functions}

\begin{classdesc}{StokesProblemCartesian}{domain}
opens the Stokes problem\index{Stokes problem} on the \Domain domain. The approximation
order needs to be two.
\end{classdesc}

\begin{methoddesc}[StokesProblemCartesian]{initialize}{\optional{f=Data(), \optional{fixed_u_mask=Data(), \optional{eta=1, \optional{surface_stress=Data(), \optional{stress=Data()}}}}}}
assigns values to the model parameters. In any call all values must be set.
\var{f} defines the external force $f$, \var{eta} the viscosity $\eta$,
\var{surface_stress} the surface stress $s$ and \var{stress} the initial stress $\sigma$.
The locations and compontents where the velocity is fixed are set by 
the values of \var{fixed_u_mask}. The method will try to cast the given values to appropriate 
\Data class objects.
\end{methoddesc}

\begin{methoddesc}[StokesProblemCartesian]{solve}{v,p,
\optional{max_iter=20, \optional{verbose=False, \optional{useUzawa=True}}}}
solves the problem and return approximations for velocity and pressure. 
The arguments \var{v} and \var{p} define initial guess. The values of \var{v} marked
by \var{fixed_u_mask} remain unchanged. 
If \var{useUzawa} is set to \True 
the Uzawa\index{Uszwa} scheme is used. Otherwise the problem is solved in coupled form. In most cases 
the Uzawa scheme is more efficient.
\var{max_iter} defines the maximum number of iteration steps. 
If \var{verbose} is set to \True informations on the progress of of the solver are printed.
\end{methoddesc}


\begin{methoddesc}[StokesProblemCartesian]{setTolerance}{\optional{tolerance=1.e-8}}
sets the tolerance in an appropriate norm relative to the right hand side. The tolerance must be non-negative and less than 1.
\end{methoddesc}
\begin{methoddesc}[StokesProblemCartesian]{getTolerance}{}
returns the current relative tolerance.
\end{methoddesc}
\begin{methoddesc}[StokesProblemCartesian]{setAbsoluteTolerance}{\optional{tolerance=0.}}
sets the absolute tolerance for the error in the relevant norm. The tolerance must be non-negative. Typically the
absolute talerance is set to 0.
\end{methoddesc}
\begin{methoddesc}[StokesProblemCartesian]{getAbsoluteTolerance}{}
sreturns the current absolute tolerance.
\end{methoddesc}
\begin{methoddesc}[StokesProblemCartesian]{setSubToleranceReductionFactor}{\optional{reduction=None}}
sets the reduction factor for the tolerance used to solve the PDEs. A reduction factor 
in the order of one will minimize compute time per iteration step but my slow down convergence or even lead to 
divergency. On the other hand a very small value for the PDE tolerance could result in a wast of compute time.
If \var{reduction} is set to \var{None} the sub-tolerance is solved adaptively but
in cases a very small tolerance is set ($<10^{-6}$) it is recommended to set the
reduction factor by hand. This may require some experiments.
\end{methoddesc}
\begin{methoddesc}[StokesProblemCartesian]{getSubToleranceReductionFactor}{}
return the current reduction factor for the sub-problem tolerance.
\end{methoddesc}

\subsection{Example: Lit Driven Cavity}
 The following script \file{lit\hackscore driven\hackscore cavity.py} 
\index{scripts!\file{helmholtz.py}} which is available in the \ExampleDirectory
illustrates the usage of the \class{StokesProblemCartesian} class to solve
the lit driven cavity problem~\cite{LITDRIVENCAVITY}:
\begin{python}
from esys.escript import *
from esys.finley import Rectangle
from esys.escript.models import StokesProblemCartesian
NE=25
dom = Rectangle(NE,NE,order=2)
x = dom.getX()
sc=StokesProblemCartesian(dom)
mask= (whereZero(x[0])*[1.,0]+whereZero(x[0]-1))*[1.,0] + \
      (whereZero(x[1])*[0.,1.]+whereZero(x[1]-1))*[1.,1]
sc.initialize(eta=.1, fixed_u_mask= mask)
v=Vector(0.,Solution(dom))
v[0]+=whereZero(x[1]-1.)
p=Scalar(0.,ReducedSolution(dom))
v,p=sc.solve(v,p, verbose=True)
saveVTK("u.xml",velocity=v,pressure=p)
\end{python}

\section{Darcy Flux}
We want to calculate the velocity $u$ and pressure $p$ on a domain $\Omega$ solving 
the Darcy flux problem \index{Darcy flux}\index{Darcy flow}
\begin{equation}\label{DARCY PROBLEM}
\begin{array}{rcl}
u\hackscore{i} + \kappa\hackscore{ij} p\hackscore{,j} & = & g\hackscore{i} \\
u\hackscore{k,k} & = & f
\end{array}
\end{equation} 
with the boundary conditions
\begin{equation}\label{DARCY BOUNDARY}
\begin{array}{rcl}
u\hackscore{i} \; n\hackscore{i}  = u^{N}\hackscore{i}  \; n\hackscore{i} & \mbox{ on } & \Gamma\hackscore{N} \\
p = p^{D} &  \mbox{ on } & \Gamma\hackscore{D} \\ 
\end{array}
\end{equation} 
where $\Gamma\hackscore{N}$ and $\Gamma\hackscore{D}$ are a partition of the boundary of $\Omega$ with $\Gamma\hackscore{D}$ non empty, $n\hackscore{i}$ is the outer normal field of the boundary of $\Omega$, $u^{N}\hackscore{i}$ and $p^{D}$ are given functions on $\Omega$, $g\hackscore{i}$ and $f$ are given source terms and $\kappa\hackscore{ij}$ is the given permability. We assume that $\kappa\hackscore{ij}$ is symmetric (which is not really required) and positive definite, i.e there are positive constants $\alpha\hackscore{0}$ and $\alpha\hackscore{1}$ wich are independent from the location in $\Omega$ such that
\begin{equation}
\alpha\hackscore{0} \; x\hackscore{i} x\hackscore{i} \le \kappa\hackscore{ij} x\hackscore{i} x\hackscore{j} \le \alpha\hackscore{1} \; x\hackscore{i} x\hackscore{i}
\end{equation}
for all $x\hackscore{i}$. 


\subsection{Solution Method \label{DARCY SOLVE}}
In practical applications it is an advantage to calculate the pressure $p$ as a correction of a 'static' pressure $p^{ref}$ which is the solution of
\begin{equation}
-(\kappa\hackscore{ki}\kappa\hackscore{kj} p\hackscore{,j}^{ref})\hackscore{,i} =  - (\kappa\hackscore{ki} (g\hackscore{k}- u^{N}\hackscore{k}))\hackscore{,i} 
\mbox{ with } 
p^{ref} = p^{D} \mbox{ on } \Gamma\hackscore{D}
\end{equation} 
With setting $u \leftarrow u-u^{N}$ and $p \leftarrow p-p^{ref}$ and 
\begin{equation}
\begin{array}{rcl}
g\hackscore{i} & \leftarrow & g\hackscore{i} - u^{N}\hackscore{i} -  \kappa\hackscore{ij} p^{ref}\hackscore{,j }\\
f & \leftarrow & f - u^{N}\hackscore{k,k}
\end{array}
\end{equation} 
we can assume that $u^{N}\hackscore{i}  \; n\hackscore{i}=0$ and 
$p^{D}=0$. 
We set 
\begin{equation}
V = \{ q \in H^{1}(\Omega) : q=0 \mbox{ on } \Gamma\hackscore{D} \}
\end{equation}
and 
\begin{equation}
W = \{ v \in (L^2(\Omega))^{d} : v\hackscore{k,k} \in L^2(\Omega) \mbox{ and } u\hackscore{i} \; n\hackscore{i} =0  \mbox{ on } \Gamma\hackscore{N} \}
\end{equation}
and define the operator $Q: V \rightarrow (L^2(\Omega))^{d}$ defined by
\begin{equation}
(Qp)\hackscore{i} = \kappa\hackscore{ij} p\hackscore{,j}
\end{equation}
and the operator $D: W \rightarrow L^2(\Omega)$ defined by 
\begin{equation}
Dv = v\hackscore{k,k}
\end{equation}
In operator notation the Darcy problem~\ref{DARCY PROBLEM} is written in the form
\begin{equation}
\begin{array}{rcl}
u + Qp & = & g \\
Du & = & f 
\end{array}
\end{equation} 
We solve this equation by minimising the functional
\begin{equation}
J(u,p):=\|u + Qp - g\|^2\hackscore{0} + \|Du-f\|\hackscore{0}^2 
\end{equation} 
over $W \times V$ where $\|.\|\hackscore{0}$ denotes the norm in $L^2(\Omega)$. A simple calculation shows that
one has to solve
\begin{equation}
( v + Qq , u + Qp - g) + (Dv,Du-f) =0 
\end{equation} 
for all $v\in W$ and $q \in V$.which translates back into operator notation
\begin{equation}
\begin{array}{rcl}
(I+D^*D)u + Qp & = & D^*f + g \\
Q^*u  + Q^*Q p & = & Q^*g \\ 
\end{array}
\end{equation} 
where $D^*$ and $Q^*$ denote the adjoint operators. 
In~\cite{XXX} it has been shown that this problem is continuous and coercive in $W \times V$ and therefore has a unique solution. Also standart FEM methods can be used for discretization. It is also possible 
to solve the problem is coupled form, however this approach leads in some cases to a very ill-conditioned stiffness matrix in particular in the case of a very small or large permability ($\alpha\hackscore{1} \ll 1$ or $\alpha\hackscore{0} \gg 1$)  

The approach we are taking is to eliminate the velocity $u$ from the problem. Assuming that $p$ is known we have
\begin{equation}
v= (I+D^*D)^{-1} (D^*f + g - Qp)
\end{equation} 
(notice that $(I+D^*D)$ is coercive in $W$) which is inserted into the second equation
\begin{equation}
Q^* (I+D^*D)^{-1} (D^*f + g - Qp) + Q^* Q p = Q^*g 
\end{equation} 
which is 
\begin{equation}
Q^* ( I - (I+D^*D)^{-1} ) Q p = Q^* (g-(I+D^*D)^{-1} (D^*f + g) ) 
\end{equation} 
We use the PCG \index{linear solver!PCG}\index{PCG} method to solve this. The residual $r$ ($\in V^*$) is given as
\begin{equation}
\begin{array}{rcl}
r & = & Q^*  \left( g -(I+D^*D)^{-1} (D^*f + g) - Qp + (I+D^*D)^{-1}Q p \right)\\
& =&  Q^* \left( - Qp - (I+D^*D)^{-1} (D^*f + g - Qp) \right) \\
& =&  Q^* \left( g - Qp - v \right)
\end{array}
\end{equation} 
So in a partical implementation we use the pair $(Qp,v)$ to represent the residual. This will save the
reconstruction of the velocity $v$. In this notation the right hand side is given as 
$(0,(I+D^*D)^{-1} (D^*f + g))$. The evaluation of the iteration operator for a given $p$ is then 
returning $(Qp,w)$ where $w$ is the solution of 
\begin{equation}\label{UPDATE W}
(I+D^*D)w = Qp
\end{equation}
We use $Q^*Q$ as a a preconditioner for the iteration operator $Q^* ( I - (I+D^*D)^{-1} ) Q$.

The iteration PCG \index{linear solver!PCG}\index{PCG} is terminated if
\begin{equation}\label{DARCY STOP}
\int r \cdot (Q^*Q)^{-1} r \; dx \le \mbox{ATOL}^2
\end{equation}
where ATOL is a given absolute tolerance.
The initial residual $r\hackscore{0}$ is 
\begin{equation}\label{DARCY STOP 2}
r\hackscore{0}=Q^* \left( g - v\hackscore{ref} \right) \mbox{ with } v\hackscore{ref} = (I+D^*D)^{-1} (D^*f + g)
\end{equation}
so the 
\begin{equation}\label{DARCY NORM 0}
\int r\hackscore0 \cdot (Q^*Q)^{-1} r\hackscore0 \; dx = \int \left( g - v\hackscore{ref} \right)  \cdot  Q  p\hackscore{ref} \; dx \mbox{ with }p\hackscore{ref} = (Q^*Q)^{-1} Q^* \left( g - v\hackscore{ref} \right)
\end{equation}
So we set 
\begin{equation}\label{DARCY NORM 1}
ATOL = atol + rtol \cdot \max(|g - v\hackscore{ref}|\hackscore{0}, |Q p\hackscore{ref} |\hackscore{0} )
\end{equation}
where atol and rtol a given absolute and relative tolerances, respectively. The reference flux $v\hackscore{ref}$
and reference pressure $p\hackscore{ref}$ may be calcualated from their definition which would require to solve to
PDEs but in a practical application estimates can be used for instance solutions from previous time steps or for simplified scenarious (e.g. constant permability). 

\subsection{Functions}
\begin{classdesc}{DarcyFlow}{domain}
opens the Darcy flux problem\index{Darcy flux} on the \Domain domain.
\end{classdesc}
\begin{methoddesc}[DarcyFlow]{setValue}{\optional{f=None, \optional{g=None, \optional{location_of_fixed_pressure=None, \optional{location_of_fixed_flux=None, \optional{permeability=None}}}}}}
assigns values to the model parameters. Values can be assigned using various calls - in particular 
in a time dependend problem only values that change over time needs to be reset. The permability can be defined as scalar (isotropic), a vector (orthotropic) or a matrix (anisotropic). 
\var{f} and \var{g} are the corresponding parameters in~\ref{DARCY PROBLEM}.
The locations and compontents where the flux is prescribed are set by positive values in
\var{location_of_fixed_flux}. 
The locations where the pressure is prescribed are set by 
by positive values of \var{location_of_fixed_pressure}. 
The values of the pressure and flux are defined by the initial guess.
Notice that at any point on the boundary of the domain the pressure or the normal component of
the flux must be defined. There must be at least one point where the pressure is prescribed. 
The method will try to cast the given values to appropriate 
\Data class objects.
\end{methoddesc}

\begin{methoddesc}[DarcyFlow]{setTolerance}{\optional{atol=0,\optional{rtol=1e-8,\optional{p_ref=None,\optional{v_ref=None}}}}}
sets the absolute tolerance ATOL according to~\ref{DARCY NORM 1}. If \var{p_ref} is not present $0$ is used. 
If \var{v_ref} is not present $0$ is used. If the final result ATOL is not positive an exception is thrown. 
\end{methoddesc}



\begin{methoddesc}[DarcyFlow]{solve}{u0,p0, \optional{max_iter=100, \optional{verbose=False \optional{sub_rtol=1.e-8}}}}
solves the problem. and returns approximations for the flux $v$ and the pressure $p$. 
\var{u0} and \var{p0} define initial guess for flux and pressure. Values marked
by positive values \var{location_of_fixed_flux} and \var{location_of_fixed_pressure}, respectively, are kept unchanged.
\var{sub_rtol} defines the tolerance used to solve the involved PDEs. \var{sub_rtol} needs to be choosen sufficiently small to ensure convergence but users need to keep in mind that a very small value for \var{sub_rtol} will result in a long compute time. Typically  $\var{sub_rtol}=\var{rtol}^2$ is a good choice if $\var{rtol}$ is not choosen too small.
\end{methoddesc}


\subsection{Example: Gravity Flow}
later

%================================================
% \section{Temperature Advection Diffusion\label{TEMP ADV DIFF}}

%\begin{equation}
% \rho c\hackscore{p} \left (\frac{\partial T}{\partial t} + \vec{v} \cdot \nabla T \right ) = k \nabla^{2}T
% \label{HEAT EQUATION}
% \end{equation}

% where $\vec{v}$ is the velocity vector, $T$ is the temperature, $\rho$ is the density, $\eta$ is the viscosity, % % $c\hackscore{p}$ is the specific heat at constant pressure and $k$ is the thermal conductivity.

% \subsection{Description}

% \subsection{Method}
%
% \begin{classdesc}{TemperatureCartesian}{dom,theta=THETA,useSUPG=SUPG}
% \end{classdesc}

% \subsection{Benchmark Problem}
%===============================================================================================================

%=========================================================
\section{Level Set Method}

The Level Set Method is used for tracking interfaces between two different types of fluids, which may have different physical parameter values for density or viscosity. The interface is represented by a signed distance function, $\phi(x)$, where the isocontour at $\phi(x)=0$ is used to defined the interface. A point in the domain can then be determined on which side of the interface it resides, based on the local sign of $\phi(x)$; for example positive $\phi(x)$ on one side of the interface and negative $\phi(x)$ on the other. Parameters values such as density and viscosity can then be defined for the two different mediums. The Level Set Method consists of two procedures, the advection and reinitialization of the signed distance function, $\phi$. The LevelSet class can be used in conjunction with the StokesProblemCartesian class for solving computational fluid dynamics problems involving the tracking of the interface. The advantage of the Level Set Method is that it can be used to track surfaces that break apart or intersect. Also, the Level Set Method avoids the need for remeshing, which is required by the Lagrangian-Eulerian (ALE) method. An example of using the Level Set Method is described in the tutorial Chapter, Section \ref{LEVELSET CHAP}.

\subsection{Solution Method}

The displacement of the interface at the zero isocontour of $\phi(x)$ is calculated each time-step by using the velocity field. This is achieved my solving the advection equation:
%
\begin{equation}
\frac{\partial \phi}{\partial t} + \vec{v} \cdot \nabla \phi = 0,
\label{ADVECTION MODELS}
\end{equation}
%
where $\vec{v}$ is the velocity field. The advection equation is solved using a Taylor-Galerkin scheme with the presence of diffusion; by expanding $\phi$ into a Taylor series:
%
\begin{equation}
\phi^{+} \simeq \phi^{-} + dt\frac{\partial \phi^{-}}{\partial t} + \frac{dt^2}{2}\frac{\partial^{2}\phi^{-}}{\partial t^{2}},
\label{TAYLOR EXPANSION MODELS}
\end{equation}
%
then by inserting
%
\begin{equation}
\frac{\partial \phi^{-}}{\partial t} = - \vec{v} \cdot \nabla \phi^{-},
\label{INSERT ADVECTION MODELS}
\end{equation}
%
and
%
\begin{equation}
\frac{\partial^{2} \phi^{-}}{\partial t^{2}} = \frac{\partial}{\partial t}(-\vec{v} \cdot \nabla \phi^{-}) = \vec{v}\cdot \nabla (\vec{v}\cdot \nabla \phi^{-}),
\label{SECOND ORDER MODELS}
\end{equation}
%
into Equation (\ref{TAYLOR EXPANSION MODELS}), the calculation of the level set function is given by:
%
\begin{equation}
\phi^{+} = \phi^{-} - dt\vec{v}\cdot \nabla \phi^{-} + \frac{dt^2}{2}\vec{v}\cdot \nabla (\vec{v}\cdot \nabla \phi^{-}).
\label{TAYLOR GALERKIN MODELS}
\end{equation}

If $\nabla \cdot \vec{v}=0$ is assumed, then the calculation of the second order derivatives in Equation (\ref{TAYLOR GALERKIN MODELS}) can be avoided.

As the computation of the distance function progresses, it becomes distorted, and so it needs to be updated in order to stay regular \cite{SUSSMAN1994}. This process is known as the reinitialization procedure. The aim is to iteratively find a solution to the reinitialization equation:
%
\begin{equation}
\frac{\partial \psi}{\partial \tau} + sign(\phi)(1 - \nabla \psi) = 0.
\label{REINITIALISATION MODELS}
\end{equation}
%
where $\psi$ shares the same level set with $\phi$, $\tau$ is pseudo time, and $sign(\phi)$ is the smoothed sign function. This equation is solved to meet the definition of the level set function, $\lvert \nabla \psi \rvert = 1$; the normalization condition. Equation (\ref{REINITIALISATION MODELS}) can be rewritten in a similar form to the advection equation:
%
\begin{equation}
\frac{\partial \psi}{\partial \tau} + \vec{w} \cdot \nabla \psi = sign(\phi),
\label{REINITIALISATION2 MODELS}
\end{equation}
%
where
%
\begin{equation}
\vec{w} = sign(\phi)\frac{\nabla \psi}{|\nabla \psi|}.
\label{REINITIALISATION3 MODELS}
\end{equation}
%
$\vec{w}$ is the characteristic velocity pointing outward from the free surface. Equation (\ref{REINITIALISATION2 MODELS}) can be solved by a similar technique to what was used in the advection step, using the Taylor-Galerkin procedure.
When the distance function, $\phi$, is calculated, the physical parameters, density and viscosity, are updated using the sign of $\phi$. The region along the interface is assumed to be of finite thickness of $\alpha h$, where $h$ is the size of the elements in the computational mesh and $\alpha$ is a smoothing parameter. The parameters are updated by the following expression:
%
\begin{equation}
P = 
\left \{ \begin{array}{l}
P\hackscore{1} \hspace{5cm}  where \ \ \psi < - \alpha h \\
P\hackscore{2} \hspace{5cm}  where \ \ \psi > \alpha h \\
(P\hackscore{2} - P\hackscore{1}) \psi/2\alpha h + (P\hackscore{1} + P\hackscore{2})/2 \ \ \ \ \ \ where \ \ |\psi| < \alpha h.
\end{array}
\right.
\label{UPDATE PARAMETERS MODELS}
\end{equation} 
%
where the subscripts $1$ and $2$ denote the different fluids.


\subsection{Functions}

\begin{classdesc}{LevelSet}{domain, func, reinit\_max, reinit\_each, tolerance, smooth}
opens the LevelSet \index{Level Set} on the \Domain domain. \var{func} defines the initial Level Set function representing the interface between two fluids. \var{reinit\_max} sets the maximum number of iterations to satisfy the normal condition, $|\nabla \phi|=1$, during the reinitialization of the Level Set function. \var{reinit\_each} sets the frequency of reinitialization for a number of time-steps. \var{tolerance} sets the convergence tolerance to satisfy the normal condition during the reinitialization of the Level Set function. \var{smooth} sets the bandwidth of size 2$\alpha h$ along the interface to smooth the physical parameters of density and viscosity; $h$ is the size of the elements in the mesh and $\alpha$ is the smoothing parameter, usually set to 1.
\end{classdesc}

\begin{methoddesc}[LevelSet]{update\_parameter}{par1, par2}
updates the physical parameters using the sign of $\phi$. \var{par1} and \var{par2} are the physical parameter values for fluid1 and fluid2 respectively. Usually this method is called twice during each time-step to update the density and viscosity of the two fluids.
\end{methoddesc}

\begin{methoddesc}[LevelSet]{update\_phi}{vel,  dt, t\_step}
updates the Level Set function. It performs the advection and reinitialization procedures. \var{vel} is the velocity field of the fluid domain, \var{dt} is the time-step size, and \var{t\_step} is the current time-step to determine when to reinitialize.
\end{methoddesc}


% \section{Drucker Prager Model}

\section{Isotropic Kelvin Material \label{IKM}}
As proposed by Kelvin~\ref{KELVN} material strain $D\hackscore{ij}=v\hackscore{i,j}+v\hackscore{j,i}$ can be decomposed into
an elastic part $D\hackscore{ij}^{el}$ and visco-plastic part $D\hackscore{ij}^{vp}$:
\begin{equation}\label{IKM-EQU-2}
D\hackscore{ij}=D\hackscore{ij}^{el}+D\hackscore{ij}^{vp}
\end{equation}
with the elastic strain given as 
\begin{equation}\label{IKM-EQU-3}
D\hackscore{ij}'^{el}=\frac{1}{2 \mu} \dot{\sigma}\hackscore{ij}'
\end{equation}
where $\sigma'\hackscore{ij}$ is the deviatoric stress (Notice that $\sigma'\hackscore{ii}=0$).
If the material is composed by materials $q$ the visco-plastic strain can be decomposed as
\begin{equation}\label{IKM-EQU-4}
D\hackscore{ij}'^{vp}=\sum\hackscore{q} D\hackscore{ij}'^{q} 
\end{equation}
where $D\hackscore{ij}^{q}$ is the strain in material $q$ given as 
\begin{equation}\label{IKM-EQU-5}
D\hackscore{ij}'^{q}=\frac{1}{2 \eta^{q}} \sigma'\hackscore{ij} 
\end{equation}
where $\eta^{q}$ is the viscosity of material $q$. We assume the following 
betwee the the strain in material $q$ 
\begin{equation}\label{IKM-EQU-5b}
\eta^{q}=\eta^{q}\hackscore{N} \left(\frac{\tau}{\tau\hackscore{t}^q}\right)^{\frac{1}{n^{q}}-1}
\mbox{ with } \tau=\sqrt{\frac{1}{2}\sigma'\hackscore{ij} \sigma'\hackscore{ij}}
\end{equation}
for a given power law coefficients $n^{q}$ and transition stresses $\tau\hackscore{t}^q$, see~\ref{POERLAW}.
Notice that $n^{q}=1$ gives a constant viscosity.
After inserting equation~\ref{IKM-EQU-5} into equation \ref{IKM-EQU-4} one gets:
\begin{equation}\label{IKM-EQU-6}
D\hackscore{ij}'^{vp}=\frac{1}{2 \eta^{vp}} \sigma'\hackscore{ij} \mbox{ with } \frac{1}{\eta^{vp}} = \sum\hackscore{q} \frac{1}{\eta^{q}} \;.
\end{equation}
With
\begin{equation}\label{IKM-EQU-8}
\dot{\gamma}=\sqrt{2 D\hackscore{ij} D\hackscore{ij}}
\end{equation}
one gets 
\begin{equation}\label{IKM-EQU-8b}
\tau = \eta^{vp} \dot{\gamma}^{vp} \;.
\end{equation}
With the Drucker-Prager cohesion factor $\tau\hackscore{Y}$, Drucker-Prager friction $\beta$ and total pressure $p$ we want to achieve 
\begin{equation}\label{IKM-EQU-8c}
\tau \le \tau\hackscore{Y} + \beta \; p
\end{equation}
which leads to the condition
\begin{equation}\label{IKM-EQU-8d}
\eta^{vp} \le \frac{\tau\hackscore{Y} + \beta \; p}{ \dot{\gamma}^{vp}} \; .
\end{equation}
Therefore we modify the definition of $\eta^{vp}$ to the form
\begin{equation}\label{IKM-EQU-6b}
\frac{1}{\eta^{vp}}=\max(\sum\hackscore{q} \frac{1}{\eta^{q}}, \frac{\dot{\gamma}^{vp}} {\tau\hackscore{Y} + \beta \; p})
\end{equation}
The deviatoric stress needs to fullfill the equilibrion equation
\begin{equation}\label{IKM-EQU-1}
-\sigma'\hackscore{ij,j}+p\hackscore{,i}=F\hackscore{i}
\end{equation}
where $F\hackscore{j}$ is a given external fource. We assume an incompressible media:
\begin{equation}\label{IKM-EQU-2}
-v\hackscore{i,i}=0
\end{equation}
Natural boundary conditions are taken in the form 
\begin{equation}\label{IKM-EQU-Boundary}
\sigma'\hackscore{ij}n\hackscore{j}-n\hackscore{i}p=f
\end{equation}
which can be overwritten by a constraint 
\begin{equation}\label{IKM-EQU-Boundary0}
v\hackscore{i}(x)=0
\end{equation}
where the index $i$ may depend on the location $x$ on the bondary.

\subsection{Solution Method \label{IKM-SOLVE}}
By using a first order finite difference approximation wit step size $dt>0$~\ref{IKM-EQU-3} get the form
\begin{equation}\label{IKM-EQU-3b}
D\hackscore{ij}'^{el}=\frac{1}{2 \mu dt } \left( \sigma\hackscore{ij}' - \sigma\hackscore{ij}^{'-} \right)
\end{equation}
where $\sigma\hackscore{ij}^{'-}$ is the deviatoric stress at the precious time step.
Now we can combine equations~\ref{IKM-EQU-2}, \ref{IKM-EQU-3b} and~\ref{IKM-EQU-6b} to get
\begin{equation}\label{IKM-EQU-10}
\sigma\hackscore{ij}' =  2 \eta\hackscore{eff}  \left( D\hackscore{ij}' + 
\frac{1}{  2 \mu \; dt} \sigma\hackscore{ij}^{'-}\right)  \mbox{ with }
\frac{1}{\eta\hackscore{eff}}=\frac{1}{\mu \; dt}+\frac{1}{\eta^{vp}}
\end{equation}
Notice that $\eta\hackscore{eff}$ is a function of diatoric stress $\sigma\hackscore{ij}'$.
After inserting~\ref{IKM-EQU-10} into~\ref{IKM-EQU-1} we get
\begin{equation}\label{IKM-EQU-1ib}
-\left(\eta\hackscore{eff} (v\hackscore{i,j}+ v\hackscore{i,j})
\right)\hackscore{,j}+p\hackscore{,i}=F\hackscore{i}+
\frac{\eta\hackscore{eff}}{\mu dt } \sigma\hackscore{ij,j}^{'-}
\end{equation}
Together with the incomressibilty condition~\ref{IKM-EQU-2} we need to solve a problem with a form almost identical 
to the Stokes problem discussed in section~\ref{STOKES SOLVE} but with the difference that $\eta\hackscore{eff}$ is depending on the solution. Analog to the iteration scheme~\ref{SADDLEPOINT iteration 2} we can run
\begin{equation}
\begin{array}{rcl}
-\left(\eta\hackscore{eff}(dv\hackscore{i,j}+ dv\hackscore{i,j}
)\right)\hackscore{,j} & = & F\hackscore{i}+ \sigma\hackscore{ij,j}'-p\hackscore{,i} \\
\frac{1}{\eta\hackscore{eff}} dp & = & - v\hackscore{i,i}^+
\end{array}
\label{IKM iteration 2}
\end{equation}
where $v^+=v+dv$. As this problem is non-linear the Jacobi-free Newton-GMRES method is used with the norm
\begin{equation}
\|(v, p)\|^2= \int\hackscore{\Omega} v\hackscore{i,j}^2 + \frac{1}{\bar{\eta}^2\hackscore{eff}} p^2 \; dx
\label{IKM iteration 3}
\end{equation}
where  $\bar{\eta}\hackscore{eff}$ is the caracteristic viscosity, for instance:
\begin{equation}
\frac{1}{\bar{\eta}\hackscore{eff}} = \frac{1}{\tau^{-}}+\sum\hackscore{q}  \frac{1}{\eta^{q}\hackscore{N}}
\label{IKM iteration 4}
\end{equation}
In oder to perform step~\ref{IKM iteration 2} we need to calculate the $\eta\hackscore{eff}$ as well as $\sigma\hackscore{ij}'$ while via $\tau$ the first is a function of the latter. The priority is the 
calculation of $\eta\hackscore{eff}$ with the Newton-Raphson scheme. This value can then be used to calculate 
$\sigma\hackscore{ij}'$ via~\ref{IKM-EQU-10}. We need to solve
\begin{equation}
\tau = \eta\hackscore{eff} \cdot \epsilon \mbox{ with }
\epsilon = \sqrt{ 2 \left( D\hackscore{ij}' + 
\frac{1}{  2 \mu \; dt} \sigma\hackscore{ij}^{'-}\right)^2}
\label{IKM iteration 5}
\end{equation} 
The Newton scheme takes the form 
\begin{equation}
\tau\hackscore{n+1} = \min(\tau\hackscore{n} - \frac{\tau\hackscore{n} - \eta\hackscore{eff}  \cdot \epsilon}{1 - \eta\hackscore{eff}'  \cdot  \epsilon}, \tau\hackscore{Y} + \beta \; p)
= \min(\frac{\eta\hackscore{eff} - \tau\hackscore{n}  \eta\hackscore{eff}'}
{1 - \eta\hackscore{eff}'  \cdot  \epsilon}, \frac{\tau\hackscore{Y} + \beta \; p}{\epsilon}) \epsilon
\label{IKM iteration 6}
\end{equation} 
where $\eta\hackscore{eff}'$ denotes the derivative of $\eta\hackscore{eff}$ with respect of $\tau$. The second term in $\min$ is droped of $\tau\hackscore{Y} + \beta \; p<0$ or $\epsilon=0$. In fact we have
\begin{equation}
\eta\hackscore{eff}' = - \eta\hackscore{eff}^2 \left(\frac{1}{\eta\hackscore{eff}}\right)'
\mbox{ with } 
\left(\frac{1}{\eta\hackscore{eff}}\right)' = \sum\hackscore{q} \left(\frac{1}{\eta^{q}} \right)'
\label{IKM iteration 7}
\end{equation} 
\begin{equation}\label{IKM-EQU-5XX}
\left(\frac{1}{\eta^{q}} \right)'
= \frac{1-\frac{1}{n^{q}}}{\eta^{q}\hackscore{N}} \frac{\tau^{-\frac{1}{n^{q}}}}{(\tau\hackscore{t}^q)^{1-\frac{1}{n^{q}}}}
= \frac{1-\frac{1}{n^{q}}}{ \tau \eta^{q}} 
\end{equation}
Notice that allways $\eta\hackscore{eff}'\le 0$ which makes the denomionator in~\ref{IKM iteration 6}
positive.



