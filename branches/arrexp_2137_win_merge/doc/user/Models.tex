
%%%%%%%%%%%%%%%%%%%%%%%%%%%%%%%%%%%%%%%%%%%%%%%%%%%%%%%%
%
% Copyright (c) 2003-2008 by University of Queensland
% Earth Systems Science Computational Center (ESSCC)
% http://www.uq.edu.au/esscc
%
% Primary Business: Queensland, Australia
% Licensed under the Open Software License version 3.0
% http://www.opensource.org/licenses/osl-3.0.php
%
%%%%%%%%%%%%%%%%%%%%%%%%%%%%%%%%%%%%%%%%%%%%%%%%%%%%%%%%


\chapter{Models}

The following sections give a breif overview of the model classes and their corresponding methods.

\section{Stokes Problem}
The velocity field $v$ and pressure $p$ of an incompressible fluid is given as the solution of their
Stokes problem
\begin{equation}\label{Stokes 1}
-\left(\eta(v\hackscore{i,j}+ v\hackscore{i,j})\right)\hackscore{,j}+p\hackscore{,i}=F\hackscore{i}
\end{equation}
where $\eta$ is the viscosity and $F_i$ defines an internal force. We assume an incompressible media:
\begin{equation}\label{Stokes 2}
-v\hackscore{i,i}=0
\end{equation}
Natural boundary conditions are taken in the form 
\begin{equation}\label{Stokes Boundary}
\left(\eta(v\hackscore{i,j}+ v\hackscore{i,j})\right)n\hackscore{j}-n\hackscore{i}p=f
\end{equation}
which can be overwritten by a constraint 
\begin{equation}\label{Stokes Boundary0}
v\hackscore{i}(x)=0
\end{equation}
where the index $i$ may depend on the location $x$ on the bondary.


\subsection{Solution Method \label{STOKES SOLVE}}
In block form equation equations~\ref{Stokes 1} and~\ref{Stokes 2} takes the form of a saddle point problem
\begin{equation}
\left[ \begin{array}{cc}
A     & B^{*} \\
B & 0 \\
\end{array} \right]
\left[ \begin{array}{c}
u \\
p \\
\end{array} \right]
=\left[ \begin{array}{c}
F \\
0 \\
\end{array} \right]
\label{SADDLEPOINT}
\end{equation}
where $A$ is coercive, self-adjoint linear operator in $V$, $B$ is a divergence operator and $B^{*}$ is it adjoint operator (=gradient operator)). For more details on the mathematics see references \cite{AAMIRBERKYAN2008,MBENZI2005}.
We apply the preconditioner matrix
\begin{equation}
\left[ \begin{array}{cc}
A^{-1}     & 0 \\
S^{-1} B A^{-1}  & -S^{-1} \\
\end{array} \right]
\label{SADDLEPOINT PRECODITIONER}
\end{equation}
with the Schur complement $S=BA^{-1}B^{*}$ to solve the problem iteratively. The updates $dv$ and $dp$ for 
given velocity $v$ and pressure $p$ is given as
\begin{equation}
\begin{array}{rcl}
A dv & = & F-Av-B^{*}p \\
S dp & = & B(v+dv) \\
\end{array}
\label{SADDLEPOINT iteration}
\end{equation}
This scheme is called the Uzawa scheme. 

In the case of the Stokes problem it can be shown that $S$ can be approximated by $\frac{1}{\eta}$. With this the iteration scheme can be implemented as
\begin{equation}
\begin{array}{rcl}
-\left(\eta(dv\hackscore{i,j}+ dv\hackscore{i,j})\right)\hackscore{,j} & = & F\hackscore{i}+\left(\eta(v\hackscore{i,j}+ v\hackscore{i,j})\right)\hackscore{,j}-p\hackscore{,i} \\
\frac{1}{\eta} dp & = & - \left(v\hackscore{,i}+dv\hackscore{,i} \right)\hackscore{,i} \\
\end{array}
\label{SADDLEPOINT iteration 2}
\end{equation}
To accelerate the convergence we are using the restarted $GMRES$ method using the norm
\begin{equation}
\|(v,p)\|^2= \int\hackscore{\Omega} \eta^2 v\hackscore{i,j}^2 + p^2 \; dx
\label{SADDLEPOINT iteration 3}
\end{equation}
or alternatively the $PCG$ method on the pressure only using the norm 
\begin{equation}
\|p\|^2= \int\hackscore{\Omega} \frac{1}{\eta^2} p^2 \; dx
\label{SADDLEPOINT iteration 4}
\end{equation}





\begin{classdesc}{StokesProblemCartesian}{domain,debug}
opens the stokes equations on the \Domain domain. Setting debug=True switches the debug mode to on.
\end{classdesc}

example usage:

solution=StokesProblemCartesian(mesh) \\
solution.setTolerance(TOL) \\
solution.initialize(fixed\_u\_mask=b\_c,eta=eta,f=Y) \\
velocity,pressure=solution.solve(velocity,pressure,max\_iter=max\_iter,solver=solver) \\

% \subsection{Benchmark Problem}
% 
% Convection problem


\section{Temperature Cartesian}

\begin{equation}
\rho c\hackscore{p} \left (\frac{\partial T}{\partial t} + \vec{v} \cdot \nabla T \right ) = k \nabla^{2}T
\label{HEAT EQUATION}
\end{equation}

where $\vec{v}$ is the velocity vector, $T$ is the temperature, $\rho$ is the density, $\eta$ is the viscosity, $c\hackscore{p}$ is the specific heat at constant pressure and $k$ is the thermal conductivity.

\subsection{Description}

\subsection{Method}

\begin{classdesc}{TemperatureCartesian}{dom,theta=THETA,useSUPG=SUPG}
\end{classdesc}

\subsection{Benchmark Problem}


\section{Level Set Method}

\subsection{Description}

\subsection{Method}

Advection and Reinitialisation

\begin{classdesc}{LevelSet}{mesh, func\_new, reinit\_max, reinit\_each, tolerance, smooth}
\end{classdesc}

%example usage:

%levelset = LevelSet(mesh, func\_new, reinit\_max, reinit\_each, tolerance, smooth)

\begin{methoddesc}[LevelSet]{update\_parameter}{parameter}
Update the parameter.
\end{methoddesc}

\begin{methoddesc}[LevelSet]{update\_phi}{paramter}{velocity}{dt}{t\_step}
Update level set function; advection and reinitialization
\end{methoddesc}

\subsection{Benchmark Problem}

Rayleigh-Taylor instability problem


% \section{Drucker Prager Model}

\section{Isotropic Kelvin Material \label{IKM}}
As proposed by Kelvin~\ref{KELVN} material strain $D\hackscore{ij}=v\hackscore{i,j}+v\hackscore{j,i}$ can be decomposed into
an elastic part $D\hackscore{ij}^{el}$ and visco-plastic part $D\hackscore{ij}^{vp}$:
\begin{equation}\label{IKM-EQU-2}
D\hackscore{ij}=D\hackscore{ij}^{el}+D\hackscore{ij}^{vp}
\end{equation}
with the elastic strain given as 
\begin{equation}\label{IKM-EQU-3}
D\hackscore{ij}'^{el}=\frac{1}{2 \mu} \dot{\sigma}\hackscore{ij}'
\end{equation}
where $\sigma'\hackscore{ij}$ is the deviatoric stress (Notice that $\sigma'\hackscore{ii}=0$).
If the material is composed by materials $q$ the visco-plastic strain can be decomposed as
\begin{equation}\label{IKM-EQU-4}
D\hackscore{ij}'^{vp}=\sum\hackscore{q} D\hackscore{ij}'^{q} 
\end{equation}
where $D\hackscore{ij}^{q}$ is the strain in material $q$ given as 
\begin{equation}\label{IKM-EQU-5}
D\hackscore{ij}'^{q}=\frac{1}{2 \eta^{q}} \sigma'\hackscore{ij} 
\end{equation}
where $\eta^{q}$ is the viscosity of material $q$. We assume the following 
betwee the the strain in material $q$ 
\begin{equation}\label{IKM-EQU-5b}
\eta^{q}=\eta^{q}\hackscore{N} \left(\frac{\tau}{\tau\hackscore{t}^q}\right)^{\frac{1}{n^{q}}-1}
\mbox{ with } \tau=\sqrt{\frac{1}{2}\sigma'\hackscore{ij} \sigma'\hackscore{ij}}
\end{equation}
for a given power law coefficients $n^{q}$ and transition stresses $\tau\hackscore{t}^q$, see~\ref{POERLAW}.
Notice that $n^{q}=1$ gives a constant viscosity.
After inserting equation~\ref{IKM-EQU-5} into equation \ref{IKM-EQU-4} one gets:
\begin{equation}\label{IKM-EQU-6}
D\hackscore{ij}'^{vp}=\frac{1}{2 \eta^{vp}} \sigma'\hackscore{ij} \mbox{ with } \frac{1}{\eta^{vp}} \sum\hackscore{q} \frac{1}{\eta^{q}} \;.
\end{equation}
With
\begin{equation}\label{IKM-EQU-8}
\dot{\gamma}=\sqrt{2 D\hackscore{ij} D\hackscore{ij}}
\end{equation}
one gets 
\begin{equation}\label{IKM-EQU-8b}
\tau = \eta^{vp} \dot{\gamma}^{vp} \;.
\end{equation}
With the Drucker-Prager cohesion factor $\tau\hackscore{Y}$, Drucker-Prager friction $\beta$ and total pressure $p$ we want to achieve 
\begin{equation}\label{IKM-EQU-8c}
\tau \le \tau\hackscore{Y} + \beta \; p
\end{equation}
which leads to the condition
\begin{equation}\label{IKM-EQU-8d}
\eta^{vp} \le \frac{\tau\hackscore{Y} + \beta \; p}{ \dot{\gamma}^{vp}} \; .
\end{equation}
Therefore we modify the definition of $\eta^{vp}$ to the form
\begin{equation}\label{IKM-EQU-6b}
\frac{1}{\eta^{vp}}=\max(\sum\hackscore{q} \frac{1}{\eta^{q}}, \frac{\dot{\gamma}^{vp}} {\tau\hackscore{Y} + \beta \; p})
\end{equation}
The deviatoric stress needs to fullfill the equilibrion equation
\begin{equation}\label{IKM-EQU-1}
-\sigma'\hackscore{ij,j}+p\hackscore{,i}=F\hackscore{i}
\end{equation}
where $F\hackscore{j}$ is a given external fource. We assume an incompressible media:
\begin{equation}\label{IKM-EQU-2}
-v\hackscore{i,i}=0
\end{equation}
Natural boundary conditions are taken in the form 
\begin{equation}\label{IKM-EQU-Boundary}
\sigma'\hackscore{ij}n\hackscore{j}-n\hackscore{i}p=f
\end{equation}
which can be overwritten by a constraint 
\begin{equation}\label{IKM-EQU-Boundary0}
v\hackscore{i}(x)=0
\end{equation}
where the index $i$ may depend on the location $x$ on the bondary.

\subsection{Solution Method \label{IKM-SOLVE}}
By using a first order finite difference approximation wit step size $dt>0$~\ref{IKM-EQU-3} get the form
\begin{equation}\label{IKM-EQU-3b}
D\hackscore{ij}'^{el}=\frac{1}{2 \mu dt } \left( \sigma\hackscore{ij}' - \sigma\hackscore{ij}^{'-} \right)
\end{equation}
where $\sigma\hackscore{ij}^{'-}$ is the deviatoric stress at the precious time step.
Now we can combine equations~\ref{IKM-EQU-2}, \ref{IKM-EQU-3b} and~\ref{IKM-EQU-6b} to get
\begin{equation}\label{IKM-EQU-10}
\sigma\hackscore{ij}' =  2 \eta\hackscore{eff} D\hackscore{ij}' + 
\frac{\eta\hackscore{eff}}{\mu \; dt}
\sigma\hackscore{ij}^{'-} \mbox{ with }
\frac{1}{\eta\hackscore{eff}}=\frac{1}{\mu \; dt}+\frac{1}{\eta^{vp}}
\end{equation}
Notice that $\eta\hackscore{eff}$ is a function of stress.
After inserting~\ref{IKM-EQU-10} into~\ref{IKM-EQU-1} we get
\begin{equation}\label{IKM-EQU-1ib}
-\left(\eta\hackscore{eff} (v\hackscore{i,j}+ v\hackscore{i,j})\right)\hackscore{,j}+p\hackscore{,i}=F\hackscore{i}+
\frac{\eta\hackscore{eff}}{\mu dt } \sigma\hackscore{ij,j}^{'-}
\end{equation}
Together with the incomressibilty condition~\ref{IKM-EQU-2} we need to solve a problem with a form almost identical 
to the Stokes problem discussed in section~\ref{STOKES SOLVE} but with the difference that $\eta\hackscore{eff}$ is depending on the solution. Analog to the iteration scheme~\ref{SADDLEPOINT iteration 2} we can run
\begin{equation}
\begin{array}{rcl}
-\left(\eta\hackscore{eff}(dv\hackscore{i,j}+ dv\hackscore{i,j})\right)\hackscore{,j} & = & F\hackscore{i}+ \sigma\hackscore{ij,j}' \\
\frac{1}{\eta\hackscore{eff}} dp & = & - v\hackscore{i,i}^+ \\
d\sigma\hackscore{ij}' & = & 2 \eta\hackscore{eff} D\hackscore{ij}^{+'} + \frac{\eta\hackscore{eff}}{\mu \; dt} \sigma\hackscore{ij}' -  \sigma\hackscore{ij}\\
\end{array}
\label{IKM iteration 2}
\end{equation}
where $v^+=v+dv$. As this problem is non-linear the Jacobi-free Newton-GMRES method is used with the norm
\begin{equation}
\|(\sigma, p)\|^2= \int\hackscore{\Omega} \sigma\hackscore{i,j}^2 + p^2 \; dx
\label{IKM iteration 3}
\end{equation}


