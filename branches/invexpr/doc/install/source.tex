%!TEX root = install.tex
%%%%%%%%%%%%%%%%%%%%%%%%%%%%%%%%%%%%%%%%%%%%%%%%%%%%%%%%%%%%%%%%%%%%%%%%%%%%%%
% Copyright (c) 2012-2015 by University of Queensland
% http://www.uq.edu.au
%
% Primary Business: Queensland, Australia
% Licensed under the Open Software License version 3.0
% http://www.opensource.org/licenses/osl-3.0.php
%
% Development until 2012 by Earth Systems Science Computational Center (ESSCC)
% Development 2012-2013 by School of Earth Sciences
% Development from 2014 by Centre for Geoscience Computing (GeoComp)
%
%%%%%%%%%%%%%%%%%%%%%%%%%%%%%%%%%%%%%%%%%%%%%%%%%%%%%%%%%%%%%%%%%%%%%%%%%%%%%%

% Notes about compilers

\chapter{Installing from Source}\label{chap:source}

This chapter assumes you are using a unix/posix like system (including MacOSX).

\section{Parallel Technologies}\label{sec:par}
It is likely that the computer you run \escript on, will have more than one processor core.
\escript can make use of multiple cores [in order to solve problems more quickly] if it is told to do so,
but this functionality must be enabled at compile time.
Section~\ref{sec:needpar} gives some rough guidelines to help you determine what you need.

There are two technologies which \escript can employ here.
\begin{itemize}
 \item OpenMP -- the more efficient of the two [thread level parallelism].
 \item MPI -- Uses multiple processes (less efficient), needs less help from the compiler.
\end{itemize}

Escript is primarily tested on recent versions of the GNU and Intel suites (``g++'' / ``icpc'').
However, it also passes our tests when compiled using ``clang++''.

Our current test compilers include:
\begin{itemize}
 \item g++ 4.7.2, 4.9.1
 \item clang++ (OSX 10.9 default, OSX 10.10 default)
 \item intel icpc v14
\end{itemize}

Note that:
\begin{itemize}
 \item OpenMP will not function correctly for g++ $\leq$ 4.2.1 (and is not currently supported by clang).
 \item icpc v11 has a subtle bug involving OpenMP and c++ exception handling, so this combination should not be used.
\end{itemize}

\subsection{What parallel technology do I need?}\label{sec:needpar}
If you are using any version of Linux released in the past few years, then your system compiler will support 
\openmp with no extra work (and give better performance); so you should use it.
You will not need MPI unless your computer is some form of cluster.

If you are using BSD or MacOSX and you are just experimenting with \escript, then performance is
probably not a major issue for you at the moment so you don't need to use either \openmp or MPI. 
This also applies if you write and polish your scripts on your computer and then send them to a cluster to execute.
If in the future you find escript useful and your scripts take significant time to run, then you may want to recompile 
\escript with more options.



Note that even if your version of \escript has support for \openmp or MPI, you will still need to tell the system to 
use it when you run your scripts.
If you are using the \texttt{run-escript} launcher, then this is controlled  by the \texttt{-t} and \texttt{-p} options.
If not, then consult the documentation for your MPI libraries (or the compiler documentation in the case of OpenMP
\footnote{It may be enough to set the \texttt{OMP\_NUM\_THREADS} environment variable.}).

If you are using MacOSX, then see the next section, if not, then skip to Section~\ref{sec:build}.

\section{MacOS}
This release of \escript has only been tested on OSX 10.9 and 10.10.
For this section we assume you are using either \texttt{homebrew} or \texttt{MacPorts} as a package 
manager\footnote{Note that package managers will make changes to your computer based on programs configured by other people from 
various places around the internet. It is important to satisfy yourself as to the security of those systems.}.
You can of course install prerequisite software in other other ways.
For example, we have had \emph{some} success changing the default 
compilers used by those systems. However this is more complicated and we do not provide a guide here.
Successful combinations of OSX and package managers are given in the table below.

\begin{center}
\begin{tabular}{|c|c|c|}\hline
 & \texttt{homebrew} & \texttt{MacPorts} \\\hline
OSX 10.9 & Yes & No\\\hline
OSX 10.10& Yes & Yes\\\hline
\end{tabular}
\end{center}

\noindent Both of those systems require the XCode command line tools to be installed\footnote{As of OSX10.9, the 
command \texttt{xcode-select --install} will allow you to download and install the commandline tools.}.

\section{Building}\label{sec:build}

To simplify things for people, we have prepared \texttt{_options.py} files for a number of 
systems\footnote{These are correct a time of writing but later versions of those systems may require tweaks. 
Also, these systems represent a cross section of possible platforms rather than meaning those systems get particular support.}.
The \texttt{_options.py} files are located in the \texttt{scons/templates} directory. We suggest that the file most relevant to your os 
be copied from the templates directory to the scons directory and renamed to the form XXXX_options.py where XXXX 
should be replaced with your computer's name.
If your particular system is not in the list below, or if you want a more customised 
build, 
see Section~\ref{sec:othersrc} for instructions.
\begin{itemize}
 \item Debian - \ref{sec:debsrc}
 \item Ubuntu - \ref{sec:ubsrc}
 \item OpenSuse - \ref{sec:susesrc}
 \item Centos - \ref{sec:centossrc}
 \item Fedora - \ref{sec:fedorasrc}
 \item MacOS (macports) - \ref{sec:macportsrc}
 \item MacOS (homebrew) - \ref{sec:homebrewsrc}
 \item FreeBSD - \ref{sec:freebsdsrc}
\end{itemize}

Once these are done proceed to Section~\ref{sec:cleanup} for cleanup steps.

All of these instructions assume that you have obtained the \escript source (and uncompressed it if necessary).
\subsection{Debian}\label{sec:debsrc}

\begin{shellCode}
sudo aptitude install python-dev python-numpy libboost-python-dev libnetcdf-dev 
sudo aptitude install scons lsb-release  libboost-random-dev
sudo aptitude install python-sympy python-matplotlib python-scipy
sudo aptitude install python-pyproj python-gdal 
\end{shellCode}

\noindent Once \textit{Jessie} is released (or if \textit{wheezy-backports} is in your \texttt{apt} sources) you can use:
\begin{shellCode}
sudo aptitude install gmsh 
\end{shellCode}
to add extra meshing functionality.

\begin{optionalstep}
If for some reason, you wish to rebuild the documentation, you would also need the following:
\begin{shellCode}
sudo aptitude install python-sphinx doxygen python-docutils texlive 
sudo aptitude install zip texlive-latex-extra latex-xcolor 
\end{shellCode}
\end{optionalstep}

\noindent In the source directory execute the following (substitute wheezy for XXXX):
\begin{shellCode}
scons -j1 options_file=scons/templates/XXXX_options.py
\end{shellCode}

\noindent If you wish to test your build, you can use the following:
\begin{shellCode}
scons -j1 py_tests options_file=scons/templates/XXXX_options.py 
\end{shellCode}

\subsection{Ubuntu}\label{sec:ubsrc}

If you have not installed \texttt{aptitude}, then substitute \texttt{apt-get} in the following.
\begin{shellCode}
sudo aptitude install python-dev python-numpy libboost-python-dev 
sudo aptitude install libnetcdf-dev libboost-random-dev
sudo aptitude install scons lsb-release
sudo aptitude install python-sympy python-matplotlib python-scipy
sudo aptitude install python-pyproj python-gdal gmsh
\end{shellCode}


\begin{optionalstep}
If for some reason, you wish to rebuild the documentation, you would also need the following:
\begin{shellCode}
sudo aptitude install python-sphinx doxygen python-docutils texlive 
sudo aptitude install zip texlive-latex-extra latex-xcolor 
\end{shellCode}
\end{optionalstep}

\noindent In the source directory execute the following (substitute precise, quantal or raring as appropriate for XXXX):
\begin{shellCode}
scons -j1 options_file=scons/templates/XXXX_options.py
\end{shellCode}

\noindent If you wish to test your build, you can use the following:
\begin{shellCode}
scons -j1 py_tests options_file=scons/templates/XXXX_options.py 
\end{shellCode}



\subsection{OpenSuse}\label{sec:susesrc}
These instructions were prepared using release $13.2$.

\noindent Install packages from the main distribution:
\begin{shellCode}
sudo zypper install libboost_python1_54_0 libboost_random1_54_0 
sudo zypper python-devel python-numpy libnetcdf_c++-devel
sudo zypper install python-scipy python-sympy python-matplotlib 
sudo zypper install gcc gcc-c++ scons boost-devel netcdf-devel
\end{shellCode}
These will allow you to use most features except some parts of the \downunder inversion library.
If you wish to use those, you will need some additional packages [python-pyproj, python-gdal].
This can be done now or after Escript installation.

\begin{shellCode}
sudo zypper addrepo \ 
 http://ftp.suse.de/pub/opensuse/repositories/Application:/Geo/openSUSE_13.2/ osgf
sudo zypper install python-pyproj python-gdal
\end{shellCode}

Now to build escript itself.
In the escript source directory:
\begin{shellCode}
scons -j1 options_file=scons/templates/opensuse13.1_options.py
\end{shellCode}

\noindent If you wish to test your build, you can use the following:
\begin{shellCode}
scons -j1 py_tests options_file=scons/templates/opensuse13.1_options.py 
\end{shellCode}

\noindent Now go to Section~\ref{sec:cleanup} for cleanup.

\subsection{Centos}\label{sec:centossrc}
These instructions were prepared using centos release $7.0$.
The core of escript works, however some functionality is not availible because the default packages for some dependencies in Centos are too old.

\noindent Install packages from the main distribution:
\begin{shellCode}
yum install python-devel numpy scipy scons boost-devel
yum install python-matplotlib gcc gcc-c++
yum install boost-python 
\end{shellCode}

The above packages will allow you to use most features except saving and loading files in \texttt{netCDF} 
format and the \downunder inversion library.
If you wish to use those features, you will need to install some additional packages.
NetCDF needs to be installed when you compile if you wish to use it.
\begin{optionalstep}
\noindent Add the \texttt{EPEL} repository.
\begin{shellCode}
yum install epel-release.noarch
\end{shellCode}

\begin{shellCode}
yum install netcdf-devel netcdf_cxx_devel gdal-python
\end{shellCode}
\end{optionalstep}

\noindent For some coordinate transformations, \downunder can also make use of the python interface to a tool called \texttt{proj}.
There does not seem to be an obvious centos repository for this though.
If it turns out to be necessary for your particular application, the source can be downloaded. 

\noindent Now to build escript itself.
In the escript source directory:
\begin{shellCode}
scons -j1 options_file=scons/templates/centos7_0_options.py
\end{shellCode}

\noindent Now go to Section~\ref{sec:cleanup} for cleanup.

\subsection{Fedora}\label{sec:fedorasrc}
These instructions were prepared using release $21.5$.

\noindent Install packages
\begin{shellCode}
yum install netcdf-cxx-devel gcc-c++ scipy 
yum install sympy scons pyproj gdal python-matplotlib 
yum install boost-devel
\end{shellCode}

\noindent Now to build escript itself.
In the escript source directory:
\begin{shellCode}
scons -j1 options_file=scons/templates/fedora21_5_options.py
\end{shellCode}

\noindent If you wish to test your build, you can use the following:
\begin{shellCode}
scons -j1 py_tests options_file=scons/templates/fedora21_5_options.py 
\end{shellCode}

\noindent Now go to Section~\ref{sec:cleanup} for cleanup.

\subsection{MacOS 10.10 (macports)}\label{sec:macportsrc}

The following will install the capabilities needed for the \texttt{macports_10.10_options.py} file.

\begin{shellCode}
sudo port install scons
sudo port select --set python python27
sudo port install boost
sudo port install py27-numpy
sudo port install py27-sympy
sudo port select --set py-sympy py27-sympy
sudo port install py27-scipy
sudo port install py27-pyproj
sudo port install py27-gdal
sudo port install netcdf-cxx
sudo port instal silo
\end{shellCode}

\begin{shellCode}
scons -j1 options_file=scons/templates/macports_10.10options.py 
\end{shellCode}


\subsection{MacOS 10.9, 10.10 (homebrew)}\label{sec:homebrewsrc}

The following will install the capabilities needed for the \texttt{homebrew_10.10_options.py} file.
OSX 10.9 can use the same file.

\begin{shellCode}
brew install scons
brew install boost-python
brew install homebrew/science/netcdf --with-cxx-compat
\end{shellCode}

There do not appear to be formulae for \texttt{sympy} or \texttt{pyproj} so if you wish to use those features, then
you will need to install them separately.


\begin{shellCode}
scons -j1 options_file=scons/templates/homebrew_10.10_options.py
\end{shellCode}


\subsection{FreeBSD}\label{sec:freebsdsrc}

At time of writing, \texttt{numpy} does not install correctly on FreeBSD.
Since \texttt{numpy} is a critical dependency for \escript, we have been unable to test on FreeBSD.

\begin{comment}
\subsubsection{Release 10.0}

Install the following packages:
\begin{itemize}
 \item python
 \item scons
 \item boost-python-libs
 \item bash
 \item netcdf
 \item silo
 \item py27-scipy
 \item py27-gdal
 \item py27-matplotlib
 \item py27-pyproj
 \item py27-sympy
\end{itemize}

\noindent Next choose (or create) your options file.
For the setup as above the escript source comes with a prepared file in
\texttt{scons/templates/freebsd10.0_options.py}.
Finally to build escript issue the following in the escript source directory
(replace the options file as required):
\begin{shellCode}
scons -j1 options_file=scons/templates/freebsd10.0_options.py
\end{shellCode}

\emph{Note:} Some packages installed above are built with gcc 4.7. Somewhere
in the toolchain a system-installed gcc library is pulled in which is
incompatible with the one from version 4.7 and would prevent escript from
executing successfully. As explained in the FreeBSD
documentation\footnote{see \url{http://www.freebsd.org/doc/en/articles/custom-gcc/article.html}}
this can be fixed by adding a line to \texttt{/etc/libmap.conf}:
\begin{shellCode}
libgcc_s.so.1 gcc47/libgcc_s.so.1
\end{shellCode}

\end{comment}
\subsection{Other Systems / Custom Builds}\label{sec:othersrc}

\escript has support for a number of optional packages.
Some, like \texttt{netcdf} need to be enabled at compile time, while others, such as \texttt{sympy} and the projection packages
used in \downunder are checked at run time.
For the second type, you can install them at any time (ensuring that python can find them) and they should work.
For the first type, you need to modify the options file and recompile with scons.
The rest of this section deals with this.

To avoid having to specify the options file each time you run scons, copy an existing \texttt{_options.py} file from the 
\texttt{scons/} or \texttt{scons/templates/} directories. Put the file in the \texttt{scons} directory and name 
it \textit{yourmachinename}\texttt{_options.py}.\footnote{If the name 
has - or other non-alpha characters, they must be replaced with underscores in the filename}.
For example: on a machine named toybox, the file would be \texttt{scons/toybox_options.py}.

Individual lines can be enabled/disabled, by removing or adding \# (the python comment character) to the beginning of the line.
For example, to enable OpenMP, change the line
\begin{verbatim}
#openmp = True 
\end{verbatim}
to
\begin{verbatim}
openmp = True 
\end{verbatim}

If you are using libraries which are not installed in the standard places (or have different names) you will need to 
change the relevant lines.
A common need for this would be using a more recent version of the boost::python library.
You can also change the compiler or the options passed to it by modifying the relevant lines.

\subsubsection*{MPI}
If you wish to enable or disable MPI, or if you wish to use a different implementation of MPI, you can use the \texttt{mpi}
configuration variable.
You will also need to ensure that the \texttt{mpi_prefix} and \texttt{mpi_libs} variables are uncommented and set correctly.
To disable MPI use, \verb|mpi = 'none'|.

\subsubsection{Python3}
\escript works with \texttt{python3} but until recently, many distributions have not distributed python3 versions of their packages.
You can try it out though by modifying or adding the following variables in your options file:

\begin{verbatim}
pythoncmd='python3'
\end{verbatim}

\begin{verbatim}
usepython3=True
\end{verbatim}

\begin{verbatim}
pythonlibname='whateveryourpython3libraryiscalled'
\end{verbatim}




\subsubsection{Testing}
As indicated earlier, you can test your build using \texttt{scons py_tests}.
Note however, that some features like \texttt{netCDF} are optional for using \escript, the tests will report a failure if
they are missing.

\section{Cleaning up}
\label{sec:cleanup}

Once the build (and optional testing) is complete, you can remove everything except:
\begin{itemize}
 \item bin
 \item esys
 \item lib
 \item doc
 \item CREDITS.TXT
 \item README_LICENSE
\end{itemize}
The last two aren't strictly required for operation.
The \texttt{doc} directory is not required either but does contain examples of escript scripts.

You can run escript using \texttt{\textit{path_to_escript_files}/bin/run-escript}.
Where \texttt{\textit{path_to_escript_files}} is replaced with the real path.

\begin{optionalstep}
You can add the escript \texttt{bin} directory to your \texttt{PATH} variable.
The launcher will then take care of the rest of the environment.
\end{optionalstep}

\section{Optional Extras}

Some other packages which might be useful include:
\begin{itemize}
 \item support for silo format (install the relevant libraries and enable them in the options file).
 \item Visit --- visualisation package. Can be used independently but our \texttt{weipa} library can make a Visit 
plug-in to allow direct visualisation of escript files.
 \item gmsh --- meshing software used by our \texttt{pycad} library.
 \item mayavi --- another visualisation tool.
\end{itemize}


%Need a better title but this is stuff like visit and silo (for non-debian distros)
%Perhaps - optional extras





