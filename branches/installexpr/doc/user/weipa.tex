
%%%%%%%%%%%%%%%%%%%%%%%%%%%%%%%%%%%%%%%%%%%%%%%%%%%%%%%%%%%%%%%%%%%%%%%%%%%%%%
% Copyright (c) 2003-2013 by University of Queensland
% http://www.uq.edu.au
%
% Primary Business: Queensland, Australia
% Licensed under the Open Software License version 3.0
% http://www.opensource.org/licenses/osl-3.0.php
%
% Development until 2012 by Earth Systems Science Computational Center (ESSCC)
% Development since 2012 by School of Earth Sciences
%
%%%%%%%%%%%%%%%%%%%%%%%%%%%%%%%%%%%%%%%%%%%%%%%%%%%%%%%%%%%%%%%%%%%%%%%%%%%%%%

\chapter{The \weipa Module and Data Visualization}\label{chap:weipa}
%\declaremodule{extension}{weipa}
%\modulesynopsis{Exporting escript data and domains for post-processing}

The {\it weipa} C++ library and accompanying \PYTHON module allow exporting
\escript \Data objects and their domain in a format suitable for visualization.
Besides creating output files, {\it weipa} can also interface with the \VisIt
visualization software. This allows accessing the latest simulation data while
the simulation is still running without the need to save any files.

\section{The \class{EscriptDataset} class}
\begin{classdesc}{EscriptDataset}{}
    holds an {\it escript} dataset including a domain and data variables
    for a single time step and offers methods to export the data in various
    formats.
    It is preferable to create a dataset object using the \var{createDataset}
    function from \weipa (see \Sec{sec:weipafuncs}) rather than using the (non-exposed)
    \PYTHON constructor for the class.
%     It is not recommended to create a dataset object directly in \PYTHON using
%     the constructor (it is not actually exposed) but use the
%     \var{createDataset} function from \weipa instead, see \Sec{sec:weipafuncs}.
\end{classdesc}

\noindent The following methods are available:
\begin{methoddesc}[EscriptDataset]{setDomain}{domain}
    sets the \Domain for this dataset. Note that the domain can only be set
    once and all \Data objects added to this dataset must be defined on the
    same domain.
\end{methoddesc}

\begin{methoddesc}[EscriptDataset]{addData}{data, name \optional{, units=""}}
    adds the \Data object \var{data} to this dataset which will be exported by
    the given \var{name}. Some export formats support data units which can be
    set through the \var{units} parameter, e.g. \code{"km/h"}.
    Before calling this method a domain must be set with \member{setDomain}
    and all \Data objects added must be defined on the same domain.
    There is no restriction, however, on the \FunctionSpace used.
\end{methoddesc}

\begin{methoddesc}[EscriptDataset]{setCycleAndTime}{cycle, time}
    sets the cycle and time values for this dataset.
    The cycle is an integer value which usually corresponds with the loop
    counter of the simulation script. That is, every time a new data file is
    created this counter is incremented.
    The value of \var{time} on the other hand is a floating point number that
    encodes some form of simulation time.
    Both, cycle and time may be read by analysis tools and shown alongside
    other metadata to the user.
\end{methoddesc}

\begin{methoddesc}[EscriptDataset]{setMeshLabels}{x, y \optional{, z=""}}
    sets the labels of the X, Y, and Z axis. By default, visualization tools
    display default strings such as "X-Axis" or "X" along the axes. Some export
    formats allow overriding these with more specific strings such as "Width",
    "Horizontal Distance", etc.
\end{methoddesc}

\begin{methoddesc}[EscriptDataset]{setMeshUnits}{x, y \optional{, z=""}}
    sets the units to be displayed along the X, Y, and Z axis in visualization
    tools (if supported). Not all export formats will use these values.
\end{methoddesc}

\begin{methoddesc}[EscriptDataset]{setMetadataSchemaString}{\optional{schema="" \optional{, metadata=""}}}
    adds custom metadata and/or XML schema strings to VTK files.
    The content of \var{schema} is added to the top-level \emph{VTKFile}
    element so care must be taken to keep the resulting file valid.
    As an example, \var{schema} may contain the string
    \code{xmlns:gml="http://www.opengis.net/gml"}. The content of \var{metadata}
    will be written enclosed in \code{<MetaData>} tags. Thus, a valid example
    would be \code{<dataSource>something</dataSource>}.
    Note that these values are ignored by other exporters.
\end{methoddesc}

\begin{methoddesc}[EscriptDataset]{saveSilo}{filename}
    saves the dataset in the \SILO file format to a file named \var{filename}.
    The file extension \code{.silo} will be automatically added if not present.
\end{methoddesc}

\begin{methoddesc}[EscriptDataset]{saveVTK}{filename}
    saves the dataset in the \VTK file format to a file named \var{filename}.
    The file extension \code{.vtu} will be automatically added if not present.
    Certain combinations of function spaces cannot be written to a single \VTK
    file due to format restrictions. In these cases this method will save
    separate files where each file contains compatible data.
    The function space name is appended to the filename to distinguish them.
\end{methoddesc}

\section{Functions}\label{sec:weipafuncs}
\begin{funcdesc}{createDataset}{domain, **data}
    creates an \class{EscriptDataset} object, sets its domain, populates it
    with the given \Data objects and returns it.
    Note that it is not possible to set units for the data variables added with
    this function. If this is required, it is recommended to call this function
    with a domain only and use the \member{addData} method subsequently.
\end{funcdesc}

\begin{funcdesc}{saveVTK}{filename \optional{, domain=None \optional{, metadata="" \optional{, metadata_schema=None}}}, **data}
    convenience function that creates a dataset with the given domain and \Data
    objects and saves it to a file in the \VTK file format.
    If \var{domain} is \code{None} the domain will be determined by the \Data
    objects.
    See the \member{setDomain}, \member{addData}, \member{saveVTK}, and
    \member{setMetadataSchemaString} methods of the \class{EscriptDataset}
    class for details.
    Unlike the class method, the \var{metadata_schema} parameter should be a
    dictionary that maps namespace name to URI, e.g.\\
    \code{\{"gml":"http://www.opengis.net/gml"\}}.
\end{funcdesc}

\begin{funcdesc}{saveSilo}{filename \optional{, domain=None}, **data}
    convenience function that creates a dataset with the given domain and \Data
    objects and saves it to a file in the \SILO file format.
    If \var{domain} is \code{None} the domain will be determined by the \Data
    objects.
    See the \member{setDomain}, \member{addData}, and \member{saveSilo}
    methods of the \class{EscriptDataset} class for details.
\end{funcdesc}

\begin{funcdesc}{visitInitialize}{simFile \optional{, comment=""}}
    initializes the \VisIt simulation interface which is responsible for the
    communication with a \VisIt client.
    This function will create a file by the name given via \var{simFile}
    (extension \code{.sim2}) which can be loaded by a compatible \VisIt client
    in order to connect to the simulation. The optional \var{comment} string
    is forwarded to the client.
    Note that this function only succeeds if {\it escript} was compiled with
    support for \VisIt and the appropriate libraries are found in the runtime
    environment. Clients wanting to connect can only do so if the version
    number matches the version number used to compile \weipa.
    Calling this function does not make any data available yet, see the
    \var{visitPublishData} function.
\end{funcdesc}

\begin{funcdesc}{visitPublishData}{dataset}
    publishes an \class{EscriptDataset} object through the \VisIt simulation
    interface, checks for client requests and handles any outstanding ones.
    Before publishing any data, the \var{visitInitialize} function must be
    called to set up the interface.
    Since this function not only publishes new data but polls for incoming
    connections and handles requests, it should be called as often as practical
    (even with the same dataset) to avoid timeout errors from clients.
    On the other hand it should be noted that the same process(es) deal with
    visualization requests that run your simulation. So a request for an
    expensive task by a \VisIt client will pause the simulation code while it
    is being processed.
\end{funcdesc}

\section{Visualizing {\it escript} Data}
This section gives a very brief overview on how data exported through \weipa
can be visualized. While there are many visualization packages available that
are compatible with \VTK and \SILO files produced by {\it escript}, this
discussion will refer to \VisIt~\cite{VisIt}, an actively maintained open
source package optimally suited to visualize and analyze large datasets both
interactively and through \PYTHON scripts. You can find a number of manuals,
a wiki page and links to mailing lists on the \VisIt website.
It is assumed that you have a working \VisIt installation that can be started
by entering \code{visit} on the command line.

The examples that follow will use the output produced by the Elastic
Deformation example from \Sec{ELASTIC CHAP} (\file{heatedblock.py} in the
\ExampleDirectory) which produces the file \file{deform.vtu}.
This \VTK file contains a 3D scalar variable called \code{stress} and a vector
variable called \code{disp}, among others.

\subsection{Using the \VisIt GUI}
Start the VisIt graphical user interface and open the file \file{deform.vtu}
via the 'File' menu. Alternatively, you can directly open the file on startup
by issuing
\begin{verbatim}
visit -o deform.vtu
\end{verbatim}
You should see the \VisIt GUI on the left hand side and an empty visualization
window on the right. Click on 'Add' under Plots in the GUI to bring up a menu
of plot types, then click on 'Pseudocolor' and select 'stress'.
This will add a plot to the list which maps values of the 'stress' variable to
colors. Note, that the plot will not be generated until you click on the 'Draw'
button in the GUI. You should now see a coloured box in the visualization
window which you can rotate around and inspect from different angles using your
mouse. The example uses a coarse mesh of 10 by 10 by 10 elements which are
clearly visible in this plot.

We can improve the visual effect by enabling interpolation between the elements.
To do so, bring up the plot attributes by double-clicking the
'Pseudocolor - stress' plot entry in the GUI.
Next, select 'Nodal' under 'Centering', click on 'Apply' and dismiss the dialog.
Notice how the colours now smoothly blend into each other and the element
boundaries are no longer visible.

Now we will add arrows to visualize the displacement vectors. Click on 'Add'
and under 'Vector' select 'disp'. Once again click on 'Draw' to execute the
new plot. By default only few vectors are shown but since the mesh is very
coarse we can tell \VisIt to draw all available vectors.
Bring up the Vector plot attributes (double-click on the plot as before) and
under 'Vector amount' select 'Stride', leaving the parameter as 1.
Click on 'Apply' and dismiss the dialog.

As a final step we would like to see inside the plot. One possibility to do so
is slicing. However, we want to keep all vectors while slicing only the
Pseudocolor plot. In \VisIt slicing is one of the Operators that may be added
to plots and by default, Operators are added to \emph{all} plots.
To change this behaviour, uncheck the 'Apply operators to all plots' box which
is located underneath the plot list in the GUI.
Then select the Pseudocolor plot, bring up the Operators menu by clicking on
'Operators' and select 'ThreeSlice' from the 'Slicing' submenu.
Again, click on 'Draw' to update the plots and notice how the box has now been
sliced. We can move the slices to more suitable positions by editing the
operator attributes. Click on the little triangle to the left of the
Pseudocolor plot to reveal the list of elements that have been applied to it.
Next, double-click the 'ThreeSlice' element to bring up the attribute window.
Change the values to $X=0.3$ and $Y=0.3$, leaving $Z=0$. Apply the changes and
dismiss the dialog to see the result.

You can now create an image of the plots as shown in the window. First, adjust
the save options to your needs in the 'Set Save options' dialog which is
accessible from the 'File' menu. Then select 'Save Window' and you should find
an image file with the name and location as entered in the options dialog.

\subsection{Using the \VisIt CLI (command line interface)}
We will now perform exactly the same steps as in the last section but using the
\PYTHON interface of \VisIt instead of the GUI. Start up the CLI by issuing
\begin{verbatim}
visit -cli
\end{verbatim}
You should now see an empty visualization window but unlike in the previous
section there will be no graphical user interface but a \PYTHON command line
instead.
Enter the following commands, one by one, noticing the changes in the
visualization window after every block of commands:
\begin{python}
  OpenDatabase("deform.vtu")
  AddPlot("Pseudocolor","stress")
  DrawPlots()

  p=PseudocolorAttributes()
  p.centering=p.Nodal
  SetPlotOptions(p)

  AddOperator("ThreeSlice")
  DrawPlots()

  t=ThreeSliceAttributes()
  t.x=0.3
  t.y=0.3
  SetOperatorOptions(t)

  AddPlot("Vector", "disp")
  DrawPlots()

  v=VectorAttributes()
  v.useStride=1
  SetPlotOptions(v)

  s=SaveWindowAttributes()
  #change settings as required
  SaveWindow()
  exit()
\end{python}
All but the last call to \code{DrawPlots()} is not required and was only put
there for demonstrating the effects of the commands.
You can save these commands to a file, e.g. \file{deformVis.py} and let \VisIt
process them non-interactively like so:
\begin{verbatim}
visit -cli -nowin -s deformVis.py
\end{verbatim}
The \code{-nowin} option prevents the visualization window from being shown
which is not required since the purpose of the script is to save an image file.

Obviously, we have barely touched on the powerful features of \VisIt and this
section was only meant to give you a minimal introduction. The \VisIt website
has a reference manual for the \PYTHON interface that explains how to perform
other operations programmatically, such as changing the view.

