
%%%%%%%%%%%%%%%%%%%%%%%%%%%%%%%%%%%%%%%%%%%%%%%%%%%%%%%%%%%%%%%%%%%%%%%%%%%%%%
% Copyright (c) 2003-2013 by University of Queensland
% http://www.uq.edu.au
%
% Primary Business: Queensland, Australia
% Licensed under the Open Software License version 3.0
% http://www.opensource.org/licenses/osl-3.0.php
%
% Development until 2012 by Earth Systems Science Computational Center (ESSCC)
% Development since 2012 by School of Earth Sciences
%
%%%%%%%%%%%%%%%%%%%%%%%%%%%%%%%%%%%%%%%%%%%%%%%%%%%%%%%%%%%%%%%%%%%%%%%%%%%%%%

\section{The Stokes Problem}
\label{STOKES PROBLEM} 
In this section we discuss how to solve the Stokes problem.
We want to calculate the velocity\index{velocity} field $v$ and pressure $p$
of an incompressible fluid\index{incompressible fluid}.
They are given as the solution of the Stokes problem\index{Stokes problem}
\begin{equation}\label{Stokes 1}
-\left(\eta(v_{i,j}+ v_{j,i})\right)_{,j}+p_{,i}=f_{i}-\sigma_{ij,j}
\end{equation}
where  $f_{i}$ defines an internal force\index{force, internal} and
$\sigma_{ij}$ is an initial stress\index{stress, initial}.
The viscosity $\eta$ may weakly depend on pressure and velocity.
If relevant we will use the notation $\eta(v,p)$ to express this dependency.

We assume an incompressible medium:
\begin{equation}\label{Stokes 2}
-v_{i,i}=0
\end{equation}
Natural boundary conditions are taken in the form 
\begin{equation}\label{Stokes Boundary}
\left(\eta(v_{i,j}+ v_{j,i})\right)n_{j}-n_{i}p=s_{i} - \alpha \cdot n_{i} n_{j} v_{j}+\sigma_{ij} n_{j}
\end{equation}
which can be overwritten by constraints of the form 
\begin{equation}\label{Stokes Boundary0}
v_{i}(x)=v^D_{i}(x)
\end{equation}
at some locations $x$ at the boundary of the domain.
$s_{i}$ defines a normal stress and $\alpha\ge 0$ the spring constant for
restoring normal force.
The index $i$ may depend on the location $x$ on the boundary.
$v^D$ is a given function on the domain.

\subsection{Solution Method \label{STOKES SOLVE}}
If we assume that $\eta$ is independent from the velocity and pressure,
equations~\ref{Stokes 1} and~\ref{Stokes 2} can be written in the block form
\begin{equation}
\left[ \begin{array}{cc}
A     & B^{*} \\
B & 0 \\
\end{array} \right]
\left[ \begin{array}{c}
v \\
p \\
\end{array} \right]
=\left[ \begin{array}{c}
G \\
0 \\
\end{array} \right]
\label{STOKES}
\end{equation}
where $A$ is a coercive, self-adjoint linear operator in a suitable Hilbert
space, $B$ is the $(-1) \cdot$ divergence operator and $B^{*}$ is the adjoint
operator (=gradient operator).
For more details on the mathematics see references \cite{AAMIRBERKYAN2008,MBENZI2005}.

If $v_{0}$ and $p_{0}$ are given initial guesses for velocity and pressure we
calculate a correction $dv$ for the velocity by solving the first equation of
\eqn{STOKES}
 \begin{equation}\label{STOKES ITER STEP 1}
 A dv_{1} = G - A v_{0} - B^{*} p_{0}
\end{equation}
We then insert the new approximation $v_{1}=v_{0}+dv_{1}$ to calculate a
correction $dp_{2}$ for the pressure and an additional correction $dv_{2}$ for
the velocity by solving
 \begin{equation}
 \begin{array}{rcl}
 B A^{-1} B^{*} dp_{2} & = & Bv_{1} \\
 A dv_{2} & = & B^{*} dp_{2} 
\end{array}
 \label{STOKES ITER STEP 2}
 \end{equation}
The new velocity and pressure are then given by $v_{2}=v_{1}-dv_{2}$ and
$p_{2}=p_{0}+dp_{2}$ which will fulfill the block system~\ref{STOKES}. 
This solution strategy is called the Uzawa scheme\index{Uzawa scheme}.

There is a problem with this scheme: in practice we will use an iterative
scheme to solve any problem for operator $A$.
So we will be unable to calculate the operator $ B A^{-1} B^{*}$ required for
$dp_{2}$ explicitly. In fact, we need to use another iterative scheme to solve
the first equation in~\ref{STOKES ITER STEP 2} where in each iteration step
an iterative solver for $A$ is applied. Another issue is the fact that the
viscosity $\eta$ may depend on velocity or pressure and so we need to iterate
over the three equations~\ref{STOKES ITER STEP 1} and~\ref{STOKES ITER STEP 2}. 

In the following we will use the two norms
\begin{equation}
\|v\|_{1}^2 = \int_{\Omega} v_{j,k}v_{j,k} \; dx 
\mbox{ and }
\|p\|_{0}^2= \int_{\Omega} p^2 \; dx.
\label{STOKES STOP}
\end{equation}
for velocity $v$ and pressure $p$.
The iteration is terminated if the stopping criterion
 \begin{equation} \label{STOKES STOPPING CRITERIA}
\max(\|Bv_{1}\|_{0},\|v_{2}-v_{0}\|_{1}) \le \tau \cdot \|v_{2}\|_{1} 
 \end{equation}
for a given tolerance $0<\tau<1$ is met.
Notice that because of the first equation of~\ref{STOKES ITER STEP 2} we have
that $\|Bv_{1}\|_{0}$ equals the norm of $B A^{-1} B^{*} dp_{2}$ and
consequently provides a norm for the pressure correction.

We want to optimize the tolerance choice for solving~\ref{STOKES ITER STEP 1}
and~\ref{STOKES ITER STEP 2}. To do this we write the iteration scheme as a
fixed point problem. Here we consider the errors produced by the iterative
solvers being used.
From \eqn{STOKES ITER STEP 1} we have
\begin{equation} \label{STOKES total V1}
v_{1} = e_{1} + v_{0} + A^{-1} ( G - Av_{0} - B^{*} p_{0} ) 
\end{equation}
where $e_{1}$ is the error when solving~\ref{STOKES ITER STEP 1}.
We will use a sparse matrix solver so we have not full control on the norm
$\|.\|_{s}$ used in the stopping criterion for this equation.
In fact we will have a stopping criterion of the form
\begin{equation} 
\| A e_{1} \|_{s}  = \| G - A v_{1} - B^{*} p_{0} \|_{s} \le \tau_{1} \| G - A v_{0} - B^{*} p_{0} \|_{s} 
\end{equation}
where $\tau_{1}$ is the tolerance which we need to choose.
This translates into the condition
\begin{equation} 
\| e_{1} \|_{1} \le K \tau_{1} \| dv_{1} - e_{1} \|_{1} 
\end{equation}
The constant $K$ represents some uncertainty combining a variety of unknown
factors such as the norm being used and the condition number of the stiffness matrix.
From the first equation of~\ref{STOKES ITER STEP 2} we have
\begin{equation}\label{STOKES total P2}
p_{2} =  p_{0} + (B A^{-1} B^{*})^{-1} (e_{2} + Bv_{1} )
\end{equation}
where $e_{2}$ represents the error when solving~\ref{STOKES ITER STEP 2}.
We use an iterative preconditioned conjugate gradient method
(PCG)\index{linear solver!PCG}\index{PCG} with iteration operator
$B A^{-1} B^{*}$ using the $\|.\|_{0}$ norm.
As suitable preconditioner\index{preconditioner} for the iteration operator we
use $\frac{1}{\eta}$ \cite{ELMAN}, i.e. the evaluation of the preconditioner
$P$ for a given pressure increment $q$ is the solution of
\begin{equation} \label{STOKES P PREC}
\frac{1}{\eta} (Pq) = q \; . 
\end{equation}
Note that in each evaluation of the iteration operator $q=B A^{-1} B^{*} s$
one needs to solve the problem
\begin{equation} \label{STOKES P OPERATOR}
A w = B^{*} s 
\end{equation}
with sufficient accuracy to return $q=Bw$. We assume that the desired
tolerance is sufficiently small, for instance one can take $\tau_{2}^2$ where
$\tau_{2}$ is the tolerance for~\ref{STOKES ITER STEP 2}.

In an implementation we use the fact that the residual $r$ is given as
\begin{equation} \label{STOKES RES }
 r= B (v_{1} -  A^{-1} B^{*} dp) =  B (v_{1} - A^{-1} B^{*} dp) = B (v_{1}-dv_{2}) = B v_{2}
\end{equation}
In particular we have $e_{2} = B v_{2}$.
So the residual $r$ is represented by the updated velocity $v_{2}=v_{1}-dv_{2}$.
In practice, if one uses the velocity to represent the residual $r$ there is
no need to recover $dv_{2}$ in~\ref{STOKES ITER STEP 2} after $dp_{2}$ has
been calculated.
In PCG the iteration is terminated if
\begin{equation} \label{STOKES P OPERATOR ERROR}
\| P^{\frac{1}{2}}B v_{2} \|_{0} \le \tau_{2} \| P^{\frac{1}{2}}B v_{1} \|_{0}
\end{equation}
where $\tau_{2}$ is the given tolerance. This translates into
\begin{equation} \label{STOKES P OPERATOR ERROR 2}
\|e_{2}\|_{0} = \| B v_{2} \|_{0} \le M \tau_{2} \| B v_{1} \|_{0}
\end{equation}
where $M$ is taking care of the fact that $P^{\frac{1}{2}}$ is dropped.

As we assume that there is no significant error from solving with the operator
$A$ we have 
\begin{equation} \label{STOKES total V2}
v_{2} =  v_{1} - dv_{2} 
= v_{1}  - A^{-1} B^{*}dp 
\end{equation}
Combining the equations~\ref{STOKES total V1},~\ref{STOKES total P2} and~\ref{STOKES total V2}
and setting the errors to zero we can write the solution process as a fix
point problem
\begin{equation} 
v = \Phi(v,p) \mbox{ and } p = \Psi(u,p) 
\end{equation}
with suitable functions $\Phi(v,p)$ and $ \Psi(v,p)$ representing the
iteration operator without errors. In fact, for a linear problem, $\Phi$ and
$\Psi$ are constant. With this notation we can write the update step in the
form $p_{2}= \delta p + \Psi(v_{0},p_{0})$ and $v_{2}= \delta v + \Phi(v_{0},p_{0})$
where the total error $\delta p$ and $\delta v$ are given as
\begin{equation} 
 \begin{array}{rcl}
\delta p & = &  (B A^{-1} B^{*})^{-1} ( e_{2} + B e_{1} ) \\
\delta v & = &  e_{1} -  A^{-1} B^{*}\delta p  \;.
\end{array}\label{STOKES ERRORS}
\end{equation}
Notice that $B\delta v = - e_{2}=-Bv_{2}$.
Our task is now to choose the tolerances $\tau_{1}$ and $\tau_{2}$ such that
the global errors $\delta p$ and $\delta v$ do not stop the convergence of the
iteration process.

To measure convergence we use
\begin{equation} 
\epsilon = \max(\|v_{2}-v\|_{1}, \|B A^{-1} B^{*} (p_{2}-p)\|_{0})
\end{equation}
In practice using the fact that $B A^{-1} B^{*} (p_{2}-p_{0}) = B v_{1}$
and assuming that $v_{2}$ gives a better approximation to the true $v$ than
$v_{0}$ we will use the estimate
\begin{equation} 
\epsilon = \max(\|v_{2}-v_{0}\|_{1}, \|B v_{1}\|_{0})
\end{equation}
to estimate the progress of the iteration step after the step is completed.
Note that the estimate of $\epsilon$ is used in the stopping
criterion~\ref{STOKES STOPPING CRITERIA}.
If $\chi^{-}$ is the convergence rate assuming exact calculations, i.e.
$e_{1}=0$ and $e_{2}=0$, we are expecting to maintain $\epsilon \le \chi^{-} \cdot \epsilon^{-}$.
For the pressure increment we get:
\begin{equation} \label{STOKES EST 1}
\begin{array}{rcl}
\|B A^{-1} B^{*} (p_{2}-p)\|_{0}
 & \le & \|B A^{-1} B^{*} (p_{2}-\delta p-p)\|_{0} +
\|B A^{-1} B^{*} \delta p\|_{0} \\
 & = & \chi^{-} \cdot \epsilon^{-} + \|e_{2} + B e_{1}\|_{0}  \\
& \approx & \chi^{-} \cdot \epsilon^{-} + \|e_{2}\|_{0} \\
& \le & \chi^{-} \cdot \epsilon^{-} + M \tau_{2} \|B v_{1}\|_{0} \\  
\end{array}
\end{equation}
So we choose the value for $\tau_{2}$ from
\begin{equation} \label{STOKES TOL2}
 M \tau_{2} \|B v_{1}\|_{0}  \le (\chi^{-})^2 \epsilon^{-}
\end{equation}
in order to make the perturbation for the termination of the pressure
iteration a second order effect. We use a similar argument for the velocity:
\begin{equation}\label{STOKES EST 2}
\begin{array}{rcl}
\|v_{2}-v\|_{1} & \le & \|v_{2}-\delta v-v\|_{1} + \| \delta v\|_{1} \\
& \le &  \chi^{-} \cdot \epsilon^{-}  + \| e_{1} -  A^{-1} B^{*}\delta p \|_{1} \\
& \approx &  \chi^{-} \cdot \epsilon^{-}  + \| e_{1} \|_{1} \\
& \le &  \chi^{-} \cdot \epsilon^{-}  +  K \tau_{1} \| dv_{1} - e_{1} \|_{1}
\\
& \le &  ( 1  + K \tau_{1}) \chi^{-} \cdot \epsilon^{-}
\end{array}
\end{equation}
So we choose the value for $\tau_{1}$ from
\begin{equation} \label{STOKES TOL1}
K \tau_{1} \le \chi^{-}
\end{equation}
Assuming we have estimates for $M$ and $K$\footnote{if no estimates are
available, we use the value $1$} we can use~\ref{STOKES TOL1} and
\ref{STOKES TOL2} to get appropriate values for the tolerances.
After the step has been completed we can calculate a new convergence rate
$\chi =\frac{\epsilon}{\epsilon^{-}}$.
For partial reasons we restrict $\chi$ to be less or equal a given maximum
value $\chi_{max}\le 1$.
If we see $\chi \le \chi^{-} (1+\chi^{-})$ our choices for the tolerances were
suitable. Otherwise, we need to adjust the values for $K$ and $M$.
From the estimates~\ref{STOKES EST 1} and~\ref{STOKES EST 2} we establish
\begin{equation}\label{STOKES EST 3}
\chi \le ( 1 + \max(M \frac{\tau_{2} \|B v_{1}\|_{0}}{\chi^{-} \epsilon^{-}},K \tau_{1}  ) ) \cdot \chi^{-} 
\end{equation}
If we assume that this inequality would be an equation if we would have chosen
the right values $M^{+}$ and $K^{+}$ then we get
\begin{equation}\label{STOKES EST 3b}
\chi =  ( 1 + \max(M^{+} \frac{\chi^{-}}{M},K^{+} \frac{\chi^{-}}{K}) ) \cdot \chi^{-} 
\end{equation}
From this equation we see if our choice for $K$ was not good enough.
In this case we can calculate a new value
 \begin{equation}
K^{+} =  \frac{\chi-\chi^{-}}{(\chi^{-})^2} K
\end{equation}
In practice we will use
 \begin{equation}
K^{+}  = \max(\frac{\chi-\chi^{-}}{(\chi^{-})^2} K,\frac{1}{2}K,1)
\end{equation}
where the second term is used to reduce a potential overestimate of $K$.
The same identity is used for to update $M$. The updated $M^{+}$ and $K^{+}$
are then use in the next iteration step to control the tolerances.

In some cases one can observe that there is a significant change in the
velocity but the new velocity $v_{1}$ has still a small divergence, i.e.
we have $\|Bv_{1}\|_{0} \ll \|v_{1}-v_{0}\|_{1}$.
In this case we will get a small pressure increment and consequently only very
small changes to the velocity as a result of the second update step which
therefore can be skipped and we can directly repeat the first update step
until the increment in velocity becomes significant relative to its divergence.
In practice we will ignore the second half of the iteration step as long as
 \begin{equation}\label{STOKES LARGE BV1}
\|Bv_{1}\|_{0} \le \theta \cdot \|v_{1}-v_{0}\| 
\end{equation}
where $0<\theta<1$ is a given factor. In this case we will also check the
stopping criterion with $v_{1}\rightarrow v_{2}$ but we will not correct $M$
in this case.

Starting from an initial guess $v_{0}$ and $p_{0}$ for velocity and pressure
the solution procedure is implemented as follows:
\begin{enumerate}
 \item calculate viscosity $\eta(v_{0},p)_{0}$ and assemble operator $A$ from $\eta$
 \item calculate the tolerance $\tau_{1}$ from \eqn{STOKES TOL1}
 \item solve \eqn{STOKES ITER STEP 1} for $dv_{1}$ with tolerance $\tau_{1}$
 \item update $v_{1}= v_{0}+ dv_{1}$
 \item if $Bv_{1}$ is large (see~\ref{STOKES LARGE BV1}):
 \begin{enumerate}
   \item calculate the tolerance $\tau_{2}$ from~\ref{STOKES TOL2}
   \item solve~\ref{STOKES ITER STEP 2} for $dp_{2}$ and $v_{2}$ with tolerance $\tau_{2}$
   \item update $p_{2}\leftarrow p_{0}+ dp_{2}$
 \end{enumerate}
 \item else:
  \begin{itemize}
    \item update $p_{2}\leftarrow p$ and $v_{2}\leftarrow v_{1}$
  \end{itemize}
 \item calculate convergence measure $\epsilon$ and convergence rate $\chi$
 \item if stopping criterion~\ref{STOKES STOPPING CRITERIA} holds:
 \begin{itemize}
   \item return $v_{2}$ and $p_{2}$
 \end{itemize}
 \item else:
 \begin{enumerate}
   \item update $M$ and $K$
   \item goto step 1 with $v_{0}\leftarrow v_{2}$ and $p_{0}\leftarrow p_{2}$.
 \end{enumerate}
\end{enumerate}

\subsection{Functions}

\begin{classdesc}{StokesProblemCartesian}{domain}
opens the Stokes problem\index{Stokes problem} on the \Domain domain.
The domain needs to support LBB compliant elements for the Stokes problem, see~\cite{LBB} for details\index{LBB condition}.
For instance one can use second order polynomials for velocity and first order
polynomials for the pressure on the same element.
Alternatively, one can use macro elements\index{macro elements} using linear
polynomials for both pressure and velocity with a subdivided element for the
velocity. Typically, the macro element is more cost effective.
The fact that pressure and velocity are represented in different ways is
expressed by
\begin{python}
  velocity=Vector(0.0, Solution(mesh))
  pressure=Scalar(0.0, ReducedSolution(mesh))
\end{python}
\end{classdesc}

\begin{methoddesc}[StokesProblemCartesian]{initialize}{\optional{f=Data(),
    \optional{fixed_u_mask=Data(), \optional{eta=1,%
    \optional{surface_stress=Data(), \optional{stress=Data()},%
    \optional{restoration_factor=0}}}}}}
assigns values to the model parameters. In any call all values must be set.
\var{f} defines the external force $f$, \var{eta} the viscosity $\eta$,
\var{surface_stress} the surface stress $s$ and \var{stress} the initial stress $\sigma$.
The locations and components where the velocity is fixed are set by the values
of \var{fixed_u_mask}.
\var{restoration_factor} defines the restoring force factor $\alpha$.
The method will try to cast the given values to appropriate \Data class objects.
\end{methoddesc}

\begin{methoddesc}[StokesProblemCartesian]{solve}{v,p
\optional{, max_iter=100 \optional{, verbose=False \optional{, usePCG=True }}}}
solves the problem and returns approximations for velocity and pressure. 
The arguments \var{v} and \var{p} define initial guesses.
\var{v} must have function space \var{Solution(domain)} and \var{p} must have
function space \var{ReducedSolution(domain)}.
The values of \var{v} marked by \var{fixed_u_mask} remain unchanged.
If \var{usePCG} is set to \True then the preconditioned conjugate gradient
method (PCG)\index{preconditioned conjugate gradient method!PCG} scheme is
used. Otherwise the problem is solved with the generalized minimal residual
method (GMRES)\index{generalized minimal residual method!GMRES}.
In most cases the PCG scheme is more efficient.
\var{max_iter} defines the maximum number of iteration steps.
If \var{verbose} is set to \True information on the progress of of the solver
is printed.
\end{methoddesc}

\begin{methoddesc}[StokesProblemCartesian]{setTolerance}{\optional{tolerance=1.e-4}}
sets the tolerance in an appropriate norm relative to the right hand side.
The tolerance must be non-negative and less than 1.
\end{methoddesc}

\begin{methoddesc}[StokesProblemCartesian]{getTolerance}{}
returns the current relative tolerance.
\end{methoddesc}

\begin{methoddesc}[StokesProblemCartesian]{setAbsoluteTolerance}{\optional{tolerance=0.}}
sets the absolute tolerance for the error in the relevant norm.
The tolerance must be non-negative.
Typically the absolute tolerance is set to 0.
\end{methoddesc}

\begin{methoddesc}[StokesProblemCartesian]{getAbsoluteTolerance}{}
returns the current absolute tolerance.
\end{methoddesc}

\begin{methoddesc}[StokesProblemCartesian]{getSolverOptionsVelocity}{}
returns the solver options used to solve \eqn{STOKES ITER STEP 1} and \eqn{STOKES P OPERATOR}) for velocity.
\end{methoddesc}

\begin{methoddesc}[StokesProblemCartesian]{getSolverOptionsPressure}{}
returns the solver options used to solve the preconditioner \eqn{STOKES P PREC} for pressure.
\end{methoddesc}

\begin{methoddesc}[StokesProblemCartesian]{getSolverOptionsDiv}{}
sets the solver options for solving the equation to project the divergence of
the velocity onto the function space of pressure.
\end{methoddesc}

\begin{methoddesc}[StokesProblemCartesian]{setStokesEquation}{\optional{f=None,
    \optional{fixed_u_mask=None, \optional{eta=None,%
    \optional{surface_stress=None, \optional{stress=None},%
    \optional{restoration_factor=None}}}}}}
assigns new values to the model parameters, see method \function{initialize} for description of the 
parameter list. In contrast to \function{initialize} only values given in the argument list are set. 
Typically this method is called to update parameters such as viscosity $\eta$ within a time integration scheme
after the problem has been set up by an initial call of method \function{initialize}.
\end{methoddesc}

\begin{methoddesc}[StokesProblemCartesian]{updateStokesEquation}{v, p}
this method is called by the solver to allow for updating the model parameter during the iteration process for 
incompressibility. In order to implement a non-linear dependence of model parameters such as viscosity $\eta$
from the current velocity \var{v} or pressure field \var{p} this function can be overwritten in the following way:
\begin{python}
    class MyStokesProblem(StokesProblemCartesian):
       def updateStokesEquation(self, v, p):
           my_eta=<eta derived from v and p>
           self.setStokesEquation(eta=my_eta)
\end{python}
Note that \function{setStokesEquation} to update the model. \warning{It is not guaranteed that the iteration converges if model parameters are altered.}
\end{methoddesc}

\subsection{Example: Lid-driven Cavity}
The following script \file{lid_driven_cavity.py}\index{scripts!\file{lid_driven_cavity.py}}
which is available in the \ExampleDirectory illustrates the usage of the
\class{StokesProblemCartesian} class to solve the lid-driven cavity problem:
\begin{python}
  from esys.escript import *
  from esys.finley import Rectangle
  from esys.weipa import saveVTK
  from esys.escript.models import StokesProblemCartesian
  NE=25
  dom = Rectangle(NE,NE,order=2)
  x = dom.getX()
  sc=StokesProblemCartesian(dom)
  mask= (whereZero(x[0])*[1.,0]+whereZero(x[0]-1))*[1.,0] + \
        (whereZero(x[1])*[0.,1.]+whereZero(x[1]-1))*[1.,1]
  sc.initialize(eta=.1, fixed_u_mask=mask)
  v=Vector(0., Solution(dom))
  v[0]+=whereZero(x[1]-1.)
  p=Scalar(0.,ReducedSolution(dom))
  v,p=sc.solve(v, p, verbose=True)
  saveVTK("u.vtu", velocity=v, pressure=p)
\end{python}

