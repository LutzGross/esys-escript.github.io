%%%%%%%%%%%%%%%%%%%%%%%%%%%%%%%%%%%%%%%%%%%%%%%%%%%%%%%%%%%%%%%%%%%%%%%%%%%%%%
% Copyright (c) 2003-2013 by University of Queensland
% http://www.uq.edu.au
%
% Primary Business: Queensland, Australia
% Licensed under the Open Software License version 3.0
% http://www.opensource.org/licenses/osl-3.0.php
%
% Development until 2012 by Earth Systems Science Computational Center (ESSCC)
% Development since 2012 by School of Earth Sciences
%
%%%%%%%%%%%%%%%%%%%%%%%%%%%%%%%%%%%%%%%%%%%%%%%%%%%%%%%%%%%%%%%%%%%%%%%%%%%%%%

\chapter{Introduction}
This document describes how to install \emph{esys-Escript}\footnote{For the rest of the document we will drop the \emph{esys-}} on to your computer.
To learn how to use \esfinley please see the Cookbook, User's guide or the API documentation.
If you use the Debian or Ubuntu and you have installed the \texttt{python-escript-doc} package then the documentation 
will be available in the directory\\
\file{/usr/share/doc/python-escript-doc}, otherwise (if you haven't done so already) you can download the documentation bundle 
from launchpad.



\esfinley is primarily developed on Linux desktop, SGI ICE and \macosx systems.
It can be installed in two ways:
\begin{enumerate}
  \item Binary packages -- ready to run with no compilation required.
      Bundles are available for:
      \begin{itemize}
	  \item .deb files for Debian and Ubuntu Linux
% 	  \item $32$bit Windows (requires some other packages to be installed).
      \end{itemize}
    Please see Appendix~\ref{chap:winstall} for Windows instructions.
    The rest of this guide assumes you are using a posix like system.
  \item From source -- that is, it must be compiled for your machine.
  This will be required if there is no binary package 
    for your machine or if extra functionality is required such as \mpi parallelisation.
\end{enumerate}

\emph{The major change from the point of view of installation is that we no longer provide binary packages compiled
against the ``support bundles''.
This means, there are no longer binary packages for MacOS or generic Linux.
}
One can still build from source and we have endevoured to make this process as straight forward as possible.

See the site \url{https://answers.launchpad.net/escript-finley} for online help.
Chapter~\ref{chap:bin} describes how to install binary packages on Debian/Ubuntu systems.
Chapter~\ref{chap:source} covers installing from source.
Appendix~\ref{chap:winstall} gives brief instructions for Windows.


% \section{Significant changes since version 3.3}
% 
% \begin{itemize}
%  \item The minimum Python version is now $2.6$.   
% This means among other things that you can no longer build escript using the system Python from OSX Leopard.
%  \item New [optional] symbolic support requires SymPy.   
%  \item This release contains the \module{downunder} inversion module. Certain operations in this module \emph{may} require either 
% \module{pyproj}\footnote{python-pyproj in Debian and Ubuntu} or \module{gdal}\footnote{python-gdal in Debian and Ubuntu}.
% Escript will warn you if you try to do something which requires either package.
% \end{itemize}

