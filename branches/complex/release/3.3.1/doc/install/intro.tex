%%%%%%%%%%%%%%%%%%%%%%%%%%%%%%%%%%%%%%%%%%%%%%%%%%%%%%%%%%%%%%%%%%%%%%%%%%%%%%
% Copyright (c) 2003-2013 by University of Queensland
% http://www.uq.edu.au
%
% Primary Business: Queensland, Australia
% Licensed under the Open Software License version 3.0
% http://www.opensource.org/licenses/osl-3.0.php
%
% Development until 2012 by Earth Systems Science Computational Center (ESSCC)
% Development since 2012 by School of Earth Sciences
%
%%%%%%%%%%%%%%%%%%%%%%%%%%%%%%%%%%%%%%%%%%%%%%%%%%%%%%%%%%%%%%%%%%%%%%%%%%%%%%

\chapter{Introduction}
This document describes how to install \emph{esys-Escript}\footnote{For the rest of the document we will drop the \emph{esys-}} on your computer.
To learn how to use \esfinley please see the Cookbook, User's guide or the API documentation.
If you use the Debian or Ubuntu packages to install then the documentation will be available in
\file{/usr/share/doc/escript}, otherwise (if you haven't done so already) you can download the documentation bundle 
from launchpad.

\esfinley is primarily developed on Linux desktop, SGI ICE and \macosx systems.
It is distributed in two forms:
\begin{enumerate}
  \item Binary bundles -- these are great for first time users or for those who want to start using 
    \esfinley immediately.
      Bundles are available for:
      \begin{itemize}
	  \item Debian and Ubuntu Linux distributions ($32$/$64$-bit i686) (.deb package)
	  \item Linux desktop systems with gcc (stand-alone bundle)
	  \item \macosx Leopard systems (also tested on Lion) with gcc (stand-alone bundle)
	  \item $32$bit Windows (requires some other packages to be installed).
      \end{itemize}    
    Please see Chapter~\ref{chap:bin} for instructions on how to install the binary bundles \esfinley.
  \item Source bundles -- these require compilation and should be used if the binary bundles 
    don't work on the target machine or if extra functionality is required such as \mpi parallelisation.
    See Chapter~\ref{chap:compiler} for detailed instructions.
\end{enumerate}

See the site \url{https://answers.launchpad.net/escript-finley} for online help.

\section{Significant changes since version 3.3}
\begin{itemize}
 \item The minimum Python version is now $2.6$.   
This means among other things that you can no longer build escript using the system Python from OSX Leopard.
 \item New [optional] symbolic support requires SymPy.   \\- This is not in the support bundle, so these features will be unavailable in the
stand-alone bundle unless you install SymPy yourself.
 \item This release contains the \module{downunder} inversion module. Certain operations in this module \emph{may} require either 
\module{pyproj}\footnote{python-pyproj in Debian and Ubuntu} or \module{gdal}\footnote{python-gdal in Debian and Ubuntu}.
Escript will warn you if you try to do something which requires either package.

Please note however, that these packages are not included in the support bundle so if you do require them you will need to install
them yourself.
\end{itemize}

% \noindent If you choose to compile from source your options are to
% \begin{itemize}
%     \item install dependencies (e.g. using your package manager) and only compile \esfinley, OR
%     \item compile everything from source.
% \end{itemize}
% Either way, please see Chapter~\ref{chap:compiler} for a discussion of compiler features.
% Compiling \esfinley when its dependencies are already installed is discussed in Chapter~\ref{chap:essrc}.
% To compile \esfinley and all dependencies from source please see Chapter~\ref{chap:allsrc}.
% The latter option takes a significant amount of time and is only required if the versions of the dependent libraries available on your system do not work with \esfinley.
% 
% Once everything is installed you can test your installation using the Python scripts in \file{examples.zip} or \file{examples.tar.gz}\footnote{These should either be in \file{escript.d/release/doc} or in the case of Debian, in \file{/usr/share/doc/escript}.}.
% Unpack the examples and try to run the following from a terminal:
% \begin{shellCode}
%  run-escript poission.py
% \end{shellCode}
% If this produces a VTK file called \file{u.vtu} then you are likely to have a functional \esfinley installation.
% You can try and visualize the VTK data or delete the file.
% For visualization we suggest using \file{VisIt}\footnote{\url{https://wci.llnl.gov/codes/visit/}} or \file{MayaVi}\footnote{\url{http://mayavi.sourceforge.net}} which are both freely available.





