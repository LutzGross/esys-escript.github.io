\chapter{Geodetic Coordinates}\label{sec:geodetic}

The geodetic coordinates are more appropriate for describing near surface data of the Earth treating the 
the Earth as an ellipsoid. The location of a point 
is described by its geodetic latitude $\phi$, longitude $\lambda$ and geodetic height $h$ where we assume
\begin{equation}
  -\frac{\pi}{2} \le \phi \le \frac{\pi}{2} \mbox{ and } -\pi \le \lambda \le \pi \;.
\end{equation} 
In the following we refer to the $(\phi, \lambda, h)$ as the Geodetic Coordinate system\index{geodetic coordinates}.
The Cartesian coordinates $(x_0,x_1,x_2)$ of a point are given as 
\begin{equation}
\begin{array}{rcll}
   x_0 & = &  (N + h) & \cdot \cos(\phi) \cdot  \cos(\lambda) \\
   x_1 & = &  (N + h) & \cdot  \cos(\phi) \cdot  \sin(\lambda) \\
   x_2 & = &  (N \cdot (1-e^2) + h ) & \cdot  \sin(\phi)\\
\end{array}
\label{equ:geodetic:1}
\end{equation} 
where $N$ is given as 
\begin{equation}
 N = \frac{a}{\sqrt{1- e^2 \cdot \sin^2(\phi) }}
\label{equ:geodetic:2}
\end{equation}
with the semi major axis length $a$ and $b$ ($ a \ge b$), and the eccentricity 
\begin{equation}
e = \sqrt{2f - f^2} \mbox{ with flattening } f = 1-\frac{b}{a} \ge 0
\label{equ:geodetic:3}
\end{equation}
Notice that the surface of the ellipsoid (Earth) is described by the case $h=0$.  The following 
table shows values for  flattening semi-major axis for the major reference systems of the Earth:
\begin{center}
\begin{tabular}{c|lll}
Ellipsoid reference & Semi-major axis a & Semi-minor axis b & Inverse flattening (1/f)\\
\hline
GRS 80 & 6 378 137.0 m & 6 356 752.314 140 m & 298.257 222 101\\
WGS 84 & 6 378 137.0 m & 6 356 752.314 245 m & 298.257 223 563
\end{tabular}
\end{center}
We need to translate any object from the Cartesian coordinates $(x_i)$ in terms of Geodetic Coordinate system
$(\phi, \lambda, h)$. In the following indexes running through $(\phi, \lambda, h)$ are denoted by little Greek letters $\alpha$, 
$\beta$
to separate them from indexes running through the components of the Cartesian coordinates system in little Latin letters.
With this convection the derivative of a function $g$ with respect to the Geodetic coordinates which is given as
\begin{equation}
\begin{array}{rcl}
  g_{,\phi} & =  & g_{,0} \cdot x_{0,\phi} + g_{,1} \cdot x_{1,\phi} + g_{,2} \cdot x_{2,\phi} \\
  g_{,\lambda} & =  & g_{,0} \cdot x_{0,\lambda} + g_{,1} \cdot x_{1,\lambda} + g_{,2} \cdot x_{2,\lambda} \\  
  g_{,h} & =  & g_{,0} \cdot x_{0,h} + g_{,1} \cdot x_{1,h} + g_{,2} \cdot x_{2,h} \\
\end{array}
\end{equation}
via chain rule can be written in the compact form 
\begin{equation}
  g_{,\alpha}   =    g_{,i} \cdot x_{i,\alpha} 
\end{equation}

\begin{equation}
 x_{i,\alpha}
= 
\left[
\begin{array}{ccc}
-(M + h) \cdot \sin(\phi) \cdot  \cos(\lambda) & -(M + h) \cdot  \sin(\phi) \cdot  \sin(\lambda) & (M  + h )  \cdot  \cos(\phi) \\
 - (N + h) \cdot \cos(\phi) \cdot  \sin(\lambda) & (N + h)  \cdot  \cos(\phi) \cdot  \cos(\lambda) & 0 \\
 \cos(\phi) \cdot  \cos(\lambda)  & \cos(\phi) \cdot  \sin(\lambda) &   \sin(\phi) \\
\end{array}
\right]
\end{equation}
with 
\begin{equation}
 M = \frac{a \cdot  (1-e^2) }{(1- e^2 \cdot \sin^2(\phi))^{\frac{3}{2}}}
\label{equ:geodetic:5}
\end{equation}
With the coordinate vectors $(u_{\alpha})$ defined as 
\begin{equation}
 u_{\alpha i}
= 
\left[
\begin{array}{ccc}
-\sin(\phi) \cdot  \cos(\lambda)  & -  \sin(\lambda) &  \cos(\phi) \cdot  \cos(\lambda)  \\
- \sin(\phi) \cdot  \sin(\lambda)  &  \cos(\lambda)  &  \cos(\phi) \cdot  \sin(\lambda)  \\
\cos(\phi)                       &   0               &   \sin(\phi)  \\
\end{array}
\right]
\end{equation}
and scaling factors 
\begin{equation}
s_{\phi \phi} = (M + h) \; , 
s_{\lambda \lambda} =  (N + h) \cdot \cos(\phi) \mbox{ and }
s_{h h} = 1
\end{equation}
we get 
\begin{equation}
 x_{i,\alpha} =  u_{\alpha i} s_{\alpha \alpha} \mbox{ and }   g_{,\alpha}   =    g_{,i} u_{\alpha i} s_{\alpha \alpha} 
\end{equation}
With the fact that 
\begin{equation}
 u_{\alpha i} u_{\alpha j} = \delta_{ij} \mbox{ and }  u_{\alpha i} u_{\beta i} = \delta_{\alpha \beta}
\end{equation}
\begin{equation}
g_{,i} = \frac{1}{ s_{\alpha \alpha}} g_{,\alpha}  u_{\alpha i} 
\end{equation}
or 
\begin{equation}
g_{,i} = \frac{1}{M + h} g_{,\phi}  u_{\phi i} + 
\frac{1}{(N + h) \cdot \cos(\phi) } g_{,\lambda}  u_{\lambda i} +
g_{,h}  u_{h i} 
\end{equation} 
Moreover for integrals we get by substitution rule 
\begin{equation}
dx_0 \; dx_1  \;  dx_2 =\det((x_{i,\alpha}))  \;  d \phi  \;   d\lambda   \;  dh 
=  (M + h) \cdot (N + h) \cdot \cos(\phi)  \;  d \phi  \;  d\lambda  \;  dh 
\end{equation} 
Notice that for a spherical Earth $e=0$ for which $M=N$ is the radius of the Earth and $M+h$ is the distance from
the center.  