%%%%%%%%%%%%%%%%%%%%%%%%%%%%%%%%%%%%%%%%%%%%%%%%%%%%%%%%
%
% Copyright (c) 2003-2010 by University of Queensland
% Earth Systems Science Computational Center (ESSCC)
% http://www.uq.edu.au/esscc
%
% Primary Business: Queensland, Australia
% Licensed under the Open Software License version 3.0
% http://www.opensource.org/licenses/osl-3.0.php
%
%%%%%%%%%%%%%%%%%%%%%%%%%%%%%%%%%%%%%%%%%%%%%%%%%%%%%%%%

\section{\macosx binary installation}
\label{sec:binmac}

\esfinley can be installed as a stand-alone bundle, containing all the required tools.

For more information on using the \file{run-escript} command please see the User Guide.

\subsection{Stand-alone bundle \macosx 10.5 (``Leopard'')}

You will need to download both escript (\file{escript_3.2_osx.dmg}) and the support files (\file{escript-support_3.0_osx.dmg}).
This point release uses the same support bundle as previous releases so if you already have it you don't need a new version.
\begin{itemize}
\item Create a folder to hold escript (no spaces in the name please).
\item Open the \file{.dmg} files and copy the contents to the folder you just created.
\end{itemize}

To use escript, open a terminal\footnote{If you do not know how to open a terminal on Mac, then just type \texttt{terminal} in the spotlight (search tool on the top of the right corner) and once found, just click on it.} and type
\begin{shellCode}
x/escript.d/bin/run-escript
\end{shellCode}
where \textit{x} is the absolute path to your install.

If you wish to save on typing you can add \file{x/escript.d/bin} to your PATH variable (where \textit{x} is the absolute path to your install). 


The previous point release (3.1) installed sucessfully on \macosx 10.6.2 (``Snow Leopard'') but we have not tested this one. 

