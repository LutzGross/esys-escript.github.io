
%%%%%%%%%%%%%%%%%%%%%%%%%%%%%%%%%%%%%%%%%%%%%%%%%%%%%%%%
%
% Copyright (c) 2003-2010 by University of Queensland
% Earth Systems Science Computational Center (ESSCC)
% http://www.uq.edu.au/esscc
%
% Primary Business: Queensland, Australia
% Licensed under the Open Software License version 3.0
% http://www.opensource.org/licenses/osl-3.0.php
%
%%%%%%%%%%%%%%%%%%%%%%%%%%%%%%%%%%%%%%%%%%%%%%%%%%%%%%%%

\section{Darcy Flux}
\label{DARCY FLUX}
We want to calculate the velocity $u$ and pressure $p$ on a domain $\Omega$
solving the Darcy flux problem\index{Darcy flux}\index{Darcy flow}
\begin{equation}\label{DARCY PROBLEM}
\begin{array}{rcl}
u_{i} + \kappa_{ij} p_{,j} & = & g_{i} \\
u_{k,k} & = & f
\end{array}
\end{equation} 
with the boundary conditions
\begin{equation}\label{DARCY BOUNDARY}
\begin{array}{rcl}
u_{i} \; n_{i}  = u^{N}_{i}  \; n_{i} & \mbox{ on } & \Gamma_{N} \\
p = p^{D} &  \mbox{ on } & \Gamma_{D} \\ 
\end{array}
\end{equation} 
where $\Gamma_{N}$ and $\Gamma_{D}$ are a partition of the boundary of
$\Omega$ with $\Gamma_{D}$ non-empty, $n_{i}$ is the outer normal field of the
boundary of $\Omega$, $u^{N}_{i}$ and $p^{D}$ are given functions on $\Omega$,
$g_{i}$ and $f$ are given source terms and $\kappa_{ij}$ is the given
permeability.
We assume that $\kappa_{ij}$ is symmetric (which is not really required) and
positive definite, i.e. there are positive constants $\alpha_{0}$ and
$\alpha_{1}$ which are independent from the location in $\Omega$ such that
\begin{equation}
\alpha_{0} \; x_{i} x_{i} \le \kappa_{ij} x_{i} x_{j} \le \alpha_{1} \; x_{i} x_{i}
\end{equation}
for all $x_{i}$.


\subsection{Solution Method \label{DARCY SOLVE}}
Unfortunate equation~\ref{DARCY PROBLEM} can not solved directly in an easy way and requires mixed FEM.  
We consider a few options to solve equation~\ref{DARCY PROBLEM}
\subsubsection{Simple Solver}\label{SEC DARCY SIMPLE}
The first equation of equation~\ref{DARCY PROBLEM} is inserted into the second one:
\begin{equation}\label{DARCY PROBLEM SIMPLE}
- (\kappa_{ij} p_{,j})_{,i}  =  f  - (g_{i})_{,i}
\end{equation} 
with boundary conditions
\begin{equation}\label{DARCY BOUNDARY SIMPLE}
\begin{array}{rcl}
\kappa_{ij} p_{,j} \; n_{i}  = ( g_{i} - u^{N}_{i} )  \; n_{i} & \mbox{ on } & \Gamma_{N} \\
p = p^{D} &  \mbox{ on } & \Gamma_{D} \\ 
\end{array}
\end{equation} 
Then the flux field is recovred by solving
\begin{equation}\label{DARCY PROBLEM SIMPLE FLUX}
 u_{j} = g_j -  \kappa_{ij} p_{,j}  
\end{equation} 
with boundary conditions
\begin{equation}\label{DARCY SIMPLE BOUNDARY FLUX}
u_{i} = u^{N}_{i}  \mbox{ on }  \Gamma_{N}
\end{equation} 

\subsubsection{Global Postprocessing \label{SEC DARCY POST}}
An improved flux recovery can be achieved by solving a modified version of equation~\ref{DARCY PROBLEM SIMPLE FLUX}
adding the the gradient of the divergence of the flux:
\begin{equation}\label{DARCY PROBLEM POST FLUX}
\kappa^{-1}_{ij} u_{j} - 
(\lambda \cdot u_{k,k} )_{,i}= 
\kappa^{-1}_{ij} g_j- p_{,i} 
- (\lambda \cdot f )_{,i} 
\end{equation} 
where
\begin{equation}\label{DARCY PROBLEM POST FLUX A}
\lambda = \omega \cdot |\kappa^{-1}| \cdot vol(\Omega)^{1/d} \cdot h 
\end{equation} 
with a non-negative factor $\omega$, $d$ is the spatial dimension and $h$ is the local element size.
\begin{equation}\label{DARCY PROBLEM POST FLUX BOUNDARY}
\begin{array}{rcl}
u_{i} \; n_{i}  = u^{N}_{i}  \; n_{i} & \mbox{ on } & \Gamma_{N} \\
u_{k,k} = f & \mbox{ on } & \Gamma_{D} \\ 
\end{array}
\end{equation}   
Notice that the second condition is a natural boundray condition.

\subsubsection{Stabilization \label{SEC DARCY STAB}}
Stabilization term can be added to the original problem~\ref{DARCY PROBLEM SIMPLE FLUX}, see~\cite{MasudHughes2002a}.
Firstly we subtract half of the first equation from the first eqaution
\begin{equation}\label{DARCY STAB PROBLEM A}
\kappa^{-1}_{ij} u_{j} + (p)_{,i} - \frac{1}{2} \left(  \kappa^{-1}_{ij} u_{j} + p_{,i} \right) = \kappa^{-1}_{ij} g_j -\frac{1}{2} \kappa^{-1}_{ij} g_j 
\end{equation} 
and then subtract half of the dievregence of the first equation from the second equation:
\begin{equation}\label{DARCY STAB PROBLEM}
\begin{array}{rcl}
\frac{1}{2} \kappa^{-1}_{ij} u_{j} + (p)_{,i} -  \frac{1}{2}  p_{,i}  & = &  \frac{1}{2} \kappa^{-1}_{ij} g_j \\
u_{i,i} - \frac{1}{2}  \left(u_{i} +   \kappa_{ij} p_{,j} \right)_{,i} & = & f - \frac{1}{2} \left(g_{i} \right)_{,i}   
\end{array}
\end{equation} 
We apply the boundary conditions:
\begin{equation}\label{DARCY SYM STAB PROBLEM BOUNDARY}
\begin{array}{rcl}
u_{i} \; n_{i}  = u^{N}_{i}  \; n_{i} & \mbox{ on } & \Gamma_{N} \\
p = p^{D} &  \mbox{ on } & \Gamma_{D} \\ 
\frac{1}{2} \left( u_{i} +   \kappa^{-1}_{ij} p_{,j} - g_{i} \right ) \; n_{i} = 0 &  \mbox{ on } & \Gamma_{N} \\
\end{array}
\end{equation} 
Notice that due to the term $(p)_{,i}$ boundary condition $p = p^{D}$ acts as a natural boundary condition for the
first equation of~\ref{DARCY STAB PROBLEM} (in the form $p  \;  n_i = p^{D}  \;  n_i$ while the last two boundary conditions act as constraint and natural boundary conditions for the second equation.

\subsubsection{Symmetric Stabilization \label{SEC DARCY SYM STAB}}
Symmetric stabilization term can be added to the original problem~\ref{DARCY PROBLEM SIMPLE FLUX}, see~\cite{LoulaCorrea2006a}. In this approach divergence of the first equation
is subtracted from $2 \times$ the second equation  
\begin{equation}\label{DARCY SYM STAB PROBLEM A}
- (u_{i})_{,i} + \frac{1}{2} \left( u_{i} + \kappa_{ij} p_{,j} \right)_{,i} = -  f + \frac{1}{2} \left( g_{i}\right)_{,i} 
\end{equation} 
The first equation is rescaled by the factor $\frac{1}{2}$ to maintain symmetry. This leads to
\begin{equation}\label{DARCY SYM STAB PROBLEM}
\begin{array}{rcl}
\frac{1}{2}   \kappa^{-1}_{ij} u_{j} + \frac{1}{2}  p_{,i}& = &  \frac{1}{2}  \kappa^{-1}_{ij} g_{j} \\
\frac{1}{2}  \left(  \kappa_{ij} p_{,j} - u_{i} \right)_{,i} & = & -  f + \frac{1}{2}  \left(g_{i}\right)_{,i} 
\end{array}
\end{equation} 
We add the boundary conditions 
\begin{equation}\label{DARCY SYM STAB PROBLEM BOUNDARY}
\begin{array}{rcl}
u_{i} \; n_{i}  = u^{N}_{i}  \; n_{i} & \mbox{ on } & \Gamma_{N} \\
p = p^{D} &  \mbox{ on } & \Gamma_{D} \\ 
\frac{1}{2} \left( \kappa_{ij} p_{,j} - g_{i} - u_{i} \right ) \; n_{i}
=  - u^{N}_{i}  \; n_{i} & \mbox{ on } & \Gamma_{N} \\
\end{array}
\end{equation} 
Notice that the last contion provides natural boundary conditons for the last eqaution of~\ref{DARCY SYM STAB PROBLEM}.



\subsection{Functions}
\begin{classdesc}{DarcyFlow}{domain, \optional{w=1., \optional{solver=\member{DarcyFlow.SYMSTAB},  \optional{
useReduced=\True, \optional{ verbose=\True} } }}}
opens the Darcy flux problem\index{Darcy flux} on the \Domain domain. 
Reduced approximations for pressure and flux are used if \var{useReduced} is set.
Argument \var{solver} defines the solver method. 
If \var{verbose} is set some information are printed.
\var{w} defines the weighting factor $\omega$ for global postprocessing pf the flux (see equation~\ref{DARCY PROBLEM POST FLUX A}.)
\end{classdesc}

\begin{memberdesc}[DarcyFlow]{SIMPLE}
simple solver, see section~\ref{SEC DARCY SIMPLE}.
\end{memberdesc}

\begin{memberdesc}[DarcyFlow]{POST}
solver using global postprocessing of flux, see section~\ref{SEC DARCY POST}. 
\end{memberdesc}

\begin{memberdesc}[DarcyFlow]{STAB}
 solver uses (non-symmetric) stabilization, see section~\ref{SEC DARCY STAB}. 
\end{memberdesc}

\begin{memberdesc}[DarcyFlow]{SYMSTAB}
 solver uses symmetric stabilization, see section~\ref{SEC DARCY SYM STAB}.   
\end{memberdesc}

\begin{methoddesc}[DarcyFlow]{setValue}{\optional{f=None, \optional{g=None, \optional{location_of_fixed_pressure=None, \optional{location_of_fixed_flux=None, 
\\\optional{permeability=None}}}}}}
assigns values to the model parameters. Values can be assigned using various
calls -- in particular in a time dependent problem only values that change
over time need to be reset. The permeability can be defined as a scalar
(isotropic), or a symmetrix matrix (anisotropic).
\var{f} and \var{g} are the corresponding parameters in~\ref{DARCY PROBLEM}.
The locations and components where the flux is prescribed are set by positive
values in \var{location_of_fixed_flux}.
The locations where the pressure is prescribed are set by by positive values
of \var{location_of_fixed_pressure}.
The values of the pressure and flux are defined by the initial guess.
Notice that at any point on the boundary of the domain the pressure or the
normal component of the flux must be defined. There must be at least one point
where the pressure is prescribed.
The method will try to cast the given values to appropriate \Data class objects.
\end{methoddesc}

\begin{methoddesc}[DarcyFlow]{setTolerance}{\optional{rtol=1e-4}}
sets the relative tolerance \var{rtol} for the pressure in the stabilized solvers.
\end{methoddesc}


\begin{methoddesc}[DarcyFlow]{getSolverOptionsFlux}{}
returns the solver options used to solve the flux problems.
Use this \SolverOptions object to control the solution algorithms.
\end{methoddesc}

\begin{methoddesc}[DarcyFlow]{getSolverOptionsPressure}{}
returns a \SolverOptions object with the options used to solve the pressure
problems.
Use this object to control the solution algorithms.
\end{methoddesc}

\begin{methoddesc}[DarcyFlow]{solve}{u0,p0, \optional{ iter_restart=20, \optional{max_iter=100}}}
solves the problem and returns approximations for the flux $v$ and the pressure $p$.
\var{u0} and \var{p0} define initial guesses for flux and pressure.
Values marked by positive values \var{location_of_fixed_flux} and
\var{location_of_fixed_pressure}, respectively, are kept unchanged.
\var{max_iter} sets the maximum number of outer iteration steps allowed for solving the stabilzed 
problems. \var{iter_restart} sets the number of iteration after which the 
iteration is restarted\footnote{the larger \var{iter_restart} the more memory is required. The smaller
\var{iter_restart} the more iteration steps are required.}.
\end{methoddesc}

\begin{methoddesc}[DarcyFlow]{getFlux}{p, \optional{ u0 = None}}
returns the flux for a given pressure \var{p} where the flux is equal to \var{u0}
on locations where \var{location_of_fixed_flux} is positive,  see \member{setValue}.
Notice that \var{g} and \var{f} are used.
\end{methoddesc}



%\subsection{Example: Gravity Flow}
%later

