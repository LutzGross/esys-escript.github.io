%%%%%%%%%%%%%%%%%%%%%%%%%%%%%%%%%%%%%%%%%%%%%%%%%%%%%%%%
%
% Copyright (c) 2003-2009 by University of Queensland
% Earth Systems Science Computational Center (ESSCC)
% http://www.uq.edu.au/esscc
%
% Primary Business: Queensland, Australia
% Licensed under the Open Software License version 3.0
% http://www.opensource.org/licenses/osl-3.0.php
%
%%%%%%%%%%%%%%%%%%%%%%%%%%%%%%%%%%%%%%%%%%%%%%%%%%%%%%%%

\chapter{Introduction}
This document describes how to install \esfinley on your computer.
To learn how to use \esfinley please see the User's guide or, for more detailed
information, the API documentation.

\esfinley is primarily developed on Linux desktop, SGI ICE and \macosx systems.
It is distributed in two forms:
\begin{enumerate}
    \item Binary bundles -- these are great for first time users or for those who want to start using \esfinley immediately
    \item Source bundles -- these require compilation and should be used if the binary bundles don't work on the target machine or if extra functionality is required such as \mpi parallelisation.
\end{enumerate}

The binary bundles are currently available for the following platforms:
\begin{itemize}
    \item Debian and Ubuntu Linux distributions ($32$-bit i686) (.deb package)
    \item Linux desktop systems with gcc (stand-alone bundle)
    \item \macosx Leopard systems with gcc (stand-alone bundle)
\end{itemize}

Hopefully, a Windows version(stand-alone) of this release will be available soon.

See \Chap{chap:bin} for instructions on how to set these up and run \esfinley.
If you choose to compile from source your options are to
\begin{itemize}
    \item install dependencies (e.g. using your package manager) and only compile \esfinley, OR
    \item compile everything from source.
\end{itemize}

Compiling \esfinley when its dependencies are already installed is discussed in \Chap{chap:essrc}.
To compile \esfinley and all dependencies from source please see \Chap{chap:allsrc}.
The latter option takes a significant amount of time and is only required if the versions of the dependent libraries available on your system do not work with \esfinley.

Once everything is installed you can test your installation using the Python scripts in \filename{examples.zip} or \filename{examples.tar.gz}\footnote{These should either be in \filename{escript.d/release/doc} or in the case of Debian, in \filename{/usr/share/doc/escript}.}.
Unpack the examples and try to run the following from a terminal:
\begin{shellCode}
 escript poission.py
\end{shellCode}
If this produces a VTK file called \filename{u.vtu} then you are likely to have a functional \esfinley installation.
You can try and visualize the VTK data or delete the file.
For visualization we suggest using \filename{VisIt}\footnote{\url{https://wci.llnl.gov/codes/visit/}} or \filename{MayaVi}\footnote{\url{http://mayavi.sourceforge.net}} which are both freely available.

See the site \url{https://answers.launchpad.net/escript-finley} for online help.
