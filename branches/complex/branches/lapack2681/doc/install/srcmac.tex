%%%%%%%%%%%%%%%%%%%%%%%%%%%%%%%%%%%%%%%%%%%%%%%%%%%%%%%%
%
% Copyright (c) 2003-2009 by University of Queensland
% Earth Systems Science Computational Center (ESSCC)
% http://www.uq.edu.au/esscc
%
% Primary Business: Queensland, Australia
% Licensed under the Open Software License version 3.0
% http://www.opensource.org/licenses/osl-3.0.php
%
%%%%%%%%%%%%%%%%%%%%%%%%%%%%%%%%%%%%%%%%%%%%%%%%%%%%%%%%

\section{Installing from source for \macosx}
\label{sec:srcmac}

Before you start installing from source you will need \macosx development tools installed on your Mac.
This will ensure that you have the following available:
\begin{itemize}
\item \filename{g++} and associated tools.
\item \filename{make}
\end{itemize}

Here are the instructions on how to install these.
\begin{enumerate}
\item Insert the \macosx 10.5 (Leopard) DVD
\item Double-click on XcodeTools.mpkg, located inside Optional Installs/Xcode Tools
\item Follow the instructions in the Installer
\item Authenticate as the administrative user (the first user you create when setting up \macosx has administrator privileges by default)
\end{enumerate}

You will also need a copy of the \esfinley source code.
If you retrieved the source using subversion, don't forget that one can use the export command instead of checkout to get a smaller copy.
For additional visualization functionality see \Sec{sec:addfunc}.

These instructions will produce the following directory structure:
\begin{itemize}
 \item[] \filename{stand}: \begin{itemize}
  \item[] \filename{escript.d}
  \item[] \filename{pkg}
  \item[] \filename{pkg_src}
  \item[] \filename{build}
  \item[] \filename{doc}
 \end{itemize}
\end{itemize}

The following instructions assume you are running the \filename{bash} shell.
Comments are indicated with \# characters. 

Open a terminal~\footnote{If you do not know how to open a terminal on Mac, then just type terminal in the spotlight (search tool on the top of the right corner) and once found just click on it.} and type

\begin{shellCode}
mkdir stand
cd stand
export PKG_ROOT=`pwd`/pkg
\end{shellCode}

Copy compressed source bundles into \filename{stand/package_src}.
Copy documentation files into \filename{doc}.

\begin{shellCode}
mkdir packages
mkdir build
cd build
tar -jxf ../pkg_src/Python-2.6.2.tar.bz2
tar -jxf ../pkg_src/boost_1_39_0.tar.bz2
tar -zxf ../pkg_src/scons-1.2.0.tar.gz
tar -zxf ../pkg_src/numpy-1.3.0.tar.gz
tar -zxf ../pkg_src/netcdf-4.0.tar.gz
tar -zxf ../pkg_src/matplotlib-0.98.5.3.tar.gz
\end{shellCode}

\begin{itemize}

\item Build python:
\begin{shellCode}
cd Python*
./configure --prefix=$PKG_ROOT/python-2.6.2 --enable-shared 2>&1 \
  | tee tt.configure.out
make 
make install 2>&1 | tee tt.make.out

cd ..

export PATH=$PKG_ROOT/python/bin:$PATH
export PYTHONHOME=$PKG_ROOT/python
export LD_LIBRARY_PATH=$PKG_ROOT/python/lib:$LD_LIBRARY_PATH

pushd ../pkg
ln -s python-2.6.2/ python
popd
\end{shellCode}

Run the new python executable to make sure it works.

\item Now build NumPy:
\begin{shellCode}
cd numpy-1.3.0
python setup.py build
python setup.py install --prefix $PKG_ROOT/numpy-1.3.0
cd ..
pushd ../pkg
ln -s numpy-1.3.0 numpy
popd
export PYTHONPATH=$PKG_ROOT/numpy/lib/python2.6/site-packages:$PYTHONPATH
\end{shellCode}

\item Next build scons:
\begin{shellCode}
cd scons-1.2.0
python setup.py install --prefix=$PKG_ROOT/scons-1.2.0

export PATH=$PKG_ROOT/scons/bin:$PATH
cd ..
pushd ../pkg
ln -s scons-1.2.0 scons
popd
\end{shellCode}

\item The Boost libraries...:
\begin{shellCode}
pushd ../pkg
mkdir boost_1_39_0
ln -s boost_1_39_0 boost
popd
cd boost_1_39_0
./bootstrap.sh --with-libraries=python --prefix=$PKG_ROOT/boost
./bjam
./bjam install --prefix=$PKG_ROOT/boost --libdir=$PKG_ROOT/boost/lib
export LD_LIBRARY_PATH=$PKG_ROOT/boost/lib:$LD_LIBRARY_PATH
cd ..
pushd ../pkg/boost/lib/
ln -s libboost_python*-1_39.dylib libboost_python.dylib
popd
\end{shellCode}

\item ...and NetCDF:
\begin{shellCode}
cd netcdf-4.0
CFLAGS="-O2 -fPIC -Df2cFortran" CXXFLAGS="-O2 -fPIC -Df2cFortran" \
FFLAGS="-O2 -fPIC -Df2cFortran" FCFLAGS="-O2 -fPIC -Df2cFortran" \
./configure --prefix=$PKG_ROOT/netcdf-4.0

make 
make install

export LD_LIBRARY_PATH=$PKG_ROOT/netcdf/lib:$LD_LIBRARY_PATH
cd ..
pushd ../pkg
ln -s netcdf-4.0 netcdf
popd
\end{shellCode}

\item Finally matplotlib:
\begin{shellCode}
cd matplotlib-0.98.5.3
python setup.py build
python setup.py install --prefix=$PKG_ROOT/matplotlib-0.98.5.3
cd ..
pushd ../pkg
ln -s matplotlib-0.98.5.3 matplotlib
popd
cd ..
\end{shellCode}
\end{itemize}

\subsection{Compiling escript}\label{sec:compileescriptmac}

Change to the directory containing your escript source (\filename{stand/escript.d}), then:

\begin{shellCode}
cd escript.d/scons
cp mac_standalone_options_example.py YourMachineName_options.py

echo $PKG_ROOT
\end{shellCode}

Edit the options file and put the value of PKG_ROOT between the quotes in the PKG_ROOT= line.
For example:
\begin{shellCode}
PKG_ROOT="/home/bob/stand/pkg"
\end{shellCode}

\begin{shellCode}
cd ../bin
\end{shellCode}

Modify the STANDALONE line of \filename{escript} to read:
 
STANDALONE=1

Start a new terminal and go to the \filename{stand} directory.

\begin{shellCode}
export PATH=$(pwd)/pkg/scons/bin:$PATH
cd escript.d
eval $(bin/escript -e)
scons
\end{shellCode}

If you wish to test your build, then you can do the following. 
Note this may take a while if you have a slow processor and/or less than 1GB of RAM.
\begin{shellCode}
scons all_tests
\end{shellCode}

Once you are satisfied, the \filename{build} and \filename{\$PKG_ROOT/build} directories can be removed.
Within the \filename{packages} directory, the \filename{scons}, \filename{scons-1.2.0}, \filename{cmake-2.6.3} and \filename{cmake} entries can also be removed.
If you are not redistributing this bundle you can remove \filename{\$PKG_ROOT/package_src}.

If you do not plan to edit or recompile the source you can remove it.
The only entries which are required in \filename{escript.d} are:
\begin{itemize}
 \item \filename{bin}
 \item \filename{esys}
 \item \filename{include}
 \item \filename{lib}
 \item \filename{README_LICENSE}
\end{itemize}

Hidden files can be removed with
\begin{shellCode}
find . -name '.?*' | xargs rm -rf
\end{shellCode}

%Mac ports removed until Artak has a chance to update it

% \section{Installing from source for \macosx using Macports}
% \label{sec:srcmacports}
% 
% As we mentioned in \Sec{sec:srcmac}, before you start installing from source you will need \macosx development tools installed on your Mac. 
% If you do not have Macports already, please install Macports from \url{www.macports.org}. You can also install porticus (GUI for Macports).
%  
% Once you have Macports working install boost using porticus or from the terminal 
% \begin{shellCode}
% sudo port install boost@1.35.0_2+complete
% \end{shellCode}
% Sometimes this fails due to unknown reasons, but to overcome this problem you need to run
% \begin{shellCode}
% sudo port clean boost@1.35.0_2+complete
% 
% sudo port install boost@1.35.0_2+complete
%  \end{shellCode}
%   
% Download scons source scons-0.98.5.tar.gz from \url{www.scons.org}.
% \begin{shellCode}
% tar xfz scons-0.98.5.tar.gz
% cd scons-0.98.5
% python setup.py install
% \end{shellCode}
% 
% Note: Do not try to install scons using porticus or \texttt{sudo port install scons}, because it automatically installs another python version and you are likely run into problems with different python versions.  
%  
% Download numarray-1.5.2.tar.gz from \\
% \url{http://www.stsci.edu/resources/software_hardware/numarray/numarray.html}. 
% \begin{shellCode}
% tar xfz numarray-1.5.2.tar.gz
% cd numarray-1.5.2
% python setup.py install --gencode  2>&1 | tee tt.install.out
% \end{shellCode}
% 
% You can run a test to check the numarray installation by
% \begin{shellCode}
% python
% import numarray.testall as testall
% testall.test()
% \end{shellCode}
%  
% Install netcdf, gsl and fltk using Macports 
% \begin{shellCode}
% sudo port install netcdf
% sudo port install gsl
% sudo port intall fltk
% \end{shellCode}
% Note: If this fails, download and install from sources. 
%  
% Downlaod gmsh-2.2.3-source.tar and install from sources.
% \begin{shellCode}
% ./configure --with-gsl-prefix=/opt/local/  --with-fltk    --prefix=/usr/local/
% \end{shellCode}
% Note: if you install using porticus or sudo port  it automaticall installs in \filename{/opt/local/}, but if you install from sources it installs in /usr/local. So, make sure these paths are right.
% \begin{shellCode}
% sudo make -j2
% sudo make install
% \end{shellCode} 
%  
% Download and install Mesa-7.0.3 (required for VTK) from sources
% \begin{shellCode}
% tar xjf MesaDemos-7.0.3.tar.bz2
% tar xjf MesaGLUT-7.0.3.tar.bz2
% tar xjf MesaLib-7.0.3.tar.bz2
% cd Mesa-7.0.3
%  
% sudo make -j 2
% make install
% \end{shellCode} 
%  
% % Install vtk-5.0.4 from source. If you install from ports it won't configure to use shared libraries.
% % Once you untar it you will have (assume user is john) /Users/john/Downloads/VTK folder, then run the following:
% %  
% % \begin{shellCode}
% % sudo mkdir /usr/local/VTKBuild/
% % cd /usr/local/VTKBuild/
% % sudo ccmake /Users/john/Downloads/VTK/ 
% % # It will create CMakeCache.txt. Make sure  you use the following configurations.
% %  
% % #       Advanced options			ON
% % #       BUILD_EXAMPLES				ON
% % #       BUILD_SHARED_LIBS			ON
% % #       VTK_WRAP_PYTHON				ON
% % #       CMAKE_VERBOSE_MAKEFILE			TRUE
% % #       VTK_OPENGL_HAS_OSMESA			ON
% % #       VTK_USE_MANGLED_MESA			OFF
% % #       VTK_USE_OFFSCREEN			OFF
% %  
% % sudo make -j2
% % sudo make install
% % \end{shellCode}
% %  
% % Note: you need to set the following ENV variables in your \filename{/Users/john/.profile} for VTK to work:
% % \begin{shellCode}
% % export LD_LIBRARY_PATH = /usr/local/VTKBuild/bin: \
% %              /usr/local/VTKBuild/bin:${LD_LIBRARY_PATH}
% % export PYTHONPATH = /usr/local/VTKBuild/Wrapping/Python:\ 
% %             /usr/local/VTKBuild/bin:${PYTHONPATH}
% %      
% % #     For testing you can run:
% % python
% % import vtk
% % \end{shellCode}
%  
% All that's left to install is triangle and netpbm (required for ppmtompeg) using Macports:
% \begin{shellCode}
% sudo port install triangle
% sudo port install netpbm
% \end{shellCode}

