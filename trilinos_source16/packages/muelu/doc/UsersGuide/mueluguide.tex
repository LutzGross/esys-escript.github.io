%
% $Id: SANDExampleReportNotstrict.tex,v 1.26 2009-05-01 20:59:19 rolf Exp $
%
% This is an example LaTeX file which uses the SANDreport class file.
% It shows how a SAND report should be formatted, what sections and
% elements it should contain, and how to use the SANDreport class.
% It uses the LaTeX report class, but not the strict option.
%
% Get the latest version of the class file and more at
%    http://www.cs.sandia.gov/~rolf/SANDreport
%
% This file and the SANDreport.cls file are based on information
% contained in "Guide to Preparing {SAND} Reports", Sand98-0730, edited
% by Tamara K. Locke, and the newer "Guide to Preparing SAND Reports and
% Other Communication Products", SAND2002-2068P.
% Please send corrections and suggestions for improvements to
% Rolf Riesen, Org. 9223, MS 1110, rolf@cs.sandia.gov
%
\documentclass[pdf,12pt,report]{SANDreport}
\usepackage{algpseudocode}
\usepackage{amsthm}
\usepackage{booktabs}
\usepackage{calc}
\usepackage{color}
\usepackage[table]{xcolor}
\usepackage{eso-pic}
\usepackage{fancyhdr}
\usepackage{float}
\usepackage{ifthen}
\usepackage{indentfirst}
\usepackage{geometry}
\usepackage{graphicx}
\usepackage[colorlinks, bookmarksopen, %pagebackref=true, backref=page,
             linkcolor={blue},
             anchorcolor={black},
             citecolor={blue},
             filecolor={magenta},
             menucolor={blue},
             pagecolor={red},
             plainpages=false,pdfpagelabels,
             pdfauthor={Luc Berger-Vergiat, Christian A. Glusa, Jonathan J. Hu, Andrey Prokopenko, Christopher M. Siefert, Raymond S. Tuminaro, Tobias A. Wiesner},
             pdftitle={MueLu User's Guide},
             pdfkeywords={MueLu,AMG,multigrid,guide,user},
             urlcolor={blue}]{hyperref}
\usepackage{listings}
\usepackage{mathptmx}	% Use the Postscript Times font
\usepackage{multirow}
\usepackage{pifont}
\usepackage[FIGBOTCAP,normal,bf,tight]{subfigure}
\usepackage{tabularx}
\usepackage{verbatim}
\usepackage{xspace}
\usepackage{flowchart} % also loads tikz
\usepackage{algorithm}
\usetikzlibrary{arrows}

%\usepackage{draftwatermark}
%\SetWatermarkScale{.5}

\algrenewcommand{\algorithmiccomment}[1]{\hskip3em // #1}
\newcommand{\monthWord}{\ifcase \month \or January\or February\or March\or April\or May%
\or June\or July\or August\or September\or October\or November\or December\fi}


% If you want to relax some of the SAND98-0730 requirements, use the "relax"
% option. It adds spaces and boldface in the table of contents, and does not
% force the page layout sizes.
% e.g. \documentclass[relax,12pt]{SANDreport}
%
% You can also use the "strict" option, which applies even more of the
% SAND98-0730 guidelines. It gets rid of section numbers which are often
% useful; e.g. \documentclass[strict]{SANDreport}



% ---------------------------------------------------------------------------- %
%
% Set the title, author, and date
%
\title{MueLu User's Guide}


\author{
  Luc Berger-Vergiat\\
  Computational Mathematics\\
  Sandia National Laboratories\\
  Mailstop 1320, P.O.~Box 5800 \\
  Albuquerque, NM 87185-1320\\
  lberge@sandia.gov
  \and
  Christian A. Glusa \\
  Scalable Algorithms\\
  Sandia National Laboratories\\
  Mailstop 1318, P.O.~Box 5800 \\
  Albuquerque, NM 87185-1318\\
  caglusa@sandia.gov
  \and
  Graham Harper\\
  Computational Mathematics\\
  Sandia National Laboratories\\
  Mailstop 1320, P.O.~Box 5800 \\
  Albuquerque, NM 87185-1320\\
  gbharpe@sandia.gov
  \and
  Jonathan J. Hu \\
  Scalable Algorithms \\
  Sandia National Laboratories\\
  Mailstop 9060, P.O.~Box 0969 \\
  Livermore, CA 94551-0969\\
  jhu@sandia.gov
  \and
  Matthias Mayr \\
  Institute for Mathematics \\
  and Computer-Based Simulation\\
  University of the Bundeswehr Munich\\
  Werner-Heisenberg-Weg 39\\
  85577 Neubiberg, Germany\\
  matthias.mayr@unibw.de
  \and
  Peter Ohm\\
  Complex Phenomena Unified Simulation Research Team \\
  RIKEN Center for Computational Science \\
  7 Chome-1-26 Minatojima Minamimachi, \\
  Chuo Ward, Kobe, Hyogo 650-0047, Japan \\
  peter.ohm@riken.jp
  \and
  Andrey Prokopenko \\
  Oak Ridge National Laboratory\\
  P.O.~Box 2008\\
  Bldg 5700, MS 6164\\
  Oak Ridge, TN 37831\\
  \and
  Christopher M. Siefert\\
  Scalable Algorithms\\
  Sandia National Laboratories\\
  Mailstop 1322, P.O.~Box 5800 \\
  Albuquerque, NM 87185-1322\\
  csiefer@sandia.gov
  \and
  Raymond S. Tuminaro\\
  Computational Mathematics\\
  Sandia National Laboratories\\
  Mailstop 9060, P.O.~Box 0969 \\
  Livermore, CA 94551-0969\\
  rstumin@sandia.gov
  \and
  Tobias Wiesner \\
  Leica Geosystems AG\\
  Heinrich-Wild-Strasse 201\\
  9435 Heerbrugg, Switzerland\\
  tobias.wiesner@leica-geosystems.com
}

% There is a "Printed" date on the title page of a SAND report, so
% the generic \date should generally be empty.
\date{}

\newcommand{\amesos}       {\textsc{Amesos}\xspace}
\newcommand{\amesostwo}    {\textsc{Amesos2}\xspace}
\newcommand{\anasazi}      {\textsc{Anasazi}\xspace}
\newcommand{\aztecoo}      {\textsc{AztecOO}\xspace}
\newcommand{\belos}        {\textsc{Belos}\xspace}
\newcommand{\epetra}       {\textsc{Epetra}\xspace}
\newcommand{\epetraext}    {\textsc{EpetraExt}\xspace}
\newcommand{\galeri}       {\textsc{Galeri}\xspace}
\newcommand{\ifpack}       {\textsc{Ifpack}\xspace}
\newcommand{\ifpacktwo}    {\textsc{Ifpack2}\xspace}
\newcommand{\isorropia}    {\textsc{Isorropia}\xspace}
\newcommand{\loca}         {\textsc{Loca}\xspace}
\newcommand{\ml}           {\textsc{ML}\xspace}
\newcommand{\muelu}        {\textsc{MueLu}\xspace}
\newcommand{\nox}          {\textsc{NOX}\xspace}
\newcommand{\stratimikos}  {\textsc{Stratimikos}\xspace}
\newcommand{\teuchos}      {\textsc{Teuchos}\xspace}
\newcommand{\teko}         {\textsc{Teko}\xspace}
\newcommand{\thyra}        {\textsc{Thyra}\xspace}
\newcommand{\tpetra}       {\textsc{Tpetra}\xspace}
\newcommand{\trilinos}     {\textsc{Trilinos}\xspace}
\newcommand{\xpetra}       {\textsc{Xpetra}\xspace}
\newcommand{\zoltan}       {\textsc{Zoltan}\xspace}
\newcommand{\zoltantwo}    {\textsc{Zoltan2}\xspace}


\newcommand{\klu}          {\textsc{Klu}\xspace}
\newcommand{\metis}        {\textsc{Metis}\xspace}
\newcommand{\mumps}        {\textsc{Mumps}\xspace}
\newcommand{\umfpack}      {\textsc{Umfpack}\xspace}
\newcommand{\superlu}      {\textsc{SuperLU}\xspace}
\newcommand{\superludist}  {\textsc{SuperLU\_dist}\xspace}
\newcommand{\parmetis}     {\textsc{ParMetis}\xspace}
\newcommand{\paraview}     {\textsc{ParaView}\xspace}

\newcommand{\parameterlist}{\texttt{ParameterList}\xspace}

\newcommand \trilinosWeb   {trilinos.sandia.gov\xspace}

%\newcommand{\be}  {\begin{enumerate}}
%\newcommand{\ee}  {\end{enumerate}}
%\newcommand{\cba}[3]{\choicebox{\texttt{#1}}{[{\texttt #2}] #3}}
%\newcommand{\cbb}[4]{\choicebox{\texttt{#1}}{[{\texttt #2}] #4 {\bf Default:~}#3.}}
%\newcommand{\cbc}[4]{\choicebox{\texttt{\color{red}#1}}{[{\texttt #2}] #4 {\bf Default:~}#3.}}
%
%\newcommand{\comm}[2]{\bigskip
%                      \begin{tabular}{|p{4.5in}|}\hline
%                      \multicolumn{1}{|c|}{{\bf Comment by #1}}\\ \hline
%                      #2\\ \hline
%                      \end{tabular}\\
%                      \bigskip
%                     }


\newtheorem*{mycomment}{\ding{42}}
\newtheoremstyle{plain}
  {\topsep}   % ABOVESPACE
  {\topsep}   % BELOWSPACE
  {\normalfont}  % BODYFONT
  {0pt}       % INDENT (empty value is the same as 0pt)
  {\bfseries} % HEADFONT
  {}         % HEADPUNCT
  {5pt plus 1pt minus 1pt} % HEADSPACE
  {}          % CUSTOM-HEAD-SPEC

% further declarations and additional commands
\definecolor{hellgelb}{rgb}{1,1,0.8}   % background color for C++ listings
\definecolor{darkgreen}{rgb}{0.0, 0.2, 0.13}
%\definecolor{hellrot}{HTML}{FFA4C2}    % background color for xml files
\definecolor{SANDgreen}{RGB}{163, 213, 199}

% settings for listings package
\lstset{
  backgroundcolor=\color{hellgelb},
  basicstyle=\ttfamily\small,
  breakautoindent=true,
  breaklines=true,
  captionpos=b,
  columns=flexible,
  commentstyle=\color{darkgreen},
  extendedchars=true,
  float=hbp,
  frame=single,
  identifierstyle=\color{black},
  keywordstyle=\color{blue},
  numbers=none,
  numberstyle=\tiny,
  showspaces=false,
  showstringspaces=false,
  stringstyle=\color{purple},
  tabsize=2,
}


% ---------------------------------------------------------------------------- %
% Set some things we need for SAND reports. These are mandatory
%
\SANDnum{SAND2023-12265}
\SANDprintDate{February 2023}
\SANDauthor{Luc Berger-Vergiat, Christian A. Glusa, Graham Harper, Jonathan J. Hu, Matthias Mayr, Peter Ohm, Andrey Prokopenko, Christopher M. Siefert, Raymond S. Tuminaro, Tobias A. Wiesner}


% ---------------------------------------------------------------------------- %
% Include the markings required for your SAND report. The default is "Unlimited
% Release". You may have to edit the file included here, or create your own
% (see the examples provided).
%
% \include{MarkUR} % Not needed for unlimted release reports


% ---------------------------------------------------------------------------- %
% The following definition does not have a default value and will not
% print anything, if not defined
%
%\SANDsupersed{SAND1901-0001}{January 1901}
%\input{MarkOUO}


% ---------------------------------------------------------------------------- %
%
% Start the document
%
\begin{document}
    \maketitle

    % ------------------------------------------------------------------------ %
    % An Abstract is required for SAND reports
    %
    \begin{abstract}
	%This is the definitive user guide for the \muelu{} library in Trilinos version XX.YY.
%\muelu{} is a C++ multigrid framework that can work with either the \epetra or \tpetra linear
%algebra libraries.
%\muelu{} provides a variety of aggregation-based multigrid algorithms,
%including smoothed aggregation algebraic multigrid (AMG), Petrov-Galerkin AMG, and AMG for
%Maxwell's equations, as well as many different types of smoothers.
%\muelu{} is templated on the index, scalar, and compute node types.
%Thus it is possible to use \muelu{} on problems with scalar types other than double, on very
%large problems, and to exploit node-level parallelism.

This is the official user guide for \muelu{} multigrid library in Trilinos
version~\input{version}. This guide provides an overview of \muelu, its capabilities, and
instructions for new users who want to start using \muelu{} with a minimum of
effort. Detailed information is given on how to drive \muelu{} through its XML
interface. Links to more advanced use cases are given. This guide gives
information on how to achieve good parallel performance, as well as how to
introduce new algorithms. Finally, readers will find a comprehensive listing of
available \muelu{} options.  {\em Any options not documented in this manual
should be considered strictly experimental.}

%%% Local Variables:
%%% mode: latex
%%% TeX-master: "mueluguide"
%%% End:

    \end{abstract}


    % ------------------------------------------------------------------------ %
    % An Acknowledgement section is optional but important, if someone made
    % contributions or helped beyond the normal part of a work assignment.
    % Use \section* since we don't want it in the table of context
    %
    \clearpage
    \chapter*{Acknowledgment}
	Many people have helped develop \muelu{} and/or provided valuable feedback, and
we would like to acknowledge their contributions here: Tom Benson, Julian
Cortial, Eric Cyr, Stefan Domino, Travis Fisher, Jeremie Gaidamour, Axel
Gerstenberger, Chetan Jhurani, Mark Hoemmen, Paul Lin, Eric Phipps, Siva
Rajamanickam, Nico Schl{\"o}mer, and Paul Tsuji.

%%% Local Variables:
%%% mode: latex
%%% TeX-master: "mueluguide"
%%% End:



    % ------------------------------------------------------------------------ %
    % The table of contents and list of figures and tables
    % Comment out \listoffigures and \listoftables if there are no
    % figures or tables. Make sure this starts on an odd numbered page
    %
    \cleardoublepage		% TOC needs to start on an odd page
    \tableofcontents
    \listoffigures
    \listoftables


    % ---------------------------------------------------------------------- %
    % An optional preface or Foreword
    %\clearpage
    %\chapter*{Preface}
    %\addcontentsline{toc}{chapter}{Preface}
	%\input{CommonPreface}


    % ---------------------------------------------------------------------- %
    % An optional executive summary
    %\clearpage
    %\chapter*{Summary}
    %\addcontentsline{toc}{chapter}{Summary}
	%\input{CommonSummary}


    % ---------------------------------------------------------------------- %
    % An optional glossary. We don't want it to be numbered
    %\clearpage
    %\chapter*{Nomenclature}
    %\addcontentsline{toc}{chapter}{Nomenclature}
    %\begin{description}
	%\item[dry spell]
	%    using a dry erase marker to spell words
	%\item[dry wall]
	%    the writing on the wall
	%\item[dry humor]
	%    when people just do not understand
	%\item[DRY]
	%    Don't Repeat Yourself
    %\end{description}


    % ---------------------------------------------------------------------- %
    % This is where the body of the report begins; usually with an Introduction
    %
    \SANDmain		% Start the main part of the report

    %-----------------------------%
    \chapter{Introduction}\label{sec:introduction}
    %-----------------------------%
    % $Id$


\chapter{Introduction}
\label{INTRO}

\subsection{Getting the software}


\escript, \ESyS, all freely available.  Where do people get \finley from?



\begin{enumerate}
 \item general structure 
 \item how to get the software
 \item a few words about the general structure
\item installation
\end{enumerate}

\subsection{Acknowlegements}
\begin{itemize}
\item Margeret Kahn Australian Nationional Unversity, Canberra.
\end{itemize}


    %-----------------------------%
    \chapter{Multigrid background}\label{sec:multigrid}
    %-----------------------------%
    \label{sec:multigrid intro}
Here we provide a brief multigrid introduction (see~\cite{MGTutorial}
or~\cite{OwlBook} for more information). A multigrid solver tries to approximate
the original problem of interest with a sequence of smaller (\textit{coarser})
problems. The solutions from the coarser problems are combined in order to
accelerate convergence of the original (\textit{fine}) problem on the finest
grid. A simple multilevel iteration is illustrated in
Algorithm~\ref{multigrid_code}.

\begin{algorithm}
\centering
\begin{algorithmic}[0]
  \State{$A_0 = A$}
  \Function{Multilevel}{$A_k$, $b$, $u$, $k$}
    \State{// Solve $A_k$ u = b (k is current grid level)}
    \State $ u = S^{1}_m (A_k, b, u)$
      \If{$(k \ne {\bf N-1})$}
        \State{$P_k = $ determine\_interpolant( $A_k$ )}
        \State{$R_k = $ determine\_restrictor( $A_k$ )}
        \State{$\widehat{r}_{k+1} = R_k (b - A_k u )$}
        \State{$A_{k+1} = R_k A_k P_k$}
        \State{$v = 0$}
        \State{}\Call{Multilevel}{$\widehat{A}_{k+1}$, $\widehat{r}_{k+1}$, $v$, $k+1$}
        \State{$ u = u + P_{k} v$}
        \State{$ u = S^{2}_m (A_k, b, u )$}
      \EndIf
  \EndFunction
\end{algorithmic}
\caption{V-cycle multigrid with $N$ levels to solve $Ax=b$.}
\label{multigrid_code}
\end{algorithm}

In the multigrid iteration in Algorithm~\ref{multigrid_code}, the $S^{1}_m()$'s
and $S^{2}_m()$'s are called \textit{pre-smoothers} and \textit{post-smoothers}.
They are approximate solvers (e.g., symmetric Gauss-Seidel), with the subscript
$m$ denoting the number of applications of the approximate solution method. The
purpose of a smoother is to quickly reduce certain error modes in the
approximate solution on a level $k$. For symmetric problems, the pre-
and post-smoothers should be chosen to maintain symmetry (e.g., forward
Gauss-Seidel for the pre-smoother and backward Gauss-Seidel for the
post-smoother). The $P_k$'s are \textit{interpolation} matrices that transfer
solutions from coarse levels to finer levels. The $R_k$'s are
\textit{restriction} matrices that restrict a fine level residual to a coarser
level. In a geometric multigrid, $P_k$'s and $R_k$'s are determined
by the application, whereas in an algebraic multigrid they are automatically
generated. For symmetric problems, typically $R_k=P_k^T$. For nonsymmetric
problems, this is not necessarily true. The $A_k$'s are the coarse level
problems, and are generated through a Galerkin (triple matrix) product.

Please note that the algebraic multigrid algorithms implemented in \muelu{}
generate the grid transfers $P_k$ automatically and the coarse problems $A_k$
via a sparse triple matrix product. \trilinos{} provides a wide selection of
smoothers and direct solvers for use in \muelu through the \ifpack,
\ifpacktwo, \amesos, and \amesostwo packages (see \S\ref{sec:options}).

%%% Local Variables:
%%% mode: latex
%%% TeX-master: "mueluguide"
%%% End:



    %-----------------------------%
    \chapter{Getting Started}\label{sec:getting started}
    %-----------------------------%
    This section is meant to get you using \muelu{} as quickly as possible.  \S\ref{sec:overview} gives a
summary of \muelu's design.  \S\ref{sec:configuration and build} lists \muelu's dependencies on other
\trilinos libraries and provides a sample cmake configuration line.  Finally, code examples using the XML
interface are given in \S\ref{sec:examples in code}.

\section{Overview of \muelu}
\label{sec:overview}
%algorithm types
%problems types
\muelu{} is an extensible algebraic multigrid (AMG) library that is part of the
\trilinos{} project. \muelu{} works with \epetra (32-bit version) and
\tpetra matrix types. The library is written in C++ and allows for different
ordinal (index) and scalar types.  \muelu{} is designed to be efficient on many
different computer architectures, from workstations to supercomputers, relying
on ``MPI+X" principle, where ``X" can be threading, CUDA, or any other back-end provided by the \kokkos package.

\muelu{} provides a number of different multigrid algorithms:
\be
  \item smoothed aggregation AMG (for Poisson-like and elasticity problems);
  \item Petrov-Galerkin aggregation AMG (for convection-diffusion problems);
  \item energy-minimizing AMG;
  \item aggregation-based AMG for problems arising from the eddy current
    formulation of Maxwell's equations.
\ee
\muelu's software design allows for the rapid introduction of new multigrid algorithms.
The most important features of \muelu{} can be summarized as:
\begin{description}
  \item \textbf{Easy-to-use interface}

    \muelu{} has a user-friendly parameter input deck which covers
    most important use cases.  Reasonable defaults are provided for common problem types
    (see Table \ref{t:problem_types}).

  \item \textbf{Modern object-oriented software architecture}

    \muelu{} is written completely in C++ as a modular object-oriented multigrid
    framework, which provides flexibility to combine and reuse existing
    components to develop novel multigrid methods.

  \item \textbf{Extensibility}

    Due to its flexible design, \muelu{} is an excellent toolkit for
    research on novel multigrid concepts. Experienced multigrid users have full
    access to the underlying framework through an advanced XML based interface.
    Expert users may use and extend the C++ API directly.

  \item \textbf{Integration with \trilinos{} library}

    As a package of \trilinos, \muelu{} is well integrated into the \trilinos
    environment. \muelu{} can be used with either the \tpetra{} or \epetra{}
    (32-bit) linear algebra stack. It is templated on the local index, global
    index, scalar, and compute node types. This makes \muelu{} ready for
    future developments.

  \item \textbf{Broad range of supported platforms}

    \muelu{} runs on wide variety of architectures, from desktop workstations to
    parallel Linux clusters and supercomputers (\cite{lin2014}).

  \item \textbf{Open source}

    \muelu{} is freely available through a simplified BSD license (see Appendix~\ref{sec:license}).
\end{description}

\section{Configuration and Build}
\label{sec:configuration and build}

\muelu{} has been compiled successfully under Linux with the following C++
compilers: GNU versions 4.1 and later, Intel versions 12.1/13.1, and clang versions 3.2 and later.
In the future, we recommend using compilers supporting C++11 standard.

\subsection{Dependencies}

\noindent{\bf Required Dependencies}

\muelu{} requires that \teuchos{} and either \epetra/\ifpack or \tpetra/\ifpacktwo
are enabled.

\noindent{\bf Recommended Dependencies}

We strongly recommend that you enable the following \trilinos libraries along with \muelu:

\begin{itemize}
  \item \epetra stack: \aztecoo, \epetra, \amesos, \ifpack, \isorropia, \galeri,
    \zoltan;
  \item \tpetra stack: \amesostwo, \belos, \galeri, \ifpacktwo, \tpetra,
    \zoltantwo.
\end{itemize}

\noindent{\bf Tutorial Dependencies}

In order to run the \muelu{} Tutorial \cite{MueLuTutorial} located in \verb!muelu/doc/Tutorial!, \muelu{} must be configured with the following
dependencies enabled:

  \aztecoo, \amesos, \amesostwo, \belos, \epetra, \ifpack, \ifpacktwo, \isorropia,
  \galeri, \tpetra, \zoltan, \zoltantwo.

\begin{mycomment}
Note that the \muelu{} tutorial \cite{MueLuTutorial} comes with a VirtualBox image with a pre-installed
Linux and \trilinos{}.   In this way, a user can immediately begin experimenting with \muelu{} without
having to install the \trilinos{} libraries. Therefore, it is an ideal starting point to get in touch with \muelu{}.
\end{mycomment}

\noindent{\bf Complete List of Direct Dependencies}

\begin{table}[ht]
  \centering
  \begin{tabular}{p{3.5cm} c c c c}
    \toprule
    \multirow{2}{*}{Dependency} & \multicolumn{2}{c}{Required} & \multicolumn{2}{c}{Optional} \\
    \cmidrule(r){2-3} \cmidrule(l){4-5}
                   & Library  & Testing  & Library  & Testing        \\
    \hline
    \amesos        &          &          & $\times$ & $\times$  \\
    \amesostwo     &          &          & $\times$ & $\times$  \\
    \aztecoo       &          &          &          & $\times$  \\
    \belos         &          &          &          & $\times$  \\
    \epetra        &          &          & $\times$ & $\times$  \\
    \ifpack        &          &          & $\times$ & $\times$  \\
    \ifpacktwo     &          &          & $\times$ & $\times$  \\
    \isorropia     &          &          & $\times$ & $\times$  \\
    \galeri        &          &          &          & $\times$  \\
    \kokkosclassic &          &          & $\times$ & \\
    \teuchos{}     & $\times$ & $\times$ &          & \\
    \tpetra        &          &          & $\times$ & $\times$  \\
    \xpetra        & $\times$ & $\times$ &          & \\
    \zoltan        &          &          & $\times$ & $\times$  \\
    \zoltantwo     &          &          & $\times$ & $\times$  \\
    \midrule
    Boost          &          &          & $\times$ & \\
    BLAS           & $\times$ & $\times$ &          & \\
    LAPACK         & $\times$ & $\times$ &          & \\
    MPI            &          &          & $\times$ & $\times$  \\
    \bottomrule
  \end{tabular}
  \caption{\label{tab:dependencies}\muelu's required and optional dependencies,
    subdivided by whether a dependency is that of the \muelu{} library itself
    (\textit{Library}), or of some \muelu{} test (\textit{Testing}). }
\end{table}

Table~\ref{tab:dependencies} lists the dependencies of \muelu, both required and
optional. If an optional dependency is not present, the tests requiring that
dependency are not built.

\begin{mycomment}
\amesos{}/\amesostwo{} are necessary if one wants to use a sparse direct solve on the coarsest level.
\zoltan{}/\zoltantwo{} are necessary if one wants to use matrix rebalancing in parallel runs (see~\S\ref{sec:performance}).
\aztecoo{}/\belos{} are necessary if one wants to test \muelu{} as a preconditioner instead of a solver.
\end{mycomment}

\begin{mycomment}
\muelu{} has also been successfully tested with SuperLU 4.1 and SuperLU 4.2.
\end{mycomment}
\begin{mycomment}
Some packages that \muelu{} depends on may come with additional requirements for
third party libraries, which are not listed here as explicit dependencies of \muelu{}.
It is highly recommended to install ParMetis 3.1.1 or newer for \zoltan{},
ParMetis 4.0.x for \zoltantwo{}, and SuperLU 4.1 or newer for
\amesos{}/\amesostwo{}.
\end{mycomment}

\subsection{Configuration}
The preferred way to configure and build \muelu{} is to do that outside of the source directory.
Here we provide a sample configure script that will enable \muelu{} and all of its optional dependencies:
\begin{lstlisting}
  export TRILINOS_HOME=/path/to/your/Trilinos/source/directory
  cmake \
      -D BUILD_SHARED_LIBS:BOOL=ON \
      -D CMAKE_BUILD_TYPE:STRING="RELEASE" \
      -D CMAKE_CXX_FLAGS:STRING="-g" \
      -D Trilinos_ENABLE_EXPLICIT_INSTANTIATION:BOOL=ON \
      -D Trilinos_ENABLE_TESTS:BOOL=OFF \
      -D Trilinos_ENABLE_EXAMPLES:BOOL=OFF \
      -D Trilinos_ENABLE_MueLu:BOOL=ON \
      -D   MueLu_ENABLE_TESTS:STRING=ON \
      -D   MueLu_ENABLE_EXAMPLES:STRING=ON \
      -D TPL_ENABLE_BLAS:BOOL=ON \
      -D TPL_ENABLE_MPI:BOOL=ON \
  ${TRILINOS_HOME}
\end{lstlisting}

\noindent
More configure examples can be found in \texttt{Trilinos/sampleScripts}.
For more information on configuring, see the \trilinos CMake Quickstart guide \cite{TrilinosCmakeQuickStart}.

\section{Examples in code}
\label{sec:examples in code}
% simple scaling test
%   galeri
%   XML input
%   belos/aztecoo or stand-alone solver
%   look @ tutorial or elsewhere for more advanced usage

The most commonly used scenario involving \muelu{} is using a multigrid
preconditioner inside an iterative linear solver. In \trilinos{}, a user has a
choice between \epetra and \tpetra for the underlying linear algebra library.
Important Krylov subspace methods (such as preconditioned CG and GMRES) are
provided in \trilinos{} packages \aztecoo (\epetra{}) and \belos
(\epetra{}/\tpetra{}).

At this point, we assume that the reader is comfortable with \teuchos{} referenced-counted
pointers (RCPs) for memory management (an introduction to RCPs can be found
in~\cite{RCP2010}) and the \texttt{Teuchos::ParameterList} class~\cite{TeuchosURL}.

\subsection{\muelu{} as a preconditioner within \belos}
\label{sec:tpetraexample}
The following code shows the basic steps of how to use a \muelu{}
multigrid preconditioner with \tpetra{} linear algebra library and with a linear
solver from \belos{}. To keep the example short and clear, we skip the template
parameters and focus on the algorithmic outline of setting up
a linear solver. For further details, a user may refer to the \texttt{examples} and
\texttt{test} directories.

First, we create the \muelu{} multigrid preconditioner. It can be done in a few
ways. For instance, multigrid parameters can be read from an XML file
(e.g., \textit{mueluOptions.xml} in the example below).
\begin{lstlisting}[language=C++]
    Teuchos::RCP<Tpetra::CrsMatrix<> > A;
    // create A here ...
    std::string optionsFile = "mueluOptions.xml";
    Teuchos::RCP<MueLu::TpetraOperator> mueLuPreconditioner =
       MueLu::CreateTpetraPreconditioner(A, optionsFile);
\end{lstlisting}
The XML file contains multigrid options. A typical file with \muelu{} parameters
looks like the following.
\begin{lstlisting}[language=XML]
<ParameterList name="MueLu">

  <Parameter name="verbosity" type="string" value="low"/>

  <Parameter name="max levels" type="int" value="3"/>
  <Parameter name="coarse: max size" type="int" value="10"/>

  <Parameter name="multigrid algorithm" type="string" value="sa"/>

  <!-- Damped Jacobi smoothing -->
  <Parameter name="smoother: type" type="string" value="RELAXATION"/>
  <ParameterList name="smoother: params">
    <Parameter name="relaxation: type"  type="string" value="Jacobi"/>
    <Parameter name="relaxation: sweeps" type="int" value="1"/>
    <Parameter name="relaxation: damping factor" type="double" value="0.9"/>
  </ParameterList>

  <!-- Aggregation -->
  <Parameter name="aggregation: type" type="string" value="uncoupled"/>
  <Parameter name="aggregation: min agg size" type="int" value="3"/>
  <Parameter name="aggregation: max agg size" type="int" value="9"/>

</ParameterList>
\end{lstlisting}
It defines a three level smoothed aggregation multigrid algorithm. The
aggregation size is between three and nine(2D)/27(3D) nodes.  One sweep with a
damped Jacobi method is used as a level smoother. By default, a direct solver is
applied on the coarsest level. A complete list of available parameters and valid
parameter choices is given in \S\ref{sec:muelu_options} of this User's Guide.

Users can also construct a multigrid preconditioner using a provided \parameterlist
without accessing any files in the following manner.
\begin{lstlisting}[language=C++]
  Teuchos::RCP<Tpetra::CrsMatrix<> > A;
  // create A here ...
  Teuchos::ParameterList paramList;
  paramList.set("verbosity", "low");
  paramList.set("max levels", 3);
  paramList.set("coarse: max size", 10);
  paramList.set("multigrid algorithm", "sa");
  // ...
  Teuchos::RCP<MueLu::TpetraOperator> mueLuPreconditioner =
     MueLu::CreateTpetraPreconditioner(A, paramList);
\end{lstlisting}

Besides the linear operator $A$, we also need an initial guess vector for the
solution $X$ and a right hand side vector $B$ for solving a linear system.
\begin{lstlisting}[language=C++]
    Teuchos::RCP<const Tpetra::Map<> > map = A->getDomainMap();

    // Create initial vectors
    Teuchos::RCP<Tpetra::MultiVector<> > B, X;
    X = Teuchos::rcp( new Tpetra::MultiVector<>(map,numrhs) );
    Belos::MultiVecTraits<>::MvRandom( *X );
    B = Teuchos::rcp( new Tpetra::MultiVector<>(map,numrhs) );
    Belos::OperatorTraits<>::Apply( *A, *X, *B );
    Belos::MultiVecTraits<>::MvInit( *X, 0.0 );
\end{lstlisting}
To generate a dummy example, the above code first declares two vectors. Then, a
right hand side vector is calculated as the matrix-vector product of a random vector
with the operator $A$. Finally, an initial guess is initialized with zeros.

Then, one can define a \texttt{Belos::LinearProblem} object where the
\texttt{mueLuPreconditioner} is used for left preconditioning
\begin{lstlisting}[language=C++]
    Belos::LinearProblem<> problem( A, X, B );
    problem->setLeftPrec(mueLuPreconditioner);
    bool set = problem.setProblem();
\end{lstlisting}

Next, we set up a \belos{} solver using some basic parameters
\begin{lstlisting}[language=C++]
    Teuchos::ParameterList belosList;
    belosList.set( "Block Size", 1 );
    belosList.set( "Use Single Reduction", true );
    belosList.set( "Maximum Iterations", 100 );
    belosList.set( "Convergence Tolerance", 1e-10 );
    belosList.set( "Output Frequency", 1 );
    belosList.set( "Verbosity", Belos::TimingDetails + Belos::FinalSummary );

    Belos::BlockCGSolMgr<> solver( rcp(&problem,false), rcp(&belosList,false) );
\end{lstlisting}

Finally, we solve the system.
\begin{lstlisting}[language=C++]
    Belos::ReturnType ret = solver.solve();
\end{lstlisting}

\subsection{\muelu{} as a preconditioner for \aztecoo}

For \epetra, users have two library options: \belos{} (recommended) and \aztecoo{}.
\aztecoo{} and \belos both provide fast and mature implementations of common iterative Krylov linear solvers.
\belos has additional capabilities, such as Krylov subspace recycling and ``tall skinny QR".

Constructing a \muelu{} preconditioner for Epetra operators is done in a similar
manner to Tpetra.
\begin{lstlisting}[language=C++]
    Teuchos::RCP<Epetra_CrsMatrix> A;
    // create A here ...
    Teuchos::RCP<MueLu::EpetraOperator> mueLuPreconditioner;
    std::string optionsFile = "mueluOptions.xml";
    mueLuPreconditioner = MueLu::CreateEpetraPreconditioner(A, optionsFile);
\end{lstlisting}
\muelu{} parameters are generally Epetra/Tpetra agnostic, thus the XML parameter file
could be the same as~\S\ref{sec:tpetraexample}.

Furthermore, we assume that a right hand side vector and a solution vector with
the initial guess are defined.
\begin{lstlisting}[language=C++]
    Teuchos::RCP<const Epetra_Map> map = A->DomainMap();
    Teuchos::RCP<Epetra_Vector> B = Teuchos::rcp(new Epetra_Vector(map));
    Teuchos::RCP<Epetra_Vector> X = Teuchos::rcp(new Epetra_Vector(map));
    X->PutScalar(0.0);
\end{lstlisting}

Then, an \texttt{Epetra\_LinearProblem} can be defined.
\begin{lstlisting}[language=C++]
    Epetra_LinearProblem epetraProblem(A.get(), X.get(), B.get());
\end{lstlisting}

The following code constructs an \aztecoo{} CG solver.
\begin{lstlisting}[language=C++]
    AztecOO aztecSolver(epetraProblem);
    aztecSolver.SetAztecOption(AZ_solver, AZ_cg);
    aztecSolver.SetPrecOperator(mueLuPreconditioner.get());
\end{lstlisting}

Finally, the linear system is solved.
\begin{lstlisting}[language=C++]
    int maxIts = 100;
    double tol = 1e-10;
    aztecSolver.Iterate(maxIts, tol);
\end{lstlisting}


\subsection{\muelu's structured algorithms}

Some users might use structured meshes to discretize their problems. In such cases it can be advantageous to use the structured grid algorithms provided in \muelu. To use these algorithms the user has to provide extra information to \muelu such as the number of spatial dimensions in the problem and the number of nodes in each direction on the local rank. As demonstrated in the code bellow \muelu expect these additional inputs to be stored in a sublist called ``user data".
\begin{lstlisting}[language=C++]
  const std::string userName = "user data";
  Teuchos::ParameterList& userParamList = paramList.sublist(userName);
  userParamList.set<int>("int numDimensions", numDimensions);
  userParamList.set<Teuchos::Array<LO> >("Array<LO> lNodesPerDim", lNodesPerDim);
  userParamList.set<RCP<RealValuedMultiVector> >("Coordinates", coordinates);
  H = MueLu::CreateXpetraPreconditioner(A, paramList, paramList);
\end{lstlisting}
Full examples demonstrating the structured capabilities of \muelu can be found in the \trilinos source directories
\begin{itemize}
  \setlength{\itemsep}{-10pt}
\item \texttt{packages/muelu/test/structured},
\item \texttt{packages/trilinoscouplings/examples/scaling}.
\end{itemize}


\subsection{\muelu's Maxwell solver}

\muelu can be used to solve Maxwell's equations in eddy current formulation which can be written as
\begin{equation}
  \nabla\times \left(\alpha \nabla  \times \vec{E}\right) + \beta \vec{E} = \vec{f}, \label{eq:maxwell}
\end{equation}
where \(\vec{E}\) is the unknown electric field, \(\alpha\) and \(\beta\) are material parameters,
and \(\vec{f}\) is the known right-hand side.
In order to deal with the large nullspace of the curl-curl operator a specialized multigrid approach
is required.
For a detailed description of the solver see ~\cite{RefMaxwell2008}.

A preconditioner for equation~\ref{eq:maxwell} can be constructed as follows:
\begin{lstlisting}[language=C++]
  RCP<Matrix> SM_Matrix = ... ;    \\ Edge-mass + curl-curl
  RCP<Matrix> D0_Matrix = ... ;    \\ Discrete gradient matrix
  RCP<Matrix> M0inv_Matrix = ... ; \\ Approximate inverse of node-mass matrix with weight 1/alpha
  RCP<Matrix> M1_Matrix = ... ;    \\ Edge-mass matrix with constant weight 1
  RCP<MultiVector> coords = ...;   \\ Nodal coordinates
  Teuchos::ParameterList params = ...; \\ Parameters

  RCP<MueLu::RefMaxwell> preconditioner
  = rcp( new MueLu::RefMaxwell(SM_Matrix, D0_Matrix, M0inv_Matrix,
         M1_Matrix, Teuchos::null, coords, params) );
\end{lstlisting}
An example set of parameters is given below:
\begin{lstlisting}[language=XML]
  <ParameterList name="MueLu">

  <Parameter name="refmaxwell: mode"  type="string" value="additive"/>

  <Parameter name="smoother: type" type="string" value="RELAXATION"/>
  <ParameterList name="smoother: params">
    <Parameter name="relaxation: type" type="string" value="Symmetric Gauss-Seidel"/>
    <Parameter name="relaxation: sweeps" type="int" value="2"/>
  </ParameterList>

  <ParameterList name="refmaxwell: 11list">
    <Parameter name="number of equations"   type="int"    value="3"/>
    <Parameter name="aggregation: type"     type="string" value="uncoupled"/>
    <Parameter name="coarse: max size"      type="int"    value="2500"/>
    <Parameter name="smoother: type" type="string" value="RELAXATION"/>
    <ParameterList name="smoother: params">
      <Parameter name="relaxation: type" type="string" value="Symmetric Gauss-Seidel"/>
      <Parameter name="relaxation: sweeps" type="int" value="2"/>
    </ParameterList>
  </ParameterList>

  <ParameterList name="refmaxwell: 22list">
    <Parameter name="aggregation: type"     type="string" value="uncoupled"/>
    <Parameter name="coarse: max size"      type="int"    value="2500"/>
    <Parameter name="smoother: type" type="string" value="RELAXATION"/>
    <ParameterList name="smoother: params">
      <Parameter name="relaxation: type" type="string" value="Symmetric Gauss-Seidel"/>
      <Parameter name="relaxation: sweeps" type="int" value="2"/>
    </ParameterList>
  </ParameterList>

</ParameterList>
\end{lstlisting}
Further examples of how to use \muelu to solve Maxwell's equations can be found in the \trilinos source directories
\begin{itemize}
  \setlength{\itemsep}{-10pt}
\item \texttt{packages/muelu/test/maxwell},
\item \texttt{packages/panzer/mini-em/example/BlockPrec} and
\item \texttt{packages/trilinoscouplings/examples/scaling}.
\end{itemize}




\subsection{Further remarks}

This section is only meant to give a brief introduction on how to use \muelu{}
as a preconditioner within the \trilinos{} packages for iterative solvers. There
are other, more complicated, ways to use \muelu{} as a preconditioner for \belos
and \aztecoo through the \xpetra interface. Of course, \muelu{} can also work as
standalone multigrid solver. For more information on these topics, the reader
may refer to the examples and tests in the \muelu{} source directory
(\texttt{packages/muelu/example} and \texttt{packages/muelu/test}) and in the trilinosCouplings source directory
(\texttt{packages/trilinosCouplings}), as well as to the \muelu{}
tutorial~\cite{MueLuTutorial}.
For in-depth details of \muelu applied to multiphysics problems, please see~\cite{Wiesner2014}.

%%% Local Variables:
%%% mode: latex
%%% TeX-master: "mueluguide"
%%% End:


    %-----------------------------%
    \chapter{Performance tips}\label{sec:performance}
    %-----------------------------%
    \section{How to wring the last bit of performance out of Ifpack2 (jhu,csiefer)}
\section{Published results}
Cite the PPL paper \cite{Lin2014}.


    %-----------------------------%
    \chapter{\muelu{} options} \label{sec:options}
    %-----------------------------%
    \label{sec:options}
In this section, we report the complete list of input parameters. Input
parameters are passed to \ifpacktwo in a single \parameterlist.

In some cases, the parameter types may depend on runtime template parameters.
In such cases, we will follow the conventions in Table~\ref{tab:conventions}.

\begin{table}[htbp]
  \centering
  \begin{tabular}{p{13.3cm} p{2.5cm}}
    \toprule
    \verb!MatrixType::local_ordinal_type!                                  & \verb!local_ordinal! \\
    \verb!MatrixType::global_ordinal_type!                                 & \verb!global_ordinal! \\
    \verb!MatrixType::scalar_type!                                         & \verb!scalar! \\
    \verb!MatrixType::node_type!                                           & \verb!node! \\
    \verb!Tpetra::Vector<scalar,local_ordinal,global_ordinal,node>!        & \verb!vector!\\
    \verb!Tpetra::MultiVector<scalar,local_ordinal,global_ordinal,node>!   & \verb!multi_vector!\\
    \verb!vector::mag_type!                                                & \verb!magnitude! \\
    \bottomrule
  \end{tabular}
  \caption{\label{tab:conventions}Conventions for option types that depend on templates.}
\end{table}

\noindent\textbf{Note:} if \verb!scalar! is \texttt{double}, then \verb!magnitude! is also \texttt{double}.

\section{Point relaxation}\label{s:relaxation}

\textbf{Preconditioner type:} ``RELAXATION''.

\ifpacktwo{} implements the following classical relaxation methods: Jacobi (with
optional damping), Gauss-Seidel, Successive Over-Relaxation (SOR), symmetric
version of Gauss-Seidel and SOR. \ifpacktwo{} calls both Gauss-Seidel and SOR
"Gauss-Seidel". The algorithmic details can be found in~\cite{Saad2003}.

Besides the classical relaxation methods, \ifpacktwo{} also implements $l_1$
variants of Jacobi and Gauss-Seidel methods proposed in~\cite{Baker2011}, which
lead to a better performance in parallel applications.

\noindent{\bf Note:} if a user provides a \texttt{Tpetra::BlockCrsMatrix}, the point relaxation
methods become block relaxation methods, such as block Jacobi or block
Gauss-Seidel.

The following parameters are used in the point relaxation methods:

\ccc{relaxation: type}
    {string}
    {``Jacobi''}
    {Relaxation method to use. Accepted values: ``Jacobi'',
     ``Gauss-Seidel'', ``Symmetric Gauss-Seidel''.}
\ccc{relaxation: sweeps}
    {int}
    {1}
    {Number of sweeps of the relaxation.}
\ccc{relaxation: damping factor}
    {scalar}
    {1.0}
    {The value of the damping factor $\omega$ for the relaxation.}
\ccc{relaxation: backward mode}
    {bool}
    {\false}
    {Governs whether Gauss-Seidel is done in forward-mode (\false) or
     backward-mode (\true). Only valid for ``Gauss-Seidel'' type.}
\ccc{relaxation: use l1}
    {bool}
    {\false}
    {Use the $l_1$ variant of Jacobi or Gauss-Seidel.}
\ccc{relaxation: l1 eta}
    {magnitude}
    {1.5}
    {$\eta$ parameter for $l_1$ variant of Gauss-Seidel. Only used if
     {\tt "relaxation: use l1"} is \true.}
\ccc{relaxation: zero starting solution}
    {bool}
    {\true}
    {Governs whether or not \ifpacktwo{} uses existing values in the left hand
     side vector. If true, \ifpacktwo{} fill it with zeros before applying
     relaxation sweeps which may make the first sweep more efficient.}
\ccc{relaxation: fix tiny diagonal entries}
    {bool}
    {\false}
    {If true, the compute() method will do extra work (computation only, no MPI
     communication) to fix diagonal entries. Specifically, the diagonal values
     with a magnitude smaller than the magnitude of the threshold \texttt{relaxation: min
     diagonal value} are increased to threshold for the diagonal inversion. The
     matrix is not modified, instead the updated diagonal values are stored. If the
     threshold is zero, only the diagonal entries that are exactly zero are replaced
     with a small nonzero value (machine precision).}
\ccc{relaxation: min diagonal value}
    {scalar}
    {0.0}
    {The threshold value used in {\tt "relaxation: fix tiny diagonal entries"}.
     Only used if {\tt "relaxation: fix tiny diagonal entries"} is \true.}
\ccc{relaxation: check diagonal entries}
    {bool}
    {\false}
    {If true, the \texttt{compute()} method will do extra work (both computation
     and communication) to count diagonal entries that are zero, have negative
     real part, or are small in magnitude. This information can be later shown
     in the description.}
\ccc{relaxation: mtgs cluster size}
    {int}
    {1}
    {Only has an effect if {\tt "relaxation: type"} is {\tt "MT Gauss-Seidel"}
     or {\tt "MT Symmetric Gauss-Seidel"}. If equal to 1 (default), point
     coloring parallel Gauss-Seidel is used. This has a faster \texttt{compute()}
     but may cause the preconditioned solver
     to converge more slowly. If set to $k > 1$, then multicolor block Gauss-Seidel
     is used with blocks of size $k$ (see \cite{Saad1999}).
     In the \texttt{apply()} there is significantly less
     error due to parallel updates of the LHS vector.}
\ccc{relaxation: local smoothing indices}
    {Teuchos::ArrayRCP<local\_ordinal>}
    {empty}
    {}

%Teuchos::ArrayRCP MatrixType::local_ordinal_type}{\texttt{Teuchos::null}}
    {A given method will only relax on the local indices listed in the
     \texttt{ArrayRCP}, in the order that they are listed. This can be used to
     reorder the relaxation, or to only relax on a subset of ids.}

\section{Block relaxation}\label{s:block_relaxation}

\textbf{Preconditioner type:} ``BLOCK\_RELAXATION''.

% \info[inline]{AP}{ILUTP cannot be constructed through {\tt Ifpack2::Factory},
% only through additive Schwarz}

\ifpacktwo{} supports block relaxation methods. Each block corresponds to a set
of degrees of freedom within a local subdomain. The blocks can be
non-overlapping or overlapping. Block relaxation can be considered as domain
decomposition within an MPI process, and should not be confused with additive
Schwarz preconditioners (see~\ref{s:schwarz}) which implement domain
decomposition across MPI processes.

There are several ways the blocks are constructed:
\begin{itemize}
  \item Linear partitioning of unknowns

    The unknowns are divided equally among a specified number of
    partitions $L$ defined by {\tt "partitioner: local parts"}. In other words,
    assuming number of unknowns $n$ is divisible by $L$, unknown $i$ will belong
    to block number $\lfloor iL/n \rfloor$.

  \item Line partitioning of unknowns

    The unknowns are grouped based on a geometric criteria which tries to
    identify degrees of freedom that form an approximate geometric line.
    Current approach uses a local line detection inspired by the work of
    Mavriplis~\cite{Mavriplis1999} for convection-diffusion. \ifpacktwo uses
    coordinate information provided by {\tt "partitioner: coordinates"} to pick
    "close" points if they are sufficiently far away from the "far" points. It
    also makes sure the line can never double back on itself.

    These "line" partitions were found to be very beneficent to problems on
    highly anisotropic geometries such as ice-sheet simulations.

  \item User partitioning of unknowns

    The unknowns are grouped according to a user provided partition. A user
    may provide a non-overlapping partition {\tt "partitioner: map"} or an
    overlapping one {\tt "partitioner: parts"}.

    A particular example of a smoother using this approach is a Vanka
    smoother~\cite{Vanka1986}, where a user may in {\tt "partition: parts"} pressure
    degrees of freedom, and request a overlap of one thus constructing Vanka
    blocks.
\end{itemize}
The original partitioning may be further modified with {\tt "partitioner: overlap"}
parameter which will use the local matrix graph to construct overlapping
partitions.

The blocks are applied in the order they were constructed. This means that in
the case of overlap the entries in the solution vector relaxed by one block may
later be overwritten by relaxing another block.

The following parameters are used in the block relaxation methods:

\cccc{relaxation: type}
    {See~\ref{s:relaxation}.}
\cccc{relaxation: container}
    {string}
    {``TriDi''}
    {Containers are used to store and solve block matrices. These container
     types are always available: ``Dense'', ``TriDi''
     (equivalent to ``Tridiagonal''), ``Banded'' and ``SparseILUT''.
     ``Dense'', ``TriDi'' and ``Banded'' block matrices are
     solved exactly LAPACK routines, and ``SparseILUT'' blocks are solved approximately
     using an incomplete LU factorization with thresholding.

     If Amesos2 is enabled, ``SparseAmesos'' (equivalent to ``SparseAmesos2'') is available.
     The default Amesos2 sparse solver is KLU2, but this can be configured by setting
     ``Amesos2 solver name'' (see the Amesos2 documentation for all available solvers).

     If experimental kokkos-kernels features are enabled (true by default), the ``BlockTriDi''
     container (equivalent to ``Block Tridiagonal'') is available. This container's solver is the damped Jacobi method, using
     block tridiagonal matrices as the diagonal D.
     For a block size of 1, this is equivalent to standard damped Jacobi.
     This container is designed for high performance on KNL and GPU.}
\cccc{relaxation: sweeps}
    {See~\ref{s:relaxation}.}
\cccc{relaxation: damping factor}
    {See~\ref{s:relaxation}.}
\cccc{relaxation: zero starting solution}
    {See~\ref{s:relaxation}.}
\cccc{relaxation: backward mode}
    {See~\ref{s:relaxation}. Currently has no effect. }
\ccc{block relaxation: decouple dofs}
    {bool}
    {false}
    {Whether to separate blocks according to the different degrees of
     freedom (PDEs) at each node. This assumes that dofs/node is constant
     throughout the matrix. Each block will have the same sparsity
     pattern as the mesh graph's corresponding diagonal block.
     For example, when using a line partitioner this
     enables the use of the tridiagonal container even if the matrix's
     bandwidth is greater than 3.
     Decoupling matches the behavior of line smoothing in ML.}
\ccc{partitioner: type}
    {string}
    {``linear''}
    {The partitioner to use for defining the blocks.  This can be either
     ``linear'', ``line'' or ``user''.}
\ccc{partitioner: overlap}
    {int}
    {0}
    {The amount of overlap between partitions (0 corresponds to no overlap).
     Only valid for ``Jacobi'' relaxation.}
\ccc{partitioner: local parts}
    {int}
    {1}
    {Number of local partitions (1 corresponds to one local partition, which
     means "do not partition locally"). Only valid for ``linear'' partitioner
     type.}
\ccc{partitioner: map}
    {Teuchos::ArrayRCP<local\_ordinal>}
    {empty}
    {An array containing the partition number for each element.
     The $i$th entry in the \texttt{ArrayRCP} is the part (block) number that
     row $i$ belongs to. Use this option if the parts (blocks) do not
     overlap. Only valid for ``user'' partitioner type.}
\ccc{partitioner: parts}
    {Teuchos::Array<Teuchos::ArrayRCP\\<local\_ordinal>>}
    {empty}
    {Use this option if the parts (blocks) overlap. The $i$th entry in the
     \texttt{Array} is an \texttt{ArrayRCP} that contains all the rows in part
     (block) $i$. Only valid for ``user'' partitioner type.}
\ccc{partitioner: line detection threshold}
    {magnitude}
    {0.0}
    {Threshold used in line detection. If the distance between two connected
     points $i$ and $j$ is within the threshold times maximum distance of all
     points connected to $i$, then point $j$ is considered close enough to line
     smooth. Only valid for ``line'' partition type.}
\ccc{partitioner: PDE equations}
    {int}
    {1}
    {Number of equations per node. Only used for ``line'' partitioning, and
     decoupled BlockRelaxation.}
\ccc{partitioner: coordinates}
    {Teuchos::RCP<multi\_vector>}
    {null}
    {Coordinates of local nodes. Only valid for ``line'' partitioner type.}
\ccc{partitioner: maintain sparsity}
    {bool}
    {\false}
    {For OverlappingPartitioner, whether to sort the entries in each partition.}

\section{Chebyshev}\label{s:Chebyshev}

\textbf{Preconditioner type:} ``CHEBYSHEV''.

% Mark Hoemmen (2016/05/31):
%   The "textbook version" of Chebyshev doesn't really
%   work; we need to get rid of it.

\ifpacktwo{} implements a variant of Chebyshev iterative method following
\ifpack{}'s implementation.  \ifpack{} has a special-case modification of the
eigenvalue bounds for the case where the maximum eigenvalue estimate is close to
one. Experiments show that the \ifpack{} imitation is much less sensitive to the
eigenvalue bounds than the textbook version.

\ifpacktwo{} uses the diagonal of the matrix to precondition the linear system on the
left. Diagonal elements less than machine precision are replaced with machine
precision.

\ifpacktwo{} requires can take any matrix $A$ but can only guarantee convergence
for real valued symmetric positive definite matrices.
\iffalse
If users could provide the ellipse parameters ($d$ and $c$ in the literature,
where $d$ is the real-valued center of the ellipse, and $d-c$ and $d+c$ the two
foci), the iteration itself would work fine with nonsymmetric real-valued $A$,
as long as the eigenvalues of $A$ can be bounded in an ellipse that is entirely
to the right of the origin.
\unsure[inline]{AP}{Really unsure about Chebyshev nonsymmetric matrices. There does not
seem anything in the code to work with ellipse. I need to ask Mark Hoemmen
about this.}
\fi

The following parameters are used in the Chebyshev method:

\ccc{chebyshev: degree}
    {int}
    {1}
    {Degree of the Chebyshev polynomial, or the number of iterations. This
     overrides parameters {\tt "relaxation: sweeps"} and {\tt "smoother: sweeps"}.}
\cccc{relaxation: sweeps}
    {Same as {\tt "chebyshev: degree"}, for compatibility with \ifpack{}.}
\cccc{smoother: sweeps}
    {Same as {\tt "chebyshev: degree"}, for compatibility with \ml{}.}
\ccc{chebyshev: max eigenvalue}
    {scalar|double}
    {computed}
    {An upper bound of the matrix eigenvalues. If not provided, the value will
     be computed by power method (see parameters {\tt "eigen-analysis: type"} and
     {\tt "chebyshev: eigenvalue max iterations"}).}
\ccc{chebyshev: min eigenvalue}
    {scalar|double}
    {computed}
    {A lower bound of the matrix eigenvalues.  If not provided, \ifpacktwo{}
     will provide an estimate based on the maximum eigenvalue and the ratio.}
\ccc{chebyshev: ratio eigenvalue}
    {scalar|double}
    {30.0}
    {The ratio of the maximum and minimum estimates of the matrix
     eigenvalues.}
\cccc{smoother: Chebyshev alpha}
    {Same as {\tt "chebyshev: ratio eigenvalue"}, for compatibility with \ml{}.}
% \ccc{chebyshev: textbook algorithm}
    % {bool}
    % {\false}
    % {If true, use the textbook variant; otherwise, use the \ifpack{} variant.}
\ccc{chebyshev: compute max residual norm}
    {bool}
    {\false}
    {The \texttt{apply} call will optionally return the norm of the residual.}
\ccc{eigen-analysis: type}
    {string}
    {"power-method"}
    {The algorithm for estimating the max eigenvalue. Currently only supports
     power method ("power-method" or "power method"). The cost of the procedure is
     roughly equal to several matrix-vector multiplications.}
\ccc{chebyshev: eigenvalue max iterations}
    {int}
    {10}
    {Number of iterations to be used in calculating the estimate for the maximum
     eigenvalue, if it is not provided by the user.}
\cccc{eigen-analysis: iterations}
    {Same as {\tt "chebyshev: eigenvalue max iterations"}, for compatibility with \ml{}.}
\ccc{chebyshev: min diagonal value}
    {scalar}
    {0.0}
    {Values on the diagonal smaller than this value are increased to this value
     for the diagonal inversion.}
\ccc{chebyshev: boost factor}
    {double}
    {1.1}
    {Factor used to increase the estimate of matrix maximum eigenvalue to ensure
    the high-energy modes are not magnified by a smoother.}
\ccc{chebyshev: assume matrix does not change}
    {bool}
    {\false}
    {Whether \texttt{compute()} should assume that the matrix has not changed
     since the last call to \texttt{compute()}. If true, \texttt{compute()}
     will not recompute inverse diagonal or eigenvalue estimates.}
\ccc{chebyshev: operator inv diagonal}
    {Teuchos::RCP<const vector>|\\Teuchos::RCP<vector>|const vector*|\\vector}
    {Teuchos::null}
    {If nonnull, a deep copy of this vector will be used as the inverse
     diagonal of the matrix, instead of computing it. Expert use only.}
\ccc{chebyshev: min diagonal value}
    {scalar}
    {machine precision}
    {If any entry of the matrix diagonal is less that this in magnitude, it will
     be replaced with this value in the inverse diagonal used for left scaling.}
\cccc{chebyshev: zero starting solution}
    {See {\tt "relaxation: zero starting solution"}.}

\section{Incomplete factorizations}

\subsection{ILU($k$)}\label{s:ILU}

\textbf{Preconditioner type:} ``RILUK''.

\ifpacktwo{} implements a standard and modified (MILU) variants of the
ILU($k$) factorization~\cite{Saad2003}. In addition, it also provides an
optional \textit{a priori} modification of the diagonal entries of a matrix to
improve the stability of the factorization.

The following parameters are used in the ILU($k$) method:

\ccc{fact: iluk level-of-fill}
    {int|global\_ordinal|magnitude|double}
    {0}
    {Level-of-fill of the factorization.}
\ccc{fact: relax value}
    {magnitude|double}
    {0.0}
    {MILU diagonal compensation value. Entries dropped during factorization
     times this factor are added to diagonal entries.}
\ccc{fact: absolute threshold}
    {magnitude|double}
    {0.0}
    {Prior to the factorization, each diagonal entry is updated by adding
     this value (with the sign of the actual diagonal entry). Can be combined
     with {\tt "fact: relative threshold"}. The matrix remains unchanged.}
\ccc{fact: relative threshold}
    {magnitude|double}
    {1.0}
    {Prior to the factorization, each diagonal element is scaled by this factor
     (not including contribution specified by {\tt "fact: absolute
     threshold"}). Can be combined with {\tt "fact: absolute threshold"}.
     The matrix remains unchanged.}
% All overlap-related code was removed by M. Hoemmen in
%
% commit 162f64572fbf93e2cac73e3034d76a3db918a494
% Author: Mark Hoemmen <mhoemme@sandia.gov>
% Date:   Fri Jan 24 17:16:19 2014 -0700
%
%     Ifpack2: RILUK: Removed all overlap-related code.
%
%     Overlap never had a correct implementation in RILUK.  Furthermore,
%     AdditiveSchwarz is the proper place for overlap to be implemented, not
%     RILUK.  Ifpack2's incomplete factorizations are local (per MPI
%     process) solvers and don't need to know anything about overlap across
%     processes.  Thus, this commit removes all overlap-related code from
%     RILUK.
%
% So, older parameter "fact: iluk level-of-overlap" is no longer valid and is ignored.

\subsection{ILUT}\label{s:ILUT}

\textbf{Preconditioner type:} ``ILUT''.

\ifpacktwo{} implements a slightly modified variant of the standard ILU factorization with specified fill and
drop tolerance ILUT($p,\tau$)~\cite{Saad1994}. The modifications follow the \aztecoo implementation.
The main difference between the \ifpacktwo implementation and the algorithm in \cite{Saad1994} is the definition of
\texttt{fact: ilut level-of-fill}.

The following parameters are used in the ILUT method:

\ccc{fact: ilut level-of-fill}
    {int|magnitude|double}
    {1}
    {Maximum number of entries to keep in each row of $L$ and $U$. Each row of
     $L$ ($U$) will have at most $\lceil\frac{(\mbox{\small\tt
     level-of-fill}-1)nnz(A)}{2n}\rceil$ nonzero entries, where $nnz(A)$ is the
     number of nonzero entries in the matrix, and $n$ is the number of rows.
     ILUT always keeps the diagonal entry in the current row, regardless of the
     drop tolerance or fill level. \textbf{Note:} \textit{This is
     different from the $p$ in the classic algorithm in~\cite{Saad1994}.}}
\ccc{fact: drop tolerance}
    {magnitude|double}
    {0.0}
    {A threshold for dropping entries ($\tau$ in~\cite{Saad1994}).}
\cccc{fact: absolute threshold}
    {See~\ref{s:ILU}.}
\cccc{fact: relative threshold}
    {See~\ref{s:ILU}.}
\cccc{fact: relax value}
    {Currently has no effect. For backwards compatibility only.}

\subsection{ILUTP}\label{s:ILUTP}

\textbf{Preconditioner type:} ``AMESOS2''.

% \info[inline]{AP}{ILUTP cannot be constructed through {\tt Ifpack2::Factory},
% only through additive Schwarz}

\ifpacktwo{} implements a standard ILUTP factorization~\cite{Saad2003}. This is
done through is through the \amesostwo interface to SuperLU~\cite{Li2011}. We
reproduce the \amesostwo options here for convenience. {\em You should consider
the \href{http://trilinos.org/docs/dev/packages/amesos2/doc/html/group__amesos2__solver__parameters.html#superlu_parameters}{\amesostwo
documentation} to be the final authority.}

The following parameters are used in the ILUTP method:

\ccc{ILU\_DropTol}
    {double}
    {1e-4}
    {ILUT drop tolerance.}
\ccc{ILU\_FillFactor}
    {double}
    {10.0}
    {ILUT fill factor.}
\ccc{ILU\_Norm}
    {string}
    {``INF\_NORM''}
    {Norm to be used in factorization. Accepted values: ``ONE\_NORM'', ``TWO\_NORM'', or ``INF\_NORM''.}
\ccc{ILU\_MILU}
    {string}
    {``SILU''}
    {Type of modified ILU to use. Accepted values: ``SILU'', ``SMILU\_1'', ``SMILU\_2'', or ``SMILU\_3''.}

\subsection{ShyLU FastILU}\label{s:FastILU}
\ifpacktwo{} provides an interface to the FastILU family of factorizations provided by ShyLU.
They are available if Trilinos was configured with the

\texttt{-D Trilinos\_ENABLE\_ShyLU\_Node=ON}

option. There are three values of ``Preconditioner type:'' that use the FastILU subpackage:

\begin{table}[h!]
\centering
\begin{tabular}{|l|l|}
\hline
``Preconditioner type:'' & Factorization   \\ \hline \hline
FAST\_ILU                & Incomplete LU       \\ \hline
FAST\_IC                 & Incomplete Cholesky \\ \hline
FAST\_ILDL               & Incomplete LDL*     \\ \hline
\end{tabular}
\end{table}

FAST\_ILU, FAST\_IC, and FAST\_ILDL all use iterative factorization algorithms in compute(). \texttt{"sweeps"} controls
this iteration count. A higher number of sweeps improves the quality of the factorization. All three
preconditioners also use an triangular block Jacobi solver in apply().
The Jacobi iteration count is controlled by \texttt{"triangular solve iterations"}.
The valid set of parameters is the same for FAST\_ILU, FAST\_IC, and FAST\_ILDL:

\ccc{sweeps}
    {int}
    {5}
    {Number of iterations of ILU/IC/ILDL factorization algorithm.}
\ccc{triangular solve iterations}
    {int}
    {1}
    {Number of iterations of the block Jacobi triangular solver.}
\ccc{level}
    {int}
    {0}
    {Level of fill.}
\ccc{damping factor}
    {double}
    {0.5}
    {Damping factor $\omega$ for the Jacobi triangular solver. $0 < \omega \leq 1$. A lower $\omega$ slows convergence but improves stability.}
\ccc{shift}
    {double}
    {0}
    {Manteuffel shifting parameter $\alpha$.}
\ccc{guess}
    {bool}
    {true}
    {Whether to run another instance of FastILU/IC/ILDL (but with a lower level of fill) to compute the initial guess (only has an effect if level of fill $> 0$).} 
\ccc{block size}
    {int}
    {1}
    {Block size for the block Jacobi solver.}

\section{Additive Schwarz}\label{s:schwarz}

\textbf{Preconditioner type:} ``SCHWARZ''.

\ifpacktwo{} implements additive Schwarz domain decomposition with optional
overlap. Each subdomain corresponds to exactly one MPI process in the given
matrix's MPI communication. For domain decomposition within an
MPI process see~\ref{s:block_relaxation}.

One-level overlapping domain decomposition preconditioners use local solvers of
Dirichlet type. This means that the inverse of the local matrix (possibly with
overlap) is applied to the residual to be preconditioned. The preconditioner can
be written as:
$$ P_{AS}^{-1} = \sum_{i=1}^M P_i A_i^{-1} R_i, $$
where $M$ is the number of subdomains (in this case, the number of (MPI)
processes in the computation), $R_i$ is an operator that restricts the global
vector to the vector lying on subdomain $i$, $P_i$ is the prolongator
operator, and $A_i = R_i A P_i$.

Constructing a Schwarz preconditioner requires defining two components.

{\bf Definition of the restriction and prolongation operators.}
Users may control how the data is combined with existing data by setting {\tt
"combine mode"} parameter. Table~\ref{t:combine_mode} contains a list of modes to
combine overlapped entries. The default mode is ``ZERO'' which is equivalent to
using a restricted additive Schwarz~\cite{Cai1999} method.

\begin{table}[htbp]
  \centering
  \begin{tabular}{p{3.5cm} p{12.0cm}}
    \toprule
    Combine mode name & Description \\
    \midrule
    ``ADD''           & Sum values into existing values \\
    ``ZERO''          & Replace old values with zero \\
    ``INSERT''        & Insert new values that don't currently exist \\
    ``REPLACE''       & Replace existing values with new values \\
    ``ABSMAX''        & Replace old values with maximum of magnitudes of old and new values \\
    \bottomrule
  \end{tabular}
  \caption{\label{t:combine_mode}Combine mode descriptions.}
\end{table}

{\bf Definition of a solver for subdomain linear system.}
Some preconditioners may benefit from local modifications to the subdomain
matrix. It can be filtered to eliminate singletons and/or reordered.
Reordering will often improve performance during incomplete factorization setup,
and improve the convergence. The matrix reordering algorithms specified in {\tt
"schwarz: reordering list"} are provided by \zoltantwo.  At the present time,
the only available reordering algorithm is RCM (reverse Cuthill-McKee). Other
orderings will be supported by the Zoltan2 package in the future.

To solve linear systems involving $A_i$ on each subdomain, a user can specify
the inner solver by setting {\tt "inner preconditioner name"} parameter (or any
of its aliases) which allows to use any \ifpacktwo preconditioner. These include
but are not necessarily limited to the preconditioners in
Table~\ref{t:schwarz_inner}.

\begin{table}[htbp]
  \centering
  \begin{tabular}{p{5.0cm} p{10.5cm}}
    \toprule
    Inner solver type       & Description \\
    \midrule
    ``DIAGONAL''            & Diagonal scaling \\
    ``RELAXATION''          & Point relaxation (see~\ref{s:relaxation}) \\
    ``BLOCK\_RELAXATION''   & Block relaxation (see~\ref{s:block_relaxation}) \\
    ``CHEBYSHEV''           & Chebyshev iteration (see~\ref{s:Chebyshev}) \\
    ``RILUK''               & ILU($k$) (see~\ref{s:ILU}) \\
    ``ILUT''                & ILUT (see~\ref{s:ILUT}) \\
    ``FAST\_ILU''             & FastILU (see~\ref{s:FastILU}) \\
    ``FAST\_IC''              & FastIC (see~\ref{s:FastILU}) \\
    ``FAST\_ILDL''            & FastILDL(see~\ref{s:FastILU}) \\
    ``AMESOS2''             & \amesostwo's interface to sparse direct solvers \\
    ``DENSE'' or ``LAPACK'' & LAPACK's LU factorization for a dense representation of a subdomain matrix \\
    ``CUSTOM''              & User provided inner solver \\
    % ``RBILUK''
    \bottomrule
  \end{tabular}
  \caption{\label{t:schwarz_inner}Additive Schwarz solver preconditioner types.}
\end{table}

The following parameters are used in the Schwarz method:

\ccc{schwarz: inner preconditioner name}
    {string}
    {none}
    {The name of the subdomain solver.}
\cccc{inner preconditioner name}
    {Same as {\tt "schwarz: inner preconditioner name"}.}
\cccc{schwarz: subdomain solver name}
    {Same as {\tt "schwarz: inner preconditioner name"}.}
\cccc{subdomain solver name}
    {Same as {\tt "schwarz: inner preconditioner name"}.}
\ccc{schwarz: inner preconditioner parameters}
    {\parameterlist}
    {empty}
    {Parameters for the subdomain solver. If not provided, the subdomain solver
     will use its specific default parameters.}
\cccc{inner preconditioner parameters}
    {Same as {\tt "schwarz: inner preconditioner parameters"}.}
\cccc{schwarz: subdomain solver parameters}
    {Same as {\tt "schwarz: inner preconditioner parameters"}.}
\cccc{subdomain solver parameters}
    {Same as {\tt "schwarz: inner preconditioner parameters"}.}
\ccc{schwarz: combine mode}
    {string}
    {``ZERO''}
    {The rule for combining incoming data with existing data in overlap regions.
     Accepted values: see Table~\ref{t:combine_mode}.}
\ccc{schwarz: overlap level}
    {int}
    {0}
    {The level of overlap (0 corresponds to no overlap).}
\ccc{schwarz: num iterations}
    {int}
    {1}
    {Number of iterations to perform.}
\ccc{schwarz: use reordering}
    {bool}
    {\false}
    {If true, local matrix is reordered before computing subdomain solver. \trilinos must have been built with
     \zoltantwo and \xpetra enabled.}
\ccc{schwarz: reordering list}
    {\parameterlist}
    {empty}
    {Specify options for a \zoltantwo reordering algorithm to use. See {\tt
     "order\_method"}. {\em You should consider the
     \href{http://trilinos.org/docs/dev/packages/zoltan2/doc/html/z2_parameters.html}{\zoltantwo
     documentation} to be the final authority.}}
\ccc{order\_method}
    {string}
    {``rcm''}
    {Reordering algorithm. Accepted values: ``rcm'', ``minimum\_degree'',
     ``natural'', ``random'', or ``sorted\_degree''. Only used in {\tt
     "schwarz: reordering list"} sublist.}
\cccc{schwarz: zero starting solution}
    {See {\tt "relaxation: zero starting solution"}.}
\ccc{schwarz: filter singletons}
    {bool}
    {\false}
    {If true, exclude rows with just a single entry on the calling process.}
\cccc{schwarz: subdomain id}
    {Currently has no effect.}
\cccc{schwarz: compute condest}
    {Currently has no effect. For backwards compatibility only.}
\ccc{schwarz: update damping}
    {double}
    {1.0}
    {The amount by which to damp the updates from the Schwarz solve
      (1.0 is no damping).}
\section{Hiptmair}

\ifpacktwo{} implements Hiptmair algorithm of~\cite{Hiptmair1997}. The method
operates on two spaces: a primary space and an auxiliary space. This situation
arises, for instance,  when preconditioning Maxwell's equations discretized by
edge elements. It is used in \muelu~\cite{MueLu} ``RefMaxwell''
solver~\cite{RefMaxwell}.

Hiptmair's algorithm does not use \texttt{Ifpack2::Factory} interface for
construction.  Instead, a user must explicitly call the constructor
\begin{lstlisting}[language=C++]
  Teuchos::RCP<Tpetra::CrsMatrix<> > A, Aaux, P;
  // create A, Aaux, P here ...
  Teuchos::ParameterList paramList;
  paramList.set("hiptmair: smoother type 1", "CHEBYSHEV");
  // ...
  RCP<Ifpack2::Ifpack2Preconditioner<> > ifpack2Preconditioner =
    Teuchos::rcp(new Ifpack2::Hiptmair(A, Aaux, P);
  ifpack2Preconditioner->setParameters(paramList);
\end{lstlisting}
\noindent Here, $A$ is a matrix in the primary space, $Aaux$ is a matrix in
auxiliary space, and $P$ is a prolongator/restrictor between the two spaces.

The following parameters are used in the Hiptmair method:

\ccc{hiptmair: smoother type 1}
    {string}
    {"CHEBYSHEV"}
    {Smoother type for smoothing the primary space.}
\ccc{hiptmair: smoother list 1}
    {\parameterlist}
    {empty}
    {Smoother parameters for smoothing the primary space.}
\ccc{hiptmair: smoother type 2}
    {string}
    {"CHEBYSHEV"}
    {Smoother type for smoothing the auxiliary space.}
\ccc{hiptmair: smoother list 2}
    {\parameterlist}
    {empty}
    {Smoother parameters for smoothing the auxiliary space.}
\ccc{hiptmair: pre or post}
    {string}
    {``both''}
    {\ifpacktwo{} always relaxes on the auxiliary space. ``pre'' (``post'') means
     that it relaxes on the primary space before (after) the relaxation on the
     auxiliary space. ``both'' means that we do both ``pre'' and ``post''.}
\cccc{hiptmair: zero starting solution}
    {See {\tt "relaxation: zero starting solution"}.}


    %-----------------------------%
    \chapter{\muemex: The MATLAB Interface for \muelu} \label{sec:muemex}
    %-----------------------------%
    
%%%%%%%%%%%%%%%%%%%%%%%%%%%%%%%%%%%%%%%%%%%%%%%%%%%%%%%%%%%%%%%%%%%%
\muemex is \muelu's interface to the MATLAB environment. It allows access
to a limited set of routines either \muelu as a preconditioner,
Belos as a solver and Epetra or Tpetra for data structures.
It is designed to provide access to \muelu's aggregation and
solver routines from MATLAB and does little else. \muemex allows users to
setup and solve arbitrarily many problems, so long as memory suffices.
More than one problem can be set up simultaneously.

\section{CMake Configure and Make}\label{sec:muemex:cmake}
To use \muemex, Trilinos must be configured with (at least) the
following options:

\begin{lstlisting}
  export TRILINOS_HOME=/path/to/your/Trilinos/source/directory
  cmake \
      -D Trilinos_ENABLE_EXPLICIT_INSTANTIATION:BOOL=ON  \
      -D Trilinos_ENABLE_Amesos:BOOL=ON \
      -D Trilinos_ENABLE_Amesos2:BOOL=ON \
      -D Amesos2_ENABLE_KLU2:BOOL=ON \
      -D Trilinos_ENABLE_AztecOO:BOOL=ON \
      -D Trilinos_ENABLE_Epetra:BOOL=ON \
      -D Trilinos_ENABLE_EpetraExt:BOOL=ON \
      -D Trilinos_ENABLE_Fortran:BOOL=OFF \
      -D Trilinos_ENABLE_Ifpack:BOOL=ON \
      -D Trilinos_ENABLE_Ifpack2:BOOL=ON \
      -D Trilinos_ENABLE_MueLu:BOOL=ON \
      -D Trilinos_ENABLE_Teuchos:BOOL=ON \
      -D Trilinos_ENABLE_Tpetra:BOOL=ON \
      -D TPL_ENABLE_MPI:BOOL=OFF \
      -D TPL_ENABLE_MATLAB:BOOL=ON \
      -D MATLAB_ROOT:STRING="<my matlab root>" \
      -D MATLAB_ARCH:STRING="<my matlab os string>" \
      -D Trilinos_EXTRA_LINK_FLAGS="-lrt -lm -lgfortran" \
  ${TRILINOS_HOME}
\end{lstlisting}

Since \muemex supports both the Epetra and Tpetra linear algebra
libraries, you have to have both enabled in order to build \muemex.
\begin{mycomment}
If you turn off either Epetra or Tpetra then you will run into an error message: \textit{MueMex requires Epetra, Tpetra and MATLAB}.
\end{mycomment}

Most additional options can be specified as well.  It is important to
note that \muemex does not work properly with MPI, hence MPI must be
disabled in order to compile \muemex.  The MATLAB\_ARCH option is new to
the cmake build system, and involves the MATLAB-specific architecture
code for your system.  There is currently no automatic way to extract
this, so it must be user-specified.  As of MATLAB 7.9 (R2009b), common
arch codes are:
\begin{center}
\begin{tabular}{l|l}
Code& OS\\
\hline
glnx86& 32-bit Linux (intel/amd)\\
glnxa64& 64-bit Linux (intel/amd)\\
maci64& 64-bit MacOS\\
maci& 32-bit MacOS\\
\end{tabular}
\end{center}

On 64-bit Intel/AMD architectures, Trilinos and all relevant TPLs
(note: this includes BLAS and LAPACK)
must be compiled with the \texttt{-fPIC} option.  This necessitates adding:
\begin{lstlisting}
    -D CMAKE_CXX_FLAGS:STRING="-fPIC" \
    -D CMAKE_C_FLAGS:STRING="-fPIC" \
    -D CMAKE_Fortran_FLAGS:STRING="-fPIC" \
\end{lstlisting}
to the cmake configure line.

The additional linker flags specified in \texttt{Trilinos\_EXTRA\_LINK\_FLAGS} may slightly vary depending on the system and the exact configuration. But the given parameters may work for most Linux based systems.
If you encounter an error message like \textit{Target "muemex.mexa64" links to item "-Wl,-rpath-link,/opt/matlab/bin/glnxa64 " which has leading or trailing whitespace.} you have to add some options to the \texttt{Trilinos\_EXTRA\_LINK\_FLAGS} variable. At least adding \texttt{-lm} should be safe and fix the error message.

\subsection{BLAS \& LAPACK Option \#1: Static Builds}
Trilinos does not play nicely with MATLAB's default LAPACK and BLAS on
64-bit machines.
If \muemex randomly crashes when you run with any Krylov method that
has orthogonalization, chances are \muemex is finding the wrong
BLAS/LAPACK libraries.
This leaves you
with one of two options.  The first is to build them both \textit{statically}
and then specify them as follows:
\begin{lstlisting}
    -D LAPACK_LIBRARY_DIRS:STRING="<path to my lapack.a>" \
    -D BLAS_LIBRARY_DIRS:STRING="<path to my blas.a>" \
\end{lstlisting}
Using static linking for LAPACK and BLAS prevents MATLAB's default libraries to take precedence.

\subsection{BLAS \& LAPACK Option \#2: LD$\_$PRELOAD}
\label{sec:preload}
The second option is to use \texttt{LD\_PRELOAD} to tell MATLAB exactly
which libraries to use.  For this option, you can use the dynamic
libraries installed on your system.
Before starting MATLAB, set
\texttt{LD\_PRELOAD} to the paths of libstdc++.so corresponding to the version of GCC used
to build Trilinos, and the paths of libblas.so and liblapack.so on your local system.

For example, if you use bash, you'd do something like this
\begin{lstlisting}
  export LD_PRELOAD=<path>/libstdc++.so:<path>/libblas.so:<path>/liblapack.so
  \end{lstlisting}

For csh / tcsh, do this
\begin{lstlisting}
  setenv LD_PRELOAD <path>/libstdc++.so:<path>/libblas.so:<path>/liblapack.so
\end{lstlisting}

\subsection{Running MATLAB}

Before you run MATLAB you have to make sure that MATLAB is using the same libraries that have been used for compiling \muemex.
This includes the \texttt{libstdc++.so} and depending whether you turned on/off fortran also \texttt{libgfortran.so}. Please make sure that the correct libraries and paths are declared in the \texttt{LD\_PRELOAD} variable. You can refer to section \ref{sec:preload} to see how the \texttt{LD\_PRELOAD} variable is set.

For a 64 bit Linux system using the bash the command should look like

\begin{lstlisting}
export LD_PRELOAD=/usr/lib64/libstdc++.so.6:/usr/lib64/libgfortran.so.3:$LD_PRELOAD
\end{lstlisting}

to add the \texttt{libstdc++.so} and \texttt{libgfortran.so} to the existing \texttt{LD\_PRELOAD} variable. Then run the MATLAB executable in the same shell window.

\begin{mycomment}
Note, that this step is necessary even if you statically linked BLAS and LAPACK.
\end{mycomment}

If you are unsure which libraries have to be set in the \texttt{LD\_PRELOAD} variable you will find out latest if you start MATLAB and try to run \muemex. It will throw some error messages with the missing library names. For a 64 bit Linux system the standard libraries usually can be found in \texttt{/usr/lib64} or \texttt{/usr/lib} (for a 32 bit system).

\section{Using \muemex}\label{sec:muemex:usage}
\muemex is designed to be interfaced with via the MATLAB script
\texttt{muelu.m}.  There are five modes in which \muemex can be run:
\begin{enumerate}
\item Setup Mode --- Performs the problem setup for \muelu.
  Depending on whether or not the \texttt{Linear Algebra} option is
  used, \muemex creates either an unpreconditioned Epetra problem,
  an Epetra problem with \muelu, or a Tpetra problem with \muelu.
  The default is \texttt{tpetra}. The \texttt{epetra} mode only supports
  real-valued matrices, while \texttt{tpetra}
  supports both real and complex and will infer the scalar type
  from the matrix passed during setup.  This call returns a problem
  handle used to reference the problem in the future, and (optionally)
  the operator complexity, if a preconditioner is being used.
\item Solve Mode --- Given a problem handle and a right-hand side, \muemex
  solves the problem specified.  Setup mode must be called before
  solve mode.
\item Cleanup Mode --- Frees the memory allocated to internal \muelu,
  Epetra and Tpetra objects.  This can be called with a particular
  problem handle, in which case it frees that problem, or without one,
  in which case all \muemex memory is freed.
\item Status Mode --- Prints out status information on problems which
  have been set up.  Like cleanup, it can be called with or without a
  particular problem handle.
\item Get Mode --- Get information from a MueLu hierarchy that has been
  generated. Given the problem handle, a level number and the name of the
  field, returns the appropriate array or scalar as a MATLAB object.
\end{enumerate}
All of these modes, with the exception of status and cleanup take
option lists which will be directly converted into
\texttt{Teuchos::ParameterList} objects by \muemex, as key-value pairs.
Options passed during setup will apply to the \muelu preconditioner, and
options passed during a solve will apply to Belos.

\subsection{Setup Mode}
Setup mode is called as follows:
\begin{lstlisting}[language=Matlab]
  >> [h, oc] = muelu('setup', A[, 'parameter', value,...])
\end{lstlisting}
The parameter \texttt{A} represents the sparse matrix to perform aggregation on
and the parameter/value pairs represent standard \muelu options.

The routine returns a problem handle, \texttt{h}, and the operator
complexity \texttt{oc} for the operator.  In addition to the standard
options, setup mode has one unique option of its own:

\choicebox{\tt Linear Algebra}{[{\tt string}] Whether to use
  'epetra unprec', 'epetra', or 'tpetra'. Default is 'epetra' for
  real matrix and 'tpetra' for complex matrix.}

\subsection{Solve Mode}
Solve mode is called as follows:
\begin{lstlisting}[language=Matlab]
  >> [x, its] = muelu(h[, A], b[, 'parameter', value,...])
\end{lstlisting}
The parameter \texttt{h} is a problem handle returned by the
setup mode call, \texttt{A} is the sparse matrix with which to
solve and \texttt{b} is the right-hand side.  Parameter/value pairs
to configure the Belos solver are listed as above. If A is not supplied,
the matrix provided when setting up the problem will be used. \texttt{x} is
the solution multivector with the same dimensions as \texttt{b}, and \texttt{its}
is the number of iterations Belos needed to solve the problem.

All of these options are taken directly from Belos, so consult its
manual for more information. Belos output style and verbosity settings
are implemented as enums, but can be set as strings in \muemex. For example:

\begin{lstlisting}[language=Matlab]
  >> x = muelu(0, b, 'Verbosity', 'Warnings + IterationDetails', ...
                       'Output Style', 'Brief');
\end{lstlisting}

Verbosity settings can be separated by spaces, '+' or ','. Belos::Brief
is the default output style.

\subsection{Cleanup Mode}
Cleanup mode is called as follows:
\begin{lstlisting}[language=Matlab]
  >> muelu('cleanup'[, h])
\end{lstlisting}
The parameter \texttt{h} is a problem handle returned by the
setup mode call and is optional.  If \texttt{h} is provided, that
problem is cleaned up.  If the option is not provided all currently
set up problems are cleaned up.

\subsection{Status Mode}
Status mode is called as follows:
\begin{lstlisting}[language=Matlab]
  >> muelu('status'[, h])
\end{lstlisting}
The parameter \texttt{h} is a problem handle returned by the
setup mode call and is optional.  If \texttt{h} is provided, status
information for that problem is printed.  If the option is not provided all currently
set up problems have status information printed.

\subsection{Get Mode}
Get mode is called as follows:
\begin{lstlisting}[language=Matlab]]
  >> muelu('get', h, level, fieldName[, typeHint])
\end{lstlisting}
The parameter \texttt{h} is the problem handle, and \texttt{level}
is an integer that identifies the level within the hierarchy containing
the desired data. \texttt{fieldName} is a string that identifies the
field within the level, e.g. 'Nullspace'. \texttt{typeHint} is an optional
parameter that tells MueMex what data type to expect from the level. This
is a string, with possible values 'matrix', 'multivector', 'lovector' (ordinal
vector), or 'scalar'. MueMex will attempt to guess the type from \texttt{fieldName}
but \texttt{typeHint} may be required.

\subsection{Tips and Tricks }\label{sec:muemex:tips}

Internally, MATLAB represents all data as doubles unless you go
through efforts to do otherwise.  \muemex detects integer parameters by
a relative error test, seeing if the relative difference between the
value from MATLAB and the value of the \texttt{int}-typecast value are
less than 1e-15.  Unfortunately, this means that \muemex will choose the
incorrect type for parameters which are doubles that happen to have an
integer value (a good example of where this might happen would be the parameter
`smoother Chebyshev: alpha', which defaults to 30.0).  Since \muemex does no
internal typechecking of
parameters (it uses \muelu's internal checks), it has no way of detecting
this conflict.  From the user's perspective, avoiding this is as
simple as adding a small perturbation (greater than a relative 1e-15)
to the parameter that makes it non-integer valued.


%%% Local Variables:
%%% mode: latex
%%% TeX-master: "mueluguide"
%%% End:


    %-----------------------------%
    %\chapter{YAML Parameter Lists}\label{sec:yaml}
    %-----------------------------%
    %YAML is a human-readable data serialization format. MueLu provides a
YAML parameter list interpreter. It produces Teuchos::ParameterList
objects equivalent to those produced by the Teuchos XML helper functions.

Here is a simple example XML parameter list:
\begin{verbatim}
<ParameterList>
  <ParameterList Input>
    <Parameter name="values" type="Array(double)" value="{54.3 -4.5 2.0}"/>
    <Parameter name="myfunc" type="string" value="
def func(a, b):
  return a * 2 - b"/>
  </ParameterList>
  <ParameterList Solver>
    <Parameter name="iterations" type="int" value="5"/>
    <Parameter name="tolerance" type="double" value="1e-7"/>
    <Parameter name="do output" type="bool" value="true"/>
    <Parameter name="output file" type="string" value="output.txt"/>
  </ParameterList>
</ParameterList>
\end{verbatim}

Here is an equivalent YAML parameter list:
\begin{verbatim}
%YAML 1.1
---
ANONYMOUS:
  Input:
    values: [54.3, -4.5, 2.0]
    myfunc: |-

      def func(a, b):
        return a * 2 - b
  Solver:
    iterations: 5
    tolerance: 1e-7
    do output: yes
    output file: output.txt
...
\end{verbatim}

The nested structure and key-value pairs of these two lists are identical.
To a program querying them for settings, they are indistinguishable.

These are the general rules for creating a YAML parameter list:
\begin{itemize}
\item First line must be ``\%YAML 1.1'', second must be ``---'', and last must be ``...''
\item Nested map structure is determined by indentation. SPACES ONLY, NO TABS!
\item As with the above example, for a top-level anonymous parameter list, ``ANONYMOUS:'' must be explicit
\item Type is inferred. 5 is an int, 5.0 is a double, and '5.0' is a string
\item Quotation marks (single or double) are optional for strings, but required for strings with special characters: \verb.:{}[],&*#?|-<>=!%@\.
\item Quotation marks also turn non-string types into strings: '3' is a string
\item As with XML parameter lists, keys are regular strings
\item Even though YAML supports several names for bool true/false, only ``true'' and ``false'' are supported by the parameter list reader.
\item Arrays of int, double and string supported. exampleArray: {[}hello, world, goodbye{]}
\item {[}3, 4, 5{]} is an int array, {[}3, 4, 5.0{]} is a double array, and {[}3, '4', 5.0{]} is a string array
\item For multi-line strings, place ``$|-$'' after the ``key:'' and then indent the string one level deeper than the key
\item To preserve indentation in a multiline string, place ``$|2-$'' and then indent your string's content by 2 spaces relative to the key.
\end{itemize}


    %\nocite{*}

    % ---------------------------------------------------------------------- %
    % References
    %
    \clearpage
    % If hyperref is included, then \phantomsection is already defined.
    % If not, we need to define it.
    \providecommand*{\phantomsection}{}
    \phantomsection
    \addcontentsline{toc}{chapter}{References}
    \bibliographystyle{plain}
    \bibliography{mueluguide}


    % ---------------------------------------------------------------------- %
    %
    \appendix
    \chapter{Copyright and License}
    \label{sec:license}
\begin{center}
MueLu: A package for multigrid based preconditioning

Copyright 2012 Sandia Corporation
\end{center}

\noindent
Under the terms of Contract DE--AC04--94AL85000 with Sandia Corporation,
the U.S. Government retains certain rights in this software.

\noindent
Redistribution and use in source and binary forms, with or without
modification, are permitted provided that the following conditions are
met:

\begin{enumerate}
  \item Redistributions of source code must retain the above copyright
    notice, this list of conditions and the following disclaimer.

\item Redistributions in binary form must reproduce the above copyright
  notice, this list of conditions and the following disclaimer in the
  documentation and/or other materials provided with the distribution.

\item Neither the name of the Corporation nor the names of the
  contributors may be used to endorse or promote products derived from
  this software without specific prior written permission.
\end{enumerate}

\noindent
THIS SOFTWARE IS PROVIDED BY SANDIA CORPORATION ``AS IS'' AND ANY
EXPRESS OR IMPLIED WARRANTIES, INCLUDING, BUT NOT LIMITED TO, THE
IMPLIED WARRANTIES OF MERCHANTABILITY AND FITNESS FOR A PARTICULAR
PURPOSE ARE DISCLAIMED\@. IN NO EVENT SHALL SANDIA CORPORATION OR THE
CONTRIBUTORS BE LIABLE FOR ANY DIRECT, INDIRECT, INCIDENTAL, SPECIAL,
EXEMPLARY, OR CONSEQUENTIAL DAMAGES (INCLUDING, BUT NOT LIMITED TO,
PROCUREMENT OF SUBSTITUTE GOODS OR SERVICES\@; LOSS OF USE, DATA, OR
PROFITS\@; OR BUSINESS INTERRUPTION) HOWEVER CAUSED AND ON ANY THEORY OF
LIABILITY, \\WHETHER IN CONTRACT, STRICT LIABILITY, OR TORT (INCLUDING
NEGLIGENCE OR OTHERWISE) ARISING IN ANY WAY OUT OF THE USE OF THIS
SOFTWARE, EVEN IF ADVISED OF THE POSSIBILITY OF SUCH DAMAGE\@.


%%% Local Variables:
%%% mode: latex
%%% TeX-master: "mueluguide"
%%% End:

    %\chapter{Historical Perspective}
	%\input{CommonHistory}

    \chapter{ML compatibility}
    
\label{sec:ml_options}
\muelu provides a basic compatibility layer for \ml parameter lists. This allows \ml users to quickly perform some experiments with \muelu.

\textbf{First and most important: } Long term, we would like to have users use the new \muelu interface, as that is where most of new features will be made accessible. One should make note of the fact that it may not be possible to make ML deck do exactly same things in \ml and \muelu, as internally \ml implicitly makes some decision that we have no control over and which could be different from \muelu.

\noindent There are basically two distinct ways to use \ml input parameters with \muelu:
\begin{description}
\item[MLParameterListInterpreter:] This class is the pendant of the \texttt{ParameterListInterpreter} class for the \muelu parameters. It accepts parameter lists or XML files with \ml parameters and generates a \muelu multigrid hierarchy. It supports only a well-defined subset of \ml parameters which have a equivalent parameter in \muelu.
\item[ML2MueLuParameterTranslator:] This class is a simple wrapper class which translates \ml parameters to the corresponding \muelu parameters. It has to be used in combination with the \muelu \texttt{ParameterListInterpreter} class to generate a \muelu multigrid hierarchy. It is also meant to be used in combination with the \texttt{CreateEpetraPreconditioner} and \texttt{CreateTpetraPreconditioner} routines (see \S\ref{sec:examples in code}). It supports only a small subset of the \ml parameters.
\end{description}

\section{Usage of \ml parameter lists with \muelu}

\subsection{MLParameterListInterpreter}

The \texttt{MLParameterListInterpreter} directly accepts a \texttt{ParameterList} containing \ml parameters. It also interprets the \texttt{null space: vectors} and the \texttt{null space: dimension} \ml parameters. However, it is recommended to provide the near null space vectors directly in the \muelu way as shown in the following code snippet.

\begin{lstlisting}[language=C++]
    Teuchos::RCP<Tpetra::CrsMatrix<> > A;
    // create A here ...

    // XML file containing ML parameters
    std::string xmlFile = "mlParameters.xml"
    Teuchos::ParameterList paramList;
    Teuchos::updateParametersFromXmlFileAndBroadcast(xmlFile, Teuchos::Ptr<Teuchos::ParameterList>(&paramList), *comm);

    // use ParameterListInterpreter with MueLu parameters as input
    Teuchos::RCP<HierarchyManager> mueluFactory = Teuchos::rcp(new MLParameterListInterpreter(*paramList));

    RCP<Hierarchy> H = mueluFactory->CreateHierarchy();
    H->GetLevel(0)->Set<RCP<Matrix> >("A", A);
    H->GetLevel(0)->Set("Nullspace", nullspace);
    H->GetLevel(0)->Set("Coordinates", coordinates);
    mueluFactory->SetupHierarchy(*H);
\end{lstlisting}

Note that the \texttt{MLParameterListInterpreter} only supports a basic set of \ml parameters allowing to build smoothed aggregation transfer operators (see \S\ref{sec:compatiblemlparameters} for a list of compatible \ml parameters).

\subsection{ML2MueLuParameterTranslator}

The \texttt{Ml2MueLuParameterTranslator} class is a simple wrapper translating \ml parameters to the corresponding \muelu parameters. This allows the usage of the simple \texttt{CreateEpetraPreconditioner} and \texttt{CreateTpetraPreconditioner} interface with \ml parameters:

\begin{lstlisting}[language=C++]
    Teuchos::RCP<Tpetra::CrsMatrix<> > A;
    // create A here ...

    // XML file containing ML parameters
    std::string xmlFile = "mlParameters.xml"
    Teuchos::ParameterList paramList;
    Teuchos::updateParametersFromXmlFileAndBroadcast(xmlFile, Teuchos::Ptr<Teuchos::ParameterList>(&paramList), *comm);

    // translate ML parameters to MueLu parameters
    RCP<ParameterList> mueluParamList = Teuchos::getParametersFromXmlString(MueLu::ML2MueLuParameterTranslator::translate(paramList,"SA"));

    Teuchos::RCP<MueLu::TpetraOperator> mueLuPreconditioner =
       MueLu::CreateTpetraPreconditioner(A, mueluParamList);
\end{lstlisting}

In a similar way, \ml input parameters can be used with the standard \muelu parameter list interpreter class. Note that the near null space vectors have to be provided in the \muelu way.

\begin{lstlisting}[language=C++]
    Teuchos::RCP<Tpetra::CrsMatrix<> > A;
    // create A here ...

    // XML file containing ML parameters
    std::string xmlFile = "mlParameters.xml"
    Teuchos::ParameterList paramList;
    Teuchos::updateParametersFromXmlFileAndBroadcast(xmlFile, Teuchos::Ptr<Teuchos::ParameterList>(&paramList), *comm);

    // translate ML parameters to MueLu parameters
    RCP<ParameterList> mueluParamList = Teuchos::getParametersFromXmlString(MueLu::ML2MueLuParameterTranslator::translate(paramList,"SA"));

    // use ParameterListInterpreter with MueLu parameters as input
    Teuchos::RCP<HierarchyManager> mueluFactory = Teuchos::rcp(new ParameterListInterpreter(*mueluParamList));

    RCP<Hierarchy> H = mueluFactory->CreateHierarchy();
    H->GetLevel(0)->Set<RCP<Matrix> >("A", A);
    H->GetLevel(0)->Set("Nullspace", nullspace);
    H->GetLevel(0)->Set("Coordinates", coordinates);
    mueluFactory->SetupHierarchy(*H);
\end{lstlisting}

Note that the set of supported \ml parameters is very limited. Please refer to \S\ref{sec:compatiblemlparameters} for a list of all compatible \ml parameters.

\section{Compatible \ml parameters}
\label{sec:compatiblemlparameters}
\subsection{General \ml options}

\mlcbb{ML output}{int}{0}{MLParameterListInterpreter, ML2MueLuParameterTranslator}{Control of the amount of printed information. Possible values: 0-10 with 0=no output and 10=maximum verbosity.}   
      
\mlcbb{PDE equations}{int}{1}{MLParameterListInterpreter, ML2MueLuParameterTranslator}{Number of PDE equations at each grid node. Only constant block size is considered.}   
       
\mlcbb{max levels}{int}{10}{MLParameterListInterpreter, ML2MueLuParameterTranslator}{Maximum number of levels in a hierarchy.}   
      
\mlcbb{prec type}{string}{"MGV"}{MLParameterListInterpreter, ML2MueLuParameterTranslator}{Multigrid cycle type. Possible values: "MGV", "MGW". Other values are NOT supported by MueLu.}   
      

\subsection{Smoothing and coarse solver options}

\mlcbb{smoother: type}{string}{"symmetric Gauss-Seidel"}{MLParameterListInterpreter, ML2MueLuParameterTranslator}{Smoother type for fine- and intermedium multigrid levels. Possible values: "Jacobi", "Gauss-Seidel", "symmetric Gauss-Seidel", "Chebyshev", "ILU".}

\mlcbb{smoother: sweeps}{int}{2}{MLParameterListInterpreter, ML2MueLuParameterTranslator}{Number of smoother sweeps for relaxation based level smoothers. In case of Chebyshev smoother it denotes the polynomial degree.}

\mlcbb{smoother: damping factor}{double}{1.0}{MLParameterListInterpreter, ML2MueLuParameterTranslator}{Damping factor for relaxation based level smoothers.}

\mlcbb{smoother: Chebyshev alpha}{double}{20}{MLParameterListInterpreter, ML2MueLuParameterTranslator}{Eigenvalue ratio for Chebyshev level smoother.}

 
\mlcbb{smoother: pre or post}{string}{"both"}{MLParameterListInterpreter, ML2MueLuParameterTranslator}{Pre- and post-smoother combination. Possible values: "pre" (only pre-smoother), "post" (only post-smoother), "both" (both pre-and post-smoothers).}   
      
\mlcbb{max size}{int}{128}{MLParameterListInterpreter, ML2MueLuParameterTranslator}{Maximum dimension of a coarse grid. \ml will stop coarsening once it is achieved.}   
      

\mlcbb{coarse: type}{string}{"Amesos-KLU"}{MLParameterListInterpreter, ML2MueLuParameterTranslator}{Solver for coarsest level. Possible values: "Amesos-KLU", "Amesos-Superlu" (depending on \muelu installation).}


\subsection{Transfer operator options}

\mlcbb{energy minimization: enable}{int}{0}{MLParameterListInterpreter, ML2MueLuParameterTranslator}{Enable energy minimization transfer operators (using Petrov-Galerkin approach).}   
      
\mlcbb{aggregation: damping factor}{double}{1.33}{MLParameterListInterpreter, ML2MueLuParameterTranslator}{Damping factor for smoothed aggregation.}   
      

\subsection{Rebalancing options}

\mlcbb{repartition: enable}{int}{0}{MLParameterListInterpreter}{Rebalancing on/off switch. Only limited support for repartitioning. Does not use provided node coordinates.}   
      
\mlcbb{repartition: start level}{int}{1}{MLParameterListInterpreter}{Minimum level to run partitioner. \muelu does not rebalance levels finer than this one.}   
      
\mlcbb{repartition: min per proc}{int}{512}{MLParameterListInterpreter}{Minimum number of rows per MPI process. If the actual number if smaller, then rebalancing occurs.}   
      
\mlcbb{repartition: max min ratio}{double}{1.3}{MLParameterListInterpreter}{Maximum nonzero imbalance ratio. If the actual number is larger, the rebalancing occurs.}   
      

%%% Local Variables:
%%% mode: latex
%%% TeX-master: "mueluguide"
%%% End:


    %\chapter{Some Other Appendix}
	%\input{CommonAppendix}

    % \printindex

    %
% This is an example of how to create the distribution page. Some
% distributions are required by Sandia; e.g. the housekeeping copies.
% Depending on the type of report; e.g. CRADA, Patent Caution, etc.
% additional distribution lines may have to be added. See the
% "Guide for Preparing SAND Reports"
%
% SANDdistribution takes CA or NM as an optional argument. If given,
% the approrpiate housekeeping copies are inserted autmatically.
% Inside the SANDdistribution environment, several commands can be used
% insert the distributions for CRADA, LDRD, etc. See example below.
%
% You can leave the CA or NM option off and not use any of the SANDdist*
% commands. This will allow you to create a distribution list manually.
%
\begin{SANDdistribution}[NM]
    % Housekeeping copies necessary for every unclassified report:
    % \SANDdistCRADA	% If this report is about CRADA work
    % \SANDdistPatent	% If this report has a Patent Caution or Patent Interest
    % \SANDdistLDRD	% If this report is about LDRD work

    % Some external Addresses
    %\SANDdistExternal{1}{An Address\\ 99 $99^{th}$ street NW\\City, State}
    %\SANDdistExternal{3}{Some Address\\ and street\\City, State}
    %\SANDdistExternal{12}{Another Address\\ On a street\\City, State\\U.S.A.}
    \bigskip


    % The following MUST BE between the external and internal distributions!
    % \SANDdistClassified % If this report is classified


    % Internal Addresses
    \SANDdistInternal{1}{1320}{Michael Heroux}{1426}
    \SANDdistInternal{1}{1318}{Robert Hoekstra}{1423}
    \SANDdistInternal{1}{1318}{Erik Strack}{1426}
    \SANDdistInternal{1}{1320}{Mark Hoemmen}{1426}
    \SANDdistInternal{1}{1320}{Alicia Klinvex}{1426}
    \SANDdistInternal{1}{1320}{Paul Lin}{1426}
    \SANDdistInternal{1}{1318}{Andrey Prokopenko}{1426}
    \SANDdistInternal{1}{1322}{Christopher Siefert}{1443}

\end{SANDdistribution}

\begin{SANDdistribution}[CA]
    \SANDdistInternal{1}{9159}{Jonathan Hu}{1426}
    \SANDdistInternal{1}{9159}{Raymond Tuminaro}{1442}
\end{SANDdistribution}


\end{document}
